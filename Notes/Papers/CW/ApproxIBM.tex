\documentclass[11pt,leqno]{amsart}

%-------------------------------------------------------------------------
% Packages
%-------------------------------------------------------------------------
\usepackage{amsmath, amssymb, amsthm}
\usepackage{geometry}
\usepackage{mathrsfs}
\usepackage{enumitem}
\usepackage{times} % A font closer to the classic Annals look

%-------------------------------------------------------------------------
% Geometry / Annals-like formatting
%-------------------------------------------------------------------------
\geometry{
  body={6.5in, 9in},
  left=1in,
  top=1in
}

%-------------------------------------------------------------------------
% Environments
%-------------------------------------------------------------------------
\newtheorem{theorem}{Theorem}[section]
\newtheorem{proposition}[theorem]{Proposition}
\newtheorem{lemma}[theorem]{Lemma}
\newtheorem{corollary}[theorem]{Corollary}
\newtheorem{conjecture}[theorem]{Conjecture}

\theoremstyle{definition}
\newtheorem{definition}[theorem]{Definition}
\newtheorem{remark}[theorem]{Remark}

%-------------------------------------------------------------------------
% Macros
%-------------------------------------------------------------------------
\newcommand{\R}{\mathbb{R}}
\newcommand{\C}{\mathbb{C}}
\newcommand{\Z}{\mathbb{Z}}
\newcommand{\E}{\mathbb{E}}
\renewcommand{\P}{\mathbb{P}}
\newcommand{\Rea}{\mathrm{Re}}
\newcommand{\Cov}{\mathrm{Cov}}

\title[Gaussian Approximation for Dirichlet Polynomials]{Gaussian Approximation and Block Berry--Esseen for Arithmetic Dirichlet Polynomials Attached to the Riemann Zeta Function}

\author{Matteo Cipollina}
\author{Jonathan Washburne}
\address{Recognition Physics Institute}
\email{mcipollina@recognitionphysics.com}

\begin{document}

\begin{abstract}
We consider the Dirichlet polynomial approximation
\[
X_T^{\zeta}(h)
:= \Rea\sum_{p\le T^\theta} p^{-1/2-i(\tau+h)},\qquad \tau\sim\mathrm{Unif}[T,2T],\ |h|\le 1,
\]
to $\log|\zeta(1/2+i(\tau+h))|$, and its multiscale block decomposition according to the size of $\log p$, as developed by Fyodorov--Hiary--Keating and Arguin--Bourgade--Radziwiłł (ABR/FHK).

Our main result is a \textbf{finite--dimensional, blockwise Gaussian universality theorem in the arithmetic model}: for each fixed scale $k$ in the ABR/FHK decomposition and each finite set of offsets $h_0,\dots,h_m\in[-1,1]$, the joint law of the arithmetic block vector
\[
\mathbf Y_k^{\zeta}
:=\big(Y_k^{\zeta}(h_0),\dots,Y_k^{\zeta}(h_m)\big),\qquad
Y_k^{\zeta}(h)
:=\sum_{e^{k-1}<\log p\le e^{k}}\Rea\big(p^{-1/2-i(\tau+h)} + \tfrac12 p^{-1-2i(\tau+h)}\big),
\]
under the $\tau$--average, is close—in rectangle distance and in the smooth test--function metric—to that of a Gaussian vector with matching covariance. The error decays like $e^{-c e^k}$ in the block index $k$, and is summable over scales.

The proof proceeds in three steps:
\begin{enumerate}
    \item A multivariate Berry--Esseen bound for the \textbf{prime--phase} model, where $p^{-i(\tau+h)}$ is replaced by an independent phase $Z_p p^{-ih}$.
    \item A representation of smoothed block indicators as squared moduli of Dirichlet polynomials of length $\le T^{1/100}$ using the multiscale smoothing and length estimates of ABR/FHK.
    \item A mean--value theorem for Dirichlet polynomials that transfers expectations of these smoothed observables from the arithmetic model to the prime--phase model with error $O(T^{-c})$.
\end{enumerate}

As a corollary, we obtain a \textbf{blockwise arithmetic Gaussian approximation} strong enough to control any finite-dimensional marginal of a \emph{single block} at any fixed scale.

We then prove a correct \textbf{approximate Gaussian integration--by--parts (IBP) identity} for the \textbf{random (prime--phase)} model, using independence of blocks and the block Berry--Esseen. Finally, we formulate an explicit global Dirichlet--polynomial representation conjecture under which the same IBP identity would hold for the arithmetic field $X_T^{\zeta}$. This isolates precisely what additional analytic input is needed to import Guerra/Parisi spin--glass techniques (stochastic stability, Ghirlanda--Guerra identities) into the genuine arithmetic setting.
\end{abstract}

\maketitle

\section{Introduction}

Let $\zeta(s)$ be the Riemann zeta function, and consider its values on the critical line,
\[
s=\tfrac12+i(\tau+h),\qquad \tau\in[T,2T],\ h\in[-1,1].
\]
Over the last decade, a detailed probabilistic picture of the fluctuations of $\log|\zeta(1/2+i(\tau+h))|$ on such microscopic intervals has emerged. The works of Fyodorov--Hiary--Keating and Arguin--Bourgade--Radziwiłł (ABR/FHK) show that $\log|\zeta(1/2+i(\tau+h))|$ can be modelled by a log--correlated Gaussian field, and they use this to prove the Fyodorov--Hiary--Keating conjecture on its maxima on mesoscopic intervals.

A key technical component in ABR/FHK is a \textbf{multiscale block decomposition} of the prime sum in $\log\zeta$: primes are grouped by the size of $\log p$, and the resulting block increments behave approximately like independent Gaussian increments at the level of a \textbf{prime--phase model}, where $p^{-i(\tau+h)}$ is replaced by an independent random phase $Z_p p^{-ih}$. This decomposition underlies their analysis of the maximum and free energy.

In parallel, spin--glass theory—particularly the Guerra/Parisi interpolation and the Ghirlanda--Guerra identities—offers a powerful framework for understanding the extremal process and overlap structure of log--correlated fields. To import these tools into the zeta setting, one needs an \textbf{approximate Gaussian integration--by--parts (IBP)} identity for the relevant field, of the form
\[
\mathbb E\big[X_T(h_0)F(\mathbf X_T)\big]
\approx \sum_j \mathrm{Cov}(X_T(h_0),X_T(h_j))\,\mathbb E[\partial_jF(\mathbf X_T)],
\]
for a class of test functions $F$ and sampling points $h_j$.

The present paper does \textbf{not} prove such a global IBP for the arithmetic field. Instead, it establishes the precise \textbf{blockwise Gaussian universality} for the arithmetic Dirichlet polynomial that is the main analytic ingredient, and then shows:
\begin{itemize}
    \item how to obtain an approximate IBP in the \textbf{prime--phase model} using independence of blocks, and
    \item what exact \textbf{global Dirichlet--polynomial representation} would be needed to transfer this IBP to the arithmetic model.
\end{itemize}
In other words, we separate what is currently possible from what remains an open analytic problem, sharpening the interface between ABR/FHK and the Parisi/Guerra methodology.

\subsection{The arithmetic block increments}

Let $T$ be large, and let $\tau$ be uniform on $[T,2T]$. For some fixed $\theta>0$ (small), define the Dirichlet polynomial approximation to $\log|\zeta|$ on a short interval:
\[
X_T^{\zeta}(h)
:= \Rea\sum_{p\le T^\theta} p^{-1/2-i(\tau+h)},\qquad |h|\le 1.
\]

Following FHK/ABR, we decompose the prime sum into blocks indexed by $k$, corresponding to ranges of $\log p$. Set
\[
n := \lfloor \log\log T\rfloor,\quad n_0 := \lfloor y\rfloor,\quad n_L := n-n_0
\]
for some fixed $y\ge 1$. For $n_0<k\le n_L$, define
\[
\mathcal P_k := \{p:\ e^{k-1}<\log p\le e^k\},
\]
and the \textbf{arithmetic block increment}
\[
Y_k^{\zeta}(h)
:=
\sum_{p\in\mathcal P_k}
\Rea\Big(p^{-1/2-i(\tau+h)}+\tfrac12 p^{-1-2i(\tau+h)}\Big).
\]

Then
\[
X_T^{\zeta}(h)
= \sum_{k=n_0+1}^{n_L} Y_k^{\zeta}(h) + R_T^{\zeta}(h),
\]
where the tail $R_T^{\zeta}(h)$ is a Dirichlet polynomial supported on very small and very large primes and has variance $O(1)$. In particular, the main fluctuations are carried by the block increments $Y_k^{\zeta}$ for $k\in(n_0,n_L]$.

For a finite set of points $h_0,\dots,h_m\in[-1,1]$, we denote
\[
\mathbf Y_k^{\zeta}
:=
\big(Y_k^{\zeta}(h_0),\dots,Y_k^{\zeta}(h_m)\big)\in\mathbb R^{m+1}.
\]

\subsection{Main unconditional result: arithmetic block Gaussian approximation}

Our first (and main unconditional) theorem is:

\begin{theorem}[Arithmetic block Berry--Esseen]\label{thm:A}
Fix $m\ge 0$ and points $h_0,\dots,h_m\in[-1,1]$. For each block $k\in(n_0,n_L]$, let $\mathbf Y_k^{\mathrm{rand}}$ be the corresponding block in the \textbf{prime--phase model}, obtained by replacing $p^{-i(\tau+h)}$ with $Z_p p^{-ih}$ where $Z_p$ are independent uniform phases, and let $\mathbf N_k$ be a centred Gaussian vector with covariance
\[
\mathrm{Cov}(\mathbf N_k) = \mathrm{Cov}(\mathbf Y_k^{\mathrm{rand}}).
\]
Then there exist constants $c_m,C_m>0$ (depending only on $m$) such that, for all sufficiently large $T$, all blocks $k\in(n_0,n_L]$, and all axis-aligned rectangles $R\subset \mathbb R^{m+1}$,
\[
\Big|\mathbb P_\tau(\mathbf Y_k^{\zeta}\in R)
- \mathbb P(\mathbf N_k\in R)\Big|
\le C_m e^{-c_m e^k}.
\]
As a consequence, for any smooth test function $\Phi\in C^3(\mathbb R^{m+1})$ with polynomial growth and bounded derivatives up to order 3,
\[
\big|\mathbb E_\tau \Phi(\mathbf Y_k^{\zeta})
- \mathbb E \Phi(\mathbf N_k)\big|
\le C(m,h_\bullet,\Phi)\,e^{-c'_m e^k}.
\]
In particular, the errors are summable in $k$.
\end{theorem}

Theorem A is a genuine \textbf{blockwise Gaussian universality result for the arithmetic field} $X_T^{\zeta}$. It is not stated in ABR/FHK; its proof uses their smoothing and mean--value machinery but requires a multivariate Berry--Esseen at the block level plus careful length control for Dirichlet polynomial representations.

This is the central technical result of the paper.

\subsection{Random model IBP vs arithmetic IBP}

On top of Theorem A, one would like an approximate \textbf{Gaussian integration--by--parts} identity for the full field. In this direction we prove:

\begin{theorem}[Approximate IBP in the prime--phase model]\label{thm:B_rand}
For the prime--phase field
\[
X_T^{\mathrm{rand}}(h)
:= \sum_{k} Y_k^{\mathrm{rand}}(h),
\]
and any fixed finite set of points $h_0,\dots,h_m\in[-1,1]$, there is a constant $C(F,m,h_\bullet)$ such that for all sufficiently large $T$ and all $C^3$ test functions $F:\mathbb R^{m+1}\to\mathbb R$ with polynomial growth,
\[
\bigg|
\mathbb E\big[X_T^{\mathrm{rand}}(h_0)F(\mathbf X_T^{\mathrm{rand}})\big]
-\sum_{j=0}^m \Sigma^{\mathrm{rand}}_{0j}\,\mathbb E \big[\partial_{x_j}F(\mathbf X_T^{\mathrm{rand}})\big]
\bigg|
\le C(F,m,h_\bullet),
\]
where $\Sigma^{\mathrm{rand}}_{0j}
=\mathrm{Cov}(X_T^{\mathrm{rand}}(h_0),X_T^{\mathrm{rand}}(h_j))$, and $\mathbf X_T^{\mathrm{rand}}=(X_T^{\mathrm{rand}}(h_0),\dots,X_T^{\mathrm{rand}}(h_m))$.
\end{theorem}

This is an approximate IBP with an \textbf{absolute} error $O(1)$ at scale $\log\log T$; it is proved by combining Theorem A on the random model side (where blocks are independent) with a blockwise Lindeberg replacement and exact Gaussian IBP.

For the \textbf{arithmetic} field $X_T^{\zeta}$, we do \textbf{not} claim such an IBP theorem at present. Instead, we formulate:
\begin{itemize}
\item a concrete conjecture about \textbf{global} Dirichlet--polynomial representations of IBP--type functionals with length $\ll T^\delta$; and
\item a conditional ``Theorem B${}_{\mathrm{arith}}$'' showing that, \emph{if} this conjecture holds, then the same approximate IBP identity carries over to the arithmetic model.
\end{itemize}
This clarifies exactly which analytic input is missing for a full Guerra/Parisi analysis in the zeta setting.

\section{Framework: arithmetic and random models, block decomposition}

We first recall the block decomposition and introduce the random prime--phase model and Gaussian surrogates in a unified notation.

\subsection{Block decomposition by size of $\log p$}

Let $T$ be large and $\tau\sim\mathrm{Unif}[T,2T]$. Fix a small $\theta>0$, and define
\[
X_T^{\zeta}(h)
:= \Rea\sum_{p\le T^\theta} p^{-1/2-i(\tau+h)},\qquad |h|\le 1.
\]

Set
\[
n := \lfloor \log\log T\rfloor,\qquad
n_0 := \lfloor y\rfloor,\qquad
n_L := n-n_0,
\]
where $y$ is a fixed constant (say $y\ge 1$). For $n_0<k\le n_L$, define the $k$-th block of primes
\[
\mathcal P_k :=\{p:\ e^{k-1}<\log p\le e^k\}.
\]

The block increments are
\[
Y_k^{\zeta}(h)
:=\sum_{p\in\mathcal P_k}
\Rea\Big(p^{-1/2-i(\tau+h)}+\tfrac12 p^{-1-2i(\tau+h)}\Big).
\]

One has
\[
X_T^{\zeta}(h)
= \sum_{k=n_0+1}^{n_L} Y_k^{\zeta}(h) + R_T^{\zeta}(h),
\]
where $R_T^{\zeta}(h)$ collects contributions from primes with $\log p\le e^{n_0}$ and $\log p>e^{n_L}$ and has variance $O(1)$. The main variance ($\sim \tfrac12\log\log T$) is carried by the blocks $Y_k^{\zeta}$.

\subsection{Random prime--phase model}

Let $\{\theta_p\}_{p}$ be i.i.d. uniform on $[0,2\pi]$, and define the prime--phase model
\[
X_T^{\mathrm{rand}}(h)
:= \Rea\sum_{p\le T^\theta} e^{i\theta_p}p^{-1/2-ih}.
\]

Define block increments
\[
Y_k^{\mathrm{rand}}(h)
:= \sum_{p\in\mathcal P_k}
\Rea\Big(e^{i\theta_p}p^{-1/2-ih}+\tfrac12e^{2i\theta_p}p^{-1-2ih}\Big),
\]
and set
\[
X_T^{\mathrm{rand}}(h) = \sum_{k=n_0+1}^{n_L} Y_k^{\mathrm{rand}}(h).
\]

For each fixed block index $k$, the family $\{Y_k^{\mathrm{rand}}(h)\}_{|h|\le 1}$ forms an increment of a log--correlated field; for our purposes, it is a finite family of sums of independent (in $p$) bounded random variables.

\subsection{Gaussian block surrogates}

Fix $m\ge 0$ and points $h_0,\dots,h_m\in[-1,1]$. For each block $k$, define the block vectors
\[
\mathbf Y_k^{\zeta}
:=\big(Y_k^{\zeta}(h_0),\dots,Y_k^{\zeta}(h_m)\big),
\quad
\mathbf Y_k^{\mathrm{rand}}
:=\big(Y_k^{\mathrm{rand}}(h_0),\dots,Y_k^{\mathrm{rand}}(h_m)\big).
\]

Let $\Sigma_k := \mathrm{Cov}(\mathbf Y_k^{\mathrm{rand}})$, and define a centred Gaussian vector $\mathbf N_k$ with covariance $\Sigma_k$. The field
\[
\mathbf G_T
:= \sum_{k=n_0+1}^{n_L} \mathbf N_k + \mathbf R_T
\]
(with $\mathbf R_T$ a Gaussian tail correction) is our Gaussian surrogate for the finite–dimensional marginals of $\mathbf X_T^{\zeta}$ or $\mathbf X_T^{\mathrm{rand}}$.

\section{Multivariate Berry--Esseen for blocks in the random model}

We begin by establishing a multivariate Berry--Esseen estimate for the block vector $\mathbf Y_k^{\mathrm{rand}}$, for fixed $k$ and finite set of points $h_0,\dots,h_m$.

\subsection{Prime--wise decomposition and moment bounds}

For each prime $p\in\mathcal P_k$, define the contribution
\[
W_p
:=\big(
\Rea(e^{i\theta_p}p^{-1/2-ih_0}+\tfrac12e^{2i\theta_p}p^{-1-2ih_0}),\dots,
\Rea(e^{i\theta_p}p^{-1/2-ih_m}+\tfrac12e^{2i\theta_p}p^{-1-2ih_m})
\big)\in\mathbb R^{m+1}.
\]

Then
\[
\mathbf Y_k^{\mathrm{rand}} = \sum_{p\in\mathcal P_k} W_p.
\]

The $W_p$ are independent, mean-zero random vectors. Each coordinate of $W_p$ is bounded by $Cp^{-1/2}$, hence
\[
|W_p| \ll_m p^{-1/2},\qquad \mathbb E|W_p|^3\ll_m p^{-3/2}.
\]

Let
\[
M_k^{(3)} := \sum_{p\in\mathcal P_k} \mathbb E|W_p|^3.
\]
Since $\mathcal P_k\subset\{p:\log p\ge e^{k-1}\}$,
\[
M_k^{(3)}\ll_m \sum_{p\in\mathcal P_k}p^{-3/2}
\ll \exp(-c e^k)
\]
for some $c>0$.

On the other hand, the covariance matrix $\Sigma_k = \mathrm{Cov}(\mathbf Y_k^{\mathrm{rand}})$ satisfies
\[
\mathrm{tr}(\Sigma_k)
= \sum_{j=0}^m\mathrm{Var}\big(Y_k^{\mathrm{rand}}(h_j)\big)
\asymp_m 1,
\]
since ABR/FHK show that $\mathrm{Var}(Y_k^{\mathrm{rand}}(h))=\tfrac12+O(e^{-c\sqrt{k}})$ for each fixed $h$.

Thus the Berry--Esseen ``Lyapunov ratio''
\[
\rho_k
:= \frac{M_k^{(3)}}{(\mathrm{tr}(\Sigma_k))^{3/2}}
\ll_m e^{-c e^k}.
\]

\subsection{Multivariate Berry--Esseen for rectangles}

We use a standard multivariate Berry--Esseen inequality for sums of independent vectors in $\mathbb R^d$, controlling the Kolmogorov distance over axis–aligned rectangles $\mathcal R_d$:

\begin{theorem}[Multivariate BE for rectangles]
Let $Z_1,\dots,Z_n$ be independent, mean-zero vectors in $\mathbb R^d$ with covariance $\Sigma$, and let $S:=\sum_i Z_i$, $G\sim \mathcal N(0,\Sigma)$. There exists a constant $C_d>0$ such that
\[
\sup_{R\in\mathcal R_d}
\big|\mathbb P(S\in R) - \mathbb P(G\in R)\big|
\le C_d \frac{\sum_i\mathbb E|Z_i|^3}{(\mathrm{tr}\,\Sigma)^{3/2}}.
\]
\end{theorem}

Applying this with $Z_i=W_{p_i}$ and $d=m+1$, we get:

\begin{proposition}[Block BE in the random model]
For each fixed $m\ge 0$, points $h_0,\dots,h_m\in[-1,1]$, and block index $k$,
\[
\sup_{R\in\mathcal R_{m+1}}
\big|\mathbb P(\mathbf Y_k^{\mathrm{rand}}\in R)
-\mathbb P(\mathbf N_k\in R)\big|
\ll_m e^{-c_m e^k},
\]
where $\mathbf N_k\sim\mathcal N(0,\Sigma_k)$ with $\Sigma_k=\mathrm{Cov}(\mathbf Y_k^{\mathrm{rand}})$.
\end{proposition}

This is the multivariate generalization of the 1D/2D block CLTs in FHK.

\section{Smoothing and Dirichlet polynomial representation (ABR/FHK)}

We now recall the ABR smoothing construction and show how, composed with block increments, it yields Dirichlet polynomials in $\tau$ of controlled length.

\subsection{Smoothing a one--dimensional indicator}

Let $\Delta\ge 2$, $A\ge 1$. ABR construct an entire function $G_{\Delta,A}:\mathbb R\to[0,1]$ with the following properties:

\begin{enumerate}
\item \textbf{Approximation of an interval.}
   For any interval $[u,v]\subset \mathbb R$,
   \[
   1_{[u,v]}(x)\le G_{\Delta,A}(x-u)\le 1_{[u-\Delta^{-A},v+\Delta^{-A}]}(x) + O(e^{-\Delta^{A-1}}).
   \]
\item \textbf{Dirichlet polynomial representation.}
   $G_{\Delta,A}$ can be written as the squared modulus of a trigonometric polynomial
   \[
   G_{\Delta,A}(x) = |D_{\Delta,A}(x)|^2,
   \]
   where
   \[
   D_{\Delta,A}(x)
   = \sum_{|\ell|\le \Delta^{C}} d_\ell e^{2\pi i \ell x}
   \]
   for some absolute constant $C$, and $|d_\ell|\le \Delta^{C'}$.
\end{enumerate}

By scaling and translation, we may produce for each interval $[a,b]$ a function $G_{[a,b]}$ with analogous properties:
\[
1_{[a,b]}(x)\le G_{[a,b]}(x)\le 1_{[a-\delta,b+\delta]}(x) + O(e^{-\Delta^{A-1}}),
\]
where $\delta=\Delta^{-A}$, and
\[
G_{[a,b]}(x) = |D_{[a,b]}(x)|^2
\]
with $D_{[a,b]}$ a trigonometric polynomial of degree $\ll \Delta^C$.

\subsection{Product smoothing for a rectangle in $\mathbb R^{m+1}$}

Let
\[
R=\prod_{r=0}^m [a_r,b_r]\subset\mathbb R^{m+1}.
\]
Define
\[
G_R(\mathbf x)
:=\prod_{r=0}^m G_{[a_r,b_r]}(x_r),
\qquad
\mathbf x=(x_0,\dots,x_m)\in\mathbb R^{m+1}.
\]

Then:
\begin{itemize}
\item For all $\mathbf x$,
  \[
  1_R(\mathbf x)\le G_R(\mathbf x)
  \le 1_{R^{(+\delta)}}(\mathbf x) + O(e^{-\Delta^{A-1}}),
  \]
  where $R^{(+\delta)}$ is the rectangle obtained by fattening each interval by $\delta=\Delta^{-A}$.
\item $G_R(\mathbf x)$ is a finite linear combination of products of exponentials $\exp(2\pi i\sum_r \ell_r x_r)$, with $|\ell_r|\le \Delta^C$, and coefficients of size $\ll \Delta^{C'(m)}$.
\end{itemize}

\subsection{Composition with block increments and length bounds}

Fix a block $k\in(n_0,n_L]$. For each $r$, $Y_k^{\zeta}(h_r)$ is a finite sum over primes $p\in\mathcal P_k$ of terms of the form $p^{-1/2-i(\tau+h_r)}$, $p^{-1-2i(\tau+h_r)}$, and their conjugates. Therefore, each monomial in $Y_k^{\zeta}(h_r)$ of degree $\le M$ is a sum over integers $n$ of the form
\[
n=\prod_{p\in\mathcal P_k} p^{v_p},\quad 0\le v_p\le M,
\]
with coefficients $\ll p^{-v_p/2}$, and can therefore be written as
\[
\sum_{n\le \exp(Me^k)} a_n n^{-1/2-i\tau}.
\]

By expanding $G_R(\mathbf Y_k^{\zeta}(\tau))$ as a polynomial in the coordinates $Y_k^{\zeta}(h_r)$ of degree bounded by $\ll \Delta^{C}$, we conclude:

\begin{proposition}[Dirichlet polynomial representation, block level]
Fix $m\ge 0$ and points $h_0,\dots,h_m\in[-1,1]$. For each block $k\le n_L$ and rectangle $R\subset\mathbb R^{m+1}$, there exists a Dirichlet polynomial
\[
Q_{k,R}(\tau)
=\sum_{n\le N_k} b_{k,R}(n)\,n^{-1/2-i\tau}
\]
with length
\[
N_k \le \exp(C_m \Delta_k^{C'} e^k),
\]
where $\Delta_k$ is a chosen scale parameter, such that
\[
G_R\big(\mathbf Y_k^{\zeta}(\tau)\big)
= |Q_{k,R}(\tau)|^2 + O_m(e^{-\Delta_k^{A-1}})
\]
pointwise in $\tau$.
\end{proposition}

ABR/FHK select
\[
\Delta_k := (k\wedge (n-k))^4,
\]
and show (by summing over $k$ and choosing constants appropriately) that for all $k\le n_L$,
\[
\exp(C_m \Delta_k^{C'} e^k) \le \exp(e^n/100)=T^{1/100}.
\]

We adopt this choice here, so that for all blocks $k\le n_L$,
\[
N_k \le T^{1/100}.
\]

A completely analogous construction yields a \textbf{random} Dirichlet polynomial
\[
Q_{k,R}^{\mathrm{rand}}
= \sum_{n\le N_k} b_{k,R}(n) X_n n^{-1/2},
\]
where the $X_n$ are multiplicative random phases built from the independent $\theta_p$, such that
\[
G_R(\mathbf Y_k^{\mathrm{rand}}) = |Q_{k,R}^{\mathrm{rand}}|^2 + O_m(e^{-\Delta_k^{A-1}}).
\]

\section{Arithmetic vs random: mean--value theorem and proof of Theorem A}

We now use ABR's mean--value theorem for Dirichlet polynomials to compare arithmetic and prime--phase expectations of the smoothed observables $G_R(\mathbf Y_k^{\zeta})$.

\subsection{Mean--value theorem}

We use the following mean--value theorem for Dirichlet polynomials, which is a standard consequence of Montgomery--Vaughan mean--value estimates and ABR/FHK's setup:

\begin{lemma}[Mean--value theorem]
Let $\tau\sim\mathrm{Unif}[T,2T]$, and let
\[
D(\tau)
= \sum_{n\le N} a(n) n^{-1/2-i\tau},
\]
with $N\le T^{1/4}$. Then
\[
\mathbb E_\tau |D(\tau)|^2
= \Big(1+O\Big(\frac{N}{T}\Big)\Big)
\sum_{n\le N}|a(n)|^2
= \Big(1+O\Big(\frac{N}{T}\Big)\Big)
\mathbb E_Z\Big|\sum_{n\le N}a(n)X_n n^{-1/2}\Big|^2,
\]
where $X_n$ is the multiplicative prime--phase model.
\end{lemma}

In our setting, $N_k\le T^{1/100}\le T^{1/4}$ for large $T$, so Lemma 5.1 applies to each $Q_{k,R}$.

\subsection{Expectation comparison for smoothed indicators}

Combining Proposition 4.1 and Lemma 5.1 yields:
\[
\begin{aligned}
\mathbb E_\tau G_R(\mathbf Y_k^{\zeta})
&= \mathbb E_\tau |Q_{k,R}(\tau)|^2 + O_m(e^{-\Delta_k^{A-1}})\\
&= \Big(1+O\Big(\frac{N_k}{T}\Big)\Big)\sum_{n\le N_k}|b_{k,R}(n)|^2 + O_m(e^{-\Delta_k^{A-1}})\\
&= \Big(1+O\Big(\frac{N_k}{T}\Big)\Big)\mathbb E_Z|Q_{k,R}^{\mathrm{rand}}|^2 + O_m(e^{-\Delta_k^{A-1}})\\
&= \mathbb E_Z G_R(\mathbf Y_k^{\mathrm{rand}}) + O_m\Big(\frac{N_k}{T}\Big) + O_m(e^{-\Delta_k^{A-1}}).
\end{aligned}
\]

Since $N_k\le T^{1/100}$, the mean--value error satisfies
\[
\frac{N_k}{T}\le T^{-99/100},
\]
which is much smaller than any power of $e^{-e^k}$ for $k\le n_L$ (recall $e^n=\log T$ and $e^k\le e^n$). Specifically, for each fixed $\varepsilon>0$,
\[
T^{-99/100}\le e^{-\varepsilon e^k}
\]
for all $k\le n_L$ and large $T$. Likewise, $e^{-\Delta_k^{A-1}}$ decays super-polynomially in $k$.

Thus we can write
\[
\big|\mathbb E_\tau G_R(\mathbf Y_k^{\zeta})
-\mathbb E_Z G_R(\mathbf Y_k^{\mathrm{rand}})\big|
\ll_m e^{-c_m e^k}.
\]

\subsection{From smoothed indicators to rectangles and smooth test functions}

By construction,
\[
1_R(\mathbf x)\le G_R(\mathbf x)\le 1_{R^{(+\delta)}}(\mathbf x)+O(e^{-\Delta_k^{A-1}}),
\]
with $\delta=\Delta_k^{-A}$. Therefore, for any random vector $\mathbf V$,
\[
\mathbb P(\mathbf V\in R)\le \mathbb E G_R(\mathbf V)
\le \mathbb P(\mathbf V\in R^{(+\delta)}) + O(e^{-\Delta_k^{A-1}}).
\]

Applying this to $\mathbf V=\mathbf Y_k^{\zeta}$ and $\mathbf V=\mathbf Y_k^{\mathrm{rand}}$, and using the block BE for $\mathbf Y_k^{\mathrm{rand}}$ vs $\mathbf N_k$ (Proposition 3.2), we obtain:
\[
\begin{aligned}
\big|\mathbb P_\tau(\mathbf Y_k^{\zeta}\in R)
-\mathbb P(\mathbf N_k\in R)\big|
&\le
\big|\mathbb E_\tau G_R(\mathbf Y_k^{\zeta})-\mathbb E G_R(\mathbf N_k)\big|
+O_m(\delta)+O_m(e^{-\Delta_k^{A-1}})\\
&\ll_m e^{-c_m e^k} + \delta.
\end{aligned}
\]

With $\Delta_k=(k\wedge(n-k))^4$ and $A$ large, $\delta=\Delta_k^{-A}$ decays faster than any power of $k$, so in particular
\[
\sum_{k} \delta <\infty.
\]
The leading error term is $e^{-c_m e^k}$, also summable in $k$.

This proves the rectangle version of Theorem A. The smooth test–function version follows similarly by approximating any $\Phi\in C^3$ by finite linear combinations of such smoothed rectangles and using the block BE plus the above error estimates.

This completes the proof of Theorem A.

\section{Approximate Gaussian integration--by--parts in the random model}

We now work entirely in the \textbf{random prime–phase model}, where blocks are independent, and deduce an approximate Gaussian IBP for the full field $X_T^{\mathrm{rand}}(h)$.

\subsection{Setup}

Fix $m\ge 0$ and points $h_0,\dots,h_m\in[-1,1]$. Define
\[
X_T^{\mathrm{rand}}(h)
:=\sum_{k=n_0+1}^{n_L}Y_k^{\mathrm{rand}}(h),
\qquad
\mathbf X_T^{\mathrm{rand}} := (X_T^{\mathrm{rand}}(h_0),\dots,X_T^{\mathrm{rand}}(h_m)).
\]

Similarly, let
\[
\mathbf G_T = \sum_{k=n_0+1}^{n_L} \mathbf N_k + \mathbf R_T
\]
be a Gaussian vector with covariance
\[
\Sigma_T^{\mathrm{rand}} := \mathrm{Cov}(\mathbf X_T^{\mathrm{rand}}).
\]

By Theorem A (applied to the random model) and independence of blocks, we can compare $\mathbf X_T^{\mathrm{rand}}$ and $\mathbf G_T$ at the level of expectations of smooth functionals with error $O(1)$.

\subsection{Blockwise Lindeberg replacement}

Let $F:\mathbb R^{m+1}\to\mathbb R$ be $C^3$ with polynomial growth and bounded derivatives up to order 3. Define
\[
\Phi_0(x) := x_0 F(x),\qquad
\Phi_j(x) := \partial_{x_j}F(x),\quad j=0,\dots,m.
\]

Consider hybrid fields
\[
\mathbf Z^{(\ell)}
:= \sum_{k\le \ell}\mathbf Y_k^{\mathrm{rand}} + \sum_{k>\ell}\mathbf N_k + \mathbf R_T,
\qquad
\ell=n_0,\dots,n_L.
\]

Thus $\mathbf Z^{(n_0)}=\mathbf G_T$ and $\mathbf Z^{(n_L)}$ is a version of $\mathbf X_T^{\mathrm{rand}}$ (up to Gaussian tail correction $\mathbf R_T$ of variance $O(1)$).

Write
\[
\Delta_\ell(\Phi) := \mathbb E\Phi(\mathbf Z^{(\ell+1)}) -\mathbb E\Phi(\mathbf Z^{(\ell)}).
\]

By construction, conditioning on all blocks except $\ell+1$, we can write
\[
\Delta_\ell(\Phi)
= \mathbb E\Big[\Phi(U+Y_{\ell+1}^{\mathrm{rand}})
-\Phi(U+N_{\ell+1})\Big],
\]
where $U$ is independent of $Y_{\ell+1}^{\mathrm{rand}}$, $N_{\ell+1}$. Here independence holds because the $Y_k^{\mathrm{rand}}$ are independent in $k$.

A standard application of the block BE (Theorem A in the random model) and Taylor expansion of $\Phi$ shows that
\[
|\Delta_\ell(\Phi)|
\le C(F,m,h_\bullet) e^{-c_m e^{\ell+1}}.
\]

Summing over $\ell$ yields
\[
\big|\mathbb E\Phi(\mathbf X_T^{\mathrm{rand}})
-\mathbb E\Phi(\mathbf G_T)\big|
\le \sum_{\ell} |\Delta_\ell(\Phi)| + O(1)
\le C(F,m,h_\bullet).
\]

In particular, this holds when $\Phi=\Phi_0$ and $\Phi_j$.

\subsection{Gaussian IBP and conclusion}

For the Gaussian vector $\mathbf G_T\sim \mathcal N(0,\Sigma_T^{\mathrm{rand}})$, exact Gaussian integration--by--parts gives
\[
\mathbb E[G_T(h_0) F(\mathbf G_T)]
= \sum_{j=0}^m \Sigma^{\mathrm{rand}}_{0j} \mathbb E[\partial_{x_j}F(\mathbf G_T)].
\]

Combining with the comparison estimates for $\Phi_0,\Phi_j$ yields:
\[
\begin{aligned}
&\mathbb E\big[X_T^{\mathrm{rand}}(h_0)F(\mathbf X_T^{\mathrm{rand}})\big]
-\sum_{j=0}^m\Sigma^{\mathrm{rand}}_{0j}\,\mathbb E\big[\partial_{x_j}F(\mathbf X_T^{\mathrm{rand}})\big]\\
&\quad = \big(\mathbb E\Phi_0(\mathbf X_T^{\mathrm{rand}}) -\mathbb E\Phi_0(\mathbf G_T)\big)
-\sum_{j=0}^m\Sigma^{\mathrm{rand}}_{0j}\big(\mathbb E\Phi_j(\mathbf X_T^{\mathrm{rand}}) -\mathbb E\Phi_j(\mathbf G_T)\big)\\
&\quad = O(F,m,h_\bullet).
\end{aligned}
\]

This is precisely Theorem B(${}_{\mathrm{rand}}$): an approximate Gaussian IBP for the random model with absolute error $O(1)$.

\section{Arithmetic IBP and open problems}

We now return to the \textbf{arithmetic} field $X_T^{\zeta}$. One would like to prove an analogue of Theorem B(${}_{\mathrm{rand}}$) for $\mathbf X_T^{\zeta}$. However, the Lindeberg argument of Section 6 fails in the arithmetic setting because the blocks $Y_k^{\zeta}$ are all functions of the same $\tau$ and are not independent.

We explain the obstruction, then formulate a precise global Dirichlet–polynomial representation conjecture under which an arithmetic IBP would follow.

\subsection{Why blockwise Lindeberg fails in the arithmetic model}

In the random model, $\mathbf Z^{(\ell)}$ is a sum of independent blocks up to $\ell$, plus independent Gaussian blocks from $\ell+1$ onward. Conditioning on all blocks except $\ell+1$ truly isolates the single increment $Y_{\ell+1}^{\mathrm{rand}}$, and the block BE controls the difference $\Phi(U+Y_{\ell+1}^{\mathrm{rand}})-\Phi(U+N_{\ell+1})$.

In the arithmetic model, if we write
\[
\mathbf X_T^{\zeta} = \sum_k \mathbf Y_k^{\zeta} + \mathbf R_T^{\zeta},
\]
then any hybrid
\[
\mathbf Z^{(\ell)} := \sum_{k\le \ell}\mathbf Y_k^{\zeta} + \sum_{k>\ell} \mathbf N_k + \mathbf R_T,
\]
is still a function of the \emph{same} random variable $\tau$ through the surviving $Y_k^{\zeta}$; there is no independence across blocks. Conditioning on ``all blocks except $\ell+1$'' does not yield a random variable $U$ independent of $Y_{\ell+1}^{\zeta}$: both are functions of $\tau$.

Theorem A is a \textbf{marginal} CLT for each block $\mathbf Y_k^{\zeta}$, but it says nothing about the joint distribution of $(\mathbf Y_k^{\zeta}, U)$ for arbitrary $U$. Consequently, there is no way to bound
\[
\mathbb E_\tau\Big[\Phi(U+Y_{\ell+1}^{\zeta})-\Phi(U+N_{\ell+1})\Big]
\]
by Theorem A alone: we would need a conditional CLT for $Y_{\ell+1}^{\zeta}$ given $U$, which we do not have.

Therefore \textbf{the blockwise Lindeberg replacement is not available in the arithmetic field}, and the naive attempt to mimic Section 6 fails at a conceptual level.

\subsection{A global Dirichlet–polynomial representation conjecture}

To recover an IBP identity for $X_T^{\zeta}$, one must work at the level of the \textbf{entire functional} appearing in IBP, rather than blockwise. The natural approach is:
\begin{itemize}
    \item Represent the IBP functional (e.g. $X_T^{\zeta}(h_0)F(\mathbf X_T^{\zeta})$) as a single global Dirichlet polynomial $Q_{\mathrm{tot}}(\tau)$ of controlled length $N_{\mathrm{tot}}\le T^\delta$ for some $\delta<1$.
    \item Apply a global mean–value theorem to compare $\mathbb E_\tau Q_{\mathrm{tot}}(\tau)$ with the corresponding expectation in the prime–phase model.
\end{itemize}

We formalize this as follows.

Fix $m$ and points $h_0,\dots,h_m\in[-1,1]$. Let $\mathcal F$ be a class of test functions $F:\mathbb R^{m+1}\to\mathbb R$ (e.g. smooth with certain growth, or a specific family such as free–energy observables). For each $F\in\mathcal F$, define
\[
\Phi_0^F(\mathbf x) := x_0 F(\mathbf x),\qquad \Phi_j^F(\mathbf x):=\partial_{x_j}F(\mathbf x).
\]

\begin{conjecture}[Global Dirichlet–polynomial representation]
There exists $\delta\in(0,1)$ and a class $\mathcal F$ of test functions such that for each $F\in\mathcal F$, and each $j=0,\dots,m$, there are Dirichlet polynomials
\[
Q_{F,j}(\tau) = \sum_{n\le N_{\mathrm{tot}}} a_{F,j}(n)n^{-1/2-i\tau},\qquad N_{\mathrm{tot}}\le T^\delta,
\]
with
\[
\Phi_j^F(\mathbf X_T^{\zeta}(\tau))
= Q_{F,j}(\tau) + o(\log\log T)
\quad\text{in }L^1(\tau\in[T,2T]),
\]
and such that the analogous prime–phase polynomials $Q_{F,j}^{\mathrm{rand}}$ have the same coefficients.
\end{conjecture}

If such a representation exists, then Lemma 5.1 (suitably generalized to $\mathbb E|Q_{F,j}|^2$ and slightly higher moments) yields
\[
\mathbb E_\tau \Phi_j^F(\mathbf X_T^{\zeta})
= \mathbb E_Z \Phi_j^F(\mathbf X_T^{\mathrm{rand}}) + o(\log\log T),
\]
for all $F\in\mathcal F$, $j=0,\dots,m$. Similarly for $X_T^{\zeta}(h_0)F(\mathbf X_T^{\zeta})$.

\subsection{Conditional arithmetic IBP}

Assume the above conjecture for a class $\mathcal F$ containing the Gibbs/free–energy observables of interest. The random–model IBP (Theorem B(${}_{\mathrm{rand}}$)) gives
\[
\mathbb E_Z\big[X_T^{\mathrm{rand}}(h_0)F(\mathbf X_T^{\mathrm{rand}})\big]
= \sum_{j=0}^m \Sigma^{\mathrm{rand}}_{0j}\,\mathbb E_Z\big[\partial_{x_j}F(\mathbf X_T^{\mathrm{rand}})\big] + O(1).
\]

Assuming the conjectural Dirichlet–polynomial representation and applying the mean–value transfer, we obtain
\[
\mathbb E_\tau\big[X_T^{\zeta}(h_0)F(\mathbf X_T^{\zeta})\big]
= \sum_{j=0}^m \Sigma^{\mathrm{rand}}_{0j}\,\mathbb E_\tau\big[\partial_{x_j}F(\mathbf X_T^{\zeta})\big] + O(1) + o(\log\log T).
\]

Since $\Sigma^{\mathrm{rand}}_{0j}=\Sigma^{\zeta}_{0j}+O(1)$ (ABR/FHK), we can rewrite the main term with $\Sigma^{\zeta}_{0j}$ at the cost of an additional $O(1)$ error. Thus we obtain:

\begin{theorem}[Conditional Theorem B${}_{\mathrm{arith}}$]\label{thm:B_arith}
Assume the global Dirichlet–polynomial representation conjecture above for a class $\mathcal F$ and fixed points $h_0,\dots,h_m$. Then for each $F\in\mathcal F$,
\[
\bigg|
\mathbb E_\tau\big[X_T^{\zeta}(h_0)F(\mathbf X_T^{\zeta})\big]
-\sum_{j=0}^m\Sigma^{\zeta}_{0j}\,\mathbb E_\tau\big[\partial_{x_j}F(\mathbf X_T^{\zeta})\big]
\bigg|
= O_F(1).
\]
Dividing by $\log\log T$, the relative error tends to 0 as $T\to\infty$.
\end{theorem}

This is exactly the approximate IBP identity needed to apply the Guerra/Parisi machinery (stochastic stability, Ghirlanda–Guerra identities) to the arithmetic field for the class $\mathcal F$.

\subsection{Relation to spin–glass theory}

From the spin–glass perspective, the field $X_T^{\zeta}(h)$ or its prime–phase analogue plays the role of a Hamiltonian on a one–dimensional configuration space (the interval $h\in[-1,1]$); the Guerra interpolation and Parisi's functional require two main analytic inputs:
\begin{enumerate}
    \item \textbf{Gaussian universality / approximate IBP} for the finite–dimensional distributions of the Hamiltonian at a given ``temperature'', and
    \item Precise control of the free energy (or maximum) and of overlaps.
\end{enumerate}

Theorem A provides a \textbf{blockwise} Gaussian approximation in the arithmetic model, and Theorem B(${}_{\mathrm{rand}}$) plus Theorem A provide a fully rigorous approximate IBP in the \textbf{random} model. Thus, for the random model, stochastic stability and Ghirlanda–Guerra identities at high temperature can be treated with standard methods.

For the arithmetic model, the missing piece is exactly the global Dirichlet–polynomial representation conjecture above. Proving such a representation for an appropriate class $\mathcal F$ would constitute a major analytic advance: it would bridge ABR/FHK's detailed Dirichlet-polynomial technology and the full Parisi/Guerra overlap analysis for the zeta field.

We view the present paper as a technical bridge in this direction:
\begin{itemize}
\item It \textbf{packages} the ABR/FHK machinery into a robust arithmetic block Gaussian universality theorem (Theorem A).
\item It shows how, \emph{given} a global Dirichlet–polynomial representation for IBP functionals, one could bootstrap Theorem A and the random–model IBP into a full arithmetic IBP.
\item It thereby isolates a concrete and well–posed analytic problem at the interface of analytic number theory and spin–glass theory.
\end{itemize}

We do not claim that this final step is solved here, and we believe it is a natural and challenging direction for future work.

\end{document}
