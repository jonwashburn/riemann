\documentclass[12pt,a4paper]{article}
%\usepackage[utf8]{inputenc}
%\inputencoding{utf8}
%\usepackage[english,russian]{babel}

\usepackage[T2A]{fontenc}
\usepackage{indentfirst}
\usepackage{misccorr}
\usepackage{graphicx}
\usepackage{amsmath,amssymb,amsthm}

%\begin{document}

\newtheorem{theorem}{Theorem}
\newtheorem{lemma}{Lemma}
\newcommand\tig{\tilde g}



\title{On zeroes and poles of Helson zeta functions}
\author{I. Bochkov and R. Romanov\thanks{morovom@gmail.com}\\
\\
Department of Mathematics and Computer Science,\\
St Petersburg State University, Russia}
\date{}

\begin{document}

\maketitle
\begin{abstract} We show that the analytic continuations of Helson zeta functions $  \zeta_\chi (s)= \sum_1^{\infty}\chi(n)n^{-s} $ can have essentially arbitrary poles and zeroes  in the strip $ 21/40 < \Re s < 1 $ (unconditionally), and in the whole critical strip $ 1/2 < \Re s <1 $ under Riemann Hypothesis. 
\end{abstract}


Let $ \chi \colon \mathbb N \to \mathbb T $, $ \mathbb T = \{ z \in \mathbb C \colon |z|=1 \}$ be a completely multiplicative function, and let 
\begin{equation}\label{H} \zeta_\chi (s)= \sum_1^{\infty}\chi(n)n^{-s} . \end{equation}  

This series defines an analytic function in the halfplane $ \Re s > 1 $. The Euler product formula, \[ \zeta_\chi (s)=\prod_p \frac{1}{1-\chi(p)p^{-s}} , \] 
holds. In particular, the function $ \zeta_\chi $ has no zeroes for $ \Re s > 1 $. 

The study of functions $ \zeta_\chi $ was initiated by H. Helson who showed \cite{Hels} in 1969 that for almost all $ \chi$'s the function $ \zeta_\chi $ extends analytically to the halfplane $ \Re s > 1/2 $ and has no zeroes in this halfplane. Almost all here refers to the measure on the functions $ \chi $ induced by the standard product measure on the infinite dimensional torus $ \mathbb T^\infty $ via identification of $ \chi $ with the sequence $ \{ \chi (p) \} \in \mathbb T^\infty $ of its values at primes. Recently it has been shown \cite{saks} that Helson's result is optimal in that  the function $ \zeta_\chi $ does not admit meromorphic extension to the halfplane $ \Re s > \alpha $ for any $ \alpha < 1/2 $ almost surely in the above sense.

These results bring about the question of the structure of analytic continuation for $ \chi $ from the exceptional set in the Helson theorem. Such a study has been undertaken in \cite{S} where it was shown that the set of zeroes of meromorophic continuation of $ \zeta_\chi $ in the strip $ 1/2 < \Re s<1$ can be essentially arbitrary. Namely, the following theorem holds. 

\begin{theorem}{\cite[Theorem 1.4]{S}}
For any set $ \mathfrak O $ in the strip $ 1/2  < \Re s<  39/40 $ which has no accumulation points off the line $ \Re s = 1/2 $ there exists a completely multiplicative function $ \chi $ such that the Helson zeta function $ \zeta_\chi $ admits meromophic extension to the halfplane $ \Re s > 1/2 $, and 
\[ \left\{ s : \zeta_\chi ( s ) = 0,   \Re s > \frac 12 \right\} = \mathfrak O . \] 

If the RH is assumed then the same assertion holds with $ 1$ in the place of $ 39/40 $.
\end{theorem}

This theorem is proved in \cite{S} by constructing the required function $ \chi $. Given the zero set $ \mathfrak O $, the construction in \cite{S} invloves forming "dipoles" made of a zero $ \rho \in \mathfrak O $ and an additional pole nearby. The resulting function  $ \zeta_\chi $ is meromorphic, and the set of its poles is not arbitrary. A natural further question mentioned in \cite{S} is if there are any restrictions on the set of poles of $ \zeta_\chi $ in Theorem 1. Our result  is the following theorem saying essentially that the set of poles can also be arbitrary.

To account for multiple zeroes and poles we are going to use the terminology of multisets. A multiset is a pair made of a subset $ S \subset \mathbb C $ and a function $ S \longrightarrow \mathbb N $. When we say that the set of zeroes of an analytic function $ f$ coincides with a multiset, $ M_S $, we mean that $ f ( s ) = 0 $ iff $ s \in S $, and the multiplicity of zero at $ s \in S $ is $ m ( s ) $. A similar terminology is used for poles.

Our goal is to establish the following

\begin{theorem}
 1. For any disjoint multisets $  Z$ and $ P $ in the strip $ 21/40 < \Re s<1$ having no accumulation points off the line $ \Re s = 21/40  $ there exists a completely multiplicative function $ \chi $ such that $ \zeta_\chi $ admits meromorphic continuation to the halfplane $ \Re s > 21/40 $ with $ Z $ the set of its zeroes and $ P $ the set of its poles. 
 
 2. If RH holds then for any disjoint multisets $  Z$ and $  P $ in the strip $ 1/2 < \Re s<1$ having no accumulation points off the line $ \Re s = 1/2 $ there exists a completely multiplicative function $ \chi $ such that $ \zeta_\chi $ admits meromorphic continuation to the halfplane $ \Re s > 1/2 $ with $ Z $ the set of its zeroes and $  P $ the set of its poles.
\end{theorem}

\medskip 
\textit{Acknowledgements.} We are indebted to K. Seip who suggested the question about poles to us. This work was supported by Russian Science Foundation under Grant 17-11-01064.


\bigskip 

\bfseries 
\begin{center}
Logarithmic derivative
\end{center}
\mdseries

The strategy of the proof is as follows. Let $ \zeta_\chi $ be the function we seek to find. Consider $g(s)=\frac{\zeta_\chi'(s)}{\zeta_\chi (s)}$. From the Euler product representation we have 
 \begin{align*} g(s)=-\sum_{ n=1 }^\infty  \chi(n)\Lambda(n)n^{-s} = -\sum_{p,a} \chi(p^a)\Lambda(p^a)p^{-as} = \\ -\sum_p \chi(p)\Lambda(p)p^{-s} - \sum_{p, a \ge 2} \chi(p^a )\Lambda(p^a)p^{-as}, \end{align*} where $p$ ranges over the primes, $ a $ over the naturals, and $ \Lambda $ is the von Mangoldt function. The second sum in the rhs is absolutely convergent for $ \Re s > 1/2 $, hence the poles and residues of $ g $ are exactly those of the function \[ \tig (s)= -\sum_{p}  \chi(p)p^{-s} \log p . \]

Thus the problem is reduced to the one of constructing the function $ \tig $ with the required poles and residues.

\bfseries 
\begin{center}
Mellin transform	
\end{center}
\mdseries

We seek the function $\tig $ in the form
\[ \tig (s) = h(s)+\int_1^{\infty}q(x)x^{-s} dx, \Re s>1 , \] where $h$ is analytic in the halfplane $ \Re s> 1/2 $, and $ q ( x ) = o ( 1) $, $ x \to +\infty $.
%а функция $q$ допускает оценку вида:
%\begin{equation}\label{esq} |q(x)|<\frac{C}{ |\log x| +1} .\end{equation} 
%\begin{equation}\label{esq} |q(x)|<\frac{C}{ |\log x| +1} .\end{equation} 

The following lemma extracts from \cite[8.1]{S} a part of the argument we are going to need. A proof is provided for completeness.

\begin{lemma}
Let $ q ( x ) = o ( 1) $, $ x \to +\infty $. %удовлетворяющая \eqref{esq}. 
Then there exists a completely multiplicative function $\chi $ such that the function
\[  \int_1^{\infty}q(x)x^{-s} dx + \sum_{p} \chi(p)p^{-s} \log p, \]
initially defined in the halfplane $ \Re s > 1 $, extends analytically to the halfplane $ \Re s > 21/40 $ unconditionally, and to the halfplane $ \Re s > 1/2 $ if the RH holds.
\end{lemma}

\begin{proof} Arguing as in \cite[8.1]{S} we are going to construct $ \chi $ for a subset, $ \mathcal P $, of primes using the identity 
\begin{align*}
 \int_1^{\infty}q(x)x^{-s}dx +\sum_{p \in \mathcal P} \chi(p)p^{-s}\log p = \\ s\int_1^{\infty}\left( \int_1^xq(y)dy-\sum_{p\le x, p \in \mathcal P } \chi(p) \log p \right)x^{-s-1}dx . \end{align*} 
It suffices to show that there exists a $ \chi $ such that 
\begin{equation}\label{2140-e}
   r(x):= \int_1^x q(y)dy-\sum_{p\le x, p \in \mathcal P } \chi(p)\log p = O \bigl(x^{\frac{21}{40}} \log x \bigr).
\end{equation}
Let 
\begin{equation}\label{xj} x_0=2, x_{j+1}=x_j+x_j^{\frac{21}{40}}. \end{equation} 
It is enough then to establish \eqref{2140-e} at the sequence $ x= x_j$. Indeed, if it is satisfied for $ x = x_j $ then for $x \in [ x_j , x_{ j+1} ) $ we have
$$ r(x) = r(x_j) + \int_{x_j}^x q(y)dy-\sum_{x_j<p\le x, p \in \mathcal P} \chi(p)\log p .$$ The first term in rhs is $O(x^{ 21/40 } \log x )$ by assumption, \[ \int_{x_j}^xq(y)dy = O(x^{\frac{21}{40}}) \] by the boundedness of $ q$, and the rightmost term in the right hand side is trivially $ O(x^{21/40}\log x) $.
 
 It remains to choose $ \chi $ so that $ r(x_j ) = O\bigl(x_j^{21/40}\log x_j \bigr) $. In fact we are going to choose it so that $ r ( x_j ) = O ( \log x_j ) $. This is done by induction in $ j $. Let a subset $ \mathcal P \cap [ 0 , x_j ) $ and the function $ \chi $ on it  be constructed so that $ r( x_k ) \le C \log x_k $ for some $ C $ for $ k = 1, 2, \ldots, j$. Clearly
 \begin{equation} \label{esqi-e} \int_{x_j}^{x_{j+1}} q(y)dy = o\bigl( x_j^{\frac{21}{40}}\bigr). \end{equation} 
 Let $ c_j $ be the argument of the number
 $ r( x_j ) + \int_{ x_j }^{ x_{j+1}} q $, and let $ \chi (p) = e^{ i c_j } $, $ p \in \mathcal P_j $, where $ \mathcal P_j $ is a subset of primes on the interval $ [ x_j, x_{ j+1} ) $ to be chosen as follows. By construction we have 
 \[ | r ( x_{j+1} ) | \le \left| \left|r( x_j ) + \int_{ x_j }^{ x_{j+1}} q \right| - \sum_{ p \in \mathcal P_j } \log p \right| .\]
Starting from an empty set, we shall add numbers to the $ \mathcal P_j $ until the rhs becomes $ \le \log x_{ j+1} $. This is possible since 
\[ \left|r( x_j ) + \int_{ x_j }^{ x_{j+1}} q \right| = o\bigl(x_j^{\frac{21}{40}}\bigr) \] by the induction assumption and \eqref{esqi-e}, while the interval $[x_j, x_{j+1}) $ contains at least $ C x_j^{21/40}/\log x_j $ primes (\cite{BH}, p. 562) for some constant $ C>0 $ independent of $ j $ . Thus $ r_{j+1} \le C \log x_{j+1} $, as required.
 
The construction above defines $ \chi $ on a subset $ \mathcal P = \cup_j \mathcal P_j $ of primes. Extending it to all primes is done verbatim as in \cite[Lemma 5.3]{S}. This proves the unconditional part of the assertion.

Assuming the Riemann hypothesis we follow the same argument with $ x_{i+1}=x_i + 4x_i^{1/2} \log x_i $ instead of \eqref{xj}, and take into account that  the number of primes in the interval $ [ x , x + c \sqrt{x} \log x ] $ is estimated below by $ \sqrt x $ for all $ c > 3 $ \cite{Dudek}. This gives  \[  r ( x ) = O( \sqrt{x} \log^2 x ) \] which implies the required assertion in the conditional case.
 \end{proof}
 
 The function $ q$ in \cite{S} is "reverse engineered"{} from its Mellin transform, $ R $, given explicitly, and has power decay, $ q ( x) = O ( x^{ -1/40} \log^2  x$). Our construction will be done in terms of $ q $ rather than $ R $. The price to pay is that we are not going to have any control over the decay of $ q$ in the power scale. It is for this reason our stip in the unconditional result is shifted by $ 1/40 $ to the right as compared to \cite{S}. 
 
% Заметим, что условие \eqref{esq} в этой лемме можно заменить более слабым условием убывания $ q ( x ) = o ( 1 )$,  $ x \to \infty $, но это не приводит к какому-либо улучшению основного результата. 
 
Thus it remains to find an analytic function $g_1 $ in the halfplane 
$ \Re s > 1 $ of the form \[ g_1(s)=\int_1^{\infty}q(x)x^{-s}dx,\] with $ q $ vanishing as $ x\to +\infty $ which admits meromorphic extension to the halfplane $\Re s>21/40 $ with the prescribed poles and respective residues. 

\begin{lemma}
Let $g$ be an analytic function in the halfplane $ \Re z > 1 $ such that 
$\sup |z|^2 |g(z)| < \infty $. Then there exists a continuous function, $q$, $ q ( s ) = o ( 1 ) $, $ s\to +\infty $, such that \[ g(s)=\int_1^{\infty}q(s)x^{-s}dx, \; \Re s > 1 . \] 
\end{lemma}

\begin{proof}
Consider the function $h(t)=g(-it+1)$. The function $h$ belongs to the Hardy class $ H^2_+ $, hence the restriction $\left. h \right|_{\mathbb R } $ is the inverse Fourier transform of a certain function $ p \in L^2 ( \mathbb R ) $, vanishing on the negative real axis. Since $ \left.h \right|_{ \mathbb R } \in L^2 \cap L^1 $, the function $ p $ is the classical Fourier transform of $ h $, hence it is continuous and vanishes as $ x \to +\infty $
by the Riemann--Lebesgue lemma. The function $q(s):= p(\log s)$ then also vanishes as $ s \to +\infty $. Finally we have (recall that $s=-it+1$):  
\begin{align*}\int_1^{\infty}q(x)x^{-s} dx =\int_0^{\infty}q(e^y)e^{(1-s)y}dy =\int_0^{\infty}p(y)e^{ity}d y=h(t)=g(s) , \end{align*} as required.
\end{proof}

\medskip
\bfseries 
\begin{center}
Construction of $ g $
\end{center}
\mdseries

Let $\alpha = 21/40 $ in the unconditional case, and $ \alpha = 1/2 $ if RH is satisifed. Assume first that the sets $ Z $ and $ P $ have no accumulation points at finite distance. In view of lemma 2, Theorem 2 for this case will be proven if we manage to find a function $g$ analytic in the halfplane $ \Re s > 1 $ and satisfying $\sup_{ \Re z > 1 } |g(z)|\, |z|^2< \infty $, which admits meromorphic extension to the halfplane $ \Re z > \alpha $ with the given poles and residues in the strip $\alpha < \Re z <1 $.

Notice first that, given a point $z_0$, $\alpha< \Re z_0< 1 $, and a number $ C > 0 $, one can choose $ n = n ( C , z_0 )$ large enough so that the function \[ g_{z_0}(z) =\frac{1}{(z-z_0)(z-z_0+1)^{2n}} \] 
has the following properties, 

(i) $|g_{z_0}(z)| <C$ for $ \Re z >1 $.

(ii) $ g_{ z_0 } $ is analytic in $\{ \Re z > \alpha \} $ except at $ z_0 $,  has a simple pole at $z_0$ with $ \operatorname{Res}_{ z_0 } g_{ z_0 } = 1 $. 

(iii) $|g_{z_0}(z)|<C$ for $|z-z_0|>20 $, $ \Re z >\alpha$. 

\begin{lemma}
Let $ \Sigma $ be a subset of the strip $\alpha< \Re z < 1 $ having no accumulation points at finite distance, and $ m \colon \Sigma \longrightarrow \mathbb Z \setminus \{ 0 \} $ be an arbitrary function. Then there exists a meromorphic function $ g $ in the halfplane $\Re z >\alpha$, whose set of poles coincides with $ \Sigma $, all poles are simple, $ \operatorname{Res}_z g =  m ( z ) $ for $ z \in \Sigma$, and \[\sup_{ \Re z > 1 } |g(z)| |z|^2 < \infty .\]
\end{lemma}

\begin{proof} Let $ G_1 $ be an arbitrary function analytic in the halfplane $\Re z>\alpha$, having no zeroes and satisfying $ G_1 ( z ) = O ( |z|^{-2} ) $ as $ |z| \to \infty $. $G_1(z)=e^{-z}z^{-2}$ will do.

Fix an arbitrary enumeration of $ \Sigma $. Given a $p_i \in \Sigma $, define $ g_i $ to be the function $ g_{ p_i } $ satisfying properties (i)--(iii) with \[ C=\frac{G_1(p_i)}{ |m (p_i)| 2^{i+1}}. \] Let
\begin{equation}\label{g} g(z)= G_1(z) \sum_i m ( p_i) \frac{g_i(z)}{G_1(p_i)}. \end{equation} Let us first check that the series in the rhs converge absolutely at any point $ z \notin \Sigma $ in the halfplane $ \Re z > \alpha $, and thus the function $ g $ is meromorphic with simple poles at $ \Sigma $ and no other singularities.

Indeed let $z \notin \Sigma$, $ \Re z > \alpha $. Clearly $|z-p_i|\le 20$ for at most finitely many $ i $'s. If $|z-p_k|>20$ for some $ k $, then \[ |g_k(z)| < \frac{G_1(p_k)}{|m ( p_k )| 2^{k+1}}, \] by (iii) from whence \[ \left| m ( p_k ) \frac{g_k(z)}{G_1(p_k)}\right|< 2^{-k-1} ,\] and the convergence is proven. The equality $ \operatorname{Res}_{ p_i } g = m ( p_i ) $ is obvious. It remains to notice that for $\Re z> 1 $ we have \[ \left| m ( p_i ) \frac{g_i(z)}{G_1(p_i)} \right|\le 2^{-i-1} ,\] hence $ | g ( z) | \le | G_1 ( z )| $, and thus $ \sup_{ \Re z \ge 1} |G_1(z)||z|^2 < \infty $, as required.
\end{proof}

Theorem 2 is thus proved in the partial case when the sets $ Z $ and $ P $ do not accumulate at finite distance. The general case is reduced to this one via dyadic decomposition of the strip in the same way as in \cite[Section 8.1]{S}.

\begin{thebibliography}{90}
\bibitem{S} K. Seip, Universality and distribution of zeros and poles of some zeta functions,  J. Anal. Math. \textbf{141}(2020), no. 1, 331--381. arXiv:1812.11729.

\bibitem{Hels} H. Helson, Compact groups and Dirichlet series, Ark. Mat. \textbf{8}(1969), 139--143.

\bibitem{BH} R. C. Baker, G. Harman, and J. Pintz, The difference between consecutive primes. II, Proc. London
Math. Soc. 83 (2001), 532--562.

\bibitem{saks} E. Saksman and C. Webb, The Riemann zeta function and Gaussian multiplicative chaos: statistics on
the critical line, arXiv:1609.00027.

\bibitem{Dudek} A. Dudek, On the Riemann hypothesis and the difference between primes, Int. J. Number Theory
\textbf{11}(2015), 771--778.

\end{thebibliography}

\end{document}



\thispagestyle{empty}
 \begin{center}
 	\vfill
 	\phantom{}
 	
 	\vspace{10pt}
	\large 	
 	Санкт-Петербургский государственный университет

 	\vspace{50pt}
 	
 	
 	{\Large\bf --- Бочков Иван Алексеевич  ---}\\
 	
 	\vspace{20pt}
 	 	
 	{\Large\bf Выпускная квалификационная работа}\\
 	
 	\vspace{50pt}
 	
 	{\Huge \bf --- <<Спектральные свойства матриц Якоби в случае предельного круга
>> ---} 
 	
 	\vspace{40pt}
 	
 	Уровень образования: магистратура\\
 	Направление: 01.04.01 <<Математика>>\\
 	Основная образовательная программа: ВМ.5832.2019\\
 	Профиль (при наличии): нет
 	
 	\vspace{20pt}
 	
 	 	\begin{flushright}
 	Научный руководитель:\\
 	{\bf  к.ф-м.н., в.н.с. ПОМИ � АН \\  � оманов � оман Владимирович }
 	
 	\vspace{20pt}
 	
 	� ецензент:\\
 	{\bf д.ф-м.н., с.н.с. ПОМИ � АН \\  Капустин Владимир Владимирович 
 	
 	ведущий научный сотрудник ПОМИ � АН } 	
 	\vfill
%\inputencoding{utf8}

 	\end{flushright}
 	\vspace{20pt}
 	\small{Санкт-Петербург\\
 		   2021г}
 \end{center}

\vspace{60pt}

\bfseries 
\begin{center}
Введение	
\end{center}
\mdseries
  
В статье \cite{S} К. Сейпом был рассмотрен некоторый класс дзета-функций, включающий в себя дзета-функ\-цию � имана. А именно, функции вида:

\begin{equation}\label{H} F(s)= \sum_1^{\infty}\chi(n)n^{-s}, \end{equation} где $ \chi \colon \mathbb N \to \mathbb T $, $ \mathbb T = \{ z \in \mathbb C \colon |z|=1 \}$, -- вполне мультипликативная функция, т. е. такая что $ \chi ( nm ) = \chi ( n ) \chi ( m ) $, $ n , m \in \mathbb N $. Будем называть такую функцию обобщенной дзета-функцией по Сейпу (ОДС). При $\Re$ $s>1$ этот ряд сходится абсолютно и задает аналитическую функцию, при $ \chi \equiv 1 $ она совпадает с дзета--функцией � имана. Одна из возможных точек зрения на дзета-функцию � имана состоит в рассмотрении ее как элемента класса рядов Дирихле, отвечающих различным функциям $ \chi $. При такой точке зрения естественным становится вопрос о нулях и полюсах продолжения функции $ F $ за прямую $\Re s=1$. Повторяя рассуждения из вывода формулы Эйлера, легко видеть, что при $ \Re s > 1 $ \[ F(s)=\prod_p \frac{1}{1-\chi(p)p^{-s}} . \] 
Из этого представлени немедленно вытекает, что функция $ F $ не имеет нулей в полуплоскости $ \Re s>1$.

Изучение дзета-функций вида \eqref{H} было начато Х. Хелсоном, который в 1969 году показал \cite{Hels}, что при почти всех значениях характера $ \chi $ функция $ F $ аналитична в полуплоскости $ \Re s > 1/2 $ и не имеет нулей при $ \Re s > 1/2 $. Плчти всюду в этом утверждении относится к характерам, параметризованным последовательностью $ \{ \chi (p) \} \subset \mathbb T^\infty $, и подразумевает стандартную меру произведения для тора $ \mathbb T^\infty $. Сравнительно недавно было показано \cite{saks}, что для почти всех характеров $ \chi $ функция $ F $ не допускает мероморфного продолжения в полуплоскость $ \Re s > \alpha $ c каким-либо $ \alpha < 1/2 $, и в этом смысле результат Хелсона неулучшаем.

 Будем считать, что функция $ F $ продолжается мероморфно хотя бы до прямой $Re$ $s=\frac{1}{2}$. В работе К. Сейпа исследовался вопрос о том, каким может быть множество нулей функции $ F $. В терминах задачи Хелсона это вопрос о структуре аналитического продолжения для характеров из исключительного с точки зрения теории меры множества. Легко видеть, что никакой аналог гипотезы � имана для нулей функции $ F $ не справедлив. В \cite{S} было gоказано, что множество нулей в полосе $\frac{1}{2} < \Re s<1$ может быть практически любым. А именно, была установлена следующая 

\begin{theorem}{\cite[Theorem 1.4]{S}}
(i) Для любого множества нулей и полюсов в полосе  $\frac{1}{2} < \Re s<\frac{3}{4}$, такого что в прямоугольнике $\frac{1}{2} < $$\Re$ $s<1$, $|\Im$ $s|<n$ не более $O(n^2)$ нулей и полюсов, существует обобщенная по Сейпу дзета-функция с данным множеством нулей и полюсов.

(ii) Для любого локально конечного множества $ \mathfrak O $  в полосе $\frac{1}{2} < 1 - \Re s<1 - \epsilon $, $ \epsilon = \frac 1{40}$, существует ОДС, множество нулей которой свпадает с $ \mathfrak O $.
\end{theorem}

Любопытно отметить, что, если гипотеза � имана справедлива, утверждение пункта (ii) этой теоремы справедливо при произвольном $\epsilon > 0 $, а при некоторых еще более сильных гипотезах о регулярности распределения простых чиcел и при $ \epsilon = 0 $.

К. Сейпом был задан естественный вопрос - верно ли, что для любого докально конечного множества нулей и полюсов в полосе $\frac{21}{40} < $$\Re$ $s<\frac{39}{40}$ существует ОДС с ровно такими нулями и полюсами? 

Данная работа имеет цель доказать это утверждение. Справедлива следующая

\begin{theorem}
 
 1)Для любого  локально конечного множества нулей и полюсов в полосе  $\frac{21}{40} < \Re s<1$,   существует обобщенная по Сейпу дзета-функция с данным множеством нулей и полюсов.
 
 2)Если гипотеза � имана верна, то
 для любого $\epsilon >0$ и любом локально конечного множества нулей и полюсов в полосе  $\frac{1}{2}+\epsilon < \Re s<1$,   существует ОДС с данным множеством нулей и полюсов.
 
\end{theorem}

Мы приведем доказательство только первого пункта, доказательство второго аналогично.

\medskip 

\bfseries 
\begin{center}
Логарифмическая производная	
\end{center}
\mdseries

Опишем основную идею доказательства. Пусть $ F $ -- искомая функция. � ассмотрим функцию $g(s)=\frac{F'(s)}{F(s)}$. Если исходная функция имела ноль кратности $a$, то $g$ будет иметь в той же точке полюс с вычетом, равным $a$. Если же у исходной функции был полюс кратности $a$, то $g$ будет иметь в той же точке полюс с вычетом, равным $-a$.

С другой стороны, из произведения Эйлера функция $g$ может быть выражена явно: \begin{align*} g(s)=-\sum \chi(n)\Lambda(n)n^{-s} = -\sum_{p,a} \chi(p^a)\Lambda(p^a)p^{-as} = \\ -\sum_p \chi(p)\Lambda(p)p^{-s} - \sum_{p, a \ge 2} \chi(p^a )\Lambda(p^a)p^{-as}, \end{align*} где $p$ - простое, а $ \Lambda (n )$ -- функция фон Мангольдта, $ \Lambda ( p^k ) = \log p $ при $ p$ простом и $ k \in \mathbb N $; $ \Lambda ( n ) = 0 $ во всех остальных случаях. Вторая сумма в правой части представляет собой ряд, абсолютно сходящийся при $ \Re s > 1/2 $, поэтому полюса функции $ g $ и вычеты в них совпадают с соответствующими полюсами и вычетами функции \[ \tig (s)= -\sum_{p}  \chi(p)p^{-s} \log p . \]

Таким образом, достаточно построить функцию $ \tig $, обладающую требуемыми полюсами и вычетами.

\bfseries 
\begin{center}
Приближение преобразованием Меллина	
\end{center}
\mdseries

Мы будем искать функцию $\tig $ в виде: \[ \tig (s) = h(s)+\int_1^{\infty}q(x)x^{-s}, \Re s>1 , \]  где $h$ - функция, аналитичная в полуплоскости $ \Re s>\frac{1}{2}$, а функция $ q ( x ) = o ( 1) $, $ x \to +\infty $.
%а функция $q$ допускает оценку вида:
%\begin{equation}\label{esq} |q(x)|<\frac{C}{ |\log x| +1} .\end{equation} 
\begin{equation}\label{esq} |q(x)|<\frac{C}{ |\log x| +1} .\end{equation} 

\begin{lemma}
Пусть $q$ -- измеримая функция такая, что $ q ( x ) = o ( 1) $, $ x \to +\infty $. %удовлетворяющая \eqref{esq}. 
Тогда существует вполне мультипликативная функция $ \chi $ такая, что функция
\[ \int_1^{\infty}q(x)x^{-s} + \sum_{p} \chi(p)p^{-s} \log p, \]
изначально заданная в полуплоскости $ \Re s > 1 $, допускает аналитическое продолжение на полуплоскость $ \Re s > \frac{21}{40} $. 
\end{lemma}

\begin{proof}
Построим сначала функцию $ \chi $ для некоторого подмножества $ \mathcal P $ простых чисел. Имеем: 
\begin{align*}
 \int_1^{\infty}q(x)x^{-s} +\sum_{p \in \mathcal P} \chi(p)p^{-s}\log p = \\ s\int_1^{\infty}\left( \int_1^xq(y)dy-\sum_{p\le x, p \in \mathcal P } \chi(p) \log p \right)x^{-s-1}dx . \end{align*} 
В силу этого представления достаточно доказать, что существует вполне мультипликативная функция $ \chi $ такая, что
\begin{equation}\label{2140}
   r(x):= \int_1^x q(y)dy-\sum_{p\le x, , p \in \mathcal P } \chi(p)\log p = O \bigl(x^{\frac{21}{40}} \log x \bigr).
\end{equation}

 Для этого возьмем последовательность $$x_0=2, x_{i+1}=x_i+x_i^{\frac{21}{40}}. $$ Теперь нам достаточно обеспечить  \eqref{2140} только в точках $x_i$. Действительно, пусть $ r(x_i ) = O\bigl(x_i^{\frac{21}{40}}\bigr) $. Любое достаточно большое $x$ может быть записано в виде $ x=x_k+y$ для какого-то $k$ и $ y \in [0, x_k^{\frac{21}{40}}] $. Тогда $$ r(x) = r(x_k) + \int_{x_k}^x q(y)dy-\sum_{x_k<p\le x, p \in \mathcal P} \chi(p)\log p .$$ Первое слагаемое есть  $O(x_i^{\frac{21}{40}})$ по предположению. Заметим далее, что $\int_{x_k}^xq(y)dy = O(x^{\frac{21}{40}})$, так как подынтегральная функция ограничена, а длина промежутка интегрирования не превосходит $ x_k^{\frac{21}{40}} $. Аналогичным образом, оценивая число простых чисел на промежутке $ [x_k,x]$ его длиной, получим: \[\sum_{x_k<p\le x , p\in \mathcal P} \chi(p) \log p = O(x^{\frac{21}{40}}\log x) .\]
 
 Остается выбрать функцию $ \chi $ так, чтобы $ r(x_i ) = O\bigl(x_i^{\frac{21}{40}}\bigr) $. Мы докажем, что на самом деле функцию $ \chi $ можно выбрать так, что $ r ( x_i ) = O ( \log x_i ) $. Для этого воспользуемся индукцией по $ i $. Пусть $ r( x_k ) \le C \log x_k $ c некоторой константой $ C $ при всех $ k = 1, 2, \ldots, i$. Требуется получить ту же оценку при $ k= i+1$. Ясно, что
 \begin{equation} \label{esqi} \int_{x_i}^{x_{i+1}} q(y)dy = o\bigl( x_i^{\frac{21}{40}}\bigr). \end{equation} 
 Покажем, что можно шаг индукции можно сделать, выбирая значения функции $ \chi $ на простых числах из промежутка $ [ x_i , x_{ i+1 }) $. Для этого достаточно выбрать подходящее подмножество $ \mathcal P_i $ простых чисел на интервале $ [ x_i, x_{ i+1} ) $ и для всех чисел $ p \in \mathcal P_i $ положить $ \chi (p) = e^{ i c_i } $, где $ c_i $ -- аргумент числа
 $ r( x_i ) + \int_{ x_i }^{ x_{i+1}} q $. Выбор подмножества $ \mathcal P_i $ производится следующим образом. По построению имеем:
 \[ | r ( x_{i+1} ) | \le \left| \left|r( x_i ) + \int_{ x_i }^{ x_{i+1}} q \right| - \sum_{ p \in \mathcal P_i } \log p \right| .\]
Начиная с пустого множества, будем добавлять точки в множество $ \mathcal P_i $ до тех пор, пока правая часть не станет $ \le \log x_{ i+1} $. Это возможно, поскольку \[ \left|r( x_i ) + \int_{ x_i }^{ x_{i+1}} q \right| = o\bigl(x_i^{\frac{21}{40}}\bigr) \] в силу индуктивного предположения и \eqref{esqi}, а промежуток $[x_i, x_{i+1}) $ содержит $ \gg C\frac{x_i^\frac{21}{40}}{\log x_i} $ простых чисел (\cite{BH}, с. 562). Таким образом, $ r_{i+1} \le C \log x_{i+1} $, что и требовалось.
 
 Приведенное выше построение задает функцию $ \chi $ на некотором подмножестве $ \mathcal P = \cup_i \mathcal P_i $ простых чисел. Для, того, чтобы функция $ \chi $ была полностью определена, требуется задать ее на остальных простых числах, не испортив достигнутой аналитичности логарифмической производной. Соответствующая конструкция, основанная на соображениях из теории вероятности, стандартна \cite[Lemma 5.3]{S} и будет опущена.
 \end{proof}
 
 Приведенное доказательство Леммы 1 представляет собой  модификацию рассуждения из \cite[Sect. 8]{S}, пригодную в рассматриваемой ситуации. Функция $ q$ в \cite{S} строится из иных соображений и обладает степенным убыванием ($ q ( x) = O ( x^{ -1/40} \log^2  x$).  
 
% Заметим, что условие \eqref{esq} в этой лемме можно заменить более слабым условием убывания $ q ( x ) = o ( 1 )$,  $ x \to \infty $, но это не приводит к какому-либо улучшению основного результата. 
 
Из Леммы 1
следует, что если если удастся найти аналитическую в полуплоскости $ \Re s > 1 $ функцию $g_1$ вида \[ g_1(s)=\int_1^{\infty}q(x)x^{-s}dx,\] где $ q = o(1) $ при $ x\to +\infty $ %удовлетворяет \eqref{esq}
, допускающую мероморфное продолжение в полуплоскость $\Re s>\alpha$ с заданными нулями и полюсами, то найдется вполне мультипликативная функция $\chi$, такая что функция \[ g_1(s) - \sum_{p}\chi(p) p^{-s}\log p \] аналитична в $\Re s>\alpha$. Тогда функция $\sum_{p}\chi(p)\log p p^{-s} $ будет мероморфной в полуплоскости $\Re s>\alpha$, и иметь те же полюса и вычеты, что и $g_1$. 

%Значит, нам достаточно подобрать $g_1$ с данными полюсами и вычетами, давайте переобозначим $g_1$ за $g$.
\medskip
\bfseries 
\begin{center}
Оценка в классе Харди
\end{center}
\mdseries

\begin{lemma}
Пусть $g$ - аналитическая функция в полуплоскости $ \Re z > 1 $, %$\Re z>1-\epsilon$ для любого $\epsilon>0$ 
 такая, что $ |g(z)| \le C (|z|^2+1)^{-1} $ для некоторой константы $C$. Тогда существует непрерывная функция $q$, такая что при $\Re s>1$ \[ g(s)=\int_1^{\infty}q(s)x^{-s}dx, \] и $ q ( s ) = o ( 1 ) $, $ s\to +\infty $.
\end{lemma}

\begin{proof}

Пусть $g$ удовлетворяет условию леммы. � ассмотрим функцию $h(t)=g(-it+1)$. Тогда функция $h$ аналитична при $\Im t>0$ и принадлежит классу Харди $ H^2_+ $. В силу теоремы Пэли--Винера сужение $\left. h \right|_{\mathbb R } $ совпадает с обратным преобразованием Фурье некоторой функции $ p \in L^2 ( \mathbb R ) $, обращающейся в нуль на отрицательной полуоси. 
%Заметим, что для любой производной функции $ h $ в полуплоскости $ \Im t > - \epsilon / 2 $ справедлива оценка $ h^{(k)} ( t ) \le C_k ( 1 + |z|^2 )^{ -1 } $. Чтобы убедиться в этом, выберем произвольное положительное $\kappa < \epsilon / 2 $ и рассмотрим для достаточно большого $|t| $ интегральное представление  \[ h^{(k)} ( t ) = C_k \int_{ |z-w| = \kappa } \frac{ h(w) dw }{ ( t - w )^{k+1} } , \; \Im t \ge 0 . \] Из него немедленно следует, что \[ | h^{(k)} ( t ) |\le C_k \tau^{ -k-1} ( |t| - \tau )^{ -2} \ll |t|^{ -2} . \]
Поскольку $ \left.h \right|_{ \mathbb R } \in L^2 \cap L^1 $, функция $ p $ есть классическое преобразование Фурье функции $ h $: 
\[ p ( x ) = C \int_{\mathbb R} e^{ -itx} h(t) dt .\]
По условию, $ t h (t ) = O ( t^{-1} )$, а, значит, $ t h ( t) \in L^2 $. Следовательно, функция $ p $ обладает квадратично суммируемой обобщенной производной (т. е. принадлежит пространству Соболева $ W^{2,1} $). Интегрируя тождество $ (p^2 )^\prime = 2 p p^\prime $ и применяя неравенство КБШ, получим
\[ |p^2 ( y ) - p^2 ( x ) | \le 2 \left(\int_{x}^y p^2 (s) ds  \int_{x}^y (p^\prime)^2 (s) ds \right)^{ 1/2 } , \]
откуда следует, что $ p (x ) \to 0 $ при $ x \to +\infty $, а, значит, и функция $q(s):=p(\log s)$ стремится к нулю при $  s \to +\infty $. 
% Интегрируя в этом равенстве по частям нужное число раз, получим, что $ p( x) = O ( |x|^{-k} ) $ при любом $ k $. 
%В частности, это означает, что функция $q(s):=p(\log s)$ удовлетворяет условию \eqref{esq}. 
Наконец (напомним, что $s=-it+1$):  
\begin{align*}\int_1^{\infty}q(x)x^{-s} dx =\int_0^{\infty}q(e^y)e^{(1-s)y}dy =\int_0^{\infty}p(y)e^{ity}d y=h(t)=g(s) , \end{align*} что и требовалось.

%Для того  она допускает представление вида $ h ( t ) $ тогда по условию $(*)|h(z)|(|z|^2+1)<C$ для какой-то константы $C$. Пусть $H(z)=zh(z)$. Тогда $(*)|H(z)|(|z|+1)<C$  для какой-то константы $C$.
%Заметим, что тогда и $(*)|H'(z)|(|z|+1)<C$ (возможно, с другой константой $C$). Действительно, возьмем какую-то точку $z_0$ с достаточно большим модулем. Тогда $H'(z_0)=\int_S\frac{H(z)}{z-z_0}dz$, где $S$ - круг радиуса $\frac{\epsilon}{2}$ с центром в точке $z_0$. Но тогда $|H'(z_0)|\le \int_S\frac{|H(z)}|{z-z_0}dz = \int_S\frac{2}{z\epsilon}dz =O(\frac{1}{z_0})$, что и требовалось. Аналогичное верно для любой $H^{(k)}(z)$ при фиксированном $z$.

%Обозначим через $H_{0,k}$ $(k=0,1,2,3)$ функцию $h^{(k)}(z)$, а через  $H_{1,k}$ $(k=0,1,2,3)$ функцию $H^{(k)}(z)$. Тогда для любых $i, j$ и для любого $a>0$  $\int_{-\infty}^{\infty}H_{i,j}^2(x+ia)dx< \infty$, а значит, выполнено условие теоремы Пэли-Винера (ССЫЛКА?). Отсюда существуют такие функции $p_{i, j} \in L_2$, что $H_{i, j}(z)=\int_0^{\infty}p_{i, j}e^{izt}dt$ в верхней полуплоскости. Более того, так как $H_{1,j}=zH_{0,j}$, то $p_{1,j}$ дифференцируемо, и $p_{1,j}=-ip'_{0,j}$. Аналогично, так как $H_{1,j}'=H_{1,j+1}$, то $p_{1, j+1}=izp_{1,j}$. Итого мы получаем, что, раз $p_{1,3} \in L_2$, то $p_{0,0}'z^3 \in L_2$. Обозначим $p_{0,0}$ за $p$.

%Оценим $p(x_0)$. C 1 cтороны, для любого $\delta >0$ существует сколь угодно большое чило $x$, такое что $p(x) < \delta$ из сходимости $p$ в $L_2$. С другой стороны, $$ p(x_0) \le p(x)+ \int_{x_0}^x|p'(t)|dt \le \delta+ \int_{x_0}^{\infty}|p'(t)|dt $$. Но по неравенству КБШ,$(\int_{x_0}^{\infty}|p'(t)|dt)^2  \le \int_{x_0}^{\infty}|p'(t)|^2t^6dt \int_{x_0}^{\infty}|t^{-6}dt \le Cx_0^{-5}$ для константы $C$, так как 1 интеграл справа ограничен, а второй - $O(x_0^{-5})$. Отсбда предельным переходом по $\delta$ получаем $p_{x_0}<\frac{C}{x^2+1}$ для некой константы $C$.

%Возьмем теперь $q(s)=p(\log s)$. Тогда условие $|q(x)|<\frac{C_1}{\log^2x+1}$ выполняется из вышесказанного. С другой стороны, 
%$\int_1^{\infty}q(x)x^{-s}=\int_1^{\infty}q(x)e^{-s \log x }dx=\int_0^{\infty}q(x)e^{-y (-it+1)}d e^y=\int_0^{\infty}q(e^y)e^{ity}d =\int_0^{\infty}p(y)e^{ity}d y=h(t)=g(s)$, где $t=is+1$. Что и требовалось.

\end{proof}

\medskip
\bfseries 
\begin{center}
Построение функции
\end{center}
\mdseries

Пусть $\alpha = \frac{21}{40} $. В силу Леммы 2 основная теорема будет доказана, если нам удастся построить функцию $g$ с данными полюсами и вычетами в них в полосе $\alpha < \Re z <1 $ и такую, что $\sup_{ \Re z > 1 } |g(z)|\, |z|^2< \infty $.

\begin{lemma}\label{gz}
Для любого $C>0$ и любой точки $z_0$ такой, что  $\alpha< \Re z_0< 1 $, существует мероморфная в полуплоскости $\Re z >\alpha$ функция $g_{z_0}$ со следующими свойствами:

(i) $|g_{z_0}(z)| <C$ при $ \Re z >1 $.

(ii) единственная особенность функции $ g_{ z_0 } $ в полуплоскости $ \Re z > \alpha $ -- простой полюс в точке $z_0$, и $ \operatorname{Res}_{ z_0 } g_{ z_0 } = 1 $. 

(iii) $|g_{z_0}(z)|<C$ при $|z-z_0|>20 $, $ \Re z >\alpha$. 
\end{lemma}

\begin{proof}
Покажем, что функция \[ g_{z_0}(z) =\frac{1}{(z-z_0)(z-z_0+1)^{2n}} \] при достаточно большом $n$ обладает всеми требуемыми свойствами. Свойство (ii) очевидно. Далее, при 
 $\Re z > 1 $ имеем: \[ |g_{z_0}(z)|\le\frac{1}{| 1 - \Re z_0 | |2 -\Re z_0|^{2n}} . \] Поскольку $|2 -\Re z_0|> 1 $, правая часть может быть сделана сколь угодно малой за счет выбора достаточно большого $ n $. Это доказывает (i). Наконец, при $|z-z_0|>20 $ очевидным образом $|(z-z_0+1)|>19$, и, еще увеличивая $ n $ при необходимости, можно добиться выполнения условия (iii). 
\end{proof}

\begin{lemma}
Пусть $ \Sigma $ -- произвольное локально конечное счетное подмножество точек полосы $\alpha< \Re z < 1 $, а $ S = \{ ( p_i , n_i ) \} $ -- счетное множество пар вида (точка из $ \Sigma $, ненулевое целое число). Тогда существует мероморфная в полуплоскости $\Re z >\alpha$ функция $g$ со следующими свойствами:

(i) Для любого $i$ функция $g$ имеет простой полюс в точке $p_i$, и $ \operatorname{Res}_{ z_i } g = n_i$.

(ii) Функция $ g $ не имеет особенностей в полосе $\alpha< \Re z < 1 $, отличных от $ p_i $.

(iii) \[\sup_{ \Re z > 1 } |g(z)| |z|^2 < \infty .\]
\end{lemma}

\begin{proof} Обозначим через $ G_1 $ произвольную непостоянную аналитическую в полуплоскости $\Re z>\alpha$ функцию, не имеющую в этой полуплоскости корней и такую, что $ G_1 ( z ) = O ( |z|^{-2} ) $ при $ |z| \to \infty $. Можно взять $G_1(z)=e^{-z}z^{-2}$.

Применим для точки $p_i$ Лемму 4, выбирая $z_0 = p_i$, \[ C=\frac{G_1(p_i)}{|n_i|2^{i+1}}. \] Обозначим через $g_i$ соответствующую функцию, существование которой утверждается в Лемме 4. Положим:
\begin{equation}\label{g} g(z)= G_1(z) \sum_i n_i \frac{g_i(z)}{G_1(p_i)}. \end{equation} Докажем cначала, что ряд в правой части сходится абсолютно в любой точке $ z_0 \notin \Sigma $ полуплоскости $ \Re z > \alpha $, и таким образом функция $ g $  корректно определена, а в точках множества $\Sigma $ она имеет простой полюс.

Действительно, рассмотрим точку $z_0 \notin \Sigma$, $ \Re z_0 > \alpha $. Ясно, что найдется разве лишь конечное число индексов $i$, таких что $|z_0-p_i|\le 20$. Но если для какого-то индекса $k$ выполнено неравенство $|z_0-p_k|>20$, то по свойству (iii) Леммы \ref{gz} \[ |g_k(z_0)| < \frac{G_1(p_k)}{|n_k|2^{k+1}}, \] откуда получим: \[ \left|n_k\frac{g_k(z)}{G_1(p_k)}\right|< 2^{-k-1} .\]  Абсолютная сходимость суммы в \eqref{g} доказана. 

Убедимся теперь, что функция $ g $ обладает требуемыми свойствами. В самом деле:

1) по построению для любого $i$ слагаемое $n_i\frac{g_i(z)}{G_1(p_i)}G_1(z)$ имеет в точке $p_i$ простой полюс, вычет в котором равен $n_i$, а остальные слагаемые в точке $ p_i $ аналитичны;

2) функция $ g$ не имеет особенностей вне $ \Sigma $;

3) при $\Re z> 1 $ очевидным образом будем иметь: \[ \left| n_k\frac{g_k(z)}{G_1(p_k)}\right|\le 2^{-k-1} ,\] а значит $ | g ( z) | \le | G_1 ( z )| $. Следовательно, $ \sup_{ \Re z \ge 1} ||G_1(z)||z|^2 < \infty $.

Что и требовалось.
\end{proof}

Для доказательства Теоремы 2 остается собрать все воедино. Пусть мы имеем какое-то локально конечное множество нулей и полюсов в полосе  $\frac{21}{40} < \Re s<1 $. Чтобы построить функцию с данными нулями и полюсами, достаточно построить функцию с данными полюсами в полосе в полосе  $\frac{21}{40} < \Re s<1 $ и данными вычетами в них. Эти полюса и вычеты мы подставляем в Лемму 4 в качестве пар $p_i, n_i$. Лемма даст нам функцию $g$, которая по Лемме 2 даст функцию $q$, которая в свою очередь даст нам нужную ОДС в силу Леммы 1.

Для доказательства версии теоремы в условии гипотезы � имана нужно доказать аналог Леммы 1 и заменой $\frac{21}{40}$ на $\frac{1}{2}$. Это достигается точно такой же конструкцией с заменой $\frac{21}{40}$ на $\frac{1}{2} $.

\begin{thebibliography}{90}
\bibitem{S} K. Seip, Universality and distribution of zeros and poles of some zeta functions, arXiv:1812.11729.

\bibitem{Hels} H. Helson, Compact groups and Dirichlet series, Ark. Mat. \textbf{8}(1969), 139–143.

\bibitem{BH} R. C. Baker, G. Harman, and J. Pintz, The difference between consecutive primes. II, Proc. London
Math. Soc. 83 (2001), 532–562.

\bibitem{saks} E. Saksman and C. Webb, The Riemann zeta function and Gaussian multiplicative chaos: statistics on
the critical line, arXiv:1609.00027
\end{thebibliography}


%\begin{document}

Let $ \chi \colon \mathbb N \to \mathbb T $, $ \mathbb T = \{ z \in \mathbb C \colon |z|=1 \}$ be a completely multiplicative function, and let 
\begin{equation}\label{H} \zeta_\chi (s)= \sum_1^{\infty}\chi(n)n^{-s} . \end{equation}  

Let $ \mathcal Z$ and $ \mathcal P $ be two disjoint multisets in the the complex plane.

\begin{theorem}
 For any locally finite disjoint multisets $ \mathcal Z$ and $ \mathcal P $ in the strip $\frac{21}{40} < \Re s<1$ there exists a function $ \chi $ such that $ \zeta_\chi $ admits meromorphic continuation into this strip with $ \mathcal Z $ the set of its zeroes and $ \mathcal P $ the set of its poles. 
\end{theorem}

\medskip 

\bfseries 
\begin{center}
Logarithmic derivative
\end{center}
\mdseries

The idea of the proof is as follows. Let $ \zeta_\chi $ be the function we seek to find. Consider $g(s)=\frac{\zeta_\chi'(s)}{\zeta_\chi (s)}$. From the Euler product representation we have 
 \begin{align*} g(s)=-\sum \chi(n)\Lambda(n)n^{-s} = -\sum_{p,a} \chi(p^a)\Lambda(p^a)p^{-as} = \\ -\sum_p \chi(p)\Lambda(p)p^{-s} - \sum_{p, a \ge 2} \chi(p^a )\Lambda(p^a)p^{-as}, \end{align*} where $p$ ranges over the primes, and $ \Lambda $ is the von Mangoldt function. The second sum in the rhs is absolutely convergent for $ \Re s > 1/2 $, hence the poles and residues of $ g $ are exactly those of the function \[ \tig (s)= -\sum_{p}  \chi(p)p^{-s} \log p . \]

Thus the problem is reduced to the one of constructing the function $ \tig $ with the required poles and residues.

\bfseries 
\begin{center}
Mellin transform	
\end{center}
\mdseries

We seek the function $\tig $ in the form
\[ \tig (s) = h(s)+\int_1^{\infty}q(x)x^{-s}, \Re s>1 , \] where $h$ is analytic in $ \Re s>\frac{1}{2}$, and $ q ( x ) = o ( 1) $, $ x \to +\infty $.
%а функция $q$ допускает оценку вида:
%\begin{equation}\label{esq} |q(x)|<\frac{C}{ |\log x| +1} .\end{equation} 
%\begin{equation}\label{esq} |q(x)|<\frac{C}{ |\log x| +1} .\end{equation} 

The following lemma extracts from \cite[8.1]{S} a part of the argument we are going to need. A proof is provided for completeness.

\begin{lemma}
Let $ q ( x ) = o ( 1) $, $ x \to +\infty $. %удовлетворяющая \eqref{esq}. 
Then there exists a completely multiplicative function $\chi $ such that the function
\[ \int_1^{\infty}q(x)x^{-s} + \sum_{p} \chi(p)p^{-s} \log p, \]
initially defined in the halfplane $ \Re s > 1 $, extends analytically to the halfplane $ \Re s > \frac{21}{40} $. 
\end{lemma}

\begin{proof} Arguing as in \cite[8.1]{S} we are going to construct $ \chi $ for a subset, $ \mathcal P $, of primes using the identity 
\begin{align*}
 \int_1^{\infty}q(x)x^{-s} +\sum_{p \in \mathcal P} \chi(p)p^{-s}\log p = \\ s\int_1^{\infty}\left( \int_1^xq(y)dy-\sum_{p\le x, p \in \mathcal P } \chi(p) \log p \right)x^{-s-1}dx . \end{align*} 
It suffices to show that there exists a $ \chi $ such that 
\begin{equation}\label{2140-e}
   r(x):= \int_1^x q(y)dy-\sum_{p\le x, , p \in \mathcal P } \chi(p)\log p = O \bigl(x^{\frac{21}{40}} \log x \bigr).
\end{equation}

Let $$x_0=2, x_{i+1}=x_i+x_i^{\frac{21}{40}}. $$ It is enough then to establish \eqref{2140-e} at the sequence $ x= x_i$. Indeed for $x \in [ x_i  x_{ i=1} ) $ 
$$ r(x) = r(x_k) + \int_{x_k}^x q(y)dy-\sum_{x_k<p\le x, p \in \mathcal P} \chi(p)\log p .$$ The first term in rhs is $O(x_i^{\frac{21}{40}})$ by assumption, $\int_{x_k}^xq(y)dy = O(x^{\frac{21}{40}})$ by boundedness of $ q$, and the rightmost term in rhs is trivially $ O(x^{\frac{21}{40}}\log x) $.
 
 It remains to choose $ \chi $so that $ r(x_i ) = O\bigl(x_i^{\frac{21}{40}}\bigr) $. In fact we are going to choose it so that $ r ( x_i ) = O ( \log x_i ) $. This is done by induction in $ i $. Let $ r( x_k ) \le C \log x_k $ for some $ C $ for $ k = 1, 2, \ldots, i$. Clearly
 \begin{equation} \label{esqi-e} \int_{x_i}^{x_{i+1}} q(y)dy = o\bigl( x_i^{\frac{21}{40}}\bigr). \end{equation} 
 Define $ \chi (p) = e^{ i c_i } $ where $ c_i $ is the argument of the number
 $ r( x_i ) + \int_{ x_i }^{ x_{i+1}} q $, for a subset $ \mathcal P_i $ of primes on the interval $ [ x_i, x_{ i+1} ) $ to be chosen as follows. By construction we have 
 \[ | r ( x_{i+1} ) | \le \left| \left|r( x_i ) + \int_{ x_i }^{ x_{i+1}} q \right| - \sum_{ p \in \mathcal P_i } \log p \right| .\]
Starting from an empty set, we shall add numbers to the $ \mathcal P_i $ until the rhs becomes $ \le \log x_{ i+1} $. This is possible since 
\[ \left|r( x_i ) + \int_{ x_i }^{ x_{i+1}} q \right| = o\bigl(x_i^{\frac{21}{40}}\bigr) \] by the induction assumption and \eqref{esqi-e}, while the interval $[x_i, x_{i+1}) $ contains at least $ C\frac{x_i^\frac{21}{40}}{\log x_i} $ primes (\cite{BH}, с. 562). Thus $ r_{i+1} \le C \log x_{i+1} $, as required.
 
The construction above defines $ \chi $ on a subset $ \mathcal P = \cup_i \mathcal P_i $ of primes. Extending it to all primes is done verbatim as in \cite[Lemma 5.3]{S}.
 \end{proof}
 
 The function $ q$ in \cite{S} is "reverse engeneered" from its Mellin transform, $ R $, given explicitly, and has power decay ($ q ( x) = O ( x^{ -1/40} \log^2  x$). Our construction will be done in terms of $ q $ rather than $ R $. The price to pay is that we are not going to have any control over the decay of $ q$ in the power scale. It is for this reason our stip is shifted by $ 1/40 $ to the right as compared to \cite{S}. 
 
% Заметим, что условие \eqref{esq} в этой лемме можно заменить более слабым условием убывания $ q ( x ) = o ( 1 )$,  $ x \to \infty $, но это не приводит к какому-либо улучшению основного результата. 
 
Thus it remains to find an analytic function, $g_1 $, in the halfplane 
$ \Re s > 1 $ of the form \[ g_1(s)=\int_1^{\infty}q(x)x^{-s}dx,\] with $ q $ vanishing as $ x\to +\infty $ which admits meromorphic extension to the halfplane $\Re s>21/40 $ with the prescribed zeroes and poles. 

\begin{lemma}
Let $g$ be an analytic function in the halfplane $ \Re z > 1 $ such that 
$\sup |z|^2 |g(z)| < \infty $. Then there exists a continuous function, $q$, $ q ( s ) = o ( 1 ) $, $ s\to +\infty $, such that \[ g(s)=\int_1^{\infty}q(s)x^{-s}dx, \; \Re s > 1 . \] 
\end{lemma}

\begin{proof}
Consider the function $h(t)=g(-it+1)$. The function $h$ belongs to the Hardy class $ H^2_+ $, hence the restriction $\left. h \right|_{\mathbb R } $ is the inverse Fourier transform of a certain function $ p \in L^2 ( \mathbb R ) $, vanishing on the negative real axis. Since $ \left.h \right|_{ \mathbb R } \in L^2 \cap L^1 $, the function $ p $ is the classical Fourier transform of $ h $, hence it is continuous and vanishes as $ x \to +\infty $
by the Riemann--Lebesgue lemma. The function $q(s):= p(\log s)$ then also vanishes as $ s \to +\infty $. Finally we have (recall that $s=-it+1$):  
\begin{align*}\int_1^{\infty}q(x)x^{-s} dx =\int_0^{\infty}q(e^y)e^{(1-s)y}dy =\int_0^{\infty}p(y)e^{ity}d y=h(t)=g(s) , \end{align*} as required.
\end{proof}

\medskip
\bfseries 
\begin{center}
Construction of $ g $
\end{center}
\mdseries

Let $\alpha = \frac{21}{40} $. By lemma just established the theorem will be proven if we manage to find a function $g$ analytic in the halfplane $ \Re s > 1 $ satisfying $\sup_{ \Re z > 1 } |g(z)|\, |z|^2< \infty $, which admits meromorphic extension to the halfplane $ \Re x > \alpha $ with the given poles and residues in the strip $\alpha < \Re z <1 $.

Notice first that, given a point $z_0$, $\alpha< \Re z_0< 1 $, and a number $ C > 0 $, one can choose $ n = n ( C , z_0 )$ large enough so that the function \[ g_{z_0}(z) =\frac{1}{(z-z_0)(z-z_0+1)^{2n}} \] 
has the following properties, 

(i) $|g_{z_0}(z)| <C$ for $ \Re z >1 $.

(ii) $ g_{ z_0 } $ is analytic in $\{ \Re z > \alpha \} $ except at $ z_0 $,  has a simple pole at $z_0$ with $ \operatorname{Res}_{ z_0 } g_{ z_0 } = 1 $. 

(iii) $|g_{z_0}(z)|<C$ for $|z-z_0|>20 $, $ \Re z >\alpha$. 

\begin{lemma}
Let $ \Sigma $ be an arbitrary locally finite subset in the strip $\alpha< \Re z < 1 $, and $ S = \{ ( p_i , n_i ) \} $ be a set of pairs of the form   (a point from $ \Sigma $, a nonzero integer). Then there exists a meromorphic function, $ g $, in the halfplane $\Re z >\alpha$, whose set of poles coincides with $ \Sigma $, all poles are simple, $ \operatorname{Res}_{ z_i } g = n_i$, and \[\sup_{ \Re z > 1 - \epsilon } |g(z)| |z|^2 < \infty .\]
\end{lemma}

\begin{proof} Let $ G_1 $ be an arbitrary function analytic in the halfplane $\Re z>\alpha$, having no zeroes and satisfying $ G_1 ( z ) = O ( |z|^{-2} ) $ as $ |z| \to \infty $. $G_1(z)=e^{-z}z^{-2}$ will do.

Now, given a $p_i \in \Sigma $, define $ g_i $ to be the function $ g_{ p_i } $ satisfying properties (i)--(iii) with \[ C=\frac{G_1(p_i)}{|n_i|2^{i+1}}. \] Let
\begin{equation}\label{g} g(z)= G_1(z) \sum_i n_i \frac{g_i(z)}{G_1(p_i)}. \end{equation} Let us first check that the series in the rhs converge at any point $ z \notin \Sigma $ in the halfplane $ \Re z > \alpha $, and thus the function $ g $ is meromorphic with simple poles at $ \Sigma $ and no other singularities.

Indeed let $z \notin \Sigma$, $ \Re z > \alpha $. Clearly $|z-p_i|\le 20$ for at most finitely many $ i $'s. If $|z-p_k|>20$ for some $ k $, then \[ |g_k(z)| < \frac{G_1(p_k)}{|n_k|2^{k+1}}, \] by (iii) from whence \[ \left|n_k\frac{g_k(z)}{G_1(p_k)}\right|< 2^{-k-1} ,\] and the convergence is proven. The equality $ \operatorname{Res}_{ z_i } g = n_i$ is obvious. It remains to notice that for $\Re z> 1 $ we have \[ \left| n_i \frac{g_i(z)}{G_1(p_i)} \right|\le 2^{-i-1} ,\] hence $ | g ( z) | \le | G_1 ( z )| $, and thus $ \sup_{ \Re z \ge 1} |G_1(z)||z|^2 < \infty $, as required.
\end{proof}

%Для доказательства Теоремы 2 остается собрать все воедино. Пусть мы имеем какое-то локально конечное множество нулей и полюсов в полосе  $\frac{21}{40} < \Re s<1 $. Чтобы построить функцию с данными нулями и полюсами, достаточно построить функцию с данными полюсами в полосе в полосе  $\frac{21}{40} < \Re s<1 $ и данными вычетами в них. Эти полюса и вычеты мы подставляем в Лемму 4 в качестве пар $p_i, n_i$. Лемма даст нам функцию $g$, которая по Лемме 2 даст функцию $q$, которая в свою очередь даст нам нужную ОДС в силу Леммы 1.

%Для доказательства версии теоремы в условии гипотезы � имана нужно доказать аналог Леммы 1 и заменой $\frac{21}{40}$ на $\frac{1}{2}$. Это достигается точно такой же конструкцией с заменой $\frac{21}{40}$ на $\frac{1}{2} $.

  
\end{document} 

