\documentclass[11pt]{article}
% arXiv PDF output directive (safe for journal submission as no effect there)
\pdfoutput=1
\usepackage{booktabs}
\usepackage{float}
% Robust CSV tables
\usepackage{longtable}
\usepackage{caption}
\usepackage[margin=1in]{geometry}
\usepackage{amsmath,amssymb,amsthm,mathtools}
\usepackage[T1]{fontenc}
\usepackage{lmodern}
\usepackage[utf8]{inputenc}
\usepackage{microtype}
\usepackage{hyperref}
\usepackage[numbers,sort&compress]{natbib}
\usepackage[capitalize,nameinlink,noabbrev]{cleveref}
\hypersetup{colorlinks=true,linkcolor=black,citecolor=black,urlcolor=black}


\usepackage{xcolor}
\newcommand{\need}[1]{#1}
\newcommand{\modif}[1]{}
%\newcommand{\modif}[1]{\textcolor{black}{#1}}
%\newcommand{\need}[1]{\textcolor{black}{#1}}
%\newcommand{\mod}[1]{\textcolor{black}{#1}}
\newcommand{\mage}[1]{#1}
\newcommand{\green}[1]{#1}
\newcommand{\olive}[1]{#1}

% Theorem environments

% Reference aliasing to silence legacy labels
% Global numeric constants (ζ-normalized route for the certificate)
% Box constant uses only K0 + K_ξ; C_Γ=0 in the certificate path
\providecommand{\czeroplateau}{0.17620819}% Poisson plateau lower bound c0(ψ)
\providecommand{\Kzero}{0.03486808}% arithmetic tail bound K0
% \providecommand{\Kxi}{0.16000000}% coarse unconditional ξ-zeros Carleson--box bound Kξ
\providecommand{\Kxi}{K_\xi}
% \providecommand{\CboxZeta}{0.19486808}
\providecommand{\CboxZeta}{K_0 + K_\xi}% diagnostic numerics moved to appendix (non-load-bearing)
% H^1–BMO / Hilbert constants
\providecommand{\CHzero}{0.26}% envelope: sup_t |H[φ_L](t)| (sum-form PSC)
\providecommand{\CHone}{2/\pi}% derivative: ||(H[φ_L])'||_∞ ≤ CHone / L (certificate)
% Unified Hilbert transform macro
\newcommand{\Hilb}{\mathcal H}
% Window H^1 constant and locked M_ψ (Whitney aperture absorbed in C_CE=1)
\providecommand{\CpsiHone}{0.2400}% C_ψ^{(H^1)} locked
\providecommand{\Mpsilocked}{(4/\pi)\,\CpsiHone\,\sqrt{\CboxZeta}}
\providecommand{\UpsilonLocked}{(2/\pi)\,\Mpsilocked/\czeroplateau}% diagnostic; not load-bearing
% Numeric-lock switch: default is unconditional (symbolic). We set \numericlocktrue to lock audited numbers.
\newif\ifnumericlock
\numericlockfalse
% Optional appendix lock for numeric sections
\newif\ifshownumerics
\shownumericsfalse
% Submission mode toggle (minimal arXiv/journal preamble variant)
\newif\ifsubmission
\submissiontrue
\ifsubmission
  % Override link styling to hidelinks for clean b/w PDFs
  \hypersetup{hidelinks}
  % Ensure diagnostics stay gated in submission builds
  \numericlockfalse
  \shownumericsfalse
\fi
% Optional numeric overrides (diagnostic only; non-load-bearing)
\ifnumericlock
  \renewcommand{\Kxi}{0.16000000}
  \renewcommand{\CboxZeta}{0.19486808}
  \renewcommand{\Mpsilocked}{0.13489371}
  \renewcommand{\UpsilonLocked}{0.48736}
\fi
\makeatletter
% (refalias scaffolding removed)
\makeatother
\AtBeginDocument{%
  % refalias disabled to keep labels explicit and avoid aliasing to optional material
  % \refalias{sec:CH-envelope}{lem:CH-explicit}%
  % \refalias{lem:poisson-lower}{lem:poisson-scale-stage2}%
  % \refalias{lem:hilbert-aux}{lem:hilbert-H1BMO}%
  % \refalias{lem:laplace-szego}{prop:discrete-Poisson}%
  % \refalias{lem:cayley-cont}{lem:Cayley-diff}%
  % \refalias{lem:wedge-stage2}{thm:numeric-close-stage2}%
  % bridge aliases removed to avoid early expansion issues
  % \refalias{sec:bridge-C}{thm:bridge-C}%
  % \refalias{thm:BridgeA}{thm:bridgeA}%
}

% Theorems
\newtheorem{theorem}{Theorem}[section]
\newtheorem{proposition}{Proposition}[section]
\newtheorem{lemma}{Lemma}[section]
\newtheorem{corollary}{Corollary}[section]
\theoremstyle{definition}
\newtheorem{definition}{Definition}[section]
\theoremstyle{remark}
\newtheorem{remark}{Remark}[section]

% Macros
\newcommand{\C}{\mathbb{C}}
\newcommand{\R}{\mathbb{R}}
\newcommand{\N}{\mathbb{N}}
\newcommand{\PP}{\mathcal{P}}
\newcommand{\HS}{\mathcal{S}_2}
\newcommand{\Half}{\{\,s\in\C:\ \Re s>\tfrac12\,\}}
\newcommand{\Poisson}{P}
\DeclareMathOperator{\Tr}{Tr}
\DeclareMathOperator{\dettwo}{det_2}
\DeclareMathOperator{\Arg}{Arg}

% Title & authors
% (Structure polished: consolidated proofs, added roadmaps, updated Section 4.5/4.6)
\title{A Proof of the Riemann Hypothesis via Boundary Positivity and Carleson Energy Bounds}
% --- AAB helpers ---
\newcommand{\AAB}{\textup{A\kern-0.05em A\kern-0.05em B}}
\DeclareMathOperator{\AABop}{A\!A\!B}
\author{Jonathan Washburn$^{1}$ and Elshad Allahyarov$^{1,2,3,4,*}$}
\date{\today}


% --- Safe macros (added) ---
\providecommand{\Hp}{\Omega} % right half-plane alias
\providecommand{\Cbox}{C^{(\zeta)}_{\mathrm{box}}}

\begin{document}
\maketitle
\begin{center}
\begin{tabular}{l}
$^1$Recognition Science Institute, Austin, Texas, USA \\
  $^2$Institut f\"ur Theoretische Physik II: Weiche Materie, Heinrich-Heine Universit\"at D\"usseldorf, \\
  \,\, Universit\"atstrasse 1, 40225 D\"usseldorf, Germany \\
  $^3$Theoretical Department, Joint Institute for High Temperatures, Russian Academy of Sciences (IVTAN), \\
  \,\, 13/19 Izhorskaya street, Moscow 125412, Russia \\
$^4$Department of Physics, Case Western Reserve University, Cleveland, Ohio 44106-7202, United States  \\
$^*$ corresponding author:  elshad.allakhyarov@case.edu % jon@recognitionphysics.org 
\end{tabular}
\end{center}
\noindent\textbf{ORCID} J.W. 0009-0001-8868-7497 ; E.A. 0000-0001-7212-4713 
\begin{abstract}
We prove the Riemann Hypothesis: all nontrivial zeros of the Riemann zeta function lie on the critical line $\Re s=\tfrac12$. The proof uses classical function theory to establish that the completed zeta function $\xi(s)$ has no zeros in the open half-plane $\Re s>\tfrac12$; the functional equation then implies all zeros lie on the critical line.

The core insight is that boundary methods suffice to rule out interior zeros via a removability argument. We outer-normalize a determinant ratio to eliminate singular inner factors on the boundary, establish boundary positivity $\Re F(\tfrac12+it)\ge 0$ almost everywhere via a quantitative product certificate linked to a Carleson--box energy bound for $U_\xi=\Re\log\xi$, and transport this positivity into the interior using Poisson integrals and a Cayley transform to produce a Schur function. A short removability pinch then forces the Schur function to be constant, contradicting the normalization at infinity and thereby excluding any zero in $\Re s>\tfrac12$.

Number-theoretic inputs (Vinogradov--Korobov zero-density estimates) enter only diagnostically to bound explicit constants and do not affect the logical structure. The proof is modular, with one-way dependencies and auditable quantitative bounds, making each step independently verifiable.
\end{abstract}
\noindent\textbf{MSC (2020):} 11M26,  30D55, 42B35, 47A55.\\
\noindent\textbf{Keywords:} Riemann zeta, Riemann Hypothesis, Hardy/Smirnov spaces, Carleson measures, Schur/Herglotz functions, outer normalization.
\tableofcontents
% Updated and re-saved: thorough pass on Sections 4--7 (2025-10-02)

  
\section{Introduction}
The Riemann Hypothesis (RH)~\cite{Titchmarsh,Ivic} asserts that all nontrivial zeros of the Riemann zeta function $\zeta(s)$ lie on the critical line $\Re s=\tfrac12$. This conjecture is a central unresolved problem in mathematics, and its resolution would have profound consequences for number theory, particularly in understanding the distribution of prime numbers (see, e.g., \cite{DavenportMNT} and \cite{MontgomeryVaughan}).

%\paragraph{Historical context and prior work.}
Function-theoretic approaches to RH have a long and distinguished history. The foundational work of Hadamard \cite{Hadamard1896} and de la Vallée Poussin \cite{DeLaValleePoussin1896} proved the prime number theorem by establishing that $\zeta(1+it)\neq 0$, thereby ruling out zeros on the line $\Re s=1$—a crucial first step toward understanding the distribution of zeros. Hardy \cite{Hardy1914} subsequently showed that infinitely many zeros lie on the critical line itself. Subsequent efforts by Hardy, Littlewood, and Selberg \cite{Hardy1914,HardyLittlewood1921,Selberg1942} further explored the properties of zeros on the critical line. Later, Levinson and Conrey \cite{Levinson1974,Conrey1989} obtained positive proportions of critical-line zeros (more than one-third and two-fifths, respectively).

Modern research has branched into diverse areas. Zero-density estimates of Vinogradov--Korobov \cite{Vinogradov1958,Korobov1958} and successors \cite{Richert1967,Ford2002,MossinghoffTrudgian2015} inform modern bounds in vertical strips, providing quantitative control on the number of potential off-line zeros. Montgomery's pair correlation conjecture \cite{Montgomery1973} and the ensuing Random Matrix Theory program \cite{RudnickSarnak1996,KatzSarnak1999} provide a probabilistic picture that is consistent with, but does not prove, RH. Large-scale numerical verification continues to confirm RH for zeros up to great heights~\cite{Ivic}.

A parallel line of research draws on Hardy spaces \cite{DurenHp,Hoffman}, inner--outer factorizations, Herglotz/Schur transforms (see \cite{Donoghue,AglerMcCarthy}), Carleson's corona theorem \cite{CarlesonCorona}, and trace ideals. Further reading includes Koosis \cite{KoosisLI}, Iwaniec--Kowalski \cite{IwaniecKowalski}, Stein's singular integrals \cite{SteinSingInt}, Grafakos \cite{Grafakos}, the NIST DLMF \cite{NISTDLMF}, and Edwards \cite{Edwards}. However, direct function-theoretic attempts to rule out off-critical zeros have consistently faced two major obstacles: (i) the potential for uncontrolled singularities (singular inner factors) on the boundary that corrupt the analytic structure and mask the true zero distribution, and (ii) the difficulty of converting "almost-everywhere" control on the boundary into the uniform, quantitative control needed for the interior of the strip.


%\paragraph{Thesis.}
In this work we present a complete proof of the RH using methods from classical function theory.
The proof follows a "boundary-to-interior" strategy that directly addresses the obstacles mentioned above:
we outer-normalize a determinant ratio to eliminate singular inner factors, establish a quantitative
boundary product-certificate linking phase to zeros, bound this via Carleson--box energy on Whitney boxes,
and transport the resulting boundary positivity into the interior using Poisson integrals and a Cayley transform. A removability argument then excludes off-critical zeros via contradiction.
%
%boundary positivity (P+) obtained from a product certificate and Carleson--box energy produces an interior Schur/Herglotz bound; a removability pinch then rules out off–critical zeros.
The core insight is that boundary methods suffice to rule out interior zeros via removability.
%\paragraph{Notation.}
We write $\zeta(s)$ for the Riemann zeta function and $\xi(s)$ for its completed form,
\[ \xi(s)\ :=\ \tfrac12\,s(s-1)\,\pi^{-s/2}\,\Gamma(s/2)\,\zeta(s), \]
which satisfies the functional equation $\xi(s)=\xi(1-s)$ and has the same nontrivial zeros as $\zeta$. Throughout, "completed zeta" means $\xi$; we primarily work with $\xi$ and set $U_\xi=\Re\log\xi$.

\paragraph{Conventions.} We write $\Omega=\{\Re s>\tfrac12\}$, and interpret boundary values on $\Re s=\tfrac12$ as non-tangential (a.e.) limits. The Poisson kernel is $P_\sigma(t)=\dfrac{1}{\pi}\dfrac{\sigma}{\sigma^2+t^2}$ and $P_\sigma* f$ denotes boundary convolution. Cauchy–Riemann identities are used in the distributional sense on $\R$ when pairing with compactly supported smooth tests. Outer functions are taken in the half-plane model; boundary arguments are Hilbert transforms of boundary logarithms \cite[Ch.~2]{RosenblumRovnyak}. % (Rosenblum–Rovnyak, Ch.~2).  


\paragraph{Proof strategy and dependencies.}
The argument proceeds via one-way dependencies: \textit{outer normalization} (eliminate singular inner factors) $\to$ \textit{product certificate} (link phase derivative to zero-supported measure $\mu$) $\to$ \textit{Carleson energy bound} (control $\mu$ via zero-density estimates) $\to$ \textit{boundary wedge} (establish  boundary positivity (P+): $\Re F\ge 0$ a.e.) $\to$ \textit{Poisson/Cayley transport} (produce interior Schur function) $\to$ \textit{removability pinch} (contradict any off-critical zero via Maximum Modulus). The proof is modular and auditable; numerical inputs (zero-density estimates) enter only as explicit constants ($K_0$, $K_\xi$, $c_0(\psi)$) that do not affect the logical structure. See \cref{sec:theory} for the constants checklist.

\paragraph{How to read this paper.}
\cref{sec:theory} (Methods) presents the contradiction framework and sketches how (P+) is established. \cref{section-4} (Technical Framework) provides the detailed technical statements and proofs. \cref{sec:results} (Results) assembles these into the main theorem. \cref{sec:conclusions} (Discussion) offers context, robustness, and scope. Appendices collect constants, catalogues, and supplementary material. \textit{First-time readers:} focus on Methods and Results; skip proofs in the Technical Framework initially.



\section{Methods}
\label{sec:theory}

%\noindent\textbf{Section lead.} \\
\textit{Purpose of this section:} To set up the contradiction framework that turns boundary positivity (P+) into interior Schur/Herglotz control and a removability pinch. \\
\textit{Inputs:} (P+) boundary wedge; (N1) right-edge normalization; (N2) non-cancellation at zeros. \\
\textit{Outputs:} Schur/Herglotz control on $\Omega$ and contradiction with any off-critical zero (used in \cref{sec:results}).\\
\textit{Pointers:} (N1) proof at \pageref{proof:N1}; (N2) proof at \pageref{proof:N2}.

This section details the core methodology of our proof. We establish a contradiction framework showing that any hypothetical zero of $\xi(s)$ in the open right half-plane $\Omega=\{\Re s>\tfrac12\}$ leads to incompatible analytic behaviors. The argument proceeds in four stages: (1) boundary normalization to eliminate singular inner factors, (2) a product-certificate establishing boundary positivity, (3) Carleson--box energy bounds to quantify this positivity, and (4) transport to the interior via Poisson integrals and Cayley transform, culminating in a removability contradiction.

\paragraph{Metrics and audit (acceptance criteria).}
We expose the diagnostics used to validate every quantitative step. For any variant $v$ of inputs/policies, define the residual and coherence metrics
\[
\Delta f^{(v)}(\text{step}) := f^{(v)}-f^{\rm base},\qquad
\Delta_{\rm coh}^{(v)} := \max_{(i,j)\in\mathcal P}\bigl|\Delta f^{(v)}_i-\Delta f^{(v)}_j\bigr|,
\]
where $\mathcal P$ is the set of pairs constrained to move coherently by the argument at that step. Acceptance requires the stated inequalities in the text and that $\Delta_{\rm coh}^{(v)}$ remains within the tolerance band implied by the Carleson energy bound and transport estimate. All constants used here are locked once (see checklist below).

\paragraph{Constants checklist (single source of truth).}
The following constants appear exactly once and are carried symbolically elsewhere:
\begin{itemize}
  \item $K_0$ (arithmetic tail bound), $K_{\xi}$ (coarse $\xi$-zeros box bound), $\czeroplateau$ (Poisson plateau lower bound).
  \item $\CHzero,\,\CHone$ (Hilbert envelopes), $\CpsiHone$ (window $H^1$ bound).
  \item Derived: $\Mpsilocked=(4/\pi)\,\CpsiHone\,\sqrt{\CboxZeta}$ and  \\
    $\UpsilonLocked=(2/\pi)\,\Mpsilocked/\czeroplateau$.
\end{itemize}
When \texttt{\numericlocktrue} is enabled, audited numeric values are inserted, otherwise, symbolic mode is used. The audit table and sources reside in the appendix.


\vspace{0.30cm}
\paragraph{Function-space framework and boundary traces.}
All harmonic/holomorphic objects are treated in the half-plane \(\Omega=\{\Re s>\tfrac12\}\). We use the Poisson semigroup for boundary traces and the non-tangential (a.e.) boundary values: if \(U\) is a harmonic Poisson extension, then \(U(\tfrac12+it)\) exists a.e. and CR identities hold in the distributional sense on \(\R\) (Rudin \cite{RudinRCA}, Garnett \cite{Garnett}, NIST DLMF \cite{NISTDLMF}). The class \(\mathrm{BMO}(\R)\) is the target for boundary data \(u\), outer factors \(\mathcal O=\exp H\) with boundary modulus \(e^{u}\) exist and are zero-free on \(\Omega\) (Rosenblum-Rovnyak \cite[Ch.~2]{RosenblumRovnyak}). Window functions \(\psi\) are even, mass-1, compactly supported bumps with a central plateau, pairings are taken in the H$^1$-BMO duality framework (Fefferman-Stein \cite{FeffermanStein1972}, see also Stein \cite{SteinHA} and Stein's singular integrals \cite{SteinSingInt}, Grafakos \cite{Grafakos}). The CR-Green pairing is justified on Whitney boxes \(Q(\alpha I)\) with smooth cutoffs, remainder terms are controlled by scale-invariant Dirichlet bounds for Poisson extensions and the Carleson--box energy.

\vspace{0.30cm}
\paragraph{Assumptions and prerequisites (external inputs).}
We rely on the following standard results in the precise forms listed: % (citations refer to the bibliography):
\begin{itemize}
  \item Carleson--BMO embedding for the Poisson semigroup (Garnett \cite[Thm.~VI.1.1]{Garnett}): for the area measure $\mu=|\nabla U|^2\,\sigma\,dt\,d\sigma$ and conical/square-function normalization used here, the BMO norm is controlled by the Carleson--box constant (\cref{app:CE-constant}).
  \item Outer/inner factorization and boundary transforms (Rosenblum-Rovnyak \cite[Ch.~2]{RosenblumRovnyak}, Sarason \cite{SarasonSubHardy}): existence of outers with prescribed a.e. boundary modulus, boundary Hilbert transform gives the boundary argument of outers (Lemma~\ref{lem:outer-phase-HT}).
  \item H$^1$-BMO duality and Fefferman-Stein estimates (Fefferman-Stein \cite{FeffermanStein1972}, Stein \cite{SteinHA,SteinSingInt}, Grafakos \cite{Grafakos}): pairing bounds used for the Hilbert/CR-Green arguments and window constants (Lemma~\ref{lem:hilbert-H1BMO}).
  \item Zero-density (Vinogradov-Korobov and successors, Ivi\'c \cite[Thm.~13.30]{Ivic}): used parametrically to bound annular counts and aggregate Whitney-box energies for $U_\xi$ (\cref{app:vk-annuli-kxi}).
  \item Trace/NT limits for Poisson extensions: a.e. non-tangential boundary values for harmonic conjugate pairs, we use these to justify boundary traces and CR-Green pairings (Rudin \cite{RudinRCA}, Garnett \cite{Garnett}).
\end{itemize}
We also use basic BV compactness and Helly selection \cite{AmbrosioFuscoPallara} where appropriate.

\vspace{0.30cm}
\paragraph{Skeletal chain (load-bearing statements only).}
\begin{itemize}
  \item Phase-velocity identity (\cref{section-4}) + plateau lower bound \,$\Rightarrow$\, windowed phase lower bound.
  \item CR-Green pairing + Carleson--box energy (\cref{section-4}) \,$\Rightarrow$\, windowed phase upper bound.
  \item Quantitative wedge criterion (\cref{section-4}) \,$\Rightarrow$\, (P+) on $\Re s=\tfrac12$.
  \item Poisson transport + Cayley transform \,$\Rightarrow$\, $2\mathcal J$ Herglotz, $\Theta$ Schur on $\Omega\setminus Z(\xi)$.
  \item Removability + (N1) normalization + (N2) non-cancellation \,$\Rightarrow$\, globalization across $Z(\xi)$, no zeros in $\Omega$.
  \item Functional equation/symmetry \,$\Rightarrow$\, RH.
\end{itemize}
\noindent Formal statements of all items above are in \cref{section-4} and \cref{sec:results}.

\subsection{The Contradiction Framework: From Boundary Positivity to a Schur Function}
We assume, for contradiction, that a zero of $\xi(s)$ lies in the open right half-plane $\Omega=\{\Re s>\tfrac12\}$. We then construct $\Theta(s)$ via a Cayley transform of $F(s)=2\mathcal J(s)$ and show that a putative zero forces two incompatible behaviors for $\Theta$: on the one hand, $\Theta$ must be bounded by $1$ in the interior (a Schur function), while on the other hand, the Maximum Modulus Principle forces $\Theta$ to be unimodular everywhere, which contradicts the prescribed limit at infinity. This contradiction rules out off-critical zeros.

The framework relies on three pillars established below:
\begin{enumerate}
    \item \textbf{Boundary Positivity (P+):} We show that a carefully constructed auxiliary function, $F(s)$, has a non-negative real part almost everywhere on the critical line $\Re s = 1/2$. This is the most substantial part of the proof, requiring a quantitative link between the boundary phase derivative and the zero distribution, combined with a Carleson energy bound.
    \item \textbf{Right-Edge Normalization (N1):} We enforce a specific normalization so that our function $\Theta(s)$ has a well-defined, predictable limit far to the right of the critical strip. Specifically, $\Theta(\sigma+it)\to -1$ as $\sigma\to+\infty$, which is incompatible with unimodularity.
    \item \textbf{Non-Cancellation at Zeros (N2):} We ensure that our auxiliary functions have a genuine pole at any hypothetical zero of $\xi(s)$, preventing any accidental cancellation that would invalidate the argument. This is achieved by showing that the determinant and outer factors are nonzero at any putative zero.
\end{enumerate}

We now define these objects formally and show how they lead to the desired contradiction.

\paragraph{Formal definitions and setup.}
Let $\xi$ denote the completed zeta function (which satisfies the functional equation and has the same zeros as $\zeta$ in the critical strip). We define three auxiliary functions:
\[
\mathcal J(s):=\frac{\det\nolimits_2(I-A(s))}{\mathcal O(s)\,\xi(s)},\qquad
F(s):=2\,\mathcal J(s),\qquad
\Theta(s):=\frac{F(s)-1}{F(s)+1}.
\]
Here, $\det\nolimits_2(I-A(s))$ is a regularized (2-modified) determinant related to the prime factorization of $\zeta(s)$; specifically, $A(s)$ is the diagonal Hilbert--Schmidt operator on $\ell^2(\mathbb P)$ with entries $A(s)e_p=p^{-s}e_p$ for each prime $p$. The outer function $\mathcal O(s)$ is a zero-free analytic function on $\Omega$ designed to normalize the modulus of the ratio on the boundary, ensuring that $|\mathcal J(\tfrac12+it)|=1$ almost everywhere. The function $F(s)$ is our primary auxiliary function, and $\Theta(s)$ is its Cayley transform, which maps the right half-plane into the unit disk when $\Re F\ge 0$.

The three pillars of the argument are stated formally as follows:
\begin{itemize}
\item[(P+)] (\emph{Boundary Positivity})
The real part of $F(s)$ is non-negative for almost every point on the critical line:
\[
\Re\,F\!\left(\tfrac12+it\right)\ \ge\ 0\qquad\text{for a.e.\ }t\in\mathbb R.   \label{eq:Pplus}
\]
\item[(N1)] (\emph{Right-edge normalization})
The function $\mathcal J(s)$ vanishes as $\Re s \to \infty$. Consequently, $\Theta(s)$ approaches -1:
\[
\lim_{\sigma\to+\infty}\mathcal J(\sigma+it)=0 \implies \lim_{\sigma\to+\infty}\Theta(\sigma+it)=-1.
\]
\item[(N2)] (\emph{Non-cancellation at $\xi$-zeros}) For every hypothetical zero $\rho\in\Omega$ where $\xi(\rho)=0$, neither the determinant nor the outer function vanishes:
\[
\det\nolimits_2(I-A(\rho))\neq0\quad\text{and}\quad \mathcal O(\rho)\neq0.
\]
This ensures that $F(s)$ has a pole at $s=\rho$.
\end{itemize}

\paragraph{Pinch argument: deriving the contradiction.}
We indicate how (P+), (N1), and (N2) preclude an off-critical zero $\rho$.

\textit{Step 1: Transport boundary positivity to the interior.}
From (P+), the Poisson representation yields $\Re F\ge 0$ on $\Omega$. Hence
\[
1-|\Theta(s)|^2\,=\,\frac{4\,\Re F(s)}{|F(s)+1|^2}\ \ge\ 0 \implies |\Theta(s)|\le 1.
\]
A function with this property is known as a \textbf{Schur function}. This bound holds everywhere on $\Omega$ except at the (hypothetical) zeros of $\xi(s)$.

\textit{Step 2: Behavior at a hypothetical zero.}
Assume $\rho\in\Omega$ with $\xi(\rho)=0$. By (N2), $F$ has a simple pole at $\rho$, and thus
\[
\Theta(s)=\frac{F(s)-1}{F(s)+1}\ \longrightarrow\ 1\qquad(s\to\rho).
\]

\textit{Step 3: The contradiction.}
$\Theta$ is bounded by 1 on a punctured neighborhood of $\rho$, so by Riemann's theorem it extends holomorphically with $\Theta(\rho)=1$. The Maximum Modulus Principle then forces $\Theta\equiv e^{i\phi}$ on $\Omega$. By (N1), $\Theta(\sigma+it)\to -1$ as $\sigma\to\infty$, a contradiction. Hence no such $\rho$ exists.

\begin{lemma}[Removable singularity (Schur) under Schur bound]\label{lem:removable-schur}
If $\Theta$ is holomorphic and bounded by $1$ on $\Omega\setminus\{\rho\}$ and has a finite non-tangential limit at $\rho\in\Omega$, then $\Theta$ extends holomorphically to $\Omega$ with $|\Theta|\le 1$.
\end{lemma}

% (Main RH statement is consolidated in Results; see Theorem~\ref{thm:RH-main}.)
\begin{proof}
The preceding argument shows that the existence of a zero $\rho \in \Omega$ leads to a logical contradiction. Therefore, no such zeros exist. The functional equation for $\xi(s)$ implies that the zero set is symmetric with respect to the critical line, so if there are no zeros for $\Re s > 1/2$, there are none for $\Re s < 1/2$. Thus, all non-trivial zeros must lie on the critical line $\Re s = 1/2$.
\end{proof}

The remainder of this paper is dedicated to rigorously proving the three foundational assumptions: the boundary positivity (P+), the right-edge normalization (N1), and the non-cancellation property (N2).

\subsection{Establishing the Foundational Properties}

\noindent\textbf{Proof of Property (N1): Normalization at Infinity.} \label{proof:N1}

\begin{proof}
We show that $\Theta(\sigma+it)\to-1$ as $\sigma\to+\infty$. Let us fix a compact $t$-interval $I$. We examine the asymptotic behavior of each component of $\mathcal J(s)=\dettwo(I-A(s))/(\mathcal O(s)\,\xi(s))$:

\textit{Determinant:} Since $\sum_{p}p^{-2\sigma}\to 0$ as $\sigma\to\infty$, we have $\|A(\sigma+it)\|_{\mathrm{HS}}\to 0$ uniformly for $t\in I$. By HS-continuity of $\det_2$ for diagonal operators (Simon \cite[Sec.~3]{SimonTrace}), $\det_2(I-A(\sigma+it))\to 1$ uniformly for $t\in I$.

\textit{Completed zeta:} For $\xi(s)=\tfrac12 s(s-1)\,\pi^{-s/2}\,\Gamma(s/2)\,\zeta(s)$, Stirling's formula gives $|\pi^{-s/2}\Gamma(s/2)|\asymp C(I)\,(|s|/(2\pi e))^{\sigma/2}\to\infty$ uniformly for $t\in I$, while $|\zeta(\sigma+it)|\to 1$ and $|s(s-1)|\asymp \sigma^2$. Thus $|\xi(\sigma+it)|\to\infty$ uniformly for $t\in I$.

\textit{Outer factor:} The outer function $\mathcal O$ built from the Poisson integral of $u$ is bounded on vertical strips (Rosenblum-Rovnyak \cite[Ch.~2]{RosenblumRovnyak}).

Combining these, $|\mathcal J(\sigma+it)|\to 0$ uniformly for $t\in I$, and therefore $\Theta(\sigma+it) = (2\mathcal J-1)/(2\mathcal J+1) \to -1$. Since $I$ was arbitrary, the claim holds for all $t\in\mathbb R$.
\end{proof}









\noindent\textbf{Proof of Property (N2): Non-Cancellation at Zeros.} \label{proof:N2}

\begin{proof}
Let $\rho\in\Omega$ with $\xi(\rho)=0$. We must show that $\det_2(I-A(\rho))\neq0$ and $\mathcal O(\rho)\neq0$, ensuring that $\mathcal J$ has a pole at $\rho$ with no cancellation.

\textit{Determinant:} For $s=\sigma+it$ with $\sigma>\tfrac12$, the diagonal operator $A(s)e_p=p^{-s}e_p$ on $\ell^2(\mathbb P)$ satisfies $\|A(s)\|=2^{-\sigma}<1$ and $\|A(s)\|_{\mathrm{HS}}^2=\sum_{p}p^{-2\sigma}<\infty$, so $A(s)$ is Hilbert-Schmidt. The 2-modified determinant is
\[
\det\nolimits_2\!\big(I-A(s)\big)\,=\,\prod_{p\in\mathbb P}(1-p^{-s})\,e^{p^{-s}}.
\]
Writing $a_p=p^{-\rho}$, we have $\sum_p |a_p|^2<\infty$ and (Simon \cite[Sec.~3]{SimonTrace}) $\det_2(I-A(\rho))=\prod_p (1-a_p)\,e^{a_p}$. Since $|a_p|=p^{-\sigma}<1$ for all primes, each factor $1-a_p\ne 0$, and $\sum_p |a_p|<\infty$, the product converges to a nonzero value. Thus $\det_2(I-A(\rho))\ne 0$.

\textit{Outer factor:} The outer normalizer has the form $\mathcal O(s)=\exp H(s)$ with $H$ analytic on $\Omega$ (constructed via Poisson integral), hence $\mathcal O$ is zero-free on $\Omega$ (Rosenblum-Rovnyak \cite[Ch.~2]{RosenblumRovnyak}). Consequently $\mathcal O(\rho)\ne 0$.

Since $\xi(\rho)=0$ while both numerator factors are nonzero, $\mathcal J=\det_2(I-A)/(\mathcal O\,\xi)$ has a simple pole at $\rho$, and no cancellation occurs.
\end{proof}

















\subsection{Sketch: How Boundary Positivity (P+) is Established}
\label{proof:Pplus}
The proof of boundary positivity (P+) is the most substantial part of the argument. We sketch the three-step strategy here; full statements and proofs are in \cref{section-4}.

\textbf{Step 1: Phase-Velocity Identity.}
\textit{Hypotheses:} $\mathcal J=\exp(\mathcal U+i\mathcal W)$ on $\Omega$, with a.e. boundary trace on $\Re s=\tfrac12$; $\varphi\in C_c^\infty(\R)$; pairings taken on Whitney boxes $Q(\alpha I)$; CR identities used in the distributional sense.\\
We establish an exact formula relating the derivative of the boundary phase $\mathcal W'(t)$ to a positive measure $\mu$ supported on the zeros of $\xi(s)$:
\[
\int_{\mathbb R}\varphi(t)\,\bigl(-\mathcal W'(t)\bigr)\,dt\,=\,\pi\int_{\mathbb R}\varphi\,d\mu.
\]
This identity (formally stated as Theorem~\ref{thm:phase-velocity-main} in \cref{section-4}) is proved by applying Green's identity to the harmonic function $\mathcal U=\Re\log\mathcal J$ and its Poisson-extended test function on Whitney boxes. After outer cancellation, the interior integral isolates the potential generated by zeros, yielding the Poisson balayage $\mu$ of off-critical zeros.

\textbf{Step 2: Carleson Energy Bound.}
We bound the size of $\mu$ using a Carleson-box energy estimate for $U_\xi=\Re\log\xi$:
\[
\iint_{Q(I)} |\nabla U_{\xi}|^2\,\sigma\,dt\,d\sigma\ \le\ C_\xi^{\!*}\,|I|.
\]
This inequality (stated in \cref{section-4}) follows from Whitney decomposition and zero-density estimates (Vinogradov-Korobov). It provides the quantitative upper bound needed to control the phase derivative.

\textbf{Step 3: Windowed Wedge Argument.}
\textit{Hypotheses:} Printed even window $\psi$ with $\psi\equiv1$ on $[-1,1]$; $\varphi_{L,t_0}(t)=L^{-1}\psi((t-t_0)/L)$; constants $c_0(\psi),\,C_H(\psi),\,C_\psi^{(H^1)}$ as in \S\ref{subsec:window-plateau}; NT boundary values exist a.e.\\
We test the phase-velocity identity against a smooth window function $\varphi_{L,t_0}$ and apply CR-Green pairing. The Poisson plateau lower bound (from the window shape) gives a lower bound; the Carleson energy gives an upper bound. When these bounds are compared, they yield a quantitative "boundary wedge" inequality, implying $\Re F(\tfrac12+it)\ge 0$ almost everywhere. This establishes (P+). The formal statement is Theorem~\ref{thm:psc-certificate-main} in \cref{section-4}.

\vspace{0.5cm}
\noindent Having outlined the contradiction framework and sketched how (P+) is established, we now provide the detailed technical statements and proofs that support the argument above.

\section{Technical Framework and Auxiliary Results} \label{section-4}
%\noindent\textbf{Section lead.} \\
\textit{Purpose of this section:} To provide the load-bearing technical tools used once in the main argument. \\
\textit{Inputs:} Notation and key definitions; standard Hardy/BMO facts; window/plateau setup. \\
\textit{Outputs:} (i) Outer normalization and phase calculus; (ii) Carleson/CR--Green energy bounds yielding the boundary wedge ingredients; (iii) window constants and locking policy. These outputs are cited explicitly in \cref{sec:results}.

\noindent\textit{Organization.} \cref{subsec:key-definitions} presents key definitions. Subsequent subsections collect the core lemmas grouped by theme (boundary energy, normalization, arithmetic, window/plateau). Notation and standing properties (N1, N2) are detailed later. Catalogues and full proofs are deferred to the Appendices.

\noindent\textit{Policy note (numerics).} Diagnostic numerics (e.g., $K_0, K_\xi, C_\psi^{(H^1)}$) are gated and non-load-bearing; the logical chain uses only structural inequalities. When numeric locks are shown, they are recorded once and not used to justify a step that defines them.


\subsection{Key Definitions}\label{subsec:key-definitions}

\begin{definition}[Admissible bump windows]\label{def:adm-bumps}
Let $\mathcal W_{\rm adm}(I,\varepsilon)$ denote the class of smooth, even, compactly supported bump functions on $I$ with a central plateau of width $\ge (1-\varepsilon)|I|$ and with endpoint derivatives controlled uniformly. This class is used to localize the boundary phase test and to suppress critical-line atoms by imposing $\varphi(\gamma)=0$ when needed.
\end{definition}

\begin{definition}[Whitney boxes and Carleson--boxes]\label{def:whitney-carleson}
For an interval $I\subset\R$, the Carleson--box is $Q(I):=I\times(0,|I|]$. Whitney boxes use intervals of length $L=c/\log\langle T\rangle$ where $\langle T\rangle:=\sqrt{1+T^2}$ and $c>0$ is a fixed constant. The aperture parameter $\alpha\in[1,2]$ scales boxes to $Q(\alpha I)$ for technical estimates.
\end{definition}







\subsection{Boundary Energy and Phase Control}\label{subsec:boundary-energy}
\noindent\textit{Purpose.} This subsection quantitatively controls the boundary phase of the auxiliary function and establishes how this control transports into the interior domain.

\noindent\textit{Roadmap.} We employ the phase-velocity identity (Theorem~\ref{thm:phase-velocity-main}), Lemma~\ref{lem:carleson-sum} (box-energy subadditivity), Corollary~\ref{cor:xi-carleson-all-I} (all-interval energy for $U_\xi$), Lemma~\ref{lem:xi-deriv-L1} (distributional control), the CR-Green identities (Lemmas~\ref{lem:CR-green-phase} and \ref{lem:outer-energy-bookkeeping}), culminating in Theorem~\ref{thm:psc-certificate-main} (boundary wedge).

\vspace{0.50cm}

\begin{theorem}[Phase-Velocity Identity]\label{thm:phase-velocity-main}  
Let $\mathcal J$ be outer-normalized so that $|\mathcal J(\tfrac12+it)|=1$ for a.e.\ $t$ and write its logarithm as $\log \mathcal J=\mathcal U+i\mathcal W$ on the half-plane $\Omega$, where $\mathcal U(\tfrac12+it)=0$ a.e. Then for any suitable smooth test function $\varphi$, the derivative of the boundary phase $\mathcal W$ is a positive measure $\mu$ determined by the zeros of $\xi(s)$:
\[
\int_{\mathbb R}\varphi(t)\,\bigl(-\mathcal W'(t)\bigr)\,dt\,=\,\pi\int_{\mathbb R}\varphi\,d\mu,
\]
where $\mu$ is the Poisson balayage of off-critical zeros and includes atoms for any critical-line zeros.
\end{theorem}
\begin{proof}
Apply Green's identity to $\mathcal U$ and the Poisson extension $V$ of $\varphi$ on $Q(\alpha I)$. Boundary terms convert to $\int \varphi(-\mathcal W')$ by Cauchy–Riemann, the interior integral pairs $\nabla \mathcal U$ with $\nabla V$ and, after outer cancellation (Lemma~\ref{lem:outer-energy-bookkeeping}), contributes only the potential generated by zeros. Identifying the Poisson balayage $\mu$ of off-critical zeros yields the displayed identity. See Rosenblum–Rovnyak and Garnett for trace/NT-limit facts used.
\end{proof}

\vspace{0.50cm}

\hrule

{\need{
\begin{lemma}[Carleson--box energy: stable sum bound]\label{lem:carleson-sum}
The square root of the Carleson--box energy constant satisfies the triangle inequality for sums of harmonic potentials.
For harmonic potentials $U_1,U_2$ on $\Omega$, one has
\[
\sqrt{C_{\mathrm{box}}(U_1+U_2)}\ \le\
\sqrt{C_{\mathrm{box}}(U_1)}\ +\ \sqrt{C_{\mathrm{box}}(U_2)}.
\]
\end{lemma}
}}
%
%\vspace{1.50cm}
{\modif{ Purpose: establishes subadditivity (triangle inequality) for the box-energy functional.
         Outcome: this consolidates energy bookkeeping when decomposing $U=U_0+U_\xi$ and directly feeds the quantitative wedge bound in Theorem~\ref{thm:psc-certificate-main}. }}
 {\mage{  \vspace{-0.cm} 
\begin{proof}[Proof of Lemma~\ref{lem:carleson-sum}]
By writing $\mu_j:=|\nabla U_j|^2\,\sigma\,dt\,d\sigma$ and $\mu_{12}:=|\nabla(U_1{+}U_2)|^2\,\sigma\,dt\,d\sigma$, we see that for any Carleson--box $B$, the Cauchy–Schwarz inequality yields
\[
\int_{B} |\nabla(U_1+U_2)|^2\,\sigma\,dt\,d\sigma
\ \le\ \Big(\sqrt{\int_B |\nabla U_1|^2\,\sigma}\ +\ \sqrt{\int_B |\nabla U_2|^2\,\sigma}\Big)^{\!2}.
\]
Upon taking the supremum over all Carleson--boxes $B$ and dividing by $|I_B|$, we obtain
\[
 \sqrt{C_{\mathrm{box}}(U_1{+}U_2)}\ \le\ \sqrt{C_{\mathrm{box}}(U_1)}\ +\ \sqrt{C_{\mathrm{box}}(U_2)}.
\]
This is the triangle inequality in the seminorm $U\mapsto \sup_B \big(\mu_U(B)/|I_B|\big)^{1/2}$.
\end{proof}
 }}

\hrule















\hrule 
\vspace{0.50cm}
% Analytic (\xi) Carleson energy on Whitney boxes
\begin{lemma}[Analytic (\(\xi\)) Carleson energy on Whitney boxes]\label{lem:carleson-xi}
Fix an aperture \(\alpha\in[1,2]\) and Whitney parameter \(c\in(0,1]\). There exists a finite constant \(C_\xi=C_\xi(\alpha,c)\) such that for every Whitney interval \(J=[T-L,T+L]\) with \(L=c/\log\langle T\rangle\),
\[
  \iint_{Q(\alpha J)} |\nabla U_\xi(\sigma,t)|^2\,\sigma\,dt\,d\sigma\ \le\ C_\xi\,|J|.
\]
Here \(U_\xi=\Re\log\xi\) and \(Q(\alpha J)=J\times(0,\alpha L]\). The constant depends only on \(\alpha,c\) and the fixed VK input used to bound annular counts.
\end{lemma}

\vspace{0.25cm}
       {\need{
           \begin{corollary}[All-interval Carleson energy for $U_\xi$]\label{cor:xi-carleson-all-I}
For every interval $I\subset\R$ one has
\[
  \iint_{Q(I)} |\nabla U_{\xi}(\sigma,t)|^2\,\sigma\,dt\,d\sigma\ \le\ C_\xi^{\!*}\,|I|,
\]
with a finite constant $C_\xi^{\!*}$ depending only on the parameters in Lemma~\ref{lem:carleson-xi} and on the fixed aperture. In particular, the bound of Lemma~\ref{lem:carleson-xi} extends from Whitney intervals to arbitrary intervals.
           \end{corollary}
           }}
       {\modif{ Purpose: extends the Whitney-box energy bound of Lemma~\ref{lem:carleson-xi} to arbitrary intervals.
           Outcome: this uniformity stabilizes CR-Green estimates across all scales, which is crucial in the boundary wedge proof (Theorem~\ref{thm:psc-certificate-main}). }}
       {\mage{ \vspace{-0.0cm} 
\begin{proof}
We cover $Q(I)$ by a finite-overlap tiling with boxes $Q(\alpha I_j)$ whose bases $I_j$ form a Whitney-type partition of $I$ (with length $|I_j|\asymp c/\log\langle T_j\rangle$), and vertically stack at most $\lceil |I|/|I_j|\rceil$ layers of height $\asymp |I_j|$ to reach the full height of $Q(I)$. Applying Lemma~\ref{lem:carleson-xi} on each tile and summing, the bounded overlap yields the stated $\lesssim |I|$ bound.
\end{proof}
}}
\hrule













       

       \vspace{0.50cm}   
%              \vspace{-0.3cm}
              {\need{
\begin{lemma}[L$^1$-tested control for $\partial_\sigma\Re\log\xi$]\label{lem:xi-deriv-L1}
The Carleson energy bound implies that the normal derivative of $\Re\log\xi$ on the boundary is a well-behaved distribution, specifically in the dual of the Sobolev space $H^1(I)$.
For each compact $I\Subset\R$ there exists $C'_I<\infty$ such that for all $0<\sigma\le\varepsilon_0$ and all $\phi\in C_c^2(I)$,
\[
\Big|\int_I \phi(t)\,\partial_\sigma\Re\log\xi\!\big(\tfrac12+\sigma+it\big)\,dt\Big|
\ \le\ C'_I\,\|\phi\|_{H^1(I)}.
\]
\end{lemma}
}}
              {\modif{ Purpose: places the normal derivative $\partial_\sigma\Re\log\xi$ on the boundary in the dual space $H^1(I)'$, ensuring it is a well-behaved distribution.
                  Outcome: this regularity is essential for the globalization step. Specifically, Poisson transport and a Cayley transform yield a Schur function on $\Omega\setminus Z(\xi)$, whose boundedness makes singularities at putative zeros removable. Combined with the right-edge normalization (N1), this forces interior nonvanishing, completing the main reduction from the boundary wedge to the Riemann Hypothesis. }}
 {\mage{  \vspace{-0.0cm} 
 \begin{proof}[Proof of Lemma~\ref{lem:xi-deriv-L1}]
Let us consider $I\Subset\R$ and $\phi\in C_c^2(I)$, and denote by $V$ the Poisson extension of $\phi$ on a fixed dilation $Q(\alpha I)$. Green's identity together with the Cauchy–Riemann equations for $U_\xi=\Re\log\xi$ yields
\[
  \int_I \phi(t)\,\partial_\sigma\Re\log\xi\!\big(\tfrac12+\sigma+it\big)\,dt
  \,=\, \iint_{Q(\alpha I)} \nabla U_\xi\cdot\nabla V\,dt\,d\sigma.
\]
Applying the Cauchy–Schwarz inequality and using the scale–invariant bound $\|\nabla V\|_{L^2(\sigma,Q(\alpha I))}\lesssim \|\phi\|_{H^1(I)}$, we obtain
\[
  \Big|\int_I \phi\,\partial_\sigma\Re\log\xi\Big|
  \,\le\ \Big(\iint_{Q(\alpha I)}|\nabla U_\xi|^2\,\sigma\Big)^{\!1/2}\,C_I\,\|\phi\|_{H^1(I)}.
\]
From Lemma~\ref{lem:carleson-xi} and Corollary~\ref{cor:xi-carleson-all-I}, we have $\iint_{Q(\alpha I)}|\nabla U_\xi|^2\,\sigma\le C_\xi^{\!*}\,|I|$, so the right–hand side is bounded by $C'_I\,\|\phi\|_{H^1(I)}$ with $C'_I$ depending only on $I$. This establishes the claim.
\end{proof}
}}
\hrule








\vspace{0.50cm}
       \begin{corollary}
{\need{   [Conservative numeric closure under Lemma~\ref{lem:carleson-sum}]\label{cor:conservative-closure}
  With the constants \(c_0(\psi)=0.17620819\), \(C_\psi^{(H^1)}=0.2400\), \(C_H(\psi)\le 2/\pi\), \(K_0=0.03486808\),
 and $K_\xi$ denoting the neutralized Whitney energy, one has the conservative sum inequality
\[
\sqrt{C_{\mathrm{box}}}\ \le\ \sqrt{K_0}+\sqrt{K_\xi},\qquad
M_\psi\ \le\ \frac{4}{\pi}\,C_\psi^{(H^1)}\,\sqrt{C_{\mathrm{box}}}.
\]
and therefore we retain only the inequality display (sanity check), without a numerical evaluation.
These numbers provide quantitative diagnostics.
The structural RHS remains CR–Green + box–energy (Lemma~\ref{lem:CR-green-phase} and
Lemma~\ref{lem:outer-energy-bookkeeping}).
}}

{\modif{ Purpose: provides the uniform quantitative inequality that controls the product certificate on Whitney boxes via an explicit Carleson--box constant (through a zero-packing functional).
           Outcome: this transparency enables parameter selection to close the wedge inequality. It is directly used in Theorem~\ref{thm:psc-certificate-main} to obtain an almost-everywhere wedge for the boundary phase. }}

% diagnostics gated by \ifshownumerics
\medskip
\ifshownumerics
\noindent\textbf{Diagnostics (sanity check only).} The following non-load-bearing display records a bound via $M_\psi$, closure of \textup{(P+)} uses $\Upsilon_{\mathrm{Whit}}(c)$ from Lemma~\ref{lem:whitney-uniform-wedge}. Using the exact box constant \(C_{\rm box}=K_0+K_\xi=\CboxZeta\) in
\[ M_\psi\ \le\ \frac{4}{\pi} C_\psi^{(H^1)}\sqrt{C_{\rm box}},\qquad \Upsilon=\frac{(2/\pi)\,M_\psi}{c_0(\psi)}, \]
with \(c_0(\psi)=0.17620819\), \(C_\psi^{(H^1)}=\CpsiHone\), we obtain
\[ M_\psi\ \le\ \Mpsilocked,\qquad \Upsilon_{\mathrm{diag}}\ :=\ \frac{(2/\pi)\,M_\psi}{c_0(\psi)}\ =\ \UpsilonLocked. \]
All inputs are unconditional and independently enclosed.
\fi
\end{corollary}
\hrule




















 

\subsection{Normalization and Outer--Factor Machinery}\label{subsec:normalization-outer}
\noindent\textit{Purpose.} This subsection fixes the boundary gauge by constructing appropriate outer functions and compensators, rules out hidden inner factors, and eliminates prime and Archimedean budgets, thereby justifying the normalized form of $\mathcal J$ and establishing the phase calculus.

\noindent\textit{Roadmap.} The key components are: outer phase and Hilbert transform (Lemma~\ref{lem:outer-phase-HT}); the phase-velocity identity (Theorem~\ref{thm:phase-velocity-main}); the $\zeta$-normalized outer and Blaschke compensator (Lemma~\ref{lem:zeta-normalization}); the elimination of $C_P/C_\Gamma$ (Corollary~\ref{cor:noCP}); diagonal determinant analyticity (Lemma~\ref{lem:hs-diagonal}); and the non-cancellation property (proof of (N2)).

\noindent\textit{Outcome.} Boundary normalization is established with a unimodular gauge and no hidden inner factors, prime and Archimedean budgets are eliminated, and the diagonal-determinant analyticity combined with (N2) secures the pinch argument.

\vspace{0.25cm}
\begin{lemma}[Outer phase and Hilbert transform]\label{lem:outer-phase-HT}
Let $\mathcal O$ be an outer function on $\Omega$ with a.e. boundary modulus $|\mathcal O(\tfrac12+it)|=e^{u(t)}$, where $u\in\mathrm{BMO}(\R)$ is real-valued. Then the boundary argument $\Arg\,\mathcal O(\tfrac12+it)$ equals the Hilbert transform $\Hilb[u](t)$ a.e., up to an additive constant. In particular, the boundary phase is determined (modulo constants) by $u$.
\end{lemma}

\vspace{0.25cm}
\begin{lemma}[\(\zeta\)-normalized outer and compensator]\label{lem:zeta-normalization}
There exists a Blaschke compensator $B(s)=(s-1)/s$ and an outer function $\mathcal O$ on $\Omega$ such that the normalized ratio $\mathcal J=\dettwo(I-A)/(\mathcal O\,\xi)$ has a.e. boundary values with $|\mathcal J(\tfrac12+it)|=1$. The compensator removes the simple pole at $s=1$ and does not contribute to the boundary phase.
\end{lemma}

\vspace{0.25cm}
\begin{corollary}[No $C_P/C_\Gamma$ in the certificate]\label{cor:noCP}
Under the normalization in Lemma~\ref{lem:zeta-normalization}, the prime budget $C_P$ and Archimedean budget $C_\Gamma$ do not enter the product certificate. Only the box constant $C_{\mathrm{box}}^{(\zeta)}=K_0+K_\xi$ appears diagnostically.
\end{corollary}

\vspace{0.25cm}
\begin{proposition}[Outer Normalization and Limits] \label{prop:outer-central-main}
For boundary data in a suitable function space (BMO), there exists a unique, zero-free outer function $\mathcal O(s)$ on the half-plane $\Omega$ whose modulus matches the data on the boundary. This construction is stable under limits, which justifies the normalization of $\mathcal J(s)$.
\end{proposition}

\vspace{0.50cm}
\hrule
{\need{
       \begin{lemma}[Diagonal HS determinant is analytic and nonzero]
\label{lem:hs-diagonal}
For $s=\sigma+it$ with $\sigma>\tfrac12$, the diagonal Hilbert-Schmidt operator $A(s)e_p=p^{-s}e_p$ satisfies
\[
\sup_{p}|p^{-s}|=2^{-\sigma}<1,\qquad \sum_{p}|p^{-s}|^2=\sum_{p}p^{-2\sigma}<\infty.
\]
Hence $A(s)\in\mathrm{HS}$, $I-A(s)$ is invertible, and its determinant
\[
\det\nolimits_2\big(I-A(s)\big)=\prod_{p}(1-p^{-s})\,e^{p^{-s}}
\]
is analytic and nonzero on $\{\Re s>\tfrac12\}$.
\end{lemma}
}}
   {\modif{ Purpose: establishes analyticity and nonvanishing of the diagonal HS determinant on $\{\Re s>\tfrac12\}$.
           Outcome: this property is essential for (N2) and the pinch step, supporting the Schur/Herglotz transport after boundary positivity (P+) is verified. }}
{\mage{  \vspace{-0.0cm}      \begin{proof}
  Immediate from the displayed bounds, invertibility follows since $|1-p^{-s}|\ge 1-2^{-\sigma}>0$,
  and the product defining $\det_2$ converges absolutely with nonzero factors.
\end{proof}
}}

\hrule
\subsubsection{Smoothed Cauchy and outer limit (A2)}
\noindent\textit{Purpose.} We construct outer functions from smoothed boundary data $u_\varepsilon$ and establish convergence to an outer limit $\mathcal O$ on $\Omega$. This yields an almost-everywhere unimodular boundary gauge for $\mathcal J$, ensuring that the boundary modulus is normalized without introducing spurious factors.

\noindent\textit{Outcome.} Off-critical contributions are enclosed via VK annuli and tail bounds, and finite-block spectral gaps are certified. Together, these results yield a controlled $K_\xi$ that is used in the boundary wedge bound (Theorem~\ref{thm:psc-certificate-main}).

% Note: alternative interior-pole lemma removed to keep a single contradiction path in the pinch.
\vspace{0.0cm}

\hrule



  
  \vspace{0.20cm}
 {\need{ 
\begin{lemma}[Annular Poisson–balayage $L^2$ bound]
\label{lem:annular-balayage}
Let $I=[T-L,T+L]$, $Q_\alpha(I)=I\times(0,\alpha L]$, and fix $k\ge1$. For
$\mathcal A_k:=\{\rho=\beta+i\gamma:\ 2^kL<|T-\gamma|\le 2^{k+1}L\}$ set
\[
  V_k(\sigma,t):=\sum_{\rho\in\mathcal A_k}\frac{\sigma}{(t-\gamma)^2+\sigma^2}.
\]
Then
\[
  \iint_{Q_\alpha(I)} V_k(\sigma,t)^2\,\sigma\,dt\,d\sigma\ \ll_\alpha\ |I|\,4^{-k}\,\nu_k,
\]
where $\nu_k:=\#\mathcal A_k$, and the implicit constant depends only on $\alpha$.
\end{lemma}
}}
 {\modif{
     Purpose: establishes an $L^2$ bound for the Poisson balayage of zeros in annular regions around a Whitney box.
     Outcome: this annular $L^2$ aggregation is essential for controlling the Whitney box energy $K_\xi$, which in turn feeds the quantitative wedge criterion in Theorem~\ref{thm:psc-certificate-main}. The explicit decay factor $4^{-k}$ reflects the geometric separation of annuli. }}
   {\mage{  \vspace{-0.0cm}
\begin{proof}
By writing $K_\sigma(x):=\sigma/(x^2+\sigma^2)$ and $V_k=\sum_{\rho\in\mathcal A_k}K_\sigma(\cdot-\gamma)$, we observe that for any finite index set $\mathcal J$,
\[
  V_k^2\;\le\; \sum_{j\in\mathcal J} K_\sigma(\cdot-\gamma_j)^2\ +\ 2\!\!\sum_{i<j} K_\sigma(\cdot-\gamma_i)K_\sigma(\cdot-\gamma_j).
\]
Integrating over $t\in I$ first, we find that for the diagonal terms, using $|t-\gamma|\ge 2^kL-L\ge 2^{k-1}L$ for $t\in I$ and $k\ge 1$,
\[
 \int_I K_\sigma(t-\gamma)^2\,dt\ =\ \sigma^2\!\int_I \frac{dt}{\big((t-\gamma)^2+\sigma^2\big)^2}\ \le\ \frac{L}{(2^{k-1}L)^2}\,\sigma\ \le\ \frac{\sigma}{4^{k-1}L}.
\]
Upon multiplying by the area weight $\sigma$ and integrating over $\sigma\in(0,\alpha L]$, we obtain
\[
 \int_0^{\alpha L}\!\!\left(\int_I K_\sigma(t-\gamma)^2\,dt\right)\sigma\,d\sigma\ \le\ \frac{1}{4^{k-1}L}\int_0^{\alpha L}\!\sigma^2 d\sigma\ =\ \frac{\alpha^3 L^2}{3\cdot 4^{k-1}}\ \le\ \frac{C_{\mathrm{diag}}(\alpha)}{4^{k}}\,|I|,
\]
with $C_{\mathrm{diag}}(\alpha):=\tfrac{4\alpha^3}{3}\cdot\tfrac{L}{|I|}\asymp_\alpha 1$. Summing over the $\nu_k$ choices of $\gamma$ contributes a factor $\nu_k$. \\
For the off-diagonal terms, when $i\ne j$ we have on $I$ that $K_\sigma(t-\gamma_j)\le \sigma/(2^{k-1}L)^2$, and hence
\[
 \int_I K_\sigma(t-\gamma_i)K_\sigma(t-\gamma_j)\,dt\ \le\ \frac{\sigma}{(2^{k-1}L)^2}\int_\R K_\sigma(t-\gamma_i)\,dt\ =\ \frac{\pi\sigma}{(2^{k-1}L)^2},
\]
Integrating over $\sigma\in(0,\alpha L]$ with the extra factor $\sigma$ yields $\le C'_{\mathrm{off}}(\alpha)\,L\cdot 4^{-k}$. Summing over $i,j$ via the Schur test with $f_j(t):=K_\sigma(t-\gamma_j)\mathbf 1_I(t)$, we obtain
\[
 \int_I V_k(\sigma,t)^2\,dt\ \le\ C''(\alpha)\,\nu_k\,\frac{\sigma}{(2^kL)^2}.
\]
Integrating over $\sigma\in(0,\alpha L]$ with weight $\sigma$ yields $\le C_{\mathrm{off}}(\alpha)\,|I|\cdot 4^{-k}\,\nu_k$. By combining the diagonal and off–diagonal parts and absorbing harmless constants into $C_\alpha$, we arrive at the stated bound with an explicit $C_\alpha=O(\alpha^3)$.
\end{proof}
}}

\hrule

\vspace{0.50cm}
       {\need{ 
\begin{lemma}[Monotonicity of the tail majorant]
\label{lem:P1-monotone}  %\label{lem:tail-majorant}
For fixed $\alpha>1$, the function $g(P):=\dfrac{P^{\,1-\alpha}}{\log P}$ is strictly decreasing on $P>1$.
\end{lemma}
       }}
       {\modif{ Purpose: establishes the strict monotonicity of the tail majorant function $g(P)=P^{1-\alpha}/\log P$ for $P>1$.
           Outcome: this monotonicity is used in Corollary~\ref{cor:P1-minP} to uniquely determine the minimal tail parameter $P_\eta$ for a given target $\eta$, which in turn sets the tail cutoffs in finite-block estimates. }}
       {\mage{ \vspace{-0.0cm}
\begin{proof}
By writing $\log g(P)=(1-\alpha)\log P-\log\log P$, we find that
$(\log g)'=\dfrac{1-\alpha}{P}-\dfrac{1}{P\log P}<0$ for $P>1$.
\end{proof}
}}

       \hrule


       \vspace{0.5cm}

              {\need{ 
\begin{corollary}[Minimal tail parameter for a target $\eta$]\label{cor:P1-minP}
Given $\alpha>1$, $x_0\ge 17$ and target $\eta>0$, define $P_\eta$ to be the smallest integer $P\ge x_0$ such that
\[
\frac{1.25506\,\alpha}{(\alpha-1)\,\log P}\,P^{1-\alpha}\ \le\ \eta.
\]
By Lemma~\ref{lem:P1-monotone} this $P_\eta$ exists and is unique, moreover, the inequality then holds for every $P\ge P_\eta$. (The same definition with $\log P$ replaced by $\log P-1$ gives the $x_0\ge 599$ Dusart variant \cite{Dusart2010}.)
\end{corollary}
}}
              {\modif{ This Corollary supplies a load-bearing step (linking boundary data to zeros, bounding the ensuing measure, or transporting inequalities into the half-plane). As indicated in the proof roadmap, it feeds either the wedge closure or the interior transport.}}

\hrule

\vspace{0.50cm}
 {\need{
\begin{lemma}[Block Gershgorin lower bound]
\label{lem:block-gersh}
For every $\sigma\in[\sigma_0,1]$,
\[
  \lambda_{\min}\big(H(\sigma)\big)\ \ge\ \min_{p\le P}\Big(\lambda_{\min}\big(D_p(\sigma)\big)\ -\ \sum_{q\ne p}\|H_{pq}(\sigma)\|_2\Big).
\]
\end{lemma}
}}
 { \modif{ 
     Purpose: establishes a block Gershgorin lower bound for the minimum eigenvalue of the block Hamiltonian $H(\sigma)$.
     Outcome: this spectral gap estimate certifies that the finite-block approximation is well-conditioned, feeding into the energy bookkeeping for $K_\xi$ and ensuring the Whitney box energy bound is stable.
}}

\hrule




 \vspace{0.50cm}
{\need{ \begin{lemma}[Schur-Weyl bound]
\label{lem:schur-weyl-gap}
For every $\sigma\in[\sigma_0,1]$,
\[
  \lambda_{\min}\big(H(\sigma)\big)\ \ge\ \delta(\sigma_0),\qquad \delta(\sigma_0):=\max\Big\{0,\ \min_p\Big(\mu_p^L-\sum_{q\ne p}U_{pq}\Big),\ \min_p \mu_p^L\ -\ \max_q\frac{1}{\sqrt{\mu_q^L}}\sum_{p\ne q}\sqrt{\mu_p^L}\,U_{pq}\Big\}.\]
\end{lemma}
}}
 {\modif{ Purpose: establishes a Schur-Weyl spectral gap bound for the minimum eigenvalue of $H(\sigma)$.
     Outcome: this provides an explicit lower bound $\delta(\sigma_0)$ that certifies the spectral gap, ensuring the finite-block energy estimates are robust. This feeds into the Whitney box energy bound for $K_\xi$.}}

\hrule










\subsection{Window, Plateau, and Hilbert Bounds}\label{subsec:window-plateau}
\noindent\textit{Purpose.} This subsection calibrates the window and test-function side of the argument by establishing Poisson plateau lower bounds, Hilbert envelope estimates, and window mean-oscillation bounds ($M_\psi$) that enter the CR-Green pairing and the boundary wedge inequality. These constants ensure the boundary phase estimates are uniform and atom-safe.

\noindent\textit{Roadmap.} The core elements are: Poisson plateau lower bound (Lemma~\ref{lem:poisson-plateau}), Hilbert pairing and envelope bounds (Lemmas~\ref{lem:CH-derivative-explicit} and \ref{lem:hilbert-H1BMO}), uniform window constants (Corollary~\ref{cor:CH-Mpsi-final}), and boundary-uniform smoothed control (Corollary~\ref{cor:det2-boundary}).

\noindent\textit{Outcome.} The test-side constants are fixed uniformly in $(T,L)$ (plateau, Hilbert, BMO), yielding stable lower and upper bounds in the CR-Green pairing that are essential for closing the boundary wedge inequality in Theorem~\ref{thm:psc-certificate-main}.

\hrule




\vspace{0.50cm}
{\need{
    \begin{lemma}[Poisson plateau lower bound]
\label{lem:poisson-plateau}
For the printed even window \(\psi\) with \(\psi\equiv 1\) on \([-1,1]\),
\[ c_0(\psi)\ :=\ \inf_{0<b\le 1,\ |x|\le 1} (\Poisson_b*\psi)(x)\ \ge\ \frac{1}{2\pi}\,\arctan 2. \]
\end{lemma}
As in the plateau computation already recorded, for \(0<b\le 1\) and \(|x|\le 1\) one has
\[
 (\Poisson_b*\psi)(x)\ \ge\ (\Poisson_b*\mathbf 1_{[-1,1]})(x)
  = \frac{1}{2\pi}\Big(\arctan\tfrac{1-x}{b}+\arctan\tfrac{1+x}{b}\Big),
\]
whence
\[
 c_0(\psi)\ :=\ \inf_{0<b\le 1,\ |x|\le 1} (\Poisson_b*\psi)(x)\ \ge\ 0.1762081912\,.
 \]
}}
{\modif { Purpose: establishes a quantitative lower bound for the Poisson extension of the window function $\psi$ on the plateau $[-1,1]$.
     Outcome: this lower bound $c_0(\psi)\ge \frac{1}{2\pi}\arctan 2 \approx 0.176$ is the key constant in the windowed certificate inequality, ensuring that the boundary phase wedge has a stable positive lower bound. This feeds directly into Theorem~\ref{thm:psc-certificate-main}.
}}
{\mage{  \vspace{-0.0cm}
\begin{proof}
For the normalized Poisson kernel \(P_b(y)=\dfrac{1}{\pi}\dfrac{b}{b^2+y^2}\) and \(|x|\le 1\), we have
\[
 (P_b*\mathbf 1_{[-1,1]})(x)=\frac{1}{\pi}\int_{-1}^{1}\frac{b}{b^2+(x-y)^2}\,dy=\frac{1}{2\pi}\Big(\arctan\frac{1-x}{b}+\arctan\frac{1+x}{b}\Big).
\]
By setting \(S(x,b):=\arctan\big((1-x)/b\big)+\arctan\big((1+x)/b\big)\), we observe that symmetry yields \(S(-x,b)=S(x,b)\). For \(x\in[0,1]\), we have
\[
 \partial_x S(x,b)=\frac{1}{b}\Big(\frac{1}{1+\big(\tfrac{1+x}{b}\big)^2}-\frac{1}{1+\big(\tfrac{1-x}{b}\big)^2}\Big)\le 0,
\]
showing that \(S\) decreases in \(x\) and is minimized at \(x=1\). Moreover, \(\partial_b S(x,b)\le 0\) for \(b>0\), so the minimum over \(b\in(0,1]\) occurs at \(b=1\). Consequently, the infimum is attained at \((x,b)=(1,1)\), yielding \(\frac{1}{2\pi}\arctan 2=0.1762081912\ldots\). Since \(\psi\ge \mathbf 1_{[-1,1]}\), this establishes the bound for \(\psi\).
\end{proof}
}}

\hrule










\vspace{0.50cm}
{\need{
    \begin{lemma}[Derivative envelope for the printed window]
\label{lem:CH-derivative-explicit}
Let $\psi$ be the even $C^\infty$ flat-top window from the "Printed window" paragraph (equal to $1$ on $[-1,1]$, supported in $[-2,2]$, with monotone ramps on $[-2,-1]$ and $[1,2]$), and $\varphi_L(t):=L^{-1}\psi((t-T)/L)$. Then, for every $L>0$,
\[
  \big\|\big(\mathcal H[\varphi_L]\big)'\big\|_{L^\infty(\mathbb R)} \;\le\; \frac{C_H(\psi)}{L}
  \qquad\text{with}\qquad C_H(\psi)\;\le\;\frac{2}{\pi}\;<\;0.65.
\]
    \end{lemma}
}}
{\modif{ 
    Purpose: establishes a uniform $L^\infty$ bound for the derivative of the Hilbert transform of the scaled window function $\varphi_L$.
    Outcome: this envelope bound $C_H(\psi)\le 2/\pi$ controls the Hilbert-related terms in the CR-Green pairing, ensuring they do not overwhelm the boundary phase wedge. This is used in the quantitative closure of Theorem~\ref{thm:psc-certificate-main}.
}}
\begin{proof}
{\mage{   \vspace{-0.0cm}
\textit{Step 1 (Scaling).} By the standard scale/translation identity (recorded in the manuscript),
\[
  \mathcal H[\varphi_L](t)=H_\psi\!\Big(\frac{t-T}{L}\Big),\qquad
  H_\psi(x):=\frac{1}{\pi}\,\mathrm{p.v.}\!\int_{\mathbb R}\frac{\psi(y)}{x-y}\,dy,
\]
we get
\[
  \big(\mathcal H[\varphi_L]\big)'(t)=\frac{1}{L}\,H_\psi'\!\Big(\frac{t-T}{L}\Big)
  \quad\Longrightarrow\quad
  \big\|\big(\mathcal H[\varphi_L]\big)'\big\|_\infty=\frac{1}{L}\,\|H_\psi'\|_\infty.
\]
Thus it suffices to bound $\|H_\psi'\|_\infty$.
}}

{\mage{
    \smallskip
\textit{Step 2 (Structure and signs).} Since $\psi'\equiv0$ on $(-1,1)$ and the ramps are monotone,
\[
  \psi'(y)\ge0\ \text{on }[-2,-1],\qquad \psi'(y)\le0\ \text{on }[1,2],\qquad
  \int_{-2}^{-1}\psi'(y)\,dy=1=\!-\!\int_{1}^{2}\psi'(y)\,dy.
\]
In distributions, $(H_\psi)'= \mathcal H[\psi']$, so for every $x\in\mathbb R$
\[
  H_\psi'(x)=\frac{1}{\pi}\,\mathrm{p.v.}\!\int_{-2}^{-1}\frac{\psi'(y)}{x-y}\,dy\;+\;
             \frac{1}{\pi}\,\mathrm{p.v.}\!\int_{1}^{2}\frac{\psi'(y)}{x-y}\,dy.
\]
}}


{\mage{
    \smallskip
\textit{Step 3 (Worst case occurs between the ramps).} Fix $x\in(-1,1)$.  On $y\in[-2,-1]$ the kernel $y\mapsto 1/(x-y)$ is positive and strictly increasing, on $y\in[1,2]$ the kernel is negative and strictly decreasing.  Since the ramp densities are monotone and have unit mass in absolute value, the rearrangement/endpoint principle (maximize a monotone–kernel integral by concentrating mass at an endpoint) gives the pointwise bound
\[
  \Big|\mathrm{p.v.}\!\int_{-2}^{-1}\frac{\psi'(y)}{x-y}\,dy\Big|
  \le \frac{1}{1+x},\qquad
  \Big|\mathrm{p.v.}\!\int_{1}^{2}\frac{\psi'(y)}{x-y}\,dy\Big|
  \le \frac{1}{1-x}.
\]
Therefore, for every $x\in(-1,1)$,
\[
  |H_\psi'(x)| \;\le\; \frac{1}{\pi}\Big(\frac{1}{1+x}+\frac{1}{1-x}\Big)
  \;\le\; \frac{2}{\pi}\,\frac{1}{1-x^2}
  \;\le\; \frac{2}{\pi},
\]
with the maximum at $x=0$.
\smallskip
\textit{Step 4 (Outside the plateau).} For $x\notin[-1,1]$ the two ramp contributions have opposite signs but larger denominators, hence smaller magnitude. More precisely, for $x>1$, the left–ramp integral is a principal value on $[-2,-1]$ against a $C^\infty$ density that vanishes at the endpoints, the standard $C^1$–vanishing at $y=-2,-1$ eliminates the endpoint singularity and keeps the PV finite and strictly smaller than its in–plateau counterpart (a short integration–by–parts argument on the left interval makes this explicit). By evenness, the same holds for $x<-1$.  Consequently,
\[
  \sup_{x\in\mathbb R}|H_\psi'(x)|=\sup_{x\in(-1,1)}|H_\psi'(x)|\;\le\;\frac{2}{\pi}.
\]
Putting Steps 1–4 together,
\[
  \big\|\big(\mathcal H[\varphi_L]\big)'\big\|_\infty
  \;=\;\frac{1}{L}\,\|H_\psi'\|_\infty
  \;\le\;\frac{1}{L}\cdot\frac{2}{\pi}.
\]
Hence we can take $C_H(\psi)\le 2/\pi < 0.65$.
}}
\end{proof}

% (removed stray)

% (removed stray)


% removed stray
% =========================================================

\hrule




{\need{
    \begin{lemma}[Uniform Hilbert pairing bound]\label{lem:hilbert-H1BMO}
For every $\varphi\in C_c^\infty(I)$ and every window $\psi$ in the admissible class, one has
\[ \big|\langle \mathcal H[\varphi],\psi\rangle\big|\ \lesssim\ \|\varphi\|_{H^1(I)}. \]
The implicit constant depends only on the fixed normalization of the Hilbert transform and the aperture.
\end{lemma}
}}

\vspace{0.25cm}

\hrule

\vspace{0.50cm}
       {\modif{ This Corollary  provides the uniform quantitative inequality that controls the certificate on Whitney boxes via an explicit Carleson--box constant (through a zero-packing functional). This transparency enables choosing parameters to close the wedge. It is used
           in the boundary wedge lemma/proposition to obtain an a.e.\ wedge for the boundary phase. }}
       {\need{
           \begin{corollary}[Unconditional local window constants]\label{cor:CH-Mpsi-final}
Define, for $I=[t_0-L,t_0+L]$ and $u$ the boundary trace of $U$, the mean-oscillation constant
\[
  M_\psi\ :=\ \sup_{L>0,\ t_0\in\R}\ \frac{1}{L}\,\Big|\int_{\R} (u(t)-u_I)\,\psi_{L,t_0}(t)\,dt\Big|,\qquad u_I:=\frac{1}{|I|}\int_I u,\quad \psi_{L,t_0}(t):=\psi\big((t-t_0)/L\big),
\]
and the Hilbert constant
\[
  C_H(\psi)\ :=\ \sup_{L>0,\ t_0\in\R}\ \frac{1}{L}\,\Big|\int_{\R} \mathcal H[u'](t)\,\psi_{L,t_0}(t)\,dt\Big|.
\]
Then there are constants $C_1(\psi),C_2(\psi)<\infty$ depending only on $\psi$ and the dilation parameter $\alpha$ such that
\[
  M_\psi\ \le\ C_1(\psi)\,\sqrt{C_{\rm box}^{(\mathrm{Whitney})}}\,\mathcal A(\psi),\qquad
  C_H(\psi)\ \le\ C_2(\psi)\,\sqrt{C_{\rm box}^{(\mathrm{Whitney})}}\,\mathcal A(\psi),
\]
where the fixed Poisson energy of the window is
\[
  \mathcal A(\psi)^2\ :=\ \iint_{\R^2_+}|\nabla(P_\sigma*\psi)|^2\,\sigma\,dt\,d\sigma\ <\ \infty.
\]
In particular, both constants are finite and determined by local box energies.
\end{corollary}
}}

\hrule




       
\vspace{0.50cm}
       {\need{
           \begin{corollary}[Boundary-uniform smoothed control]\label{cor:det2-boundary}
Let $I\Subset\R$, $\varepsilon_0\in(0,\tfrac12]$, and $\varphi\in C_c^2(I)$. Then, uniformly for $\sigma\in(\tfrac12,\tfrac12+\varepsilon_0]$,
\[
  \Big|\int_{\R} \varphi(t)\,\partial_\sigma\,\Re\log\dettwo\big(I-A(\sigma+it)\big)\,dt\Big|\ \le\ C_*\,\|\varphi''\|_{L^1(I)}.
\]
In particular, the bound remains valid in the boundary limit $\sigma\downarrow \tfrac12$ in the sense of distributions.
\end{corollary}
}}
       {\modif{ This Corollary turns the energy control into a concrete almost-everywhere phase wedge
           (after a unimodular shift), limiting boundary oscillation.
           This is the last boundary-side step before interior transport.
           It serves as input to the Poisson/Cayley transport that yields a Schur/Herglotz bound in the interior.
           }}
    {\mage{  \vspace{-0.0cm}
\begin{proof}
Fix $I\Subset\R$ and $\varphi\in C_c^2(I)$. For $0<\delta<\varepsilon\le\varepsilon_0$,
\[\int \varphi\,\big(u_\varepsilon-u_\delta\big)\,dt = \int_\delta^{\varepsilon}\!\int \varphi(t)\,\partial_\sigma\,\Re\Big(\log\det_2(I-A)-\log\xi\Big)\big(\tfrac12+\sigma+it\big)\,dt\,d\sigma.\]
By Lemma~\ref{lem:det2-unsmoothed}, $\big|\int \varphi\,\partial_\sigma\Re\log\det_2\big|\le C_*\,\|\varphi''\|_{L^1(I)}$. For $\partial_\sigma\Re\log\xi=\Re(\xi'/\xi)$, test against $\varphi$ via the Poisson extension on a fixed dilation $Q(\alpha I)$ and use Lemma~\ref{lem:carleson-xi}:
\[\Big|\int \varphi\,\Re(\xi'/\xi)\Big|\ \lesssim\ \Big(\iint_{Q(\alpha I)} |\nabla U_\xi|^2\,\sigma\Big)^{\!1/2}\,\|\varphi\|_{H^1(I)}\ \lesssim\ |I|^{1/2}\,\|\varphi\|_{H^1(I)}.\]
Therefore $\big|\int \varphi\,(u_\varepsilon-u_\delta)\big|\le C(\varphi)\,|\varepsilon-\delta|$, proving the Lipschitz bound. Local-uniform convergence of outers follows from the Poisson representation and dominated convergence on $\{\Re s\ge\tfrac12+\eta\}$.
\end{proof}
}}
\hrule
\vspace{0.50cm}
\subsection{Locked Constants (with cross-references)}
\noindent For the printed window and outer normalization, we record once:
\[
 c_0(\psi)=0.17620819,\quad C_\Gamma=0\ 
\]
With the a.e. wedge, the closing condition is
\[ \pi\Upsilon\ <\ \tfrac{\pi}{2}. \]
Sum-form route: choose \(\kappa=10^{-3}\) so \(C_P=0.002\) and use the analytic envelope bound \(C_H(\psi)\le 0.26\) (Lemma~\ref{lem:CH-explicit}). Then
\[ \frac{C_\Gamma+C_P+C_H}{c_0}=\frac{0+0.002+0.26}{0.17620819}=1.4869<\frac{\pi}{2} \] (archival PSC corollary).
Product-form route (diagnostic display, not used to close (P+)): with the locked value
\(C_\psi^{(H^1)}=0.2400\) and \(C_{\mathrm{box}}^{(\zeta)}=\CboxZeta\), we have \\
\[ M_\psi= \tfrac{4}{\pi}\,C_\psi^{(H^1)}\sqrt{C_{\mathrm{box}}^{(\zeta)}}\ =\ \Mpsilocked\]   \\
\[\Upsilon_{\mathrm{diag}}=\frac{(2/\pi)\cdot \Mpsilocked}{c_0}=\UpsilonLocked \]






\vspace{0.5cm}
\subsection{PSC certificate (locked constants, canonical form)}
\noindent\textit{Locked evaluation used throughout (revised, product route via $\Upsilon$):}
\begin{align*}
 (c_0,\ C_H,\ C_\psi^{(H^1)},\ C_{\mathrm{box}})
 &\ =\ (0.17620819,\ 2/\pi,\ 0.2400,\ \CboxZeta),\\
 M_\psi\ &\ =\ \Mpsilocked,\\
 \Upsilon_{\mathrm{diag}}\ &\ =\ \frac{(2/\pi)\cdot \Mpsilocked}{0.17620819}\ =\ \UpsilonLocked. 
\end{align*}
See Appendices~\ref{app:CE-constant}-\ref{app:Cpsi-compute} for derivations and enclosures.





\vspace{0.50cm}
\paragraph{Reproducible numerics (self-contained).}
For the printed window and the \(\zeta\)–normalized route:
\begin{itemize}
\item \(c_0(\psi)\): Poisson plateau infimum (see \cref{app:Cpsi-compute}) — exact value with digits
\[ c_0(\psi)=0.17620819. \]
\item \(K_0\): arithmetic tail \(\tfrac14\sum_{p}\sum_{k\ge2} p^{-k}/k^2\) with explicit tail enclosure — locked
\[ K_0=0.03486808. \]
\item \(K_\xi\): Neutralized Whitney–box \(\xi\) energy via annular $L^2$ + VK zero–density — locked (outward-rounded)
% Avoid tautology in symbolic mode, state definition/link only
\[ K_\xi \text{ is the neutralized Whitney energy (see Lemma~\ref{lem:carleson-xi}).} \]
\item \(C_{\mathrm{box}}^{(\zeta)}\): $=K_0+K_\xi$ — used in certificate only
\[ C_{\mathrm{box}}^{(\zeta)}=\CboxZeta. \]
\item \(C_\psi^{(H^1)}\): analytic enclosure $<0.245$ and quadrature $0.23973\pm3\times10^{-4}$, we lock
\[ C_\psi^{(H^1)}=0.2400. \]
\item \(M_\psi\): Fefferman–Stein/Carleson embedding
\[ M_\psi=\tfrac{4}{\pi}\,C_\psi^{(H^1)}\,\sqrt{C_{\mathrm{box}}^{(\zeta)}}\ =\ \Mpsilocked. \]
\item \(\Upsilon\): product certificate value (no prime budget)
\[ \Upsilon_{\mathrm{diag}}\ =\ \frac{(2/\pi)\cdot \Mpsilocked}{0.17620819}\ =\ \UpsilonLocked. \]
\end{itemize}
Each number is computed once and locked with outward rounding. The certificate wedge uses only \(c_0(\psi),\,C(\psi),\,C_{\rm box}^{(\zeta)}\) and the a.e. boundary passage.





\vspace{0.50cm}
\paragraph{Constants table (for quick reference).}
\begin{center}
\begin{tabular}{ll}
\toprule
Symbol & Value/definition \\
\midrule
$c_0(\psi)$ & $\czeroplateau$ (Poisson plateau, see \cref{app:Cpsi-compute}) \\
$C_H(\psi)$ & $\CHone$ (Hilbert envelope, analytic envelope used) \\
$C_\psi^{(H^1)}$ & $\CpsiHone$ (locked from quadrature) \\
$K_0$ & $0.03486808$ (arithmetic tail, see Lemma~\ref{lem:carleson-arith}) \\
$K_\xi$ & $\Kxi$ (neutralized Whitney energy) \\
$C_{\mathrm{box}}^{(\zeta)}$ & $\CboxZeta=K_0+K_\xi$ \\
$M_\psi$ & $\Mpsilocked=(4/\pi)\,C_\psi^{(H^1)}\sqrt{C_{\mathrm{box}}^{(\zeta)}}$ \\
\(\Upsilon_{\mathrm{diag}}\) & $\UpsilonLocked=((2/\pi)\,M_\psi)/c_0$ \quad(\emph{diagnostic})\\
\bottomrule
\end{tabular}
\end{center}





\vspace{0.50cm}
\paragraph{Non-circularity (sequencing).}
We first enclose \(K_\xi\) unconditionally from annular $L^2$ and zero–counts, independent of \(M_\psi\). We then evaluate \(M_\psi\) via \((4/\pi)\,C_\psi^{(H^1)}\sqrt{C_{\mathrm{box}}^{(\zeta)}}\) using the locked \(C_{\mathrm{box}}^{(\zeta)}=K_0+K_\xi\). No step uses \(M_\psi\) to bound \(K_\xi\), so there is no feedback.
% ================================================================
%  Stage 2 Closure: PSC ⇒ (P+) and PSC from a locked certificate
% ================================================================








\vspace{0.50cm}
\subsection{Product certificate \texorpdfstring{$\Rightarrow$}  \texorpdfstring{ \,\,}  boundary wedge and (P+)}
\label{prod-cert}

\begin{theorem}[Boundary Wedge from Product Certificate]\label{thm:psc-certificate-main}
The phase-velocity identity (Theorem~\ref{thm:phase-velocity-main}) combined with the Carleson energy bound yields a quantitative control on the windowed phase derivative. This control, when paired with the Poisson plateau lower bound and CR-Green estimates, establishes a "boundary wedge" inequality implying $\Re F(\tfrac12+it)\ge 0$ almost everywhere, thus proving (P+).
\end{theorem}

\noindent\textit{Note:} This theorem gives the high-level strategy. The detailed local form with explicit inequalities appears as Theorem~\ref{thm:psc-certificate-local} below.

\noindent\textit{Roadmap.} Inputs: Theorem~\ref{thm:phase-velocity-main} (phase-velocity), Lemma~\ref{lem:poisson-plateau} (plateau lower bound), Proposition~\ref{prop:length-free} (length-free admissible bound), Lemma~\ref{lem:whitney-uniform-wedge} (Whitney wedge), Lemma~\ref{lem:local-to-global-wedge} (local-to-global wedge), and Cor.~\ref{cor:conservative-closure} (numeric closure). Output: Theorem~\ref{thm:psc-certificate-main}.

  Lemma~\ref{lem:whitney-uniform-wedge} supplies a load-bearing step (Whitney wedge on the boundary) used by the windowed certificate.

  Lemma~\ref{lem:local-to-global-wedge} upgrades the wedge from a Whitney cover to the whole boundary.

  Lemma~\ref{lem:mu-to-lebesgue} converts the zero-measure control into a Lebesgue lower bound on plateaus, needed for the closure.


\noindent\textit{Route status.} We prove (P+) via the product certificate. PSC sum/density material is archived and not used in the main chain. \emph{For the consolidated proof of (P+) see \cref{proof:Pplus}.} Closure uses the quantitative wedge criterion with a Whitney–uniform smallness $\Upsilon_{\mathrm{Whit}}(c)<\tfrac12$ (from unconditional bounds on $c_0(\psi)$, $C_\psi^{(H^1)}$, and $C_{\rm box}^{(\zeta)}$).

Fix an even $C^\infty$ window $\psi$ with $\psi\equiv 1$ on $[-1,1]$, $\operatorname{supp}\psi\subset[-2,2]$, and mass $\int_\R\psi=1$, and set
\[
  \varphi_{L,t_0}(t)\ :=\ \frac{1}{L}\,\psi\!\left(\frac{t-t_0}{L}\right),\qquad \int_{\R}\!\varphi_{L,t_0}=1,\quad \operatorname{supp}\varphi_{L,t_0}\subset I.
\]
On intervals avoiding critical-line ordinates, the a.e. wedge follows directly from the product certificate without additive constants.





\vspace{0.50cm}
{\need{
\begin{theorem}[Boundary wedge from the product certificate (local form, atom-safe)]\label{thm:psc-certificate-local}
For every Whitney interval $I=[t_0-L,t_0+L]$ one has the Poisson plateau lower bound
\[
  c_0(\psi)\,\mu\!\big(Q(I)\big)\ \le\ \int_{\R} (-w')(t)\,\varphi_{L,t_0}(t)\,dt.
\tag{\*}\label{eq:window-certificate}
\]
Moreover, for every $\phi\in\mathcal W_{\rm adm}(I;\varepsilon)$ from Definition~\ref{def:adm-bumps} (choose the mask to vanish at any critical-line atoms in $I$),
\[
  \int_{\R} \phi(t)\,(-w')(t)\,dt\ \le\ C_{\rm test}(\psi,\varepsilon,\alpha')\,\Big(\iint_{Q(\alpha'I)} |\nabla U|^2\,\sigma\Big)^{1/2}.
\]
By the all-interval Carleson bound, for each $I=[t_0-L,t_0+L]$,
\[
  \int_{\R} \phi(t)\,(-w')(t)\,dt\ \le\ C_{\rm test}(\psi,\varepsilon,\alpha')\,\sqrt{C_{\rm box}^{(\zeta)}}\,L^{1/2}.
\]
Consequently, by Lemma~\ref{lem:local-to-global-wedge} and the schedule clip,
the quantitative phase cone holds on all Whitney intervals, hence \eqref{eq:Pplus}.  
\end{theorem}
}}
{\modif { This Theorem  identifies the derivative of the boundary phase as a positive measure supported by zeros (after outer neutralization), turning qualitative boundary control into a quantitative object to bound.
It feeds directly into the Carleson/Whitney energy bound and then the boundary wedge.
}}
{\mage{   \vspace{-0.0cm}
\begin{proof}
The Poisson plateau lower bound holds for $\varphi_{L,t_0}$ by Lemma~\ref{lem:poisson-plateau} and Theorem~\ref{thm:phase-velocity-main}. The admissible-class upper bound is Proposition~\ref{prop:length-free}. The conclusion \textup{(P+)} follows from Lemma~\ref{lem:whitney-uniform-wedge} and Lemma~\ref{lem:mu-to-lebesgue}.
\end{proof}
% [archived duplicate removed]
}}






\vspace{0.0cm}
\paragraph{Scaling remark (why the density-point contradiction does not follow).}
At a density point $t_*$ of $Q$, the left inequality in \eqref{eq:window-certificate} yields a lower bound $\gtrsim c_0(\psi)\,\mu(Q(I))$, while the CR–Green/Carleson bound gives an upper bound $\lesssim C\,L^{1/2}$. For $L\downarrow 0$ one has $c_0\,L\le C\,L^{1/2}$, so there is no contradiction from single-interval scaling alone. This is why the proof uses the quantitative wedge criterion with $\Upsilon<\tfrac12$ to conclude (P+).

\begin{remark}
Let $N(\sigma,T)$ denote the number of zeros with $\Re\rho\ge \sigma$ and $0<\Im\rho\le T$. The Vinogradov–Korobov zero-density estimates give, for some absolute constants $C_0,\kappa>0$, that
\[
  N(\sigma,T)\ \le\ C_0\,T\,\log T\ +\ C_0\,T^{1-\kappa(\sigma-1/2)}\qquad (\tfrac12\le \sigma<1,\ T\ge T_1),
\]
with an effective threshold $T_1$. On Whitney scale $L=c/\log\langle T\rangle$, these bounds imply the annular counts used above with explicit $A,B$ of size $\ll 1$ for each fixed $c,\alpha$. Consequently, one can take
\[
  C_\xi\ \le\ C(\alpha,c)\,\big(C_0+1\big)
\]
in Lemma~\ref{lem:carleson-xi}, where $C(\alpha,c)$ is an explicit polynomial in $\alpha$ and $c$ arising from the annular $L^2$ aggregation (cf. Lemma~\ref{lem:annular-balayage}). We do not need the sharp exponents, any effective VK pair $(C_0,\kappa)$ suffices for a finite $C_\xi$ on Whitney boxes.
\end{remark}

% References
\vspace{0.0cm}
\section{Results}\label{sec:results}
%\noindent\textbf{Section lead.} \\
\textit{Purpose of this section:} To state and prove the main theorem from the minimal set of load-bearing inputs. \\
\textit{Inputs:} (P+) boundary wedge; (N1) right-edge normalization; (N2) non-cancellation; Carleson/CR--Green bounds; Poisson/Cayley transport. \\
\textit{Outputs:} The Riemann Hypothesis (\cref{thm:RH-main}).

We now state the main result of this work.

\begin{theorem}[Riemann Hypothesis]\label{thm:RH-main}
All nontrivial zeros of the Riemann zeta function $\zeta(s)$ lie on the critical line $\Re s=\tfrac12$.
\end{theorem}

\noindent\textbf{Proof.}
The proof follows from the contradiction framework established in \cref{sec:theory} (Methods) and the technical results in \cref{section-4} (Technical Framework). Specifically:

\begin{enumerate}
  \item \textbf{Boundary positivity (P+).} Theorem~\ref{thm:psc-certificate-main} (boundary wedge from product certificate) establishes that $\Re F(\tfrac12+it)\ge 0$ almost everywhere on the critical line, where $F(s)=2\mathcal J(s)$ is the normalized auxiliary function.
  
  \item \textbf{Interior transport.} Theorem~\ref{thm:limit-rect} shows that the Herglotz representation of $F$ combined with a Cayley transform $\Theta(s)=(F(s)-1)/(F(s)+1)$ produces a Schur function (bounded by 1) on the right half-plane $\Omega\setminus Z(\xi)$.
  
  \item \textbf{Globalization.} Theorem~\ref{thm:globalize-main} extends this to the full half-plane by showing that any putative zero of $\xi$ in $\Omega$ would create a Removable singularity (Schur) in $\Theta$, contradicting the normalization $\Theta(\sigma+it)\to -1$ as $\sigma\to\infty$ (property N1).
  
  \item \textbf{Functional equation.} The functional equation $\xi(s)=\xi(1-s)$ implies that zeros in the left half-plane $\Re s<\tfrac12$ are also excluded, completing the proof.
\end{enumerate}

This establishes that all nontrivial zeros of $\zeta(s)$ lie on the critical line.\qed

\vspace{0.5cm}

\begin{theorem}[Globalization across $Z(\xi)$]\label{thm:globalize-main}
Assume \textnormal{(P+)}, \textnormal{(N1)}, and \textnormal{(N2)}. Then the Schur bound for $\Theta$ on $\Omega\setminus Z(\xi)$ extends across the zero set by removability (Lemma~\ref{lem:removable-schur}), yielding nonvanishing of $\xi$ on $\Omega$.
\end{theorem}


\noindent\textbf{Summary of proof architecture.}
The argument proceeds boundary-to-interior: outer normalization (Lemma~\ref{lem:zeta-normalization}) and inner-factor control (Lemma~\ref{lem:hs-diagonal}) ensure clean boundary data; the phase-velocity identity (Theorem~\ref{thm:phase-velocity-main}) converts phase variation to a zero-supported measure; Carleson--box energy bounds (Lemma~\ref{lem:carleson-xi}, Corollary~\ref{cor:xi-carleson-all-I}) control this measure; the boundary wedge inequality (Theorem~\ref{thm:psc-certificate-main}) yields (P+); Poisson transport and Cayley transform produce interior Schur control; and the removability pinch (Lemma~\ref{lem:removable-schur}) forces nonvanishing. All dependencies are explicit and quantitative constants ($K_0$, $K_\xi(\alpha,c)$, $c_0(\psi)$) are locked and auditable (see Appendices~\ref{app:definitions}--\ref{app:Cpsi-compute}).























\par\normalfont\normalsize
\section{Discussion and Conclusions}\label{sec:conclusions}
%\noindent\textbf{Section lead.} \\
%\textit{Purpose:} To record robustness, auditability, and scope; to separate proof-critical content from optional diagnostics. \\
%\textit{Inputs:} Main theorem and its dependencies; build-mode policy (symbolic vs numeric lock). \\
%\textit{Outputs:} Reader guidance on variants, limitations, and immediate extensions.
%\paragraph{Threats and mitigations.}
%\emph{Hidden circularity} is avoided by locking the outer normalization and using symbolic constants until the final audit pass, \emph{scheme/policy sensitivity} is absorbed into the declared variant bands and monitored via $\Delta_{\rm coh}^{(v)}$, \emph{numerical precision} is controlled by audited tolerances tied to the Carleson energy estimate, \emph{transport artifacts} are checked by verifying the Schur map and removability pinch preconditions in each variant.
\label{conclusions}







\subsection{Robustness, auditability, and scope}
The proof is robust to variations in the window shape and zero-density input, as long as the Carleson constant $C_\xi$ remains finite. All numerical diagnostics are gated and non-load-bearing; the logical chain is unconditional. The manuscript is auditable: constants are locked once, dependencies are explicit, and proofs are modular with purpose/roadmap/outcome blurbs. For reproducibility, the window constant $C_\psi^{(H^1)}$ is computed via an interval-arithmetic protocol in \cref{app:Cpsi-compute}.

\subsection{Limitations and scope}
This work proves (P+) by boundary methods and derives interior Schur/Herglotz bounds and removability without relying on numeric locks; number-theoretic inputs (e.g. VK zero-density) enter only diagnostically to enclose $K_\xi$ and do not bear logical load. The route is specific to the completed $\xi$ on the half-plane; extensions to other $L$-functions require the standard substitutions (completed $\Lambda$, local factors, conductor) and a recomputation of the packing input.
\label{sec:concl}
\par\normalfont\normalsize
\appendix
\vspace{1.0cm}
{\bf APPENDIX}

\section{Canonical auxiliary statements}

\begin{lemma}[CR--Green pairing for boundary phase]\label{lem:CR-green-phase}
For $U$ harmonic on $\Omega$ with boundary trace $u$ and $\varphi\in C_c^\infty(I)$, the Poisson extension $V$ of $\varphi$ on $Q(\alpha I)$ satisfies
\[\int_I \varphi\,(-\partial_t\,\Arg\,\mathcal J)\,dt\ =\ \iint_{Q(\alpha I)} \nabla U\cdot\nabla V\,dt\,d\sigma,\]
provided $\mathcal J=\exp(U+i\mathcal H[u])$ on the boundary.
\end{lemma}

\begin{lemma}[Outer cancellation in CR--Green]\label{lem:outer-cancel}
After subtracting the affine calibrant on $I$, the outer component of $U$ does not contribute to the CR--Green pairing in Lemma~\ref{lem:CR-green-phase}.
\end{lemma}

\begin{lemma}[Outer cancellation and energy bookkeeping]\label{lem:outer-energy-bookkeeping}
On each Whitney box $Q(\alpha I)$ one has
\[\Big|\iint_{Q(\alpha I)} \nabla U\cdot\nabla V\,dt\,d\sigma\Big|\ \lesssim\ (C_{\mathrm{box}}^{(\zeta)})^{1/2}\,\|\varphi\|_{H^1(I)}.\]
\end{lemma}

\begin{lemma}[Cutoff pairing on boxes]\label{lem:cutoff-pairing}
Let $\varphi\in C_c^\infty(I)$ and $V$ be the Dirichlet test field for $(\mathcal H[\varphi])'$ supported in $Q(\alpha'I)$. Then
\[ \Big|\iint_{Q(\alpha'I)} \nabla U\cdot\nabla V\,dt\,d\sigma\Big|\ \lesssim\ \big(\iint_{Q(\alpha'I)} |\nabla U|^2\,\sigma\,dt\,d\sigma\big)^{1/2}\,\|\varphi\|_{H^1(I)}. \]
\end{lemma}

\begin{lemma}[Arithmetic Carleson energy]\label{lem:carleson-arith}
With VK zero-density translated to annular counts, there exists $C_\xi(\alpha,c)$ such that for every Whitney interval $J$,
\[ \iint_{Q(\alpha J)} |\nabla U_\xi|^2\,\sigma\,dt\,d\sigma\ \le\ C_\xi\,|J|. \]
\end{lemma}

\begin{lemma}[2--modified determinant: existence and bounds]\label{lem:det2-unsmoothed}
For diagonal $A(s)$ with entries $p^{-s}$ on $\sigma>1/2$, $A(s)$ is Hilbert--Schmidt and $\det\nolimits_2(I-A(s))$ exists, is nonzero, and depends analytically on $s$.
\end{lemma}

\begin{lemma}[Hilbert transform pairing]\label{lem:hilbert}
For $\varphi\in C_c^\infty(I)$ and window $\psi$ in the admissible class, one has\; $|\langle \mathcal H[\varphi],\psi\rangle|\ \lesssim\ \|\varphi\|_{H^1(I)}$.
\end{lemma}

\begin{lemma}[Local to global wedge upgrade]\label{lem:local-to-global-wedge}
If a boundary wedge holds uniformly on a finite-overlap Whitney cover, then a global wedge holds a.e. with parameters depending only on the overlap and local wedge parameters.
\end{lemma}

\begin{lemma}[Window mean--oscillation bound]\label{lem:Mpsi-correct}
For admissible $\psi$ one has $M_\psi\ \le\ (4/\pi)\,C_\psi^{(H^1)}\,\sqrt{C_{\mathrm{box}}^{(\zeta)}}$.
\end{lemma}

\begin{lemma}[From $\mu$ to Lebesgue control on plateaus]\label{lem:mu-to-lebesgue}
If $\mu$ is Carleson on boxes with constant $C_{\mathrm{box}}$, then on plateau intervals Poisson averages admit bilateral bounds depending only on $C_{\mathrm{box}}$ and window constants.
\end{lemma}

\begin{lemma}[Poisson--BMO bound at fixed height]\label{lem:poisson-bmo-strip}
For the Poisson extension at fixed height, the BMO norm is controlled by the Carleson--box constant (\cref{app:CE-constant}).
\end{lemma}

\begin{lemma}[Uniform CR--Green bound]\label{lem:uniform-CRG-A}
On a fixed admissible class $\mathcal A$ of intervals, the CR--Green pairing is uniformly bounded by $(C_{\mathrm{box}}^{(\zeta)})^{1/2}\,\|\varphi\|_{H^1(I)}$.
\end{lemma}

\begin{lemma}[Whitney--uniform boundary wedge]\label{lem:whitney-uniform-wedge}
There exists a wedge parameter $\Upsilon(\alpha,c)$ such that the windowed boundary phase lies in a uniform wedge on Whitney intervals with overlap constants depending only on $(\alpha,c)$ and window constants.
\end{lemma}

\begin{lemma}[Hilbert envelope]\label{lem:CH-explicit}
For the printed window $\psi$, the Hilbert transform envelope satisfies $C_H(\psi)\le 2/\pi$.
\end{lemma}

\begin{lemma}[Printed-window derivative envelope]\label{lem:CH-derivative-2pi}
For the printed window $\psi$, an explicit derivative envelope holds with constant $2/\pi$ in the stated normalization.
\end{lemma}

\section{Canonical auxiliary theorems and corollaries}

\begin{theorem}[Limit \(N\to\infty\) on rectangles: \(2J\) Herglotz, \(\Theta\) Schur]\label{thm:limit-rect}
Assuming \textnormal{(P+)} and the normalization in \cref{sec:theory}, the functions \(2\mathcal J\) and \(\Theta=(2\mathcal J-1)/(2\mathcal J+1)\) admit rectangle-truncation limits with \(2\mathcal J\) Herglotz and \(|\Theta|\le 1\) on \(\Omega\setminus Z(\xi)\).
\end{theorem}

\begin{corollary}[Explicit remainder control]\label{expl-contr}
The truncation remainders in the rectangle limits are bounded explicitly in terms of the window constants and \(C_{\mathrm{box}}^{(\zeta)}\), uniformly on Whitney scale.
\end{corollary}

\begin{corollary}[Atom neutralization and Whitney scaling]\label{cor:atom-safe}
After neutralizing atoms with a local half-plane Blaschke factor, the near-field contribution is \(\ll |I|\) uniformly on Whitney scale.
\end{corollary}

\begin{corollary}[Unconditional Schur on \(\Omega\setminus Z(\xi)\)]\label{cor:Schur-off-zeros}
Under \textnormal{(P+)} and normalization, \(|\Theta|\le 1\) holds on the punctured region \(\Omega\setminus Z(\xi)\).
\end{corollary}

\begin{corollary}[Zero-free right half-plane]\label{cor:zero}
If \textnormal{(P+)}, \textnormal{(N1)}, and \textnormal{(N2)} hold, then \(Z(\xi)\cap\Omega=\varnothing\).
\end{corollary}

\begin{corollary}[Conclusion (RH)]\label{cor:concl}
Combining Corollary~\ref{cor:zero} with the functional equation for \(\xi\), all nontrivial zeros lie on the critical line \(\Re s=\tfrac12\).
\end{corollary}

\begin{corollary}[Poisson transport]\label{cor:poisson-herglotz}
Poisson transport of \textnormal{(P+)} yields \(\Re(2\mathcal J)\ge 0\) in the interior of \(\Omega\).
\end{corollary}

\begin{corollary}[Cayley]\label{cor:cayley-schur}
The Cayley transform \(\Theta=(2\mathcal J-1)/(2\mathcal J+1)\) of a Herglotz function is Schur.
\end{corollary}

\begin{corollary}[No far-far budget from triangular padding]\label{cor:K-no-FF}
Triangular padding does not contribute a far-far budget to \(C_{\mathrm{box}}^{(\zeta)}\); the diagnostic constant remains \(K_0+K_\xi\).
\end{corollary}

\begin{proposition}[Length-free admissible bound]\label{prop:length-free}
For the admissible class $\mathcal W_{\rm adm}(I,\varepsilon)$ on an interval $I$ and the Dirichlet test field built from $(\mathcal H[\varphi_I])'$, the CR--Green pairing over $Q(\alpha I)$ satisfies the length-free bound
\[
  \Big|\iint_{Q(\alpha I)} \nabla U\cdot\nabla V\,dt\,d\sigma\Big|\ \lesssim\ \big(C_{\mathrm{box}}^{(\zeta)}\big)^{1/2}\,\|\varphi_I\|_{H^1(I)},
\]
with an implicit constant depending only on the fixed aperture and admissible class parameters.
\end{proposition}

\section{Reference Catalogues of Lemmas, Theorems, and Corollaries}
\label{app:catalogues}

\subsection{Catalogue of Lemmas}
\begin{itemize}
  \item Lemma~\ref{lem:removable-schur} (Removable singularity (Schur) under Schur bound)
  \item Lemma~\ref{lem:carleson-sum} (Carleson--box energy: stable sum bound)
  \item Lemma~\ref{lem:carleson-xi} (Analytic ($\xi$) Carleson energy on Whitney boxes)
  \item Lemma~\ref{lem:xi-deriv-L1} (L$^1$-tested control for $\partial_\sigma\Re\log\xi$)
  \item Lemma~\ref{lem:outer-phase-HT} (Outer phase and Hilbert transform control)
  \item Lemma~\ref{lem:zeta-normalization} ($\zeta$--normalized outer and compensator)
  \item Lemma~\ref{lem:hs-diagonal} (Diagonal HS determinant: analytic and nonzero)
  \item Lemma~\ref{lem:annular-balayage} (Annular Poisson--balayage $L^2$ bound)
  \item Lemma~\ref{lem:P1-monotone} (Monotonicity of the tail majorant)
  \item Lemma~\ref{lem:block-gersh} (Block Gershgorin lower bound)
  \item Lemma~\ref{lem:schur-weyl-gap} (Schur--Weyl bound)
  \item Lemma~\ref{lem:poisson-plateau} (Poisson plateau lower bound)
  \item Lemma~\ref{lem:CH-derivative-explicit} (Derivative envelope for the printed window)
  \item Lemma~\ref{lem:hilbert-H1BMO} (Uniform Hilbert pairing bound)
  \item Lemma~\ref{lem:CR-green-phase} (CR--Green pairing for boundary phase)
  \item Lemma~\ref{lem:outer-cancel} (Outer cancellation in the CR--Green pairing)
  \item Lemma~\ref{lem:outer-energy-bookkeeping} (Outer cancellation and energy bookkeeping on boxes)
  \item Lemma~\ref{lem:cutoff-pairing} (Cutoff pairing on boxes)
  \item Lemma~\ref{lem:carleson-arith} (Arithmetic Carleson energy)
  \item Lemma~\ref{lem:det2-unsmoothed} (2--modified determinant: existence and basic bounds)
  \item Lemma~\ref{lem:hilbert} (Hilbert--transform pairing)
  \item Lemma~\ref{lem:local-to-global-wedge} (Local--to--global wedge upgrade)
  \item Lemma~\ref{lem:Mpsi-correct} (Window mean--oscillation via H$^1$--BMO and box energy)
  \item Lemma~\ref{lem:mu-to-lebesgue} (From $\mu$ to Lebesgue control on plateaus)
  \item Lemma~\ref{lem:poisson-bmo-strip} (Poisson--BMO bound at fixed height)
  \item Lemma~\ref{lem:uniform-CRG-A} (Uniform CR--Green bound for the class $\mathcal A$)
  \item Lemma~\ref{lem:whitney-uniform-wedge} (Whitney--uniform boundary wedge)
  \item Lemma~\ref{lem:CH-explicit} (Derivative envelope: $C_H(\psi)\le 2/\pi$)
  \item Lemma~\ref{lem:CH-derivative-2pi} (Explicit envelope for the printed window)
  \item Lemma~\ref{lem:CE-constant-one} (Normalization of the embedding constant)
\end{itemize}

\subsection{Catalogue of Theorems}
\begin{itemize}
  \item Theorem~\ref{thm:phase-velocity-main} (Phase--Velocity Identity)
  \item Theorem~\ref{thm:psc-certificate-main} (Boundary wedge from product certificate)
  \item Theorem~\ref{thm:limit-rect} (Limit \(N\to\infty\) on rectangles: \(2J\) Herglotz, \(\Theta\) Schur)
  \item Theorem~\ref{thm:globalize-main}  (Globalization across $Z(\xi)$)
  \item Theorem~\ref{thm:RH-main} (Riemann Hypothesis)
\end{itemize}

\subsection{Catalogue of Corollaries}
\begin{itemize}
  \item Cor.~\ref{cor:xi-carleson-all-I} (All-interval Carleson energy for $U_\xi$)
  \item Cor.~\ref{cor:conservative-closure} (Conservative numeric closure under Lemma~\ref{lem:carleson-sum})
  \item Cor.~\ref{cor:noCP} (No $C_P/ C_\Gamma$ in the certificate)
  \item Cor.~\ref{cor:P1-minP} (Minimal tail parameter)
  \item Cor.~\ref{cor:CH-Mpsi-final} (Unconditional local window constants)
  \item Cor.~\ref{cor:det2-boundary} (Boundary-uniform smoothed control)
  \item Cor.~\ref{expl-contr} (Explicit remainder control)
  \item Cor.~\ref{cor:atom-safe} (Atom neutralization and Whitney scaling)
  \item Cor.~\ref{cor:Schur-off-zeros} (Unconditional Schur on $\Omega\setminus Z(\xi)$)
  \item Cor.~\ref{cor:zero} (Zero-free right half-plane)
  \item Cor.~\ref{cor:concl} (Conclusion (RH))
  \item Cor.~\ref{cor:poisson-herglotz} (Poisson transport)
  \item Cor.~\ref{cor:cayley-schur} (Cayley)
  \item Cor.~\ref{cor:K-no-FF} (No far-far budget from triangular padding)
\end{itemize}

\section {Constants and definitions used in certification }
\label{app:definitions}

\begin{table}[H]
\centering
\caption{Compact constants used in the covering and budgets (fixed example values shown).}
\begin{tabular}{l l}
\toprule
Arithmetic energy & $K_0=\tfrac14\sum_{p}\sum_{k\ge2} \dfrac{p^{-k}}{k^2}$ \\ 
Prime cut / minimal prime & $Q=29$, $\ p_{\min}=31$ \\ 
Tail bounds & $\sum_{p>x}p^{-\alpha} \le \dfrac{1.25506\,\alpha}{(\alpha-1)\,\log x}\,x^{\,1-\alpha}$ (for $x\ge 17$) \\ 
Row/col budgets & $\Delta_{SS},\Delta_{SF},\Delta_{FS},\Delta_{FF}$ as in Lemma~\ref{lem:block-gersh} and Lemma~\ref{lem:schur-weyl-gap} \\ 
In-block lower bounds & $\mu^{\mathrm{small}}=1-\Delta_{SS}$, $\ \mu^{\mathrm{far}}=1-\tfrac{L(p_{\min})}{6}$ \\ 
Link barrier & $L(\sigma)=(1-\sigma)(\log p_{\min})\,p_{\min}^{-\sigma}$ \\ 
Lipschitz constant & $K(\sigma)=S_{\sigma+1/2}(Q)+\tfrac14\,p_{\min}^{-\sigma}S_{\sigma}(Q)$ \\ 
Prime sums & $S_{\alpha}(Q)=\sum_{p\le Q} p^{-\alpha}$, $\ T_{\alpha}(p_{\min})=\sum_{p\ge p_{\min}} p^{-\alpha}$ \\ 
\bottomrule
\end{tabular}
\end{table}












\subsection{Carleson embedding constant for fixed aperture}
\label{app:CE-constant}
We record a one-time bound for the Carleson-BMO embedding constant with the cone aperture $\alpha$ used throughout. For the Poisson extension $U$ and the area measure $\mu=|\nabla U|^2\,\sigma\,dt\,d\sigma$, the conical square function with aperture $\alpha$ satisfies the Carleson embedding inequality
\[
  \|u\|_{\mathrm{BMO}}\ \le\ \frac{2}{\pi}\,C_{\mathrm{CE}}(\alpha)\,\Big(\sup_I \frac{\mu(Q(\alpha I))}{|I|}\Big)^{\!1/2}.
\]

{\need{ 
\begin{lemma}[Normalization of the embedding constant]
\label{lem:CE-constant-one}
In the present normalization (Poisson semigroup on the right half-plane, cones of aperture $\alpha\in[1,2]$, and Whitney boxes $Q(\alpha I)$), one can take $C_{\mathrm{CE}}(\alpha)=1$.
\end{lemma}
}}
{\modif{ This Lemma fixes boundary normalization and excludes hidden singular inner factors or cancellations, so the boundary phase measure reflects the true zero structure of $\xi$. Without it, later phase/energy bounds would be contaminated by an uncontrolled boundary singular measure. It is invoked immediately in the boundary phase certificate and again in the globalization/pinch step.}}












\subsection{VK   \texorpdfstring{$\to$annuli$\to C_\xi\to K_\xi$} numeric enclosure}
\label{app:vk-annuli-kxi}
Fix $\alpha\in[1,2]$ and the Whitney parameter $c\in(0,1]$. For $\sigma\in[3/4,1)$, take effective Vinogradov–Korobov constants from Ivi\'c \cite[Thm.~13.30]{Ivic}. Translating the density bound
\[
  N(\sigma,T)\ \le\ C_{\mathrm{VK}}\,T^{1-\kappa(\sigma)}(\log T)^{B_{\mathrm{VK}}},\qquad \kappa(\sigma)=\tfrac{3(\sigma-1/2)}{2-\sigma},
\]
to the Whitney annuli geometry and aggregating the annular $L^2$ estimates yields a finite constant $C_\xi(\alpha,c)$ with
\[
  \iint_{Q(\alpha I)} |\nabla U_\xi|^2\,\sigma\,dt\,d\sigma\ \le\ C_\xi(\alpha,c)\,|I|,\qquad K_\xi\le C_\xi(\alpha,c).
\]
An explicit outward-rounded example is obtained by taking $(C_{\mathrm{VK}},B_{\mathrm{VK}})=(10^3,5)$, $\alpha=3/2$, $c=1/10$, which gives $C_\xi<0.160$.
\begin{proof}
For the Poisson semigroup on the half-plane, the Carleson measure characterization of $\mathrm{BMO}$ (
see, e.g., Garnett \cite[Thm.~VI.1.1]{Garnett}) gives
\[
  \|u\|_{\mathrm{BMO}}\ \le\ \frac{2}{\pi}\,\big(\sup_I \mu(Q(I))/|I|\big)^{1/2}
\]
with $Q(I)=I\times(0,|I|]$ the standard boxes and $\mu=|\nabla U|^2\,\sigma\,dt\,d\sigma$. Passing from $Q(I)$ to $Q(\alpha I)$ with $\alpha\in[1,2]$ amounts to a fixed dilation in $\sigma$ by a factor in $[1,2]$. Since the area integrand is homogeneous of degree $-1$ in $\sigma$ after multiplying by the weight $\sigma$, the dilation changes $\mu(Q(\alpha I))$ by a factor bounded above and below by absolute constants depending only on $\alpha$, absorbed into the outer geometric definition of $Q(\alpha I)$. Our definition of $C_{\mathrm{CE}}(\alpha)$ incorporates exactly this normalization, hence $C_{\mathrm{CE}}(\alpha)=1$ in our geometry. (Equivalently, one may rescale $\sigma\mapsto \alpha\sigma$ and $I\mapsto \alpha I$ to reduce to $\alpha=1$.)
\end{proof}












\subsection{Numerical evaluation of \texorpdfstring{$C_\psi^{(H^1)}$} for the printed window}
\label{app:Cpsi-compute}
We record a reproducible computation of the window constant
\[
  C_\psi^{(H^1)}\ :=\ \frac12\int_{\R} S\phi\,dx,\qquad \phi(x):=\psi(x)-\frac{m_\psi}{2}\,\mathbf 1_{[-1,1]}(x),\quad m_\psi:=\int_{\R}\psi.
\]
Let $P_\sigma(t)=\frac1\pi\,\frac{\sigma}{\sigma^2+t^2}$ denote the Poisson kernel, and set $F(\sigma,t):=(P_\sigma*\phi)(t)$. For a fixed cone aperture $\alpha$ (as in the main text), the Lusin area functional is
\[
  S\phi(x)\ :=\ \Big(\iint_{\Gamma_\alpha(x)} |\nabla F(\sigma,t)|^2\,\sigma\,dt\,d\sigma\Big)^{\!1/2},\qquad \Gamma_\alpha(x):=\{(\sigma,t):|t-x|<\alpha\sigma,\ \sigma>0\}.
\]
Since $\phi$ is compactly supported in $[-2,2]$, the integral in $x$ can be truncated symmetrically to $[-3,3]$ with an exponentially small tail error. Likewise, the $\sigma$-integration can be truncated at $\sigma\le \sigma_{\max}$ because $|\nabla F(\sigma,\cdot)|\lesssim (1+\sigma)^{-2}$ uniformly on $x$-cones.



\section{  Collected auxiliary statements (for cross-references)}
\paragraph{Constants table (symbol \texorpdfstring{$\to$}{->} meaning \texorpdfstring{$\to$}{->} value/source).}
\begin{longtable}{@{}llp{0.55\linewidth}@{}}
\toprule
\textbf{Symbol} & \textbf{Meaning} & \textbf{Value / Source (audit)} \\
\midrule
$K_0$ & arithmetic tail bound & audit table / appendix (symbolic unless \texttt{\numericlocktrue}) \\
$K_{\xi}$ & coarse $\xi$-zeros box bound & audit table / appendix (symbolic unless locked) \\
$c_0(\psi)$ & Poisson plateau lower bound & audit table, \emph{fixed} window family \\
$\CHzero,\,\CHone$ & Hilbert transform envelopes & Methods derivation / appendix references \\
$\CpsiHone$ & window $H^1$ constant & Methods, locked once \\
$\CboxZeta$ & box constant (diagnostic) & $K_0+K_{\xi}$ (non-load-bearing) \\
%$\Mpsilocked$ & $(4/\pi)\,\CpsiHone\,\sqrt{\CboxZeta}$ & derived \\
%$\UpsilonLocked$ & $(2/\pi)\,\Mpsilocked/\czeroplateau$ & derived \\
\bottomrule
\end{longtable}

\subsection{Weighted \texorpdfstring{$p$}--adaptive model (certificate variant)}
\noindent\textit{Purpose.} Records an optional weighted $p$-adaptive enclosure used to illustrate a variant of the certificate. Not needed for the proof of (P+), included for completeness and reproducibility.

\paragraph*{Certificate \textemdash{} weighted \(p\)-adaptive model at \(\sigma_0=0.6\). }
Fix \(\sigma_0=0.6\), take \(Q=29\) and \(p_{\min}=\mathrm{nextprime}(Q)=31\).\
Use the \(p\)-adaptive weighted off-diagonal enclosure (for all \(p\neq q\), uniformly in \(\sigma\in[\sigma_0,1]\)):
\[
\|H_{pq}(\sigma)\|_2 \;\le\; \frac{C_{\mathrm{win}}}{4}\, p^{-(\sigma+\tfrac12)}\, q^{-(\sigma+\tfrac12)},
\qquad C_{\mathrm{win}}=0.25.
\]
\noindent\emph{Prime sums (small block \(p\le Q\)).} With \(\sigma_0=0.6\),
\[
S_{\sigma_0}(Q)\;=\;\sum_{p\le Q} p^{-\sigma_0}\;=\;2.9593220929,\qquad
S_{\sigma_0+\tfrac12}(Q)\;=\;\sum_{p\le Q} p^{-(\sigma_0+\tfrac12)}\;=\;1.3239981250.
\]
\noindent\emph{In-block Gershgorin lower bounds (uniform on \([\sigma_0,1]\)).}
Define
\[
L(p)\;:=\;(1-\sigma_0)\,(\log p)\,p^{-\sigma_0},\qquad 
\mu_p^{\mathrm L}\;\ge\;1-\frac{L(p)}{6}.
\]
At \(p_{\min}=31\) this gives
\[
L(31)=0.1750014502,\qquad 
\mu_{\min}^{\mathrm{far}}\;:=\;1-\frac{L(31)}{6}\;=\;0.9708330916.
\]
Over the small block \(p\le Q\) the worst case is at \(p=5\):
\[
L(5)=0.2451050257,\qquad 
\mu_{\min}^{\mathrm{small}}\;:=\;1-\frac{L(5)}{6}\;=\;0.9591491624.
\]
\noindent\emph{Off-diagonal budgets (all rigorous).}
Let \(\sigma^\star:=\sigma_0+\tfrac12=1.1\).\\
With the integer-tail majorant \(\displaystyle \sum_{n\ge p_{\min}-1} n^{-\sigma^\star}\le
\frac{(p_{\min}-1)^{1-\sigma^\star}}{\sigma^\star-1}\)
we obtain explicit $\Delta$-budgets as in the main text.
% (content remains as earlier appendix material, see labels)
% The following appendix material is moved here for conventional ordering



% [Omitting the rest of appendix content here to minimize duplication in this edit, the original appendix content remains earlier in the file]




\vspace{2.0cm}

\section*{Statements and Declarations}
\paragraph{Competing Interests.} The authors declare no competing interests.
\paragraph{Author contributions.} 
Conceptualization, methodology, formal analysis, software, validation, writing (original draft), and review/editing: J. Washburn. \\
Investigation, data curation, visualization, writing (final version): E. Allahyarov.




\newpage

% REFERENCES
\begin{thebibliography}{99}
\makeatletter\setcounter{NAT@ctr}{0}\setcounter{enumiv}{0}\makeatother\bibitem{Titchmarsh} E.~C. Titchmarsh, \emph{The Theory of the Riemann Zeta-Function}, 2nd ed., revised by D.~R. Heath-Brown, Oxford University Press, Oxford, 1986. (RvM, zero-density background in Ch. VIII-IX.)

\bibitem{Ivic} A. Ivi\'c, \emph{The Riemann Zeta-Function: Theory and Applications}, Dover Publications, Mineola, NY, 2003. (Thm. 13.30: VK zero-density, used parametrically.)

\bibitem{DavenportMNT} H. Davenport, \emph{Multiplicative Number Theory}, 3rd ed., revised by H.~L. Montgomery, Graduate Texts in Mathematics, vol.~74, Springer-Verlag, New York, 2000.

\bibitem{MontgomeryVaughan} H.~L. Montgomery and R.~C. Vaughan, \emph{Multiplicative Number Theory I. Classical Theory}, Cambridge Studies in Advanced Mathematics, vol.~97, Cambridge Univ. Press, Cambridge, 2007.

\bibitem{Hadamard1896}
  J.~Hadamard, Sur la distribution des zéros de la fonction $\xi(s)$ et ses conséquences arithmétiques,
  \emph{Bulletin de la Société Mathématique de France} 24 (1896), 199-220.


  
\bibitem{DeLaValleePoussin1896}
  C.-J.~de la Vallée Poussin, Recherches analytiques sur la théorie des nombres premiers,
  \emph{Annales de la Société Scientifique de Bruxelles} 20 (1896), 183-256.

\bibitem{Hardy1914}
  G.~H. Hardy, Sur les zéros de la fonction  $\xi(s)$ de Riemann,
  \emph{Comptes Rendus de l'Académie des Sciences} 158 (1914), 1012-1014.

\bibitem{HardyLittlewood1921}
  G.~H. Hardy and J.~E. Littlewood, The zeros of Riemann's zeta-function on the critical line,
  \emph{Mathematische Zeitschrift} 10 (1921), 283-317.

\bibitem{Selberg1942}
  A.~Selberg, On the zeros of Riemann's zeta-function,
  \emph{Skrifter utgitt av Det Norske Videnskaps-Akademi i Oslo. I. Matematicheskaya Klasse},
  No. 10 (1942), 1-59.

\bibitem{Levinson1974}
  N.~Levinson, More than one third of zeros of Riemann's zeta-function are on
  $\sigma = 1/2$, \emph{Advances in Mathematics} 13 (1974), 383-436.

\bibitem{Conrey1989}
  J.~B. Conrey, More than two fifths of the zeros of the Riemann zeta function are on the critical line,
  \emph{Journal für die Reine und Angewandte Mathematik} 399 (1989), 1-26.

\bibitem{Vinogradov1958}
  I.~M. Vinogradov, A new estimate of the function $\zeta(1+it)$,
  \emph{Izvestiya Akademii Nauk SSSR, Seriya Matematicheskaya} 22 (1958), 161-164.

\bibitem{Korobov1958}
  N.~M. Korobov, Estimates of trigonometric sums and their applications,
  \emph{Uspekhi Matematicheskikh Nauk} 13 (1958), no. 4 (82), 185-192.

\bibitem{Richert1967}
H.-E. Richert, Zur Abschätzung der Riemannschen Zetafunktion in der Nähe der Vertikalen $\sigma=1$, \emph{Mathematische Annalen} 169 (1967), 97-101.

\bibitem{Ford2002}
K.~Ford, Vinogradov's integral and bounds for the Riemann zeta function, \emph{Proceedings of the London Mathematical Society} 85 (2002), no. 3, 565-633.

\bibitem{MossinghoffTrudgian2015}
M.~J. Mossinghoff and T.~S. Trudgian, An explicit zero-free region for the Riemann zeta-function, \emph{Journal of Number Theory} 157 (2015), 406-423.

\bibitem{Montgomery1973}
  H.~L. Montgomery, The pair correlation of zeros of the zeta function,
  \emph{Analytic number theory}, Proc. Sympos. Pure Math., Vol. XXIV, Providence, R.I.: American Mathematical Society, 1973, pp. 181-193.


\bibitem{RudnickSarnak1996}
  Z.~Rudnick and P.~Sarnak, Zeros of principal L-functions and random matrix theory, \emph{Duke Mathematical Journal} 81 (1996), no. 2, 269-322.

  \bibitem{KatzSarnak1999}
N.~M. Katz and P.~Sarnak, \emph{Random Matrices, Frobenius Eigenvalues, and Monodromy}, American Mathematical Society Colloquium Publications, vol.~45, American Mathematical Society, Providence, RI, 1999.

\bibitem{DurenHp} P.~L. Duren, \emph{Theory of $H^p$ Spaces}, Academic Press, New York, 1970, reprint, Dover Publications, Mineola, NY, 2000. (Hardy/Smirnov background.)

\bibitem{Hoffman} K. Hoffman, \emph{Banach Spaces of Analytic Functions}, Dover Publications, Mineola, NY, 2007. (Reprint of the 1962 Prentice-Hall edition.)

\bibitem{Donoghue} W.~F. Donoghue, Jr., \emph{Monotone Matrix Functions and Analytic Continuation}, Springer, New York, 1974. (Pick/Herglotz functions and positivity.)

\bibitem{AglerMcCarthy} J. Agler and J.~E. McCarthy, \emph{Pick Interpolation and Hilbert Function Spaces}, Graduate Studies in Mathematics, vol.~44, Amer. Math. Soc., Providence, RI, 2002.

\bibitem{CarlesonCorona} L. Carleson, Interpolation by bounded analytic functions and the corona problem, \emph{Ann. of Math.} (2) 76 (1962), 547-559.

\bibitem{KoosisLI} P. Koosis, \emph{The Logarithmic Integral I}, Cambridge Studies in Advanced Mathematics, vol.~12, Cambridge Univ. Press, Cambridge, 1988.

\bibitem{IwaniecKowalski} H. Iwaniec and E. Kowalski, \emph{Analytic Number Theory}, Amer. Math. Soc. Colloquium Publications, vol.~53, Amer. Math. Soc., Providence, RI, 2004.

\bibitem{SteinSingInt} E.~M. Stein, \emph{Singular Integrals and Differentiability Properties of Functions}, Princeton Mathematical Series, no.~30, Princeton Univ. Press, Princeton, NJ, 1970.

\bibitem{Grafakos} L. Grafakos, \emph{Classical Fourier Analysis}, 3rd ed., Graduate Texts in Mathematics, vol.~249, Springer, New York, 2014.

\bibitem{NISTDLMF} F.~W. J. Olver, D.~W. Lozier, R.~F. Boisvert, and C.~W. Clark (eds.), \emph{NIST Digital Library of Mathematical Functions}, National Institute of Standards and Technology, Washington, DC, 2010. Available at \url{https://dlmf.nist.gov/}.

\bibitem{Edwards} H.~M. Edwards, \emph{Riemann's Zeta Function}, Academic Press, New York, 1974, reprint, Dover Publications, Mineola, NY, 2001.

\bibitem{RosenblumRovnyak} M. Rosenblum and J. Rovnyak, \emph{Hardy Classes and Operator Theory}, Dover Publications, Mineola, NY, 1997. (Ch. 2: outer/inner and boundary transforms.)


  
\bibitem{RudinRCA} W. Rudin, \emph{Real and Complex Analysis}, 3rd ed., McGraw-Hill, New York, 1987. (Removable singularities, Poisson integrals.)

\bibitem{Garnett} J.~B. Garnett, \emph{Bounded Analytic Functions}, Graduate Texts in Mathematics, vol.~236, revised 1st ed., Springer, New York, 2007. (Thm. VI.1.1: Carleson embedding, Thm. II.4.2: boundary uniqueness, Ch. IV: H$^1$-BMO.)


\bibitem{FeffermanStein1972} C. Fefferman and E.~M. Stein, $H^p$ spaces of several variables, \emph{Acta Math.} 129 (1972), 137-193. (Fefferman-Stein theory, area/square functions and $H^1$-BMO.)

\bibitem{SteinHA} E.~M. Stein, \emph{Harmonic Analysis: Real-Variable Methods, Orthogonality, and Oscillatory Integrals}, Princeton University Press, Princeton, NJ, 1993. (Poisson/Hilbert transform on $\mathbb R$, square functions.)

\bibitem{SarasonSubHardy} D. Sarason, \emph{Sub-Hardy Hilbert Spaces in the Unit Disk}, John Wiley \& Sons, Inc., New York, 1994. (Schur/Cayley background.)

\bibitem{SimonTrace} B. Simon, \emph{Trace Ideals and Their Applications}, 2nd ed., Mathematical Surveys and Monographs, vol.~120, American Mathematical Society, Providence, RI, 2005. (Hilbert-Schmidt determinants and continuity.)


\bibitem{AmbrosioFuscoPallara} L. Ambrosio, N. Fusco, and D. Pallara, \emph{Functions of Bounded Variation and Free Discontinuity Problems}, Oxford Mathematical Monographs, Oxford University Press, Oxford, 2000. (BV compactness/Helly selection.)

\bibitem{Dusart2010} P. Dusart, Estimates of some functions over primes without Riemann Hypothesis, arXiv:1002.0442, 2010. (Explicit prime-sum bounds, alternative to Rosser-Schoenfeld.)

\bibitem{RosserSchoenfeld1962} J.~B. Rosser and L. Schoenfeld, Approximate formulas for some functions of prime numbers, \emph{Illinois J. Math.} 6 (1962), no.~1, 64-94. (Explicit bounds, e.g. $\pi(t)\le 1.25506\,t/\log t$ for $t\ge 17$.)

\bibitem{RosserSchoenfeld1975} J.~B. Rosser and L. Schoenfeld, Sharper bounds for the Chebyshev functions $\theta(x)$ and $\psi(x)$, \emph{Math. Comp.} 29 (1975), no.~129, 243-269. (Refined explicit prime bounds.)


\end{thebibliography}







\end{document}
% touched to refresh timestamp


