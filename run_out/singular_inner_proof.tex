\documentclass[11pt]{amsart}
\usepackage[margin=1in]{geometry}
\usepackage{amsmath,amssymb,amsthm}
\usepackage{booktabs}
\usepackage[T1]{fontenc}
\usepackage{lmodern}
\usepackage{microtype}

\newtheorem{theorem}{Theorem}
\newtheorem{lemma}[theorem]{Lemma}
\newtheorem{proposition}[theorem]{Proposition}
\theoremstyle{remark}
\newtheorem{remark}[theorem]{Remark}

\newcommand{\C}{\mathbb{C}}
\newcommand{\R}{\mathbb{R}}
\DeclareMathOperator{\dettwo}{det_2}

\title{Proof that the singular inner factor is trivial ($S\equiv 1$)}
\author{Technical companion to paper1\_zerozeta-v19}
\date{February 7, 2026}

\begin{document}
\maketitle

\section{The problem}

The inner reciprocal $\mathcal I:=B^2/\mathcal J_{\rm out}$ is an inner function
on $\Omega=\{\Re s>1/2\}$ (holomorphic, $|\mathcal I|\le 1$, $|\mathcal I^*|=1$ a.e.).
Its canonical factorization is
\[
  \mathcal I\ =\ e^{i\theta}\,B_{\mathcal I}\,S,
\]
where $B_{\mathcal I}$ is the Blaschke product over $\zeta$-zeros in $\Omega$
and $S$ is the singular inner factor (Poisson integral of a positive singular
measure $\nu_S$ on $\partial\Omega$).

\textbf{The obstacle:}
If $S\not\equiv 1$, then $-\log|S|=P_\sigma[\nu_S]$ contributes a Poisson spike
at height $\sigma\sim L$ of size $\nu_S(\text{near})/(\pi\sigma)
=O(\log^2\!\langle t_0\rangle/c_0)$
to the boundary bound $M=\sup_{\partial D}\widetilde W$.
This extra $\log$ factor in $M$ breaks the height-dependent $c=c_0/\log$
cancellation trick (the ratio Upper/Lower grows as $\log\langle\gamma_0\rangle$
instead of being constant).

\textbf{The resolution:}
We prove $S\equiv 1$ using the polynomial growth of $\zeta$ and the standard
$N^+$ (Smirnov class) criterion.

\section{The proof}

\begin{proposition}
The singular inner factor of $\mathcal I=B\,\mathcal O_\zeta\,\zeta/\dettwo(I-A)$
on $\Omega$ is trivial: $S\equiv 1$.
\end{proposition}

\begin{proof}
The singular inner of $\mathcal I$ equals the singular inner of $(s-1)\zeta(s)/s$
on $\Omega$, because:
\begin{itemize}
\item $\dettwo(I-A)$ is holomorphic and nonvanishing on $\Omega$,
  with $\log|\dettwo|$ having a BMO boundary trace (from the arithmetic Carleson
  energy bound). This implies $\dettwo\in N^+(\Omega)$
  (Fefferman--Stein/Garnett characterization), so its inner factor is trivial.
\item $\mathcal O_\zeta$ is outer by construction (trivial inner factor).
\item The rational factor $B(s)=(s-1)/s$ is in $H^\infty(\Omega)$
  (trivial inner factor on $\Omega$, since $|B|=1$ on $\partial\Omega$
  and $|B|\le C$ on $\Omega$).
\end{itemize}
Hence the singular inner of the product $\mathcal I=B\,\mathcal O_\zeta\,\zeta/\dettwo$
equals the singular inner of $\zeta$ (equivalently, of $f(s):=(s-1)\zeta(s)/s$)
on $\Omega$.

\medskip
Now we show $f\in N^+(\Omega)$, which implies $\nu_\zeta=0$.

\emph{Step 1: $f\in N(\Omega)$.}
The function $f$ is holomorphic on $\Omega$ (the pole of $\zeta$ at $s=1$
is canceled by $s-1$; $s=0\notin\Omega$).
By the convexity bound for $\zeta$:
\[
  |f(\tfrac12+\sigma+it)|\ \le\ C(1+|t|)^A
  \qquad\text{for all }\sigma\in(0,1],\;t\in\R,
\]
where $A$ and $C$ are absolute constants.
Hence $\log^+\!|f|=O(\log(2+|s|))$, and the Poisson integral
$\int\log^+\!|f|/(1+t^2)\,dt<\infty$ provides a harmonic majorant.
So $f\in N(\Omega)$.

\emph{Step 2: $f\in N^+(\Omega)$ by the uniform integrability criterion.}
By Garnett~\cite[Ch.~II, Thm.~3.2]{GarnettBAF}:
$f\in N(\Omega)$ belongs to $N^+(\Omega)$ if and only if the family
$\{\log^+\!|f(\sigma,\cdot)|\}_{\sigma>0}$ is uniformly integrable
in $L^1(\R,dt/(1+t^2))$.

Since $\log^+\!|f(\tfrac12+\sigma+it)|\le A\log(2+|t|)$ \textbf{uniformly}
for all $\sigma\in(0,1]$, and $A\log(2+|t|)/(1+t^2)\in L^1(\R)$,
the family is dominated by a fixed $L^1$ function, hence uniformly integrable.
Therefore $f\in N^+(\Omega)$.

\emph{Step 3: Conclusion.}
$f\in N^+(\Omega)$ means the inner factor of $f$ has no singular part:
$\nu_\zeta=0$.
Therefore $S\equiv 1$.
\end{proof}

\section{Why this is unconditional}

The proof uses \textbf{only}:
\begin{enumerate}
\item The polynomial growth of $\zeta$ on $\Omega$
  ($|\zeta(\tfrac12+\sigma+it)|\le C(1+|t|)^A$ for $\sigma\in(0,1]$),
  which follows from the \textbf{convexity bound} (a classical,
  unconditional result; see Titchmarsh, Ch.~V).
\item The $N^+$ criterion via uniform integrability of $\log^+$
  (Garnett, Ch.~II, Thm.~3.2).
\item The triviality of the singular inner factors of $\dettwo$ and $\mathcal O_\zeta$
  (from the Carleson energy bound and the outer construction, respectively).
\end{enumerate}

\textbf{No zero-free hypothesis is used.}
The convexity bound is a consequence of the Phragm\'en--Lindel\"of principle
applied to $\zeta$ on vertical strips, and does not assume anything about
the location of $\zeta$-zeros.

\section{Impact on the proof}

With $S\equiv 1$:
\begin{itemize}
\item The neutralized potential $\widetilde W=-\log|B_{\rm far}|$ has
  \textbf{no singular inner contribution}.
\item The boundary bound $M=\sup_{\partial D}\widetilde W\le C_*\log\langle t_0\rangle$
  comes \textbf{entirely from the Blaschke tail} (Poisson-averaged Green kernels).
\item The constant $C_*$ depends only on the apertures and the RvM density ---
  \textbf{not on $c$}.
\item The contradiction in Theorem~1 closes with $c=c_0/\log\langle\gamma_0\rangle$:
  ratio $A\sqrt{c_0}/11=11/(\sqrt{2}\cdot 11)<1$. Unconditional.
\end{itemize}

\end{document}
