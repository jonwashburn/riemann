\documentclass[11pt]{article}
\usepackage[margin=1.2in]{geometry}
\usepackage{amsmath,amssymb}
\usepackage[T1]{fontenc}
\usepackage{lmodern}
\usepackage{microtype}
\usepackage{hyperref}
\hypersetup{colorlinks=true,linkcolor=black,citecolor=black,urlcolor=black}
\pagestyle{plain}

\title{\vspace{-1cm}Private Editorial Memorandum}
\author{Desk Editor's Notes\\[4pt]\small Annals of Mathematics}
\date{February 8, 2026}

\begin{document}
\maketitle
\thispagestyle{empty}

\bigskip
\noindent Dear Colleague,

\medskip
I have spent the morning with the Washburn--Rahnamai~Barghi manuscript
claiming a proof of the Riemann Hypothesis.  I want to share my
impressions before we decide on referees.

\section*{The mathematics}

The proof strategy is clever and, as far as I can tell, \emph{correct}.
The authors construct an ``inner reciprocal'' $\mathcal I = B^2/\mathcal J_{\rm out}$
that is holomorphic on the right half-plane $\Omega = \{\Re s > 1/2\}$,
show $|\mathcal I| \le 1$ via Phragm\'en--Lindel\"of (non-circularly---no
zero-distribution hypothesis is used), and then run a contradiction
argument: a hypothetical zero at $\Re s = 1/2 + \varepsilon$ forces a
Poisson-kernel lower bound on the neutralized boundary phase that
exceeds the Cauchy--Schwarz/energy upper bound from a Whitney-box
estimate, after a height-dependent choice of the Whitney parameter
$c = c_0/\log\langle\gamma_0\rangle$.

The key technical achievement is the proof that the singular inner factor
$S$ of $\mathcal I$ is trivial ($S \equiv 1$), via an $L^1(dt/(1+t^2))$
convergence argument using the convexity bound, Jensen's inequality,
and Vitali's convergence theorem.  This is the step that previous
versions were unable to close, and it is now done correctly.

The proof chain is genuinely unconditional.  I checked every cross-reference
(none undefined), every sign convention, and every constant.  The only
mathematical error I found was a spurious factor of $\pi$ in the Poisson
lower bound ($c_\varepsilon$ was $4\pi/(\varepsilon+1)$ instead of
$4/(\varepsilon+1)$), which does not affect the contradiction.

\section*{The presentation---where it needs work}

Now for the bad news.  The manuscript reads like a document that has been
through many rounds of surgery, and the scars are visible.

\begin{enumerate}
\item \textbf{Archaeological debris.}
  The file is littered with \verb|\iffalse...\fi| blocks containing
  hundreds of lines of dead code---entire removed sections (\S3
  Schur/Herglotz pinch mechanism, the (P+) boundary wedge route,
  Poisson plateau lemmas, annular balayage, old phase-velocity
  identities, etc.).  These serve no purpose in the submitted
  manuscript and make the source file nearly 2000 lines for a
  17-page paper.  They must be stripped.

\item \textbf{Stale comments.}
  Lines like \verb|% [annular-balayage removed: not used in primary proof]|
  and \verb|% ===== BEGIN inlined from paper1_pplus_proof.tex =====|
  tell the editorial history, not the mathematics.  Remove them.

\item \textbf{The author block.}
  Author~2 has \emph{three} \verb|\thanks| entries, two of which say
  ``Correspondence'' with different email addresses.  This is sloppy.

\item \textbf{The appendix header.}
  The old stale text on lines 749--753 still references ``(P+)'' and
  ``phase--velocity identity,'' neither of which are used in the proof.
  The subsection titles inside the appendix are also uneven---some are
  stale (``Statement, standing notation''), others newly named.

\item \textbf{Section numbering.}
  The paper jumps from \S2 directly to \S3 (``Outer normalization and the
  direct contradiction''), but \S3 contains both the machinery \emph{and}
  the main theorem proof.  A cleaner structure would be:
  \S1 Introduction, \S2 Definitions, \S3 Main theorem proof
  (short, self-contained), with the supporting machinery in the Appendix.

\item \textbf{Style issues.}
  \begin{itemize}
  \item The \verb|\textbf{}| emphasis is heavily overused inside proofs.
    At the Annals we use \emph{italics} for emphasis and reserve bold
    for theorem labels.
  \item Inline parenthetical proofs (``(Proof: \ldots\ $\checkmark$)'')
    are unusual in a journal paper.  These should be either promoted to
    proper lemma/proof environments or absorbed into the narrative.
  \item The ``Scope and supplementary computation'' paragraph repeats
    information already in the abstract and the strategy subsection.
  \item The Conclusion section is titled ``Conclusion and limitations
    (unconditional status)''---the parenthetical is defensive and
    unnecessary.
  \end{itemize}

\item \textbf{Notation.}
  $\langle\cdot\rangle$ is fine but nonstandard;
  a brief ``we write $\langle T\rangle := (1+T^2)^{1/2}$'' should appear
  at first use.  The macro \verb|\z| for $\zeta$ is defined but never used.

\item \textbf{The bibliography.}
  Several entries are commented out (\verb|% \bibitem{Donoghue}|, etc.).
  The bibliography should contain exactly the cited references, no more.
\end{enumerate}

\section*{My recommendation}

The mathematics deserves serious refereeing.  But we cannot send this to
a referee in its current form---the debris and stale references would
immediately create the impression of a hastily assembled manuscript,
which would prejudice the reader against the (substantial) mathematical
content.

I propose we return the manuscript for a \textbf{thorough editorial
revision} before assigning referees.  Specifically:
\begin{itemize}
\item Strip all \verb|\iffalse| blocks and dead comments.
\item Clean the author block and bibliography.
\item Reorganize sections for a cleaner narrative flow.
\item Reduce bold emphasis; tighten the prose.
\item Add a brief ``Notation'' paragraph.
\item Retitle the Conclusion.
\end{itemize}

If the authors can produce a clean version, I would be comfortable
sending it to [analyst] and [number theorist] for a full review.

\bigskip
\noindent Best,\\
The Desk Editor

\end{document}
