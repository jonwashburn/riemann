\documentclass[11pt]{amsart}
\usepackage[margin=1in]{geometry}
\usepackage{amsmath,amssymb,amsthm}
\usepackage{booktabs}
\usepackage[T1]{fontenc}
\usepackage{lmodern}
\usepackage{microtype}

\newcommand{\C}{\mathbb{C}}
\DeclareMathOperator{\dettwo}{det_2}

\title{Final fixes and honest status\\of the zero-free region proof}
\author{Technical companion to paper1\_zerozeta-v19}
\date{February 7, 2026}

\begin{document}
\maketitle

\section{Fixes implemented in this round}

\subsection{Fix A: ``holomorphic and nonvanishing'' for harmonicity}

\textbf{Complaint:} The proof claimed $\log|\mathcal J_{\rm neut}|$ is harmonic
because $\mathcal J_{\rm neut}$ is holomorphic, but holomorphic alone doesn't imply
harmonicity of the log-modulus (zeros reintroduce singularities).

\textbf{Fix:} The theorem proof now explicitly states that $\mathcal J_{\rm neut}$
is \emph{holomorphic and nonvanishing} on~$D$, with a sentence explaining why:
the poles of $\mathcal J_{\rm out}$ are exactly canceled (with multiplicity) by the
zeros of $B_{\rm box}$, and $\mathcal J_{\rm out}$ has no zeros in~$D$ (the only zero
at $s=1$ lies outside~$D$ for large~$t_0$).

\subsection{Fix B: Proposition restated for neutralized energy}

\textbf{Complaint:} The proposition's headline bound was for
$E(I)=\iint|\nabla\log|\mathcal J_{\rm out}||^2\sigma$, which is \emph{infinite}
if there are poles in the box.

\textbf{Fix:} The proposition now defines and bounds $E_{\rm neut}(I)$,
the energy of the \emph{neutralized harmonic function}
$\log|\mathcal J_{\rm neut}|=2\log|B|+\widetilde W$.
The infinite-energy near-Blaschke singularities are explicitly excluded
from the bound.

\subsection{Fix C: Constant notation}

\textbf{Complaint:} The proposition stated $C(\alpha',c)$ but the theorem proof
treated it as $C(\alpha')$ independent of~$c$.

\textbf{Fix:} The proposition now states $C(\alpha')$ throughout, with an explicit
note that independence from~$c$ comes from the $L$-cancellation in the Poisson integral.

\subsection{Fix D: $B_{\rm box}$ definition and multiplicities}

\textbf{Complaint:} $B_{\rm box}$ was described as ``zeros with ordinate in~$D$''
without specifying both coordinates or multiplicities.

\textbf{Fix:} The theorem proof now says ``the zeros of~$\zeta$ with ordinate in~$D$
(i.e.\ $|\gamma-\gamma_0|\le\alpha''L$ and $\beta>1/2$)'' and states that each
Blaschke factor cancels the corresponding pole ``with multiplicity.''

\section{The remaining open step: the singular inner factor}

\subsection{What was claimed before}

Previous versions claimed the singular inner factor~$S$ of the inner reciprocal
$\mathcal I=B^2/\mathcal J_{\rm out}$ could be controlled by the pointwise bound
$-\log|S|\le W\le N\log(2+|t|)+C$.

\subsection{Why this is insufficient}

The bound $W(s)\le N\log(2+|t|)+C$ at fixed height $\sigma>0$ relies on a
polynomial lower bound for $|\mathcal I(s)|$, which in turn requires a lower bound
on $|\zeta(1/2+\sigma+it)|$.
For \emph{fixed}~$\sigma$, such bounds exist (Hadamard product + zero repulsion).
But on the Whitney schedule $\sigma=\alpha''L=\alpha''c_0/\log^2\!\langle t_0\rangle\to 0$,
the exponent~$N(\sigma)$ degrades (grows as~$1/\sigma$), and the resulting bound on~$M$
picks up an extra factor of $\log\langle t_0\rangle$.

Concretely: the singular measure $\nu_S$ has uniformly bounded mass
$\nu_S([t_0-1,t_0+1])\le\nu_*<\infty$ (from the bounded values of $W$ at $\Re s=3/2$).
The Poisson integral at height $\sigma=\alpha''L$ is
$P_\sigma[\nu_{S,{\rm near}}]\le\nu_*/(\pi\alpha''L)=O(\log\langle t_0\rangle/c)$.
With $c=c_0/\log$: this is $O(\log^2/c_0)$, adding one uncancelable~$\log$ to~$M$.

The energy then grows as $M^2|I|=O(\log^4/c_0^2\cdot c_0/\log^2)=O(\log^2/c_0)$,
and $\sqrt{E}\cdot L=O(\sqrt{c_0}/\log)$ while the lower bound is $O(c_0/\log^2)$.
The ratio Upper/Lower $=O(\log/\sqrt{c_0})\to\infty$.

\subsection{What would close the proof}

The proof is complete \textbf{if} $S\equiv 1$ (equivalently $\nu_S=0$).
Under this condition:
\begin{itemize}
\item $M=O(\log\langle t_0\rangle)$ with constant independent of~$c$.
\item $E_{\rm neut}=O(c_0)$ (height-independent after the $c=c_0/\log$ trick).
\item Ratio Upper/Lower $=A\sqrt{c_0}/11<1$ for $c_0<(11/A)^2/2$. Contradiction. \checkmark
\end{itemize}

Three potential routes to establishing $S\equiv 1$:
\begin{enumerate}
\item Show that the boundary log-modulus limits of each factor of~$\mathcal I$
  converge in $L^1(\mathbb R,(1+t^2)^{-1}dt)$ (not just $L^1_{\rm loc}$),
  and that the sum converges to~$0$ in that weighted sense.
  This would give $\int W(\sigma,\cdot)/(1+t^2)\,dt\to 0$ as $\sigma\to 0$,
  which is the standard criterion for $S\equiv 1$.
\item Show that $(s-1)\zeta(s)/s$ restricted to~$\Omega$ belongs to the
  \emph{Cartwright class} (entire of exponential type with $\log^+$ in
  $L^1(\mathbb R,(1+t^2)^{-1}dt)$), which implies its inner factor is a pure
  Blaschke product with no singular part (Krein/Koosis theory).
\item Bound the singular measure $\nu_S$ directly from the explicit formula
  for $\mathcal I=B\mathcal O_\zeta\zeta/\dettwo$, using the Carleson energy
  of $\dettwo$ and the boundary trace properties of~$\zeta$.
\end{enumerate}

\section{Current state of the proof}

\begin{center}
\renewcommand{\arraystretch}{1.3}
\begin{tabular}{lcc}
\toprule
\textbf{Component} & \textbf{Status} & \textbf{Unconditional?}\\
\midrule
Inner reciprocal $\mathcal I$, $|\mathcal I|\le 1$ & Proved & Yes\\
Neutralized ratio $\mathcal J_{\rm neut}$ & Proved & Yes\\
Phase-velocity lower bound & Proved & Yes\\
CR--Green on harmonic $\log|\mathcal J_{\rm neut}|$ & Proved & Yes\\
Blaschke-tail energy bound ($S\equiv 1$ case) & Proved & Yes\\
$c=c_0/\log$ cancellation algebra & Proved & Yes\\
Contradiction ($S\equiv 1$ case) & Proved & Yes\\
\midrule
Singular inner factor $S\equiv 1$ & \textbf{Open} & ---\\
\bottomrule
\end{tabular}
\end{center}

\medskip
The paper's Theorem~1 is proved \textbf{conditional on $S\equiv 1$}.
Establishing $S\equiv 1$ would make the proof fully unconditional.

\end{document}
