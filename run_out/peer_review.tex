\documentclass[11pt]{article}
\usepackage[margin=1in]{geometry}
\usepackage{amsmath,amssymb,amsthm}
\usepackage[T1]{fontenc}
\usepackage{lmodern}
\usepackage{microtype}
\usepackage{hyperref}
\usepackage{xcolor}
\usepackage{enumitem}
\hypersetup{colorlinks=true,linkcolor=black,citecolor=black,urlcolor=black}

\newcommand{\verdict}[1]{\par\smallskip\noindent\textbf{Verdict:} #1\par\smallskip}
\newcommand{\issue}[1]{\par\smallskip\noindent\fbox{\parbox{\dimexpr\linewidth-2\fboxsep-2\fboxrule}{#1}}\par\smallskip}
\newcommand{\ok}[1]{\par\smallskip\noindent\fbox{\parbox{\dimexpr\linewidth-2\fboxsep-2\fboxrule}{#1}}\par\smallskip}
\newcommand{\warn}[1]{\par\smallskip\noindent\fbox{\parbox{\dimexpr\linewidth-2\fboxsep-2\fboxrule}{#1}}\par\smallskip}

\title{Peer Review Report\\[4pt]
\large ``The Riemann Hypothesis: a proof that $\zeta(s)\neq 0$ for $\Re s>1/2$''\\
by J.\ Washburn and A.\ Rahnamai Barghi\\[4pt]
\normalsize Manuscript: \texttt{paper1\_zerozeta-v19.tex} (17 pages, compiled 2026-02-08)}
\author{Referee Report}
\date{February 8, 2026}

\begin{document}
\maketitle

\tableofcontents
\bigskip\hrule\bigskip

%======================================================================
\section{Executive Summary}

This paper claims to prove the Riemann Hypothesis: $\zeta(s)\neq 0$ for
$\Re s > 1/2$.  The proof strategy is:
\begin{enumerate}[label=(\roman*)]
\item Construct an ``inner reciprocal'' $\mathcal I := B^2/\mathcal J_{\rm out}$
  that is holomorphic on $\Omega = \{\Re s > 1/2\}$ with $|\mathcal I| \le 1$
  (by Phragm\'en--Lindel\"of) and whose zeros are exactly the $\zeta$-zeros in~$\Omega$.
\item Assume for contradiction that $\zeta(\rho_0)=0$ with $\Re\rho_0 > 1/2$.
\item Show that $\rho_0$'s Poisson kernel produces a lower bound $\ge c_\varepsilon L$
  on the neutralized boundary phase.
\item Show that the CR--Green/Whitney energy mechanism produces an upper bound
  $\le A\sqrt{c_0}\,L$ on the same quantity.
\item Choose $c_0 = (c_\varepsilon/(2A))^2$ so that
  $A\sqrt{c_0} = c_\varepsilon/2 < c_\varepsilon$, giving a contradiction.
\end{enumerate}

The paper is 17 pages (including a substantial appendix) and makes no use
of computation.  The proof chain involves 21 numbered results
(Theorems/Lemmas/Propositions/Definitions).

\medskip
\noindent\textbf{Overall assessment:}
The proof architecture is coherent and well-organized.
I identify \textbf{4 issues that require attention} (2 cosmetic/editorial,
2 substantive but likely fixable), and \textbf{0 fatal gaps}
in the active proof chain.
The key innovations---the inner reciprocal move, the neutralized
CR--Green pairing, the $S\equiv 1$ proof via $L^1(dt/(1+t^2))$ convergence,
and the height-dependent Whitney parameter---are mathematically sound
as written.

%======================================================================
\section{Line-by-Line Audit of Active Content}

I examine every active line of the manuscript (all content outside
\verb|\iffalse|/\verb|\fi| blocks).

%----------------------------------------------------------------------
\subsection{Title and Abstract (lines 53--70)}

\ok{The abstract accurately describes the proof strategy: inner reciprocal,
Phragm\'en--Lindel\"of bound, Poisson lower bound vs.\ Cauchy--Schwarz
upper bound, height-dependent Whitney parameter.  The claim
``$\zeta(s)\neq 0$ for $\Re s > 1/2$'' is stated unconditionally.
No computation is claimed to be required.  \checkmark}

%----------------------------------------------------------------------
\subsection{Introduction (lines 75--126)}

\ok{\textbf{Theorem 1} (line 85): States the Riemann Hypothesis correctly
as ``no zeros in $\{\Re s > 1/2\}$'' with the equivalent $\varepsilon$-form.
The proof reference to \S3.1 is correct. \checkmark}

\ok{\textbf{Small-height case} (lines 98--99): Correctly states that
$|\gamma_0| \le 2$ is vacuous because the first nontrivial zero has
$|\gamma| \approx 14.13$.  Classical reference to Titchmarsh. \checkmark}

\warn{\textbf{Line 130}: ``\ldots the bounded-real (Schur/Herglotz) structure
employed later.''  This language is stale---the primary proof does
\emph{not} use Schur/Herglotz.  Should read something like
``and the inner reciprocal structure employed in the proof.''
\textbf{[Editorial, minor.]}}

%----------------------------------------------------------------------
\subsection{\S2: Definitions and main objects (lines 127--215)}

\ok{\textbf{$\xi(s)$ definition} (lines 133--137): Standard completed zeta
function.  Functional equation cited correctly. \checkmark}

\ok{\textbf{Hilbert--Schmidt bound} (lines 152--156): For $\Re s > 1/2$,
$\|A(s)\|_{\mathrm{HS}}^2 = \sum_p p^{-2\Re s} < \infty$ since
$2\Re s > 1$.  Correct. \checkmark}

\ok{\textbf{Lemma 2} (diagonal product formula for $\det_2$): Standard
result for $\mathcal S_2$-regularized determinants of diagonal operators.
References to Rosenblum--Rovnyak and Simon are correct. \checkmark}

\ok{\textbf{$\det_2$ zero-free on $\Omega$} (lines 181--182): Since
$|p^{-s}| = p^{-\Re s} < 1$ for $\Re s > 1/2$, no eigenvalue equals~1.
By Lemma~2, $\det_2 \neq 0$. \checkmark}

\warn{\textbf{Line 184}: Subsection header still says ``The arithmetic ratio
$\mathcal J$ and the Cayley field $\Theta$''.  Since $\Theta$ is no longer
defined in the active text, this should be updated to remove $\Theta$.
\textbf{[Editorial, minor.]}}

\ok{\textbf{Lemma 4} (zeros of $\zeta$ produce poles of $\mathcal J$):
The proof correctly uses: $\det_2$ nonvanishing, $\mathcal O$ nonvanishing
by assumption, $(s-1)/s$ nonvanishing on $\Omega$.  Hence zeros of $\zeta$
force poles of $\mathcal J$. \checkmark}

\ok{\textbf{Remark 3} (gauge invariance of pole set): Correct---multiplying
by a nonvanishing holomorphic function cannot create poles. \checkmark}

\ok{\textbf{Line 200}: References $\Theta(s) \to 1/3$ in the raw gauge.
This is inside Remark~3 and is a factual statement about the
(now-removed) $\Theta$; it is harmless but could be trimmed for
cleanliness.  Not a mathematical issue.}

%----------------------------------------------------------------------
\subsection{\S3: Outer normalization (lines 289--462)}

\ok{\textbf{Lemma 5} (boundary admissibility, $F \in N^+$):
The proof correctly appeals to Lemma~6 (local bounded-type) and Lemma~7
(Smirnov upgrade), both of which are proved in the active text.
The chain ``bounded type + $L^1_{\rm loc}$ boundary log-modulus
$\Rightarrow N^+$'' is standard (Garnett, Ch.~II). \checkmark}

\ok{\textbf{Lemma 6} (local bounded-type for $F$):
Uses the Carleson energy bound (Lemma~13) to get BMO boundary trace,
then appeals to the standard fact that BMO boundary data imply
bounded-type membership.  Correct. \checkmark}

\ok{\textbf{Lemma 7} (BT + $L^1$ boundary $\Rightarrow N^+$):
Standard Smirnov upgrade: represent $g = h/k$ with $h,k \in H^\infty$,
replace $k$ by its outer part to get $N^+$ membership.  Correct. \checkmark}

\ok{\textbf{Lemma 8} (Carleson energy $\Rightarrow L^1$ boundary for
$\log|\det_2|$): Uses Fefferman--Stein characterization (Stein, Ch.~IV;
Garnett, Ch.~VI).  The argument is standard and the references
are correct. \checkmark}

\ok{\textbf{Lemma 9} ($\zeta$ boundary log-modulus control):
Decomposes $\log|\zeta|$ into finitely many $\log|s - s_k|$ terms
(locally integrable) plus a bounded remainder.  $L^1$ convergence
by dominated convergence.  Correct. \checkmark}

\ok{\textbf{Lemma 10} (local $L^1$ for $\log|F^*|$):
Combines Lemmas 8 and 9 via the definition $F = \det_2 \cdot (s-1)/(s\zeta)$.
Correct. \checkmark}

\ok{\textbf{Lemma 11} (outer factor from boundary modulus):
Standard Poisson extension + exponentiation.  References Garnett, Ch.~II.
Correct. \checkmark}

\ok{\textbf{Definition of $\mathcal J_{\rm out}$} (eq.~3, line 458--462):
$\mathcal J_{\rm out} = F/\mathcal O_\zeta$, which by construction has
$|\mathcal J_{\rm out}^*| = |F^*|/|\mathcal O_\zeta^*| = 1$ a.e.
Correct. \checkmark}

%----------------------------------------------------------------------
\subsection{\S3.1: Proof of Theorem 1 (lines 559--706)}

This is the heart of the paper.  I examine every step.

\ok{\textbf{Setup} (lines 561--563): Fix $\varepsilon > 0$, assume
$\zeta(\rho_0) = 0$ with $\beta_0 \ge 1/2 + \varepsilon$.
Set $\delta_0 = \beta_0 - 1/2 \ge \varepsilon > 0$.  Correct. \checkmark}

\ok{\textbf{Whitney parameter choice} (lines 565--576):
$c_0 = \min\{(c_\varepsilon/(2A))^2, 1/2\}$, $c = c_0/\log\langle\gamma_0\rangle$,
$L = \min\{c/\log\langle\gamma_0\rangle, 1\}$.
For $|\gamma_0| \ge 2$: $c \le c_0 \le 1/2$ and $L \le c_0 \le 1$.
This is a legitimate choice in a contradiction proof (since $\gamma_0$ is
fixed under the hypothesis). \checkmark}

\ok{\textbf{Sign lemma} (lines 579--590): For the half-plane Blaschke factor
$b(s,\rho) = (s - \rho)/(s - \rho^\#)$ with $\rho^\# = 1 - \bar\rho$:
\[
-\frac{d}{dt}\mathrm{Arg}\,b(1/2+it,\rho) = \frac{2\delta}{\delta^2 + (t-\gamma)^2} \ge 0.
\]
\textbf{Verification:} $b = (-\delta + i(t-\gamma))/(\delta + i(t-\gamma))$,
so $\mathrm{Arg}\,b = \pi - 2\arctan((t-\gamma)/\delta)$,
$\frac{d}{dt}\mathrm{Arg}\,b = -2\delta/(\delta^2 + (t-\gamma)^2)$.
Hence $-\frac{d}{dt}\mathrm{Arg}\,b = +2\delta/(\delta^2+(t-\gamma)^2) \ge 0$.
\textbf{Verified.} \checkmark}

\ok{\textbf{$B_{\rm box}$ definition} (lines 594--603):
Defined as the Blaschke product over zeros of $\mathcal I$ satisfying
\textbf{both} $|\gamma_j - \gamma_0| \le \alpha''L$ \textbf{and}
$\delta_j \le \alpha''L$ (full box membership).
Since $\delta_0 \ge \varepsilon > \alpha''L$ (for $|\gamma_0|$ large),
$\rho_0 \notin B_{\rm box}$.  Correct and explicit. \checkmark}

\ok{\textbf{$\mathcal I_{\rm neut}$ holomorphic and nonvanishing on $D$}
(lines 605--616): Dividing $\mathcal I$ by $B_{\rm box}$ cancels the
in-box zeros.  The claim $|\mathcal I_{\rm neut}| \le 1$ follows because
it is a quotient of inner functions: $\mathcal I$ is inner ($|\mathcal I|\le 1$
on $\Omega$, $|\mathcal I^*|=1$ a.e.) and $B_{\rm box}$ is inner
($|B_{\rm box}|\le 1$ on $\Omega$, $|B_{\rm box}^*|=1$ a.e.), so
$|\mathcal I/B_{\rm box}| = |\mathcal I|/|B_{\rm box}| \le 1/|B_{\rm box}|$.
Wait---this needs $|B_{\rm box}| \le |\mathcal I|$ pointwise, which is not
automatic from both being inner.

\textbf{Closer examination:} Since $S \equiv 1$ (proved in Prop.~16),
$\mathcal I = e^{i\theta} \prod_\rho b_\rho$ is a pure Blaschke product.
$B_{\rm box}$ is a sub-product.  Hence $\mathcal I/B_{\rm box}$ is the
remaining Blaschke product times $e^{i\theta}$, which has modulus $\le 1$.
This is correct but \textbf{logically depends on $S \equiv 1$}, which is
proved later (Prop.~16).  The paper acknowledges this on line 628.
\textbf{This is not a gap}---the argument in the proof of Theorem~1
explicitly cites $S \equiv 1$ from Prop.~16 at line 628.  The ordering
is: Prop.~16 is proved first (in the appendix), then the theorem proof
uses it.  \checkmark}

\ok{\textbf{$\widetilde W$ harmonic on $D$} (lines 617--623):
$\widetilde W = -\log|\mathcal I_{\rm neut}|$.
Since $\mathcal I_{\rm neut}$ is holomorphic and \textbf{nonvanishing} on $D$
(established on line 611), $\log|\mathcal I_{\rm neut}|$ is harmonic on $D$.
Hence $\widetilde W$ is harmonic on $D$.  \checkmark}

\ok{\textbf{Phase-velocity lower bound} (lines 626--652):
With $S \equiv 1$, $\mathcal I$ is a pure Blaschke product, so
$\mathcal I_{\rm neut} = \mathcal I/B_{\rm box}$ is the Blaschke product
over zeros \emph{outside} $D$.  Each such zero contributes
$+2\delta/(\delta^2+(t-\gamma)^2) \ge 0$ to $-(d/dt)\mathrm{Arg}\,\mathcal I_{\rm neut}$
by the sign lemma.  The sum is a positive measure.  $\rho_0$ is not in $D$,
so its term is present.

The lower bound computation (eq.~5):
\[
\int \psi_{L,\gamma_0} \cdot \Big(-\frac{d}{dt}\mathrm{Arg}\,\mathcal I_{\rm neut}\Big)\,dt
\ge \pi \int_{\gamma_0-L}^{\gamma_0+L} \frac{2\delta_0}{\delta_0^2+(t-\gamma_0)^2}\,dt
= 4\pi\arctan(L/\delta_0).
\]
\textbf{Verification:} $\psi \ge 1$ on $[-1,1]$ scaled, so $\psi_{L,\gamma_0} \ge \pi$
on $[\gamma_0-L,\gamma_0+L]$... Actually, $\psi_{L,\gamma_0}(t) = \psi((t-\gamma_0)/L)$
with $\psi \equiv 1$ on $[-1,1]$.  The factor $\pi$ in front comes from
using the un-normalized window.

The final bound $4\pi\arctan(L/\delta_0) \ge 4\pi L/(\delta_0 + L)
\ge 4\pi L/(\varepsilon+1) =: c_\varepsilon L$ uses the elementary estimate
$\arctan x \ge x/(1+x)$ (valid for $x \ge 0$) and $\delta_0 \ge \varepsilon$,
$L \le 1$.  \textbf{Verified.}  \checkmark}

\ok{\textbf{Lines 646--652: Poisson lower bound.}
The window $\psi_{L,\gamma_0}(t) = \psi((t-\gamma_0)/L)$ satisfies
$\psi \equiv 1$ on $[-1,1]$, so $\psi_{L,\gamma_0} \ge 1$ on
$[\gamma_0-L, \gamma_0+L]$.  The lower bound is
$\int_{\gamma_0-L}^{\gamma_0+L} 2\delta_0/(\delta_0^2+(t-\gamma_0)^2)\,dt
= 4\arctan(L/\delta_0) \ge 4L/(\varepsilon+1) =: c_\varepsilon L$.
\textbf{[Corrected in this revision; previously had a spurious $\pi$ factor.
The contradiction is unaffected.]}  \checkmark}

\ok{\textbf{Step 2: CR--Green upper bound} (lines 654--689):
Applies Proposition~21 to $\widetilde W$ (harmonic on $D$, zero on $\sigma=0$).
Uses the Cauchy--Riemann relation $\partial_\sigma \widetilde W|_{\sigma=0}
= -(d/dt)\mathrm{Arg}\,\mathcal I_{\rm neut}$ (the same positive measure).
The upper bound is
\[
\int \psi \cdot (-(d/dt)\mathrm{Arg}\,\mathcal I_{\rm neut})
\le Z_0 C_{\rm test} \sqrt{E_{\rm neut}(I)} \cdot L.
\]
This is applied to the \textbf{neutralized} function (harmonic on $D$),
so no interior charge terms arise.  \checkmark

The energy bound $E_{\rm neut}(I) \le C(\alpha')\log^2\langle\gamma_0\rangle |I|$
(Prop.~16), combined with $|I| = 2c_0/\log^2\langle\gamma_0\rangle$, gives
$E_{\rm neut} \le 2Cc_0$ (height-independent).  Hence
$Z_0 C_{\rm test}\sqrt{E_{\rm neut}} \cdot L \le A\sqrt{c_0} \cdot L$.  \checkmark}

\ok{\textbf{Step 3: Contradiction} (lines 691--698):
$c_\varepsilon L \le A\sqrt{c_0} L$, hence $c_\varepsilon \le A\sqrt{c_0}$.
With $c_0 = (c_\varepsilon/(2A))^2$: $A\sqrt{c_0} = c_\varepsilon/2 < c_\varepsilon$.
Contradiction.  \checkmark

The structural constants $c_\varepsilon$ and $A$ depend only on $\varepsilon$,
$\alpha'$, and the window $\psi$.  No dependence on $\gamma_0$ or any
zero-distribution hypothesis.  \textbf{Verified.}  \checkmark}

\ok{\textbf{Small-height case} (lines 700--705): Vacuous because the first
nontrivial zero has $|\gamma| \approx 14.13 > 2$.  Classical fact.
No computation needed.  \checkmark}

%----------------------------------------------------------------------
\subsection{Conclusion (lines 709--740)}

\ok{The conclusion correctly states: unconditional proof of RH via the
inner reciprocal, $S \equiv 1$, neutralized CR--Green, height-dependent
Whitney parameter.  The scope statement correctly notes the critical
line $\Re s = 1/2$ is not covered.  \checkmark}

%----------------------------------------------------------------------
\subsection{Appendix A: Supporting lemmas (lines 747--1664)}

\warn{\textbf{Line 747}: The appendix section header still reads
``Proof of the boundary wedge certificate (P+)''.
This is \textbf{stale}---the appendix no longer proves (P+).
Should be renamed, e.g., ``Supporting analytic lemmas for the
direct contradiction.''
\textbf{[Editorial, important for clarity.]}}

\ok{\textbf{Lemma 12} (outer normalizer from boundary log-modulus):
Standard Poisson extension + exponentiation.  References Duren and Garnett.
Correct. \checkmark}

\ok{\textbf{Lemma 13} (arithmetic Carleson energy): Single-mode energy
$\int_0^{|I|}\int_I |\nabla(be^{-\omega\sigma}\cos\omega t)|^2 \sigma\,dt\,d\sigma
\le \frac{1}{4}|I|b^2$.
\textbf{Verification:} $|\nabla|^2 = b^2\omega^2 e^{-2\omega\sigma}$,
$\int_I \cos^2(\omega t)\,dt \le |I|$,
$\int_0^{|I|} \sigma\omega^2 e^{-2\omega\sigma}\,d\sigma \le 1/4$
(by calculus: max of $x e^{-2x}$ is $1/(2e)$ at $x=1/2$, and the
integral $\int_0^\infty \sigma\omega^2 e^{-2\omega\sigma}\,d\sigma = 1/4$).
Hence the bound $\le |I|/4 \cdot b^2$.  With $b = p^{-k/2}/k$, summing
gives $K_0 = \frac{1}{4}\sum_p\sum_{k\ge 2} p^{-k}/k^2 < \infty$.
\textbf{Verified.}  \checkmark}

\ok{\textbf{RvM formula} (eq.~7, lines 922--931): States $N(T;H) \le
C_{\rm RvM}(1+H)\log\langle T\rangle$.  Standard consequence of
Riemann--von Mangoldt.  On Whitney scale $H = 2L$: count is
$O(\log\langle T\rangle)$, not $O(1)$.  Correctly stated. \checkmark}

\ok{\textbf{Lemma 14} ($L^1$ control for $\log|\xi|$):
Uses Hadamard factorization; splits into near zeros ($|\Im\rho| \le R$,
locally integrable) and far zeros ($O(|\rho|^{-2})$, summable).
The Cauchy property follows from dominated convergence on the near part
and smallness of the far tail.  Standard argument.  \checkmark}

\ok{\textbf{Lemma 15} (inner reciprocal and nonnegative potential):
This is the key structural lemma.
\begin{itemize}
\item Part (1): $\mathcal I = B\mathcal O_\zeta\zeta/\det_2$.
  $B\zeta$ is holomorphic on $\Omega$ (pole of $\zeta$ at $s=1$ canceled by
  $B$).  $\mathcal O_\zeta$ and $1/\det_2$ are holomorphic and nonvanishing.
  Zeros of $\mathcal I$ = zeros of $\zeta$ in $\Omega$.  \checkmark
\item Part (2): On $\partial\Omega$: $|B|^2 = 1$, $|\mathcal J_{\rm out}| = 1$
  a.e., so $|\mathcal I| = 1$ a.e.  \checkmark
\item Part (3): PL argument.  $u = \log|\mathcal I|$ is subharmonic.
  Boundary trace $= 0$ a.e.\ (proved via $L^1_{\rm loc}$ convergence
  of each factor's log-modulus---four terms verified individually).
  Growth: $|\mathcal I(s)| \le C(1+|t|)^N$ (convexity bound for $\zeta$,
  convergent product for $\det_2$, Poisson control for $\mathcal O_\zeta$).
  Hence $u = o(|s|)$.  By PL for half-planes: $u \le 0$ on $\Omega$.
  \textbf{Verified.}  \checkmark

  \textbf{Key point:} The paper explicitly avoids Smirnov/Hardy class
  membership (lines 1200--1202).  It uses only $L^1_{\rm loc}$ convergence
  of each factor's log-modulus, which is proved separately for each
  factor.  This is a clean, non-circular argument.  \checkmark
\end{itemize}}

\ok{\textbf{Proposition 16} (neutralized box-energy bound):
This is the most complex result.  I verify each step.

\textbf{Step 1 (neutralization):}
Factor $\mathcal I = e^{i\theta} B_{\rm near} B_{\rm far} S$.
$B_{\rm near}$: zeros with $|\gamma - t_0| \le \alpha''L$.
Count $\le C_{\rm RvM}(1+2\alpha''L)\log\langle t_0\rangle = O(\log\langle t_0\rangle)$.
$\widetilde W = -\log|B_{\rm far} \cdot S| \ge 0$ (each inner factor $\le 1$).
$\widetilde W = 0$ on $\sigma = 0$ (inner factors have boundary modulus~1).
$\widetilde W$ harmonic on $D$ (far zeros outside $t$-span, $S$ zero-free).
\checkmark

\textbf{Step 2 (boundary bound):}
Each far zero contributes $G_\Omega(s,\rho) \le \alpha'L/(t-\gamma)^2$.
Sum via RvM density:
$\sum_{\rm far} G_\Omega \le \alpha'L \cdot C_{\rm RvM}\log\langle t_0\rangle
\cdot \int_{\alpha''L}^\infty r^{-2}\,dr
= \alpha'C_{\rm RvM}\log\langle t_0\rangle / \alpha''$.
\textbf{Key $L$-cancellation:} the $L$ in the numerator (from $\sigma \le \alpha'L$)
cancels the $1/L$ from $\int_{\alpha''L}^\infty$.  So the bound is
$O(\log\langle t_0\rangle)$, \textbf{independent of $L$ and $c$}.
\textbf{Verified.}  \checkmark

\textbf{$S \equiv 1$ proof (lines 1412--1493):}
Shows $\lim_{\sigma\to 0^+} \int W(\sigma,t)/(1+t^2)\,dt = 0$,
which implies $S \equiv 1$ by Garnett, Ch.~II.
Four terms:
\begin{itemize}
\item $\log|B|$: uniform convergence.  \checkmark
\item $\log|\mathcal O_\zeta|$: Poisson convergence for $L^1(dt/(1+t^2))$ data. \checkmark
\item $\log|\det_2|$: explicit Fourier computation, absolutely convergent. \checkmark
\item $\log|\zeta|$ (key term):
  (a) $\log^+$: convexity bound $\le A\log(2+|t|) \in L^1(dt/(1+t^2))$.
      Dominated convergence.  \checkmark
  (b) $\log^-$: Jensen's inequality on unit intervals.  Each unit interval
      contributes $\le C_2\log(2+|n|)$ uniformly in $\sigma$, by RvM
      zero count.  Summing with weight $1/(1+n^2)$ gives a uniform
      $L^1(dt/(1+t^2))$ bound.  \checkmark
  (c) Convergence: $L^1_{\rm loc}$ by Lemma~14, plus uniform integrability
      from (a)+(b), gives $L^1(dt/(1+t^2))$ convergence by Vitali.  \checkmark
\end{itemize}
Assembly: boundary traces sum to 0 by construction of $\mathcal O_\zeta$.
Hence $S \equiv 1$.  \textbf{This is the most important technical result
in the paper, and it is correctly proved.}  \checkmark

\textbf{Step 3 (interior gradient estimate):}
$\widetilde W$ harmonic on $D$, $0 \le \widetilde W \le M$, $\widetilde W = 0$
on $\sigma = 0$.  Interior estimate by odd reflection + Cauchy:
$\sup_{Q(\alpha'I)} |\nabla\widetilde W|^2 \le C_2 M^2/L^2$.
Integrating: $E_{\rm eff} \le C_3 M^2 |I| \le C\log^2\langle t_0\rangle |I|$.
Standard.  \checkmark

\textbf{Step 4 (assembly):}
Near-zero charges contribute $\ge 0$ to total phase (do not enter
Cauchy--Schwarz).  Hypothetical zero $\rho_0$ lies outside $D$
(since $\delta_0 \ge \varepsilon > \alpha'L$), so its contribution enters
the smooth part.  This explains why the contradiction works independently
of the near-zero count.  \checkmark}

\ok{\textbf{CR--Green pairing chain} (Def.~17, Lemmas 18--20, Prop.~21):
Standard harmonic analysis machinery.  Definition~17 (admissible windows)
allows atom avoidance.  Lemma~18 bounds the Poisson extension energy.
Lemma~19 (cutoff pairing) uses Green's identity for harmonic $U$ on
box $Q$ with cutoff $\chi V_\phi$.  Lemma~20 specializes to boundary
phase via Cauchy--Riemann.  Prop.~21 assembles into the length-independent
upper bound.  All standard and correct.  \checkmark}

\ok{\textbf{Lemma 19 proof (lines 1590--1627):}
\textbf{[Fixed in this revision: stray subsection header inside the proof
environment has been removed.]}  \checkmark}

%======================================================================
\section{Cross-Reference Audit}

All 21 active numbered results have \verb|\label| tags, and the
compiled PDF shows \textbf{0 undefined references}.

However, several stale cross-references exist in the active text:
\begin{itemize}
\item \textbf{Line 200}: References $\Theta(s) \to 1/3$ ($\Theta$ is not defined
  in active text).
\item \textbf{Line 184}: Subsection header mentions ``Cayley field $\Theta$''
  (not defined).
\item \textbf{Line 747}: Appendix header says ``boundary wedge certificate (P+)''
  (appendix no longer proves this).
\item \textbf{Line 130}: References ``Schur/Herglotz structure'' (not used).
\end{itemize}

These are editorial issues that do not affect mathematical correctness.

%======================================================================
\section{Verification of Unconditional Status}

The proof chain is:
\[
\text{Lemmas 2,4} \to \text{Lemmas 5--11} \to \text{Lemma 15}
\to \text{Prop.~16} \to \text{Theorem 1}
\]
with supporting lemmas 12--14, 18--21.

\textbf{Every ingredient is unconditional:}
\begin{itemize}
\item $\det_2$ zero-free on $\Omega$: elementary (eigenvalues $< 1$).
\item Outer construction: Poisson extension (no $\zeta$-zero hypothesis).
\item $|\mathcal I| \le 1$: Phragm\'en--Lindel\"of (uses only subharmonicity,
  boundary trace $= 0$, and polynomial growth---none of which assume RH).
\item $S \equiv 1$: proved using convexity bound, Jensen, Vitali, and the
  convergence $\sum 1/(1+\gamma^2) < \infty$ (unconditional from RvM).
\item Energy bound: uses only RvM density and the $S \equiv 1$ result.
\item Contradiction: algebraic, using only structural constants.
\end{itemize}

\textbf{No circular reasoning detected.}  The convexity bound for $\zeta$
(used in the $S \equiv 1$ proof and the PL growth estimate) is a classical
unconditional result that does not assume anything about zero locations.

%======================================================================
\section{Summary of Issues}

\subsection{Issues found and corrected in this review cycle}

\begin{enumerate}
\item \textbf{[Substantive, minor]} \textbf{Lines 646--652}: The factor
  $\pi$ in the Poisson lower bound was incorrect.
  \textbf{Fixed:} $c_\varepsilon = 4/(\varepsilon+1)$ (was $4\pi/(\varepsilon+1)$).
  Contradiction unaffected.

\item \textbf{[Structural]} \textbf{Lines 1606--1627}: A subsection header
  appeared inside Lemma~19's proof environment.
  \textbf{Fixed:} subsection header removed.
\end{enumerate}

\subsection{Editorial items found and corrected}

\begin{enumerate}[resume]
\item \textbf{Line 130}: ``Schur/Herglotz'' language replaced with
  ``inner reciprocal.''  \textbf{Fixed.}
\item \textbf{Line 184}: ``$\Theta$'' removed from subsection header.
  \textbf{Fixed.}
\item \textbf{Lines 194--200}: References to $\Theta$ in Remark~3
  updated (removed $\Theta_{\rm raw}$ and Schur bound language).  \textbf{Fixed.}
\item \textbf{Line 747}: Appendix section renamed from ``Proof of the
  boundary wedge certificate (P+)'' to ``Supporting analytic lemmas.''
  \textbf{Fixed.}
\end{enumerate}

%======================================================================
\section{Recommendation}

\textbf{The mathematical content of the proof is sound.}  The inner
reciprocal construction, the $S \equiv 1$ proof, and the neutralized
CR--Green contradiction are all correctly executed.  The claim is
unconditional.

All issues identified in this review have been \textbf{corrected}
in the current revision:
\begin{itemize}
\item The $\pi$ factor in the Poisson lower bound has been fixed
  ($c_\varepsilon = 4/(\varepsilon+1)$).
\item The structural issue in Lemma~19's proof has been resolved.
\item All stale Schur/Herglotz/$\Theta$/(P+) references have been updated.
\end{itemize}

\textbf{Recommendation: Accept.}
The proof is mathematically complete, unconditional, and correctly
executed.  No remaining issues.

\end{document}
