\documentclass[11pt]{article}
\usepackage[margin=1in]{geometry}
\usepackage{amsmath}
\usepackage{enumitem}
\usepackage{hyperref}
\usepackage{xcolor}

% Inline response color (AI / editor notes)
\definecolor{AIColor}{RGB}{120,0,180}
\newcommand{\AI}[1]{\textcolor{AIColor}{\textbf{AI response:} #1}}

\title {Step 2 - referee\_package\_2026-02-03}
\author{}
\date{2026-02-03}

\begin{document}
\maketitle

\section*{Follow up with Author}
 in the uploaded
\texttt{referee\_package\_2026-02-03.zip}. It is \emph{theory-only}: no scripts or JSON artifacts were executed here.
\par\AI{Correct for this \emph{alignment report}: it is a static theory audit of the shipped bundle contents and phrasing. Note the uploaded referee package itself includes \texttt{compute/} (scripts + JSON artifacts) and includes an independent Lambda A10 compute audit bundle; this document does not re-run those computations.}

\section*{What appears resolved (green responses)}
\begin{itemize}[leftmargin=*]
\item \textbf{(AC$_\mu$) removal claim.} Multiple green blocks state that Theorem~\texttt{\textbackslash ref\{thm:Pplus\}}
is now proved \emph{directly} as a Lebesgue-a.e.\ wedge statement in Appendix~\texttt{\textbackslash ref\{app:pplus-proof\}},
so no $\mu$-a.e.\ upgrade and no (AC$_\mu$) hypothesis is used.
\par\AI{Partially resolved. The Appendix now routes Lebesgue-a.e.\ wedge via a Whitney-local oscillation/globalization step. One concrete gap that existed here (Lemma 17 ``triangular kernel'' vs arbitrary bump) has been fixed by pinning \(\varphi_I\) to an explicit hat/cutoff and using positivity of \(-w'\) (no unjustified ``integration by parts''). Remaining work is to remove contradictory (AC$_\mu$) sentences/blocks elsewhere in the main paper so the unconditional story is internally consistent.}
\item \textbf{Citations added.} Green blocks add citations for the $\det_2$ product formula and for standard half-plane
outer-function and Schur-transport facts (e.g.\ Garnett; Rosenblum--Rovnyak; Simon).
\par\AI{Resolved in the current marked draft (citations present). Still advisable for submission: ensure each cited claim points to a \emph{specific} theorem/lemma number in the reference (not just a book-level citation) wherever the referee flagged ``proof sketch''.}
\item \textbf{Gauge/domain sanity.} Green block addresses the $(s-1)/s$ compensator and regularity at $s=1$ on $\Omega$.
\par\AI{Resolved at the level of bookkeeping: the compensator is holomorphic and nonvanishing on \(\Omega=\{\Re s>\tfrac12\}\) so it does not introduce poles/zeros in the region of interest.}
\item \textbf{Blocker-mapping narrative.} The manuscript now contains an explicit map from the two analytic inputs needed for
Lemma~\texttt{\textbackslash ref\{lem:F-boundary-admissible\}} to concrete Appendix lemmas (determinant $L^1$ control,
outer stability, Carleson/Whitney control, etc.).
\par\AI{Partially resolved. The mapping narrative helps, but a referee can still object if the main-text lemma remains worded as ``Assume ...''. This needs the wording/proof upgrade described below (make the lemma unconditional + cite exact Appendix lemma numbers in a short proof).}
\end{itemize}

\section*{Still inconsistent statements (must fix for a clean ``unconditional'' story)}
\begin{enumerate}[leftmargin=*]
\item \textbf{Main Theorem statement still asserts (AC$_\mu$).}\\
In \texttt{paper1\_farfield.tex}, Theorem~\texttt{\textbackslash ref\{thm:farfield\}} still contains a blue sentence saying
the proof relies on an Appendix $\mu$-construction and the domination hypothesis (AC$_\mu$).
This directly contradicts the green claim that (AC$_\mu$) is no longer used.
\par\AI{Not resolved yet (editorial but \emph{must} be fixed). Action: delete/replace that blue sentence in Theorem~\texttt{\textbackslash ref\{thm:farfield\}} so it forward-references the Appendix Lebesgue-a.e.\ wedge proof of Theorem~\texttt{\textbackslash ref\{thm:Pplus\}} only, with no (AC$_\mu$) dependency.}
\item \textbf{One paragraph says JSON artifacts are ``used in the proof''.}\\
In the ``Audit posture'' paragraph (near the repository contents), it says the JSON artifacts recording maxima are ``used in the proof''.
This contradicts the Abstract and the ``Conclusion and limitations'' section, which state artifacts are \emph{not} used in the proof.
\par\AI{Not resolved yet (editorial but important). Action: choose one consistent posture and rewrite: either (A) artifacts are \emph{audit-only} (recommended) \(\Rightarrow\) change ``used in the proof'' to ``for independent cross-check/auditability only'', or (B) artifacts are load-bearing \(\Rightarrow\) update Abstract/Conclusion accordingly.}
\item \textbf{Obsolete conditionality blocks remain inline.}\\
Several blue blocks still describe a $\mu$-a.e.\ $\Rightarrow$ Lebesgue-a.e.\ upgrade as a required step, and cite (AC$_\mu$).
A green block says these are ``historical referee commentary'', but they remain in the main logical flow.
For a journal-ready unconditional draft, these should be removed or moved to a separate ``Referee history'' appendix.
\par\AI{Not resolved yet. Recommended: for the \emph{submission} PDF, move all obsolete \(\mu\)-upgrade infrastructure to a clearly labeled historical/referee-log appendix (or delete entirely), so a referee cannot misread it as part of the current dependency chain. Keep only what is logically used.}
\end{enumerate}

\section*{Remaining theory gaps that are \emph{not} just editorial}
\begin{enumerate}[leftmargin=*]
\item \textbf{Lemma \texttt{lem:F-boundary-admissible} is still phrased as an assumption.}\\
It begins with ``Assume the analytic inputs established in Appendix...'', then lists two inputs (boundary trace/log-modulus and a harmonic
majorant/box control). The manuscript \emph{does} provide an Appendix mapping to specific lemmas, but the main text should be updated
to a standard form:
\begin{itemize}
\item either: state the lemma unconditionally and give a short proof that explicitly cites the exact Appendix lemmas that discharge (i)--(ii), or
\item: promote those Appendix items to a named theorem and cite that theorem here.
\end{itemize}
Until this is rewritten, a referee can reasonably read the result as ``assumed'' rather than proved in-paper.
\par\AI{Not resolved yet (substantive presentation gap). I agree: rewrite the lemma unconditionally and add a short proof that explicitly discharges (i)--(ii) by citing the exact Appendix lemmas (by number) that provide each input. This is a must-have for a clean referee read.}

\item \textbf{The distributional phase--velocity lemma is only a ``proof sketch''.}\\
In \texttt{paper1\_pplus\_proof.tex}, Lemma~\texttt{lem:pv-distributional} has a proof labeled ``Proof sketch and what must be checked''.
This should be upgraded to either (a) a complete proof, or (b) a precise citation (theorem/lemma number) in a standard reference
(e.g.\ Duren or Garnett) with hypotheses matching exactly what is used later.
\par\AI{Not resolved yet (potentially load-bearing). Recommendation: either provide a complete proof in-paper (best), or replace the sketch with a precise external citation (theorem number + hypotheses) and then verify in text that the manuscript's objects satisfy those hypotheses (e.g.\ boundary integrability / nontangential limits / BV or distributional derivative framework).}

\item \textbf{Remove/neutralize unused $\mu$-infrastructure if it is truly unused.}\\
The Appendix still contains (AC$_\mu$) and $\mu$-to-Lebesgue discussion, including ``what must be checked for the present $\mu$''.
If the final (P+) proof chain no longer uses it, keep it only as an optional historical note clearly marked ``not used''; otherwise,
a referee may interpret it as part of the logical dependency chain.
\par\AI{Not resolved yet. Strongly agree: for submission, either delete this material or move it to a ``Not used / historical'' subsection with an explicit statement ``Theorem~\texttt{\textbackslash ref\{thm:Pplus\}} does not invoke any \(\mu\)-upgrade.'' Right now it invites confusion.}
\end{enumerate}

\section*{Minimal patch list to make the paper unambiguously unconditional}
\begin{itemize}[leftmargin=*]
\item Delete or rewrite the (AC$_\mu$) sentence inside Theorem~\texttt{\textbackslash ref\{thm:farfield\}}; replace with a single forward reference
to the Lebesgue-a.e.\ Appendix proof of Theorem~\texttt{\textbackslash ref\{thm:Pplus\}}.
\par\AI{Agree. This is required for internal consistency with the new Appendix wedge route.}
\item In ``Audit posture'': change ``used in the proof'' to ``for independent cross-check / auditability only'' (or, if artifacts are load-bearing,
then remove the Abstract/Conclusion statements that say they are not used).
\par\AI{Agree. Recommend option ``audit-only'' unless you explicitly want the main theorem to depend on shipped JSON outputs.}
\item Rewrite Lemma~\texttt{\textbackslash ref\{lem:F-boundary-admissible\}} so it is no longer an ``Assume ...'' lemma: explicitly discharge
(i)--(ii) by citing the exact Appendix lemmas.
\par\AI{Agree. This is the cleanest way to eliminate the ``assumed'' reading.}
\item Upgrade Lemma~\texttt{lem:pv-distributional} from ``proof sketch'' to a complete proof or a precise theorem-number citation.
\par\AI{Agree. Until this is upgraded, a referee can reasonably mark the Appendix wedge chain as ``UNCERTAIN''.}
\item Move all obsolete blue conditionality blocks into a separate ``Referee log'' appendix (or delete for the submission version).
\par\AI{Agree. This is important not for correctness, but for referee-time: remove any opportunity for dependency confusion.}
\end{itemize}

\section*{AI end-of-document status (2026-02-03)}
\AI{I was able to (i) respond inline to each issue here and (ii) fix one concrete load-bearing gap already identified in the Appendix wedge chain (Lemma 17 kernel/bump mismatch) inside the referee package. However, the broader set of issues listed in this report is \emph{not yet fully resolved} for a journal-ready unconditional draft: the main paper still contains inconsistent (AC$_\mu$)/artifact-load-bearing phrasing, and two items remain potentially load-bearing until rewritten (Lemma \texttt{lem:F-boundary-admissible} as ``assume'', and Lemma \texttt{lem:pv-distributional} as a proof sketch).}

\end{document}
