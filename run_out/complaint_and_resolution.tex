\documentclass[11pt]{amsart}
\usepackage[margin=1in]{geometry}
\usepackage{amsmath,amssymb,amsthm}
\usepackage{booktabs}
\usepackage[T1]{fontenc}
\usepackage{lmodern}
\usepackage{microtype}

\newtheorem{theorem}{Theorem}
\newtheorem{lemma}[theorem]{Lemma}
\newtheorem{proposition}[theorem]{Proposition}
\theoremstyle{remark}
\newtheorem{remark}[theorem]{Remark}

\newcommand{\C}{\mathbb{C}}
\newcommand{\R}{\mathbb{R}}
\DeclareMathOperator{\dettwo}{det_2}

\title{Complaint and Resolution:\\
The box-energy bound in the zero-free region proof}
\author{Technical companion to paper1\_zerozeta-v19}
\date{February 7, 2026}

\begin{document}
\maketitle

\section{Context}

The paper proves:

\begin{theorem}[Far-field zero-freeness]
The Riemann zeta function has no zeros in the region
$\{\,s\in\C:\ \Re s\ge 0.6\,\}$.
\end{theorem}

The proof proceeds by contradiction: assuming a zero $\rho_0=\beta_0+i\gamma_0$
with $\beta_0\ge 0.6$ exists, it derives a quantitative conflict between
a phase-velocity lower bound (from the zero's Poisson balayage) and a
CR--Green upper bound (from a Whitney-box energy estimate).

During adversarial review, three complaints were raised against the energy bound.
This document states each complaint precisely and explains how it is resolved
in the current version of the paper.

\section{Complaint 1: the $V\ge 0$ circularity}

\subsection*{The complaint}

The original proof defined $V:=-\log|\mathcal J_{\rm out}|\ge 0$ and used
this positivity to derive the energy bound via a strip comparison and
interior gradient estimate.
But $\mathcal J_{\rm out}$ is \emph{meromorphic} on $\Omega=\{\Re s>1/2\}$
with poles at zeros of~$\zeta$.
Near a pole, $|\mathcal J_{\rm out}|\to\infty$, so $V<0$.
Asserting $V\ge 0$ is equivalent to assuming $\mathcal J_{\rm out}$ has no poles,
which is the conclusion being proved.
\textbf{The proof was circular.}

\subsection*{The resolution: the inner reciprocal}

Define the \emph{analytic inner reciprocal}
\[
  \mathcal I(s)\ :=\ \frac{B(s)^2}{\mathcal J_{\rm out}(s)}
  \ =\ \frac{B(s)\,\mathcal O_\zeta(s)\,\zeta(s)}{\dettwo(I-A(s))}\,,
  \qquad B(s):=\frac{s-1}{s}.
\]

\begin{itemize}
\item $\mathcal I$ is \textbf{holomorphic} on $\Omega$:
  the pole of $\zeta$ at $s=1$ is canceled by $B$;
  zeros of $\zeta$ become \emph{zeros} (not poles) of $\mathcal I$;
  $\dettwo(I-A)$ is nonvanishing on $\Omega$.

\item $|\mathcal I(\tfrac12+it)|=1$ for a.e.\ $t$:
  on $\partial\Omega$, $|B|=1$ and $|\mathcal J_{\rm out}|=1$.

\item $|\mathcal I(s)|\le 1$ for all $s\in\Omega$:
  by the Phragm\'en--Lindel\"of principle applied to $\log|\mathcal I|$
  (subharmonic, boundary trace~$0$ a.e., growth $o(|s|)$).
\end{itemize}

The nonnegative potential $W:=-\log|\mathcal I|\ge 0$ is therefore
\textbf{unconditionally} nonnegative.
No assumption about $\zeta$-zeros is used.

\subsection*{Why the PL argument is fully justified}

A concern was raised that the PL bound ``$|\mathcal I|\le 1$ from a.e.\ boundary
unimodularity'' requires hidden Smirnov or Hardy class hypotheses.

This is not the case.
The PL principle for the half-plane states:

\begin{quote}
\emph{If $u$ is subharmonic on $\Omega$, has nontangential boundary trace
$u^*\le 0$ a.e., and satisfies $u(s)=o(|s|)$ as $|s|\to\infty$ in~$\Omega$,
then $u\le 0$ on~$\Omega$.}
\end{quote}

For $u=\log|\mathcal I|$:
\begin{enumerate}
\item $u$ is subharmonic ($\mathcal I$ holomorphic).
\item The nontangential boundary trace $u^*=0$ a.e.\ is established by
  showing that each factor of $\mathcal I=B\,\mathcal O_\zeta\,\zeta/\dettwo$
  has $L^1_{\rm loc}$-convergent log-modulus as $\sigma\to 0$:
  \begin{itemize}
  \item $\log|B|\to 0$ uniformly ($B$ is continuous, $|B^*|=1$);
  \item $\log|\mathcal O_\zeta|\to u(t)$ in $L^1_{\rm loc}$
    (Poisson extension of boundary data);
  \item $\log|\zeta(\tfrac12+\sigma+it)|\to\log|\zeta^*(t)|$ in $L^1_{\rm loc}$
    (the paper's boundary-trace lemma);
  \item $\log|\dettwo(\tfrac12+\sigma+it)|\to\log|\dettwo^*(t)|$ in $L^1_{\rm loc}$
    (BMO trace from the arithmetic Carleson energy).
  \end{itemize}
  By the algebraic identity $u=\log|\dettwo^*|-\log|\zeta^*|$
  (from the construction of $\mathcal O_\zeta$), the traces sum to~$0$.
\item $u(s)=O(\log(2+|s|))=o(|s|)$ (polynomial growth from convexity bounds).
\end{enumerate}

No Smirnov/Hardy class is invoked. Only the PL principle for subharmonic functions
with controlled growth and boundary trace.

\section{Complaint 2: the $\log^2\!\langle t_0\rangle$ growth\\
and short-interval zero control}

\subsection*{The complaint}

The energy bound gives $E(I)\le C\log^2\!\langle t_0\rangle\,|I|$ on Whitney boxes.
With a \emph{fixed} Whitney parameter~$c$, the CR--Green upper bound
$O(\sqrt{c\log\langle\gamma_0\rangle})$ grows as $\sqrt{\log}$
and eventually exceeds the lower bound~$11$ from the hypothetical zero.
So the contradiction does not close at all heights with a single fixed~$c$.

Additionally, the bound $M=\sup_{\partial D}\widetilde W\le C_*\log\langle t_0\rangle$
was claimed to require ``short-interval zero control at scale~$L$'' ---
specifically the bound $N(T;H)\le A_0+A_1H\log\langle T\rangle$ with $A_0$
independent of~$T$, which is not a consequence of the classical Riemann--von
Mangoldt formula.

\subsection*{The resolution}

\textbf{(a) The $M$ bound does not require short-interval control.}

The Poisson-averaged Green-kernel sum over far zeros is:
\[
  \sum_{|\gamma-t_0|>\alpha''L}G_\Omega(s,\rho)
  \;\le\;\alpha'L\int_{\alpha''L}^{\infty}\frac{C_{\rm RvM}\log\langle t_0\rangle}{r^2}\,dr
  \;=\;\frac{\alpha'C_{\rm RvM}\log\langle t_0\rangle}{\alpha''}.
\]

The key cancellation: the factor $\sigma\le\alpha'L$ in the Green kernel numerator
cancels the $1/(\alpha''L)$ from the integral $\int_{\alpha''L}^\infty 1/r^2\,dr$.
The result $\alpha'C_{\rm RvM}\log\langle t_0\rangle/\alpha''$ is
\textbf{independent of $L$ and $c$}.

Only the coarse $O(\log\langle t_0\rangle)$ zero count per unit ordinate interval
is used (a standard consequence of Riemann--von Mangoldt).
No short-interval control at scale~$L$ is needed.

\medskip
\textbf{(b) The $\log^2$ growth is absorbed by the height-dependent $c$ trick.}

In a proof by contradiction, $\gamma_0$ is fixed, so $c$ may depend on~$\gamma_0$.
Choosing $c=c_0/\log\langle\gamma_0\rangle$ gives $L=c_0/\log^2\!\langle\gamma_0\rangle$
and
\[
  E_{\rm eff}(I)\;=\;C\log^2\!\langle\gamma_0\rangle\cdot|I|
  \;=\;C\log^2\!\langle\gamma_0\rangle\cdot\frac{2c_0}{\log^2\!\langle\gamma_0\rangle}
  \;=\;2Cc_0
  \qquad\text{(height-independent)}.
\]

The constant $C=C_3C_*^2$ depends only on apertures $(\alpha',\alpha'')$ and the
RvM/convexity constants---\textbf{not on $c$}---because $C_*$ is independent of $L$
(see part~(a) above).

\section{Complaint 3: the singular inner factor}

\subsection*{The complaint}

The inner reciprocal $\mathcal I$ may have a nontrivial singular inner factor~$S$
in its canonical factorization $\mathcal I=e^{i\theta}B_{\mathcal I}S$.
The term $-\log|S|$ is the Poisson integral of a singular measure~$\nu_S$,
and on Whitney boxes of height $\sigma\sim L$, it could blow up as
$\nu_S(\R)/L$ without an a~priori bound on~$\nu_S$.

An earlier version of the paper attempted to prove $S\equiv 1$ via a
strip comparison for the superharmonic function $W=-\log|\mathcal I|$,
but the comparison principle goes the wrong way for superharmonic functions
(the maximum, not the minimum, can be achieved at interior singularities).

\subsection*{The resolution: the singular inner factor is absorbed by the global $W$ bound}

The key observation:
\[
  -\log|S(s)|\ \le\ -\log|\mathcal I(s)|\ =\ W(s)
  \qquad(s\in\Omega),
\]
because $|\mathcal I|=|B_{\mathcal I}|\cdot|S|\le|S|$
(the Blaschke part satisfies $|B_{\mathcal I}|\le 1$).

And $W(s)\le N\log(2+|t|)+C$ on $\{0<\sigma\le 1\}$
from the polynomial growth bound in Lemma~\texttt{inner-reciprocal}(3)
(convexity bounds for $\zeta$, convergent products for $\dettwo$,
Poisson-controlled modulus of $\mathcal O_\zeta$).

This is a \textbf{pointwise} upper bound on $-\log|S|$ that:
\begin{itemize}
\item does not depend on the singular measure $\nu_S$,
\item does not depend on $L$ or $c$,
\item does not require proving $S\equiv 1$.
\end{itemize}

On the dilated Whitney box $\partial D$: $-\log|S|\le W\le N\log\langle t_0\rangle+C$,
which is absorbed into the boundary bound $M\le C_*\log\langle t_0\rangle$ with
$C_*$ depending only on the convexity exponent~$N$ and the apertures.

\section{Complaint 4: the near-zero count and energy}

\subsection*{The complaint}

The paper originally claimed that the number of zeros of $\mathcal I$ inside the
Whitney box is $O(1)$.
The correct bound (from Riemann--von Mangoldt) is $O(\log\langle t_0\rangle)$,
not~$O(1)$, because the RvM error term $O(\log T)$ dominates on Whitney scale.

\subsection*{The resolution: near-zero charges are separate from the smooth-part bound}

The $O(\log\langle t_0\rangle)$ in-box zeros of $\mathcal I$ contribute
\emph{explicit nonnegative charges}
$2\pi\sum m_jV_\phi(\rho_j)\ge 0$
to the total windowed phase via the distributional Green identity
on the punctured domain.

These charges \textbf{add to the total phase but do not enter the
Cauchy--Schwarz energy bound for the smooth part}.

The CR--Green pairing bounds the \emph{smooth part} of the windowed phase
(the boundary normal derivative of the neutralized harmonic function~$\widetilde W$).
The hypothetical zero $\rho_0$ at $\beta_0\ge 0.6$ lies \textbf{outside} the box
(since $\delta_0=\beta_0-1/2\ge 0.1>\alpha'L$ for $t_0$ large),
so its Poisson contribution enters the smooth part, not the charge term.

The contradiction is therefore between:
\begin{itemize}
\item \textbf{Smooth-part lower bound} (from $\rho_0$): $\ge 11L$
\item \textbf{Smooth-part upper bound} (from CR--Green on $\widetilde W$):
  $\le A\sqrt{c_0}\cdot L$
\end{itemize}
For $c_0<(11/A)^2$: $A\sqrt{c_0}<11$. \textbf{Contradiction.}

The near-zero count ($O(1)$ or $O(\log T)$) is irrelevant to this comparison.

\section{Summary: the contradiction algebra}

\begin{center}
\renewcommand{\arraystretch}{1.3}
\begin{tabular}{lll}
\toprule
\textbf{Quantity} & \textbf{Value} & \textbf{Scaling}\\
\midrule
Whitney parameter & $c=c_0/\log\langle\gamma_0\rangle$ & height-dependent\\
Box half-length & $L=c_0/\log^2\!\langle\gamma_0\rangle$ & $\to 0$\\
Interval length & $|I|=2c_0/\log^2\!\langle\gamma_0\rangle$ & $\to 0$\\
Boundary bound & $M=C_*\log\langle\gamma_0\rangle$ & grows\\
Effective energy & $E_{\rm eff}=C_3C_*^2\log^2\cdot|I|=2C_3C_*^2c_0$ & \textbf{constant}\\
Smooth-part upper & $A\sqrt{c_0}\cdot L$ & $\propto L$\\
Smooth-part lower (from $\rho_0$) & $11L$ & $\propto L$\\
\textbf{Ratio upper/lower} & $A\sqrt{c_0}/11$ & \textbf{height-independent}\\
\bottomrule
\end{tabular}
\end{center}

\medskip
For $c_0=(11/A)^2/2$: ratio $=1/\sqrt{2}<1$. \textbf{Contradiction at every height.}

The $\log^2$ from the energy cancels the $1/\log^2$ from $|I|$.
The singular inner factor is absorbed by the PL bound on~$W$.
The near-zero charges don't enter the smooth-part comparison.
The constant~$A$ depends only on apertures $(\alpha',\alpha'')$,
the convexity exponent~$N$, and $C_{\rm RvM}$---none of which depend on~$c$ or~$\gamma_0$.

\end{document}
