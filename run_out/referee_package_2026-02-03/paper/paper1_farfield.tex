\documentclass[11pt]{article}
% Shared preamble for the three-paper split.
% Keep this file free of \documentclass and \begin{document}.
% Do NOT mention proof assistants anywhere in the split papers.

\usepackage[margin=1in]{geometry}
\usepackage{booktabs}
\usepackage{float}
\usepackage{amsmath,amssymb,amsthm,mathtools}
\usepackage[T1]{fontenc}
\usepackage{lmodern}
\usepackage[utf8]{inputenc}
\usepackage{microtype}
\usepackage{hyperref}
\usepackage[numbers,sort&compress]{natbib}
\hypersetup{colorlinks=true,linkcolor=black,citecolor=black,urlcolor=black}

% Theorems
\newtheorem{theorem}{Theorem}
\newtheorem{proposition}[theorem]{Proposition}
\newtheorem{lemma}[theorem]{Lemma}
\newtheorem{corollary}[theorem]{Corollary}
\newtheorem{hypothesis}[theorem]{Hypothesis}
\theoremstyle{definition}
\newtheorem{definition}[theorem]{Definition}
\theoremstyle{remark}
\newtheorem{remark}[theorem]{Remark}

% Basic macros
\newcommand{\C}{\mathbb{C}}
\newcommand{\R}{\mathbb{R}}
\newcommand{\N}{\mathbb{N}}
\newcommand{\PP}{\mathcal{P}}
\newcommand{\Hilb}{\mathcal H}
\DeclareMathOperator{\dettwo}{det_2}

% Stable angle-bracket convention
\newcommand{\angles}[1]{\langle #1\rangle}



% Editorial markup (disabled for submission)
\newcommand{\editblue}[1]{#1}
\newcommand{\editgreen}[1]{#1}

\DeclareMathOperator{\Arg}{Arg}
\DeclareMathOperator{\ntlim}{nt\!-\!lim}
\usepackage{xcolor}
% Blue text marks inserted referee-suggested additions.
\newenvironment{bluetext}{\begingroup\color{blue}}{\endgroup}

% Green text marks author revisions responding to the referee (post-review edits).
\definecolor{authorrev}{rgb}{0.0,0.45,0.0}
\newenvironment{greentext}{\begingroup\color{authorrev}}{\endgroup}
\newcommand{\rev}[1]{\textcolor{authorrev}{#1}}

\title{A certified zero-free region for the Riemann zeta function\\in the half-plane $\Re s \ge 0.6$}
\author{Jonathan Washburn\\ Referee by: Amir Rahnama}
\date{02-02-2026} 

\begin{document}
\maketitle

\begin{abstract}
We prove unconditionally that the Riemann zeta function $\zeta(s)$ has no zeros in the fixed half-plane $\{\,\Re s \ge 0.6\,\}$.
The argument is function-theoretic.
On $\Omega=\{\,\Re s>\tfrac12\,\}$ we form an arithmetic ratio $\mathcal J(s)$ whose poles encode zeros of $\zeta$, and pass to its Cayley transform $\Theta(s)=(2\mathcal J(s)-1)/(2\mathcal J(s)+1)$.
A Schur bound $|\Theta|\le 1$ on a domain forces $\mathcal J$ to be pole-free there by removability (a Schur/Herglotz pinch), hence excludes zeros.
Accordingly, the analytic task is to certify a Schur bound on a half-plane containing $\{\,\Re s\ge 0.6\,\}$.
In this version, the all-heights Schur bound is discharged by \rev{a boundary-certificate route}:
\rev{a quantitative boundary wedge \textup{(P+)} (Lebesgue-a.e.) implies that $2\mathcal J$ is Herglotz and $\Theta$ is Schur on $\Omega\setminus Z(\zeta)$,}

\begin{greentext}
\noindent\textbf{Author revision (resolving the (AC$_\mu$) dependency).}
In the current Appendix~\ref{app:pplus-proof}, \textup{(P+)} is proved \emph{directly} as a Lebesgue-a.e.\ statement via Whitney-local phase control and the local-to-global wedge lemma; no $\mu$-a.e.\ $\Rightarrow$ Lebesgue-a.e.\ upgrade (and hence no (AC$_\mu$) hypothesis) is used in the logical chain from \textup{(P+)} to the Schur/Herglotz transport.
\end{greentext}

and the pinch mechanism then excludes poles (hence zeros) on $\{\,\Re s\ge 0.6\,\}$.
For referee convenience, we also include independent rigorous ball-arithmetic artifacts on representative low-height rectangles in the handoff bundle (and mirrored in the repository), but these are not used in the proof.
\end{abstract}

\section{Introduction}

The Riemann zeta function
\[
  \zeta(s)\;=\;\sum_{n\ge 1}\frac{1}{n^s},\qquad \Re s>1,
\]
extends meromorphically to $\C$ with a simple pole at $s=1$ and satisfies a functional equation after completion.
Its nontrivial zeros govern the finest fluctuations in the distribution of prime numbers, and the Riemann Hypothesis asserts that all such zeros lie on the critical line $\Re s=\tfrac12$; see \cite{Titchmarsh,IK} for background.

This paper isolates an unconditional, fixed-strip statement in the direction of RH.
Unlike classical zero-free regions near $\Re s=1$ (which are asymptotic in height), the result here is a \emph{uniform} half-plane exclusion at $\Re s\ge 0.6$.

\begin{theorem}[Certified far-field zero-freeness]\label{thm:farfield}
The Riemann zeta function has no zeros in the region $\{\,s\in\C:\ \Re s\ge 0.6\,\}$.
\end{theorem}
\begin{greentext}
\noindent\textbf{Author revision (dependency cleanup).}
The Appendix proof of \textup{(P+)} has been organized so that it yields a \emph{Lebesgue-a.e.} wedge statement directly from Whitney-local phase control and the local-to-global wedge lemma; no (AC$_\mu$) domination hypothesis is used.
The remaining analytic-function-theory inputs (Smirnov/bounded-type regularity and the precise form of the phase--velocity identity, including atomic/singular contributions) are stated explicitly and referenced/proved within Appendix~\ref{app:pplus-proof}.
\end{greentext}

\begin{greentext}
\paragraph{Author revision (dependency map for Theorem~\ref{thm:farfield}).}
The proof is organized so that the only load-bearing inputs are proved in Appendix~\ref{app:pplus-proof} and then cited in the main text:
\begin{itemize}
\item a \emph{Lebesgue-a.e.} boundary wedge \textup{(P+)} for \(\mathcal J_{\rm out}\) (Theorem~\ref{thm:Pplus}, proved in Appendix Theorem~\ref{thm:pplus-proof-complete} via Whitney-local oscillation control and Lemma~\ref{lem:local-to-global-wedge});
\item the quantified distributional phase--velocity identity, written with explicit atomic and singular-inner terms (Theorem~\ref{thm:phase-velocity-quant} and \eqref{eq:pv-identity});
\item Smirnov/bounded-type regularity on \(\Omega\setminus Z(\zeta)\) needed for boundary-to-interior transport (Lemma~\ref{lem:F-boundary-admissible} and Lemma~\ref{lem:smirnov-regularity}).
\end{itemize}
In particular, no \(\mu\)-a.e.\(\Rightarrow\)Lebesgue-a.e.\ upgrade (and hence no domination hypothesis such as (AC$_\mu$)) is used in the final logical chain.
\end{greentext}



\subsection*{Strategy: Schur pinching via a Cayley field}
We work on the right half-plane $\Omega=\{\,\Re s>\tfrac12\,\}$.
In Section~\ref{sec:defs} we define an arithmetic ratio $\mathcal J$ (in the default \emph{raw $\zeta$-gauge}) with the following two structural properties:
\begin{itemize}
\item \textbf{(normalization at $+\infty$)} $\mathcal J(\sigma+it)\to 1$ as $\sigma\to+\infty$, hence $\Theta(\sigma+it)\to \tfrac13$ (Remark~\ref{rem:Ocan-role});
\item \textbf{(non-cancellation)} $\dettwo(I-A(s))$ is holomorphic and nonvanishing on $\Omega$, so any zero of $\zeta$ in $\Omega$ produces a pole of $\mathcal J$ (Remark~\ref{rem:poles}).
\end{itemize}
We then pass to the Cayley transform
\[
  \Theta(s)\ :=\ \frac{2\mathcal J(s)-1}{2\mathcal J(s)+1}.
\]
The analytic mechanism is a \emph{Schur/Herglotz pinch} proved in Section~\ref{sec:pinch}:
if $\Theta$ is Schur on a domain (i.e.\ $|\Theta|\le 1$) and not identically $1$, then boundedness forces removability of any isolated singularity and prevents poles of $\mathcal J$.
Since $\Theta(\sigma+it)\to \tfrac13$ as $\sigma\to+\infty$, the degenerate possibility $\Theta\equiv 1$ is excluded on the half-planes relevant here.
Therefore, to prove Theorem~\ref{thm:farfield} it suffices to certify a Schur bound for the default Cayley field $\Theta_{\rm raw}$ on some open half-plane $\{\,\Re s>0.6-\varepsilon\,\}$.

\subsection*{Certified inputs (what is rigorously checked)}
\rev{The logical implication of Theorem~\ref{thm:farfield} rests on a boundary certificate:}
\rev{we establish a boundary wedge \textup{(P+)} (Lebesgue-a.e.) for the boundary phase of $\mathcal J$ on $\Re s=\tfrac12$, which implies that $2\mathcal J$ is Herglotz and $\Theta$ is Schur on $\Omega\setminus Z(\zeta)$.}
The Schur/Herglotz pinch mechanism then excludes poles of $\mathcal J$ on $\{\,\Re s\ge 0.6\,\}$ and hence excludes zeros of $\zeta$ there.

\medskip
\noindent\emph{\textcolor{blue}{Supplementary computational cross-checks (not used in the proof).}}
The handoff bundle also contains rigorous ball-arithmetic rectangle checks and finite Pick artifacts on low-height regions; these are included as independent numerical corroboration but are not used in the all-heights proof.

\subsection*{Reproducibility and audit posture}
The certification is intended to be referee-auditable.
The handoff bundle (and repository) includes:
(i) the verifier script based on ARB ball arithmetic (`python-flint`),
and (ii) the JSON artifacts that record the certified maxima, spectral gaps, and denominator checks \rev{for independent cross-check/auditability (not used in the proof)}.
The file \texttt{README.md} provides an audit manifest mapping the manuscript’s statements to exact commands and expected outputs.

\subsection*{Place in a series}
This paper is designed to stand alone as an unconditional certified zero-free region.
Two companion papers (not required for Theorem~\ref{thm:farfield}) treat: (a) effective near-field energy barriers and Carleson budgets, and (b) a cutoff principle yielding conditional closure of RH.

\medskip
\noindent The remainder of the paper defines the arithmetic ratio $\mathcal J$ and Cayley field $\Theta$, proves the Schur pinch mechanism, and then discharges the Schur bound via the hybrid certification outlined above.

\section{Definitions and main objects}\label{sec:defs}

This section defines the analytic objects used throughout the proof and records the basic relationships between zeros of $\zeta$ and the bounded-real (Schur/Herglotz) structure.
All definitions in this section are classical.


\subsection*{The completed zeta function and the far half-plane}
Let $\zeta(s)$ denote the Riemann zeta function.
We write $\xi(s)$ for the completed zeta function
\[
  \xi(s)\ :=\ \tfrac12\,s(s-1)\,\pi^{-s/2}\Gamma(s/2)\,\zeta(s),
\]
which is entire and satisfies the functional equation $\xi(s)=\xi(1-s)$; see \cite{Titchmarsh}.
\textcolor{blue}{Note that the prefactor $s(s-1)$ cancels the pole of $\zeta$ at $s=1$ (and the $\Gamma(s/2)$ singularity at $s=0$), so $\xi$ is entire and in fact $\xi(0)=\xi(1)=\tfrac12$. In this paper all ``zeros'' refer to zeros of $\zeta$ in $\Omega$.}
We work primarily on the right half-plane
\[
  \Omega\ :=\ \{\,s\in\C:\ \Re s>\tfrac12\,\}.
\]
Theorem~\ref{thm:farfield} concerns the fixed far region $\{\,\Re s\ge 0.6\,\}\subset\Omega$.

\subsection*{The prime-diagonal operator and the regularized determinant}
Let $\PP$ denote the set of primes and write $\ell^2(\PP)$ for the Hilbert space with orthonormal basis $\{e_p\}_{p\in\PP}$.
For $s\in\C$ define the prime-diagonal operator
\[
  A(s):\ell^2(\PP)\to\ell^2(\PP),\qquad A(s)e_p:=p^{-s}e_p.
\]
For $\Re s>1/2$ we have $\|A(s)\|_{\mathrm{HS}}^2=\sum_{p}p^{-2\Re s}<\infty$, so $A(s)$ is Hilbert--Schmidt.
In particular, the regularized determinant $\dettwo(I-A(s))$ is well-defined and holomorphic on $\Omega$; see, e.g., \cite[Ch.~III]{RosenblumRovnyak}.

\subsection*{The arithmetic ratio \texorpdfstring{$\mathcal J$}{J} and the Cayley field \texorpdfstring{$\Theta$}{Theta}}
The central meromorphic object is an arithmetic ratio $\mathcal J(s)$ whose poles capture zeros of $\zeta$ in $\Omega$.
To allow numerically stable certified bounds, we permit a holomorphic nonvanishing \emph{normalizer} (or \emph{gauge}) $\mathcal O$ on the region under discussion and define
\begin{equation}\label{eq:J-def}
  \mathcal{J}(s)\ :=\ \frac{\dettwo(I-A(s))}{\zeta(s)}\cdot \frac{s-1}{s}\cdot \frac{1}{\mathcal O(s)},
\end{equation}
\textcolor{blue}{Referee note (computational interface): whenever a nontrivial gauge $\mathcal O$ is used (e.g.\ $\mathcal O=\mathcal O_{\rm proj}$),
the artifact bundle must include an explicit certificate that $\mathcal O$ is holomorphic and nonvanishing on the exact domain where the Schur bound is claimed,
since Schur bounds are not gauge-invariant.}

where $\mathcal O$ is chosen so that it is holomorphic and nonvanishing on the region where \eqref{eq:J-def} is used.
Unless explicitly stated otherwise, we work in the \emph{raw $\zeta$-gauge} $\mathcal O\equiv 1$ and denote the resulting objects by $\mathcal J_{\rm raw}$ and $\Theta_{\rm raw}$.
For readability we usually drop the subscript and simply write $\mathcal J$ and $\Theta$ in this default gauge.
On compact regions one may also divide by an auxiliary holomorphic nonvanishing normalizer to improve conditioning; when we do so we write $\mathcal J_{\rm proj}$ and $\Theta_{\rm proj}$.
Since Schur bounds are \emph{not} gauge-invariant, we keep this notation explicit whenever a certified bound is quoted or invoked in the pinch argument.
On any region where the auxiliary normalizer is nonvanishing, such a gauge change does not affect the pole set of $\mathcal J$ (hence does not change which points correspond to zeros of $\zeta$).

\begin{remark}[Role of the normalizer]\label{rem:Ocan-role}
The factor $\mathcal O$ serves only to choose a convenient gauge for $\mathcal J$.
Provided $\mathcal O$ is holomorphic and nonvanishing on a region $D\subset\Omega$, it cannot introduce poles of $\mathcal J$ on $D$.
In particular, in the raw $\zeta$-gauge $\mathcal O\equiv 1$ one has $\mathcal J(s)\to 1$ and hence $\Theta(s)\to 1/3$ as $\Re s\to+\infty$.
\end{remark}

The associated Cayley transform is
\begin{equation}\label{eq:Theta-def}
  \Theta(s)\ :=\ \frac{2\mathcal J(s)-1}{2\mathcal J(s)+1}.
\end{equation}
Heuristically, $\mathcal J$ plays the role of a Herglotz-type quantity and $\Theta$ the role of the corresponding Schur function.
The proof uses the following simple implication: a Schur bound on $\Theta$ prevents poles of $\mathcal J$ by a removability pinch.

\begin{remark}[Zeros of $\zeta$ produce poles of $\mathcal J$]\label{rem:poles}
If $\rho\in\Omega$ is a zero of $\zeta(s)$, then $\rho$ is a pole of $\mathcal J(s)$ provided the numerator factors in \eqref{eq:J-def} are nonzero at $\rho$.
For $\Re\rho>1/2$ one has $\dettwo(I-A(\rho))\neq 0$: for diagonal $A(s)$,\textcolor{blue}{(Referee note: the identity $\det_2(I-A)=\prod_n (1-\lambda_n)e^{\lambda_n}$ for Hilbert--Schmidt diagonal $A$ with eigenvalues $\lambda_n$ is standard; please cite a precise reference or lemma in \cite[Ch.~III]{RosenblumRovnyak}. Convergence is absolute when $\sum_n |\lambda_n|^2<\infty$.)}
$\dettwo(I-A(s))=\prod_{p}(1-p^{-s})\,e^{p^{-s}}$ and $\sum_{p}|\log(1-p^{-s})+p^{-s}|<\infty$ on $\Omega$; in particular $\dettwo(I-A(s))$ is holomorphic and zero-free on $\Omega$.
\begin{greentext}
\noindent\textbf{Author revision (reference for the $\det_2$ product formula).}
For regularized determinants on Hilbert--Schmidt operators (including the diagonal product formula and its absolute convergence under $\sum_n|\lambda_n|^2<\infty$), see \cite[Ch.~III]{RosenblumRovnyak}; a standard trace-ideal reference is \cite[Ch.~9]{SimonTrace}.
\end{greentext}
Also $\mathcal O(\rho)\neq 0$ by the nonvanishing assumption on the chosen gauge.
Thus zeros of $\zeta$ in $\Omega$ correspond to poles of $\mathcal J$, and hence to points where $\Theta$ cannot extend holomorphically unless the pole is ruled out.
\end{remark}

\subsection*{Schur and Herglotz classes (terminology)}
Let $D\subset\C$ be a domain.
A holomorphic function $\Theta$ on $D$ is called \emph{Schur} if $|\Theta|\le 1$ on $D$.
A holomorphic function $H$ on $D$ is called \emph{Herglotz} if $\Re H\ge 0$ on $D$.
The Cayley transform identifies these classes: if $H$ is Herglotz and $H\not\equiv -1$, then
\[
  \Theta=\frac{H-1}{H+1}
\]
is Schur.
Conversely, if $\Theta$ is Schur and $\Theta\not\equiv 1$, then $(1+\Theta)/(1-\Theta)$ is Herglotz; see \cite{Donoghue,RosenblumRovnyak}.

\subsection*{Outline of the far-field strategy in this language}
Theorem~\ref{thm:farfield} will follow once we establish that $\Theta$ is Schur on $\{\,\Re s>0.6\,\}$.
Indeed, if $|\Theta|\le 1$ holds on $\{\,\Re s>0.6\,\}$ away from the poles of $\mathcal J$, then boundedness forces removability across any isolated singularity.
Since poles of $\mathcal J$ correspond to zeros of $\zeta$ in $\Omega$ (Remark~\ref{rem:poles}), this prevents zeros of $\zeta$ in the far region.
The precise pinch argument is proved in the next section.

\section{Schur/Herglotz pinch mechanism}\label{sec:pinch}

This section records the analytic mechanism that converts a Schur bound for the Cayley field $\Theta$ into a zero-free region for $\zeta$.
The key point is simple: a holomorphic function bounded by $1$ cannot have a pole, and any isolated singularity is removable.
In our setting, poles of $\mathcal J$ in $\Omega$ encode zeros of $\zeta$ (Remark~\ref{rem:poles}), so a Schur bound forces those zeros to be absent.

\subsection*{Removable singularities under a Schur bound}
\begin{lemma}[Removable singularity under Schur bound]\label{lem:removable-schur-p1}
Let $D\subset\C$ be a disc centered at $\rho$ and let $\Theta$ be holomorphic on $D\setminus\{\rho\}$ with $|\Theta|<1$ there.
Then $\Theta$ extends holomorphically to $D$.
In particular, the Cayley inverse $(1+\Theta)/(1-\Theta)$ extends holomorphically to $D$ and has nonnegative real part on $D$.
\end{lemma}
\begin{proof}
Since $\Theta$ is bounded on the punctured disc $D\setminus\{\rho\}$, Riemann's removable singularity theorem yields a holomorphic extension of $\Theta$ to $D$.
Where $|\Theta|<1$, the Möbius map $w\mapsto (1+w)/(1-w)$ sends the unit disc into the right half-plane, hence $\Re\frac{1+\Theta}{1-\Theta}\ge 0$ on $D\setminus\{\rho\}$; continuity extends the inequality across $\rho$.
\end{proof}

\subsection*{From a Schur bound to absence of poles}
We will use Lemma~\ref{lem:removable-schur-p1} in the following form: if $\Theta$ is Schur on a domain $U$ and holomorphic on $U\setminus S$ where $S$ is a discrete set, then $\Theta$ extends holomorphically across $S$ and remains Schur on all of $U$.
Thus a Schur bound rules out poles of any meromorphic object that can be expressed as a Cayley inverse of $\Theta$.

\begin{corollary}[Schur bound prevents poles of $\mathcal J$]\label{cor:no-poles}
Let $U\subset\Omega$ be a domain and suppose that $\Theta$ is meromorphic on $U$ and satisfies $|\Theta|\le 1$ on $U$ away from its poles.
\textcolor{blue}{Referee clarification: this hypothesis is understood pointwise on punctured neighborhoods, i.e. for each pole candidate $\rho$ the bound holds on a punctured disc $D(\rho,r)\setminus\{\rho\}$; hence $\Theta$ is bounded near $\rho$ and the singularity is removable. (Equivalently one may assume $\Theta$ is holomorphic on $U\setminus S$ for some discrete $S$ and $|\Theta|\le 1$ there.)}
\textcolor{blue}{Assume additionally that $\Theta$ is not identically $1$ on any connected component of $U$.}
Then $\Theta$ extends holomorphically to $U$ and satisfies $|\Theta|\le 1$ on $U$.
Moreover, the Cayley inverse
\[
  2\mathcal J \;=\; \frac{1+\Theta}{1-\Theta}
\]
extends holomorphically to $U$ with $\Re(2\mathcal J)\ge 0$ on $U$; in particular $\mathcal J$ has no poles in $U$.
\end{corollary}
\begin{proof}
The poles of a meromorphic function form a discrete subset of $U$.
On each punctured disc around a pole, $\Theta$ is bounded by $1$, hence removable by Lemma~\ref{lem:removable-schur-p1}.
Therefore $\Theta$ extends holomorphically across all its poles and is holomorphic on $U$.
The Schur bound persists by continuity.
The Cayley inverse is holomorphic wherever $\Theta\neq 1$ and has nonnegative real part on $U$.
If $|\Theta(s_0)|=1$ at some interior point $s_0\in U$, then $|\Theta|$ attains its maximum at an interior point, so $\Theta$ is constant of unimodular value on the connected component of $U$ containing $s_0$ (Maximum Modulus Principle). In particular, if $\Theta(s_0)=1$ then $\Theta\equiv 1$ on that component.
\textcolor{blue}{The added condition rules out $\Theta\equiv 1$, so on each component one has $|\Theta|<1$ everywhere.}
In the applications below this is excluded (e.g.\ on any right half-plane $U$, Remark~\ref{rem:Ocan-role} gives $\Theta(s)\to \tfrac13$ as $\Re s\to+\infty$), hence $\Theta\neq 1$ on $U$ and the Cayley inverse extends holomorphically to $U$ with $\Re(2\mathcal J)\ge 0$.
In particular $\mathcal J$ has no poles in $U$.
\end{proof}

\subsection*{Conclusion: Schur on the far half-plane implies Theorem~\ref{thm:farfield}}
We now connect the pinching mechanism to $\zeta$.
By Remark~\ref{rem:poles}, any zero $\rho$ of $\zeta$ in $\Omega$ produces a pole of $\mathcal J$ in $\Omega$ (the numerator factors in \eqref{eq:J-def} are nonzero on $\Omega$).
\textcolor{blue}{Referee check: I think that we should verify explicitly that the working domain $\Omega$ stays away from $s=0$ and $s=1$, and that the chosen gauge $\mathcal O$ is holomorphic and nonvanishing on $\Omega$ (this must be certified whenever $\mathcal O\neq 1$). I am trying to catch all potential issues that must be addressed and then fix them all.}
\begin{greentext}
\noindent\textbf{Author revision (domain/gauge sanity check).}
On $\Omega=\{\Re s>\tfrac12\}$ we have $0\notin\Omega$, so the compensator $(s-1)/s$ introduces no pole on the working domain.
The point $s=1$ lies in $\Omega$ but the factor $(s-1)$ cancels the simple pole of $\zeta$ there, so $\mathcal J$ is holomorphic at $s=1$ in the raw gauge.
Whenever a nontrivial gauge $\mathcal O$ is introduced, the manuscript treats \rev{``$\mathcal O$ is holomorphic and nonvanishing on the stated domain''} as a load-bearing hypothesis (and in the computational cross-checks, as a separately auditable certificate).
\end{greentext}

Therefore, if we can certify a Schur bound for $\Theta$ on a half-plane $U_\varepsilon=\{\,\Re s>0.6-\varepsilon\,\}$ with some $\varepsilon>0$, Corollary~\ref{cor:no-poles} implies $\mathcal J$ has no poles in $U_\varepsilon$, hence $\zeta$ has no zeros in $U_\varepsilon$.
Since $\{\,\Re s\ge 0.6\,\}\subset U_\varepsilon$, this yields Theorem~\ref{thm:farfield}.
\rev{The next section discharges the Schur bound on $\Omega\setminus Z(\zeta)$ by a boundary-certificate route and then specializes to $U_\varepsilon$.}

\section{All-heights Schur bound via a boundary wedge certificate}\label{sec:hybrid}
We now discharge the Schur bound required in Corollary~\ref{cor:no-poles} on a half-plane $U_\varepsilon$.
The key input is an unconditional \emph{boundary wedge} \textup{(P+)} for a suitably outer-normalized version of $\mathcal J$ on the boundary line $\Re s=\tfrac12$.
This route is analytic (no large-height asymptotics) and applies for all heights.

\subsection*{Outer normalization on $\Re s=\tfrac12$}
Define
\[
  F(s)\ :=\ \frac{\dettwo(I-A(s))}{\zeta(s)}\cdot \frac{s-1}{s}.
\]

\begin{bluetext}
\begin{greentext}
\begin{lemma}[Boundary admissibility and Smirnov class for $F$]\label{lem:F-boundary-admissible}
Let \(F(s)=\dettwo(I-A(s))\,\zeta(s)^{-1}\cdot \frac{s-1}{s}\), holomorphic on \(\Omega\setminus Z(\zeta)\).
Then:
\begin{itemize}
\item \(F\) belongs to the Smirnov class \(N^+(\Omega\setminus Z(\zeta))\) (interpreted componentwise), hence admits Lebesgue-a.e.\ nontangential boundary values \(F^*(t)\) on \(\Re s=\tfrac12\).
\item \(u(t):=\log|F^*(t)|\in L^1_{\mathrm{loc}}(\mathbb R)\).
\end{itemize}
Moreover, if in addition \(|u(t)|\le C\log(2+|t|)\) for \(|t|\ge 1\), then \(u\in L^1(\mathbb R,(1+t^2)^{-1}dt)\).
\end{lemma}

\begin{proof}
\noindent\textbf{Proof (Appendix citations).}
Throughout, ``a.e.'' refers to Lebesgue-a.e.\ on \(\mathbb R\).

\smallskip
\noindent\emph{Smirnov / bounded-type input.}
Appendix Lemmas~\ref{lem:carleson-arith}, \ref{lem:carleson-xi}, and \ref{lem:annular-balayage} provide the Whitney/Carleson control of the relevant box energies,
and Lemma~\ref{lem:boundedtype-from-carleson} records the standard Carleson-energy \(\Rightarrow\) bounded-characteristic implication (with references).
Applying these on each connected component of \(\Omega\setminus Z(\zeta)\) yields that \(F\) has bounded characteristic there, hence \(F\in N^+(\Omega\setminus Z(\zeta))\) and therefore admits Lebesgue-a.e.\ nontangential boundary limits; see \cite[Ch.~II]{Garnett}.

\smallskip
\noindent\emph{Local integrability of the boundary log-modulus.}
Appendix Lemmas~\ref{lem:det2-unsmoothed}, \ref{lem:xi-deriv-L1}, and \ref{lem:desmooth-L1} give \(u_\varepsilon\to u\) in \(L^1_{\mathrm{loc}}(\mathbb R)\) for
\(u_\varepsilon(t)=\log|\dettwo(I-A(\tfrac12+\varepsilon+it))|-\log|\xi(\tfrac12+\varepsilon+it)|\).
Writing \(\xi(s)=\tfrac12 s(s-1)\pi^{-s/2}\Gamma(\tfrac s2)\,\zeta(s)\), we have on \(\Omega\setminus Z(\zeta)\)
\[
  F(s)\ =\ \frac{\dettwo(I-A(s))}{\xi(s)}\cdot \Big(\tfrac12 s(s-1)\pi^{-s/2}\Gamma(\tfrac s2)\Big)\cdot \frac{s-1}{s}.
\]
On the boundary line \(\Re s=\tfrac12\) one has \(\big|\frac{s-1}{s}\big|=1\), and the completion factor has locally integrable boundary log-modulus (in fact \(\lesssim \log(2+|t|)\) by Stirling),
so \(\log|F^*(t)|\in L^1_{\rm loc}(\mathbb R)\).
The weighted integrability under logarithmic growth is immediate since \(\int_{|t|\ge 1}\frac{\log(2+|t|)}{1+t^2}\,dt<\infty\).
\end{proof}
\end{greentext}
\begin{lemma}[Outer factor from boundary modulus on $\Omega$]\label{lem:outer-factor-halfplane}
Assume Lemma~\ref{lem:F-boundary-admissible} (together with the weighted integrability conclusion there).
Then there exists a holomorphic function $\mathcal O_\zeta$ on $\Omega$, unique up to a unimodular constant,
with no zeros on $\Omega$, such that the nontangential boundary values satisfy
\[
  \big|\mathcal O_\zeta(\tfrac12+it)\big|=\big|F^*(t)\big|\qquad\text{for a.e.\ }t\in\mathbb{R}.
\]
Moreover, $\log|\mathcal O_\zeta(s)|$ is the Poisson extension of $u(t)$ from the boundary line $\Re s=\tfrac12$.
\end{lemma}

\begin{proof}[Proof sketch (standard)]

\begin{greentext}\noindent\textbf{Author revision (citation).} A standard half-plane outer-function construction is given in \cite[Ch.~II]{Garnett} (see also \cite[Ch.~2]{RosenblumRovnyak}).\end{greentext}

Translate $\Omega$ to the right half-plane $\{\,\Re w>0\,\}$ via $w=s-\tfrac12$ and apply the classical outer-function
construction for half-planes/discs: prescribe the harmonic function $U=\mathcal P[u]$ (Poisson extension) and set
$\mathcal O_\zeta=\exp(U+iV)$ where $V$ is a harmonic conjugate.
Then $\mathcal O_\zeta$ is zero-free and has a.e.\ boundary modulus $e^{u(t)}$ by Fatou theory.
See, e.g., Garnett \cite{Garnett} or Rosenblum--Rovnyak \cite{RosenblumRovnyak}.
\end{proof}
\end{bluetext}

\begin{bluetext}
Assuming Lemma~\ref{lem:F-boundary-admissible} (together with the weighted integrability conclusion there), Lemma~\ref{lem:outer-factor-halfplane} provides an outer function $\mathcal O_\zeta$ on $\Omega$ whose a.e.\ boundary modulus satisfies
\end{bluetext}

\[
  \big|\mathcal O_\zeta(\tfrac12+it)\big|\ =\ \big|F(\tfrac12+it)\big|\qquad\text{for a.e.\ }t\in\R.
\]
Set the outer-normalized ratio
\begin{equation}\label{eq:J-out}
  \mathcal J_{\rm out}(s)\ :=\ \frac{F(s)}{\mathcal O_\zeta(s)}
  \ =\ \frac{\dettwo(I-A(s))}{\mathcal O_\zeta(s)\,\zeta(s)}\cdot \frac{s-1}{s}.
\end{equation}
Then $|\mathcal J_{\rm out}(\tfrac12+it)|=1$ for a.e.\ $t$.
Define its Cayley field
\[
  \Theta_{\rm out}(s)\ :=\ \frac{2\mathcal J_{\rm out}(s)-1}{2\mathcal J_{\rm out}(s)+1}.
\]

\subsection*{Boundary wedge (P+)}
Let $w(t):=\Arg \mathcal J_{\rm out}(\tfrac12+it)$ be the boundary phase (defined for a.e.\ $t$).
We say that \textup{(P+)} holds if there exists $m\in\R$ such that
\[
  |w(t)-m|\ <\ \frac{\pi}{2}\qquad\text{for a.e.\ }t\in\R.
\]
Equivalently, $\Re\big(e^{-im}\mathcal J_{\rm out}(\tfrac12+it)\big)\ge 0$ for a.e.\ $t$.

\begin{theorem}[\rev{Boundary wedge \textup{(P+)}}]\label{thm:Pplus}
\rev{The boundary wedge \textup{(P+)} holds for $\mathcal J_{\rm out}$.}
\end{theorem}
\begin{proof}
See Appendix~\ref{app:pplus-proof}, where we include a complete proof of the quantitative boundary certificate (phase--velocity identity, CR--Green pairing on Whitney boxes, unconditional Carleson/box-energy bounds, and the quantitative wedge criterion).
\end{proof}
\begin{greentext}
\noindent\textbf{Author revision (Theorem~\ref{thm:Pplus}).}
The Appendix proof of \textup{(P+)} uses Whitney-local oscillation control and the local-to-global wedge lemma to obtain a \emph{Lebesgue-a.e.} wedge statement directly.
Accordingly, \rev{the (AC$_\mu$) hypothesis has been removed and Theorem~\ref{thm:Pplus} is stated unconditionally.}
\end{greentext}

\begin{greentext}
\noindent\textbf{Author revision.}
Appendix~\ref{app:pplus-proof} proves \textup{(P+)} as a Lebesgue-a.e.\ statement directly (via Whitney-local oscillation control), so no $\mu$-a.e.\ upgrade is invoked in the proof of Theorem~\ref{thm:Pplus}.
\end{greentext}

\begin{bluetext}
\begin{lemma}[Smirnov regularity for $\mathcal J_{\rm out}$ and $\Theta_{\rm out}$]\label{lem:smirnov-regularity}

Assume Lemma~\ref{lem:F-boundary-admissible} and Lemma~\ref{lem:outer-factor-halfplane}, and define
$\mathcal J_{\rm out}(s):=F(s)/\mathcal O_\zeta(s)$ on $\Omega$.
Then $\mathcal J_{\rm out}$ belongs to $N^+(\Omega\setminus Z(\zeta))$ and admits nontangential boundary values
$\mathcal J_{\rm out}(\tfrac12+it)$ for a.e.\ $t$.
Consequently, its Cayley field
\[
  \Theta_{\rm out}(s):=\frac{2\mathcal J_{\rm out}(s)-1}{2\mathcal J_{\rm out}(s)+1}
\]
also lies in $N^+(\Omega\setminus Z(\zeta))$ and admits Lebesgue-a.e.\ boundary values.
\end{lemma}

\begin{proof}
By Lemma~\ref{lem:F-boundary-admissible} we have $F\in N^+(\Omega\setminus Z(\zeta))$.
By Lemma~\ref{lem:outer-factor-halfplane}, $\mathcal O_\zeta$ is outer on $\Omega$ and hence lies in $N^+(\Omega)$, is zero-free, and has a.e.\ boundary values.
Therefore the quotient $\mathcal J_{\rm out}=F/\mathcal O_\zeta$ belongs to $N^+(\Omega\setminus Z(\zeta))$.
The a.e.\ existence of boundary values follows from the Smirnov boundary theory for $N^+$.
Since $\Theta_{\rm out}$ is a rational function of $\mathcal J_{\rm out}$ with no singularities except where $2\mathcal J_{\rm out}=-1$,
it likewise belongs to $N^+(\Omega\setminus Z(\zeta))$ and has a.e.\ boundary values (excluding at most a null set where the denominator vanishes in the boundary trace).
\end{proof}
\begin{lemma}[Boundary-to-interior Schur transport on $\Omega$]\label{lem:schur-transport-omega}
Let $\Theta\in N^+(\Omega)$ admit nontangential boundary values $\Theta(\tfrac12+it)$ for a.e.\ $t$.
If $|\Theta(\tfrac12+it)|\le 1$ for a.e.\ $t$, then $|\Theta(s)|\le 1$ for all $s\in\Omega$. \textcolor{blue}{(Referee note: when used for $\Omega\setminus Z(\zeta)$, apply this on each connected component.)}
\end{lemma}
\begin{proof}[Proof sketch (standard)]

\begin{greentext}\noindent\textbf{Author revision (citation).} This boundary-to-interior Schur transport is standard for $N^+$/$H^p$ boundary theory; see \cite[Ch.~II]{Garnett} and \cite[Ch.~2]{RosenblumRovnyak}.\end{greentext}

For $\Theta\in N^+(\Omega)$ the subharmonic function $\log|\Theta|$ admits a harmonic majorant on $\Omega$.
At boundary Lebesgue points where the nontangential limit exists, the boundary inequality gives
$\log|\Theta(\tfrac12+it)|\le 0$ a.e.
Applying the Poisson domination principle for subharmonic functions yields $\log|\Theta(s)|\le 0$ in $\Omega$,
hence $|\Theta(s)|\le 1$.
References: Garnett \cite{Garnett}, Rosenblum--Rovnyak \cite{RosenblumRovnyak}.
\end{proof}
\end{bluetext}



\subsection*{From (P+) to a Schur bound on \texorpdfstring{$\Omega\setminus Z(\zeta)$}{Omega minus zeros}}
\begin{proposition}[Herglotz/Schur transport]\label{prop:herglotz-schur-transport}
\rev{Assume \textup{(P+)} for $\mathcal J_{\rm out}$ holds for Lebesgue-a.e.\ boundary $t$ on $\Re s=\tfrac12$ (e.g.\ by Theorem~\ref{thm:Pplus}, proved in Appendix~\ref{app:pplus-proof}).}

\begin{bluetext}
Assume in addition Lemma~\ref{lem:smirnov-regularity} (Smirnov boundary regularity for $\mathcal J_{\rm out}$).
\end{bluetext}

Then $2e^{-im}\mathcal J_{\rm out}$ is Herglotz on $\Omega\setminus Z(\zeta)$ and $\Theta_{\rm out}$ is Schur on $\Omega\setminus Z(\zeta)$. \textcolor{blue}{(Precisely: on each connected component, and $\Theta_{\rm out}$ is holomorphic off the discrete set where $2\mathcal J_{\rm out}=-1$.)}
\end{proposition}
\begin{proof}

\begin{bluetext}
On the boundary line $\Re s=\tfrac12$, the wedge condition \textup{(P+)} implies

\begin{bluetext}\noindent\emph{Provenance:} \textup{(P+)} is Theorem~\ref{thm:Pplus} in the main text, proved in Appendix Theorem~\ref{thm:pplus-proof-complete}.\end{bluetext}

\begin{greentext}
\noindent\textbf{Author revision (transport dependency).}
Theorem~\ref{thm:Pplus} is proved as a \emph{Lebesgue-a.e.} wedge statement directly in Appendix~\ref{app:pplus-proof}, so no $\mu$-a.e.\ upgrade is required for the boundary-to-interior transport here.
\end{greentext}

\[
  \Re\big( e^{-im}\mathcal J_{\rm out}(\tfrac12+it)\big)\ge 0 \qquad\text{for a.e.\ }t\in\mathbb{R}.
\]
Since $|\mathcal J_{\rm out}(\tfrac12+it)|=1$ for a.e.\ $t$, this is equivalent to
\[
  \Re\big(2e^{-im}\mathcal J_{\rm out}(\tfrac12+it)\big)\ge 0 \qquad\text{for a.e.\ }t\in\mathbb{R}.
\]
Define $H(s):=2e^{-im}\mathcal J_{\rm out}(s)$ on $\Omega\setminus Z(\zeta)$ and its Cayley transform
\[
  \Theta_H(s):=\frac{H(s)-1}{H(s)+1}.
\]
Noting that $H=2e^{-im}\mathcal J_{\rm out}$ differs from $2\mathcal J_{\rm out}$ by a unimodular constant,
we have $\Theta_H=\Theta_{\rm out}$.

By Lemma~\ref{lem:smirnov-regularity}, $\Theta_{\rm out}\in N^+(\Omega\setminus Z(\zeta))$ admits a.e.\ boundary values.

\begin{bluetext}
\noindent\textbf{Referee linkage:} The boundary inequality $\Re H(\tfrac12+it)\ge 0$ is asserted only for Lebesgue-a.e.\ $t$
as a direct reformulation of \textup{(P+)} on $\mathcal J_{\rm out}$ (via $H=2e^{-im}\mathcal J_{\rm out}$).
\end{bluetext}

On $\Re s=\tfrac12$, the boundary inequality $\Re H(\tfrac12+it)\ge 0$ implies
$|\Theta_{\rm out}(\tfrac12+it)|\le 1$ for a.e.\ $t$ (since the Cayley map sends the closed right half-plane into the closed unit disc).
Applying Lemma~\ref{lem:schur-transport-omega} on each connected component of $\Omega\setminus Z(\zeta)$ yields
\[
  |\Theta_{\rm out}(s)|\le 1 \qquad (s\in \Omega\setminus Z(\zeta)).
\]
Finally, on $\Omega\setminus Z(\zeta)$ the Cayley inverse
$H=(1+\Theta_{\rm out})/(1-\Theta_{\rm out})$ is holomorphic wherever $\Theta_{\rm out}\neq 1$.
But $\Theta_{\rm out}(s)=1$ is algebraically impossible for finite $\mathcal J_{\rm out}(s)$, since
\[
\frac{2\mathcal J_{\rm out}-1}{2\mathcal J_{\rm out}+1}=1 \ \Longrightarrow\ -1=1.
\]
Therefore $\Re H(s)\ge 0$ for all $s\in\Omega\setminus Z(\zeta)$, i.e.\ $2e^{-im}\mathcal J_{\rm out}$ is Herglotz there,
and $\Theta_{\rm out}$ is Schur there.
\end{bluetext}

\end{proof}


\begin{proof}[Proof of Theorem~\ref{thm:farfield}]
By Proposition~\ref{prop:herglotz-schur-transport}, $\Theta_{\rm out}$ is Schur on $\Omega\setminus Z(\zeta)$.
In particular, on the half-plane $U_\varepsilon=\{\,\Re s>0.6-\varepsilon\,\}$ it satisfies $|\Theta_{\rm out}|\le 1$ away from the poles of $\mathcal J_{\rm out}$.
Since $\Theta_{\rm out}=(2\mathcal J_{\rm out}-1)/(2\mathcal J_{\rm out}+1)$, it is algebraically impossible that $\Theta_{\rm out}\equiv 1$ on any connected component.
Therefore Corollary~\ref{cor:no-poles} applies on $U_\varepsilon$ and shows that $\mathcal J_{\rm out}$ has no poles on $U_\varepsilon$.
As $\dettwo(I-A)$ and $\mathcal O_\zeta$ are holomorphic and nonvanishing on $\Omega$, poles of $\mathcal J_{\rm out}$ in $\Omega$ can only come from zeros of $\zeta$.
Hence $\zeta$ has no zeros in $U_\varepsilon$, and therefore none in $\{\,\Re s\ge 0.6\,\}$.
\end{proof}

\begin{table}[H]
\centering
\caption{\textcolor{blue}{Supplementary computational artifacts (not used in the proof).}}\label{tab:artifact-data}
\small
\begin{tabular}{l l l}
\toprule
\textbf{Artifact} & \textbf{Parameter} & \textbf{Value} \\
\midrule
\multicolumn{3}{l}{\textit{Rectangle certification} (\texttt{theta\_certify})} \\
\quad Domain & $[\sigma_{\min}, \sigma_{\max}] \times [t_{\min}, t_{\max}]$ & $[0.6, 0.7] \times [0, 20]$ \\
\quad Certified upper bound & $\max |\Theta_{\rm proj}|$ & $0.9999928763$ \\
\quad Safety margin & $1 - \theta_{\rm hi}$ & $7.12 \times 10^{-6}$ \\
\quad Status & \texttt{ok} & \texttt{true} \\
\quad Boxes processed & & 380{,}764 \\
\quad Precision & (bits) & 260 \\
\quad Gauge & & \texttt{outer\_zeta\_proj} \\
\midrule
\multicolumn{3}{l}{\textit{Pick certificate} (\texttt{pick\_certify}, $\sigma_0 = 0.599$)} \\
\quad Matrix size & $N$ & 16 \\
\quad Spectral gap & $\delta_{\rm cert}$ & $0.594$ \\
\quad SPD at origin & $P_N \succ 0$ & \texttt{true} \\
\quad Coefficient count & $N_{\rm coeff}$ & 128 \\
\quad Tail sum (diagnostic) & $\sum_{16}^{127}|a_n|$ & $0.67$ \\
\quad Gauge & & \texttt{raw\_zeta} \\
\midrule
\multicolumn{3}{l}{\textit{Pick certificate} (\texttt{pick\_certify}, $\sigma_0 = 0.6$)} \\
\quad Matrix size & $N$ & 16 \\
\quad Spectral gap & $\delta_{\rm cert}$ & $0.594$ \\
\quad SPD at origin & $P_N \succ 0$ & \texttt{true} \\
\quad Coefficient count & $N_{\rm coeff}$ & 128 \\
\quad Gauge & & \texttt{raw\_zeta} \\
\midrule
\multicolumn{3}{l}{\textit{Pick certificate} (\texttt{pick\_certify}, $\sigma_0 = 0.7$)} \\
\quad Matrix size & $N$ & 16 \\
\quad Spectral gap & $\delta_{\rm cert}$ & $0.627$ \\
\quad SPD at origin & $P_N \succ 0$ & \texttt{true} \\
\quad Coefficient count & $N_{\rm coeff}$ & 128 \\
\quad Gauge & & \texttt{raw\_zeta} \\
\bottomrule
\end{tabular}
\end{table}

\begin{remark}[Artifact reproducibility and verification]\label{rem:artifact-repro}
The artifacts in Table~\ref{tab:artifact-data} are generated by the verifier script \texttt{scripts/verify\_attachment\_arb.py} using ARB ball arithmetic (via \texttt{python-flint}).
They are provided as independent numerical corroboration on representative low-height domains.
They do not enter the all-heights boundary-certificate proof in Section~\ref{sec:hybrid}.
\end{remark}

\section*{Conclusion and limitations (unconditional status)}

We have proved an unconditional, fixed half-plane zero-free region for the Riemann zeta function: $\zeta(s)\neq 0$ for $\Re s\ge 0.6$ (Theorem~\ref{thm:farfield}).
The argument is function-theoretic: zeros are converted into poles of an arithmetic ratio $\mathcal J$, and a Schur bound $|\Theta|\le 1$ for the associated Cayley field forces removability and rules out poles (hence zeros).
The only ``hard'' step is establishing the all-heights Schur bound, which is discharged by the boundary wedge certificate \textup{(P+)} (Section~\ref{sec:hybrid}).
\textcolor{blue}{The supplementary artifacts in Table~\ref{tab:artifact-data} provide independent numerical corroboration on low-height regions but are not used in the proof.}

\paragraph{Computer assistance and auditability.}
Although the proof is analytic, the repository also provides rigorous numerical artifacts (ball arithmetic) as cross-checks, together with a verifier and JSON outputs so that those finite checks can be independently audited.

\paragraph{Limitations and scope.}
We do not claim the Riemann Hypothesis here.
It isolates and certifies a fixed far-field exclusion $\Re s\ge 0.6$.
Pushing the boundary $0.6$ closer to $1/2$ within this framework would require sharpening the analytic boundary-certificate constants and the Carleson/box-energy bounds that enter the wedge criterion, which we do not pursue here.
The companion papers in this series treat (i) effective near-field barriers in the strip $1/2<\Re s<0.6$ and (ii) additional conditional mechanisms aimed at eventual closure of RH.

\section*{Statements and Declarations}

\paragraph{Competing interests.}
The author declares no competing interests.

\paragraph{Data and materials availability.}
\textcolor{blue}{All computational artifacts used for supplementary cross-checks are included in the handoff bundle (and mirrored in the repository):}
\begin{quote}\small\ttfamily
artifacts/theta\_certify\_sigma06\_07\_t0\_20\_outer\_zeta\_proj.json\\
artifacts/pick\_sigma0599\_raw\_zeta\_N16.json\\
artifacts/pick\_sigma06\_raw\_zeta\_N16.json\\
artifacts/pick\_sigma07\_raw\_zeta\_N16.json\\
scripts/verify\_attachment\_arb.py
\end{quote}

\paragraph{Reproducibility.}
The verifier is based on rigorous ball arithmetic (ARB via \texttt{python-flint}) and is intended to be independently auditable.
See Remark~\ref{rem:artifact-repro} and Appendix~\ref{app:audit} for a referee-facing audit manifest (commands and expected outputs).


\begin{bluetext}
\subsection*{Referee-completion checklist for Appendix~\texorpdfstring{\ref{app:pplus-proof}}{(P+)-appendix}}
The logical steps in this paper that invoke ``Smirnov/Hardy'' boundary-to-interior transport and outer normalization
are only as strong as the analytic inputs proved in the Appendix input file \texttt{paper1\_pplus\_proof.tex}.
To make the present manuscript \emph{referee-complete}, the following Appendix statements must be fully proved
with all hypotheses verified in the stated region (and their dependencies made explicit):
\begin{enumerate}
\item \textbf{Phase--velocity identity and quantification:} Appendix Theorem~\ref{thm:phase-velocity-quant}
(and its internal dependencies, notably the principal-value identity \eqref{eq:pv-identity} and Lemma~\ref{lem:outer-phase-HT}).
\item \textbf{CR--Green / cutoff pairing on Whitney boxes:} Appendix Lemma~\ref{lem:cutoff-pairing} and Lemma~\ref{lem:CR-green-phase},
including the upper bound \eqref{eq:CRG-upper-adm}.
\item \textbf{Length-free Carleson/box-energy control:} Appendix Proposition~\ref{prop:length-free},
Definition~\ref{def:adm-bumps}, Lemma~\ref{lem:uniform-test-energy}, and the Carleson lemmas
\ref{lem:carleson-arith}, \ref{lem:carleson-xi}, together with the balayage Lemma~\ref{lem:annular-balayage}.
\item \textbf{\rev{(Not used in the final proof chain).}} The $\mu$-to-Lebesgue exceptional-set upgrade (Appendix Lemma~\ref{lem:mu-to-lebesgue}) is retained only as historical referee commentary.
The Appendix proves \textup{(P+)} directly as a Lebesgue-a.e.\ statement via oscillation control and Lemma~\ref{lem:local-to-global-wedge}.
\item \textbf{Outer normalizer stability for the specific $F$:} Appendix Lemma~\ref{lem:zeta-normalization}
and Lemma~\ref{lem:outer-existence-stability}.
\item \textbf{Final wedge globalization:} Appendix Lemma~\ref{lem:local-to-global-wedge},
Lemma~\ref{lem:whitney-uniform-wedge}, and the concluding Appendix Theorem~\ref{thm:pplus-proof-complete}.
\end{enumerate}
In the current referee notes delivered separately, several of the above are marked \textbf{RED} until their proofs are expanded
to a fully checkable level (distributional boundary differentiation, interchange of limits/sums in $\log\dettwo$, and explicit constants).
\end{bluetext}


\appendix
\section{Proof of the boundary wedge certificate \textup{(P+)}}\label{app:pplus-proof}
\rev{This appendix supplies the proof of Theorem~\ref{thm:Pplus}.}
It is a self-contained analytic argument (no numerical inputs) based on:
(i) a quantitative phase--velocity identity for the boundary phase of $\mathcal J_{\rm out}$,
(ii) a Cauchy--Riemann/Green pairing on Whitney boxes,
(iii) an unconditional Carleson/box-energy bound,
and (iv) a quantitative wedge criterion converting windowed phase control into the a.e.\ wedge.

% ----------------------------------------------------------------------

% Appendix A input file: proof of the boundary wedge certificate (P+).
% This file is \input{} from Appendix~\ref{app:pplus-proof} in paper1_farfield.tex.
% ----------------------------------------------------------------------

\subsection{Statement, standing notation, and domains}

\label{appA:setup}

This subsection fixes the ambient domain, boundary conventions, Whitney geometry, and the meaning of boundary limits, so later phase and energy identities are unambiguous.

Throughout Appendix~\ref{app:pplus-proof} we work in the right half-plane
\[
  \Omega:=\{s\in\mathbb C:\Re s>\tfrac12\},
\]
with boundary line $\partial\Omega=\{\tfrac12+it:t\in\mathbb R\}$.
All analytic objects are understood componentwise on $\Omega\setminus Z$, where $Z$ denotes the relevant zero/pole set,
so that branches of $\log$ and $\Arg$ are well-defined on each connected component.

For a compact interval $I\subset\mathbb R$ and a dilation parameter $\alpha>1$ we write $Q_\alpha(I)$ for the standard Whitney box
based on $I$, and we use the weighted area measure $\sigma\,dt\,d\sigma$ on $\Omega$, where $\sigma:=\Re s-\tfrac12$.

The goal is to prove the boundary wedge certificate \textup{(P+)} stated in Theorem~\ref{thm:Pplus}.
The proof proceeds by:
(i) a quantitative phase--velocity identity for the boundary phase of $\mathcal J_{\rm out}$,
(ii) a Green/Cauchy--Riemann pairing on Whitney boxes,
(iii) a Carleson-type energy bound for a logarithmic derivative,
and (iv) a quantitative wedge criterion converting windowed phase control into a.e.\ wedge inclusion.

In the phase--velocity identity beloww, $-w'$ is a positive distribution (a locally finite measure plus a discrete atomic part)
encoding the off--critical zero data that enter the boundary phase derivative.

\subsection{A quantitative wedge criterion from Whitney-local control}

\label{app:whitney-wedge}

We state the wedge target (P+) in a form suited to local Whitney-box estimates and record the boundary conventions used throughout Appendix~A.

We work on the boundary line $\Re s=\tfrac12$ and use the following conventions.
\begin{itemize}
\item \emph{Wedge.} For an aperture parameter $\alpha\in(0,\tfrac\pi2)$ and a center angle $m\in\mathbb R$, write
\[
  W_{m,\alpha}:=\{z\in\mathbb C:\ |\Arg(e^{-im}z)|\le \alpha\}.
\]
Thus \textup{(P+)} is the Lebesgue-a.e.\ inclusion $\,\mathcal J_{\rm out}(\tfrac12+it)\in W_{m,\alpha}\,$ for some fixed $\alpha<\tfrac\pi2$
and some $m\in\mathbb R$.

\item \emph{Whitney / Carleson boxes.} For an interval $I\subset\mathbb R$, write the Carleson box
$S(I):=\{\tfrac12+\sigma+it:\ 0<\sigma\le |I|,\ t\in I\}$.
A Whitney box means a box of comparable width and height, e.g.\ $\{\tfrac12+\sigma+it:\ \sigma\in[a|I|,b|I|],\ t\in I\}$ with fixed $0<a<b$.

\item \emph{Meaning of ``a.e.''} Unless explicitly stated otherwise, ``a.e.'' refers to Lebesgue measure $dt$ on $\mathbb R$.
\end{itemize}
\begin{lemma}[Outer normalizer from boundary log-modulus]
\label{lem:outer-from-logmodulus}
Let $u\in L^1(\mathbb R,(1+t^2)^{-1}dt)$ be real-valued. Then there exists an outer function $O$ on $\Omega$
(zero-free and holomorphic on $\Omega$) whose nontangential boundary values satisfy
\[
|O(\tfrac12+it)| = e^{u(t)} \quad\text{for a.e. }t\in\mathbb R.
\]
Moreover $O$ is unique up to a unimodular constant.
\end{lemma}

\begin{proof}
Define the Poisson extension $U$ of $u$ to $\Omega$ by
\[
U(\tfrac12+\sigma+it)\ :=\ \frac{1}{\pi}\int_{\mathbb R} u(\tau)\,\frac{\sigma}{\sigma^2+(t-\tau)^2}\,d\tau,
\qquad \sigma>0.
\]
The weighted integrability $u\in L^1(\mathbb R,(1+t^2)^{-1}dt)$ ensures the integral converges and that $U$ is harmonic on $\Omega$.
Let $V$ be a harmonic conjugate of $U$ on $\Omega$ (defined up to an additive constant), and set
\[
O(s)\ :=\ \exp\!\big(U(s)+iV(s)\big).
\]
Then $O$ is holomorphic and zero-free on $\Omega$. Standard boundary theory for Poisson integrals gives that $U(\tfrac12+\varepsilon+it)\to u(t)$
for a.e.\ $t$ as $\varepsilon\downarrow 0$; therefore the nontangential boundary values satisfy $|O(\tfrac12+it)|=e^{u(t)}$ for a.e.\ $t$.
Uniqueness up to unimodular constant follows because the ratio of two such outer functions has a.e.\ boundary modulus $1$ and hence is an inner constant.
(Source.) This is the classical outer-function construction in half-planes; see Duren \emph{$H^p$ Spaces}, Ch.~II, or Garnett \emph{Bounded Analytic Functions}, Ch.~II.
\end{proof}





\subsection{Phase--velocity identity (quantitative form) and boundary passage}

\label{app:phase-velocity}
We establish the boundary phase--velocity relation for the outer-normalized ratio, and record the precise sense in which derivatives and boundary traces are taken.

\begin{lemma}[Outer--Hilbert boundary identity]\label{lem:outer-phase-HT}
Let $u\in L^1_{\mathrm{loc}}(\mathbb R)$ and let $O$ be an outer function on $\Omega$ whose boundary modulus satisfies
$|O(\tfrac12+it)|=e^{u(t)}$ for a.e.\ $t$.
Let $w(t):=\Arg O(\tfrac12+it)$ denote the boundary argument (defined modulo an additive constant).
Then, in \(\mathcal D'(\mathbb R)\),
\[
  \frac{d}{dt}w(t)=\Hilb[u'](t),
\]
where \(\Hilb\) is the boundary Hilbert transform on \(\R\) (as a continuous operator \(\mathcal D'(\R)\to\mathcal D'(\R)\))
and \(u'\) is the distributional derivative.
\end{lemma}
\begin{proof}
Write \(\log O=U+iV\) on \(\Omega\), where \(U=\Re\log O\) is harmonic and \(V=\Im\log O\) is its harmonic conjugate
(on each component of \(\Omega\setminus Z(O)\), fixing a branch of \(\log\)).
The boundary trace satisfies \(U(\tfrac12+\cdot)=u\) in \(\mathcal D'(\R)\), and the conjugate boundary trace is
\(V(\tfrac12+\cdot)=\Hilb[u]\) in \(\mathcal D'(\R)\) (up to an additive constant).
Differentiating in \(t\) in the sense of distributions gives
\[
  \frac{d}{dt}\Arg O(\tfrac12+it)=\partial_t V(\tfrac12+it)=\Hilb[\partial_t u](t)=\Hilb[u'](t),
\]
since differentiation commutes with \(\Hilb\) on \(\mathcal D'(\R)\).
\end{proof}

\begin{lemma}[Smoothed distributional bound for \(\partial_\sigma\,\Re\log\dettwo\)]\label{lem:det2-unsmoothed}
Let \(I\Subset\R\) be a compact interval and fix \(\varepsilon_0\in(0,\tfrac12]\).
There exists a finite constant
\[
  C_*\ :=\ \sum_{p}\sum_{k\ge 2}\frac{p^{-k/2}}{k^2\,\log p}\ <\ \infty
\]
such that for all \(\sigma\in(\tfrac12,\tfrac12+\varepsilon_0]\) and every \(\varphi\in C_c^2(I)\),
\[
  \Big|\int_{\R} \varphi(t)\,\partial_\sigma\Re\log\dettwo\big(I-A(\sigma+it)\big)\,dt\Big|\ \le\ C_*\,\|\varphi''\|_{L^1(I)}.
\]
\end{lemma}
\begin{proof}
For \(\sigma>\tfrac12\) one has the absolutely convergent expansion
\[
  \partial_\sigma\,\Re\log\dettwo\big(I-A(\sigma+it)\big)
  \;=\; \sum_{p}\sum_{k\ge 2} (\log p)\,p^{-k\sigma}\cos(k t\log p).
\]
For each frequency \(\omega=k\log p\ge 2\log 2\), two integrations by parts give
\[
  \Big|\int_{\R}\!\varphi(t)\cos(\omega t)\,dt\Big|\ \le\ \frac{\|\varphi''\|_{L^1(I)}}{\omega^2}.
\]
Summing the resulting majorant yields
\[
  \Big|\int \varphi\,\partial_\sigma\Re\log\dettwo\,dt\Big|
  \ \le\ \|\varphi''\|_{L^1}\sum_{p}\sum_{k\ge 2}\frac{(\log p)\,p^{-k\sigma}}{(k\log p)^2}
  \ \le\ \|\varphi''\|_{L^1}\sum_{p}\sum_{k\ge 2}\frac{p^{-k/2}}{k^2\,\log p},
\]
uniformly for \(\sigma\in(\tfrac12,\tfrac12+\varepsilon_0]\), since the rightmost double series converges.
\end{proof}



\begin{lemma}[Arithmetic Carleson energy]\label{lem:carleson-arith}
Let
\[
 U_{\det_2}(\sigma,t)\ :=\ \Re\log\dettwo\!\Big(I-A\big(\tfrac12+\sigma+it\big)\Big)
 \ =\ -\sum_{p}\sum_{k\ge 2}\frac{p^{-k/2}}{k}\,e^{-k\log p\,\sigma}\,\cos\big(k\log p\,t\big),\qquad \sigma>0,
\]
where the series converges absolutely for every \(\sigma>0\).
Then for every interval \(I\subset\R\) with Carleson box \(Q(I):=I\times(0,|I|]\),
\[
 \iint_{Q(I)} |\nabla U_{\det_2}|^2\,\sigma\,dt\,d\sigma\ \le\ \frac{|I|}{4}\,\sum_{p}\sum_{k\ge 2}\frac{p^{-k}}{k^2}
 \ =:\ K_0\,|I|,\qquad K_0:=\frac{1}{4}\sum_{p}\sum_{k\ge 2}\frac{p^{-k}}{k^2}<\infty.
\]
\end{lemma}
\begin{proof}
For a single mode \(b\,e^{-\omega\sigma}\cos(\omega t)\) one has \(|\nabla|^2=b^2\omega^2e^{-2\omega\sigma}\), hence
\[
 \int_0^{|I|}\!\int_I |\nabla|^2\,\sigma\,dt\,d\sigma
 \ \le\ |I|\cdot\sup_{\omega>0}\int_0^{|I|}\sigma\,\omega^2e^{-2\omega\sigma}d\sigma\cdot b^2
 \ \le\ \tfrac14\,|I|\,b^2.
\]
With \(b=p^{-k/2}/k\) and \(\omega=k\log p\), summing over \((p,k)\) gives the claim and the finiteness of \(K_0\).
\end{proof}

\paragraph{Whitney scale and short--interval zero counts.}
Throughout the boundary-certificate route we work on Whitney boxes based at height \(T\) with
\[
  L=L(T):=\min\Big\{\frac{c}{\log\angles{T}},\ L_\star\Big\},\qquad
  \angles{T}:=\sqrt{1+T^2},\qquad c\in(0,1]\ \text{fixed}.
\]
The only input about the \emph{number} of zeros used beloww is the classical short-interval consequence of Riemann--von Mangoldt: there exist absolute constants \(A_0,A_1>0\) such that for \(T\ge 2\) and \(0<H\le 1\),
\[
  N(T;H)\ :=\ \#\{\rho=\beta+i\gamma:\ \gamma\in[T,T+H]\}\ \le\ A_0\ +\ A_1\,H\,\log\angles{T}.
\]

\begin{lemma}[Annular Poisson--balayage \(L^2\) bound]\label{lem:annular-balayage}
Let \(I=[T-L,T+L]\), \(Q_\alpha(I)=I\times(0,\alpha L]\), and fix \(k\ge 1\).
For
\(
\mathcal A_k:=\{\rho=\beta+i\gamma:\ 2^kL<|T-\gamma|\le 2^{k+1}L\}
\)
set
\[
  V_k(\sigma,t):=\sum_{\rho\in\mathcal A_k}\frac{\sigma}{(t-\gamma)^2+\sigma^2}.
\]
Then
\[
  \iint_{Q_\alpha(I)} V_k(\sigma,t)^2\,\sigma\,dt\,d\sigma\ \ll_\alpha\ |I|\,4^{-k}\,\nu_k,
\]
where \(\nu_k:=\#\mathcal A_k\), and the implicit constant depends only on \(\alpha\).
\end{lemma}
\begin{proof}
Write \(K_\sigma(x):=\sigma/(x^2+\sigma^2)\) and \(V_k=\sum_{\rho\in\mathcal A_k}K_\sigma(\cdot-\gamma)\).
Integrate over \(t\in I\) first.
For the diagonal terms, using \(|t-\gamma|\ge 2^kL-L\ge 2^{k-1}L\) for \(t\in I\) and \(k\ge 1\),
\[
 \int_I K_\sigma(t-\gamma)^2\,dt
 = \sigma^2\!\int_I \frac{dt}{\big((t-\gamma)^2+\sigma^2\big)^2}
 \ \le\ \frac{L}{(2^{k-1}L)^2}\,\sigma.
\]
Multiplying by the area weight \(\sigma\) and integrating \(\sigma\in(0,\alpha L]\) gives a contribution \(\ll_\alpha |I|\,4^{-k}\) per \(\gamma\), hence \(\ll_\alpha |I|\,4^{-k}\nu_k\) after summing.
For off-diagonal terms, for \(i\ne j\) one has on \(I\) that \(K_\sigma(t-\gamma_j)\le \sigma/(2^{k-1}L)^2\), hence
\[
 \int_I K_\sigma(t-\gamma_i)K_\sigma(t-\gamma_j)\,dt
 \ \le\ \frac{\sigma}{(2^{k-1}L)^2}\int_\R K_\sigma(t-\gamma_i)\,dt
 = \frac{\pi\sigma}{(2^{k-1}L)^2},
\]
and integrating \(\sigma\in(0,\alpha L]\) with the extra factor \(\sigma\) yields \(\ll_\alpha |I|\,4^{-k}\).
Summing over pairs \((i,j)\) via a Schur test gives the stated bound (absorbing constants into \(\ll_\alpha\)).
\end{proof}

\subsection{Quantitative phase--velocity identity}

\label{appA:phasevelocity}
This subsection derives the quantitative identity linking the distribution $-w'$ to the off-axis zero data, in a form usable under Whitney localization.

\begin{lemma}[Distributional phase--velocity identity for outer data]\label{lem:pv-distributional}
Let \(U^*\in L^1_{\mathrm{loc}}(\mathbb R)\) and define \(w:=\Hilb[U^*]\in\mathcal D'(\mathbb R)\) by the duality
\[
  -\langle w,\varphi'\rangle \;=\; \int_{\mathbb R} U^*(t)\,(\Hilb\varphi)'(t)\,dt\qquad\forall\,\varphi\in C_c^\infty(\mathbb R).
\]
Then \(w'=\Hilb[(U^*)']\) in \(\mathcal D'(\mathbb R)\).
\end{lemma}
\begin{proof}
Let \(\eta_\varepsilon\) be a standard mollifier and set \(U_\varepsilon^*:=U^**\eta_\varepsilon\in C^\infty(\mathbb R)\).
For smooth data one has \((\Hilb[U_\varepsilon^*])'=\Hilb[(U_\varepsilon^*)']\) pointwise.
Since \(U_\varepsilon^*\to U^*\) in \(L^1_{\mathrm{loc}}\), we have \((U_\varepsilon^*)'\to (U^*)'\) in \(\mathcal D'\), and the Hilbert transform is continuous on \(\mathcal D'\).
Passing to the limit yields \(w'=\Hilb[(U^*)']\) in \(\mathcal D'\).
\end{proof}

% --- Auxiliary L^1 control for the xi term (used only locally beloww) ---
\begin{lemma}[Local $L^1$ control for $\log|\xi|$ along vertical approach]\label{lem:xi-deriv-L1}
Fix a compact interval $I\Subset\mathbb R$. Then the family
$t\mapsto \log|\xi(\tfrac12+\varepsilon+it)|$ is bounded in $L^1(I)$ uniformly for $\varepsilon\in(0,1]$.
Moreover, for $\varepsilon,\varepsilon'\downarrow 0$ the difference
$\log|\xi(\tfrac12+\varepsilon+it)|-\log|\xi(\tfrac12+\varepsilon'+it)|$ tends to $0$ in $L^1(I)$.
\end{lemma}
\begin{proof}
This is a standard boundary-approach property for $\log|\xi|$ in the half-plane, obtained from the
Poisson representation of $\log|\xi|$ together with classical zero-counting bounds for $\xi$
(equivalently for $\zeta$) on vertical lines; see, e.g., \cite[Ch.~9]{Titchmarsh}.
\end{proof}


\begin{theorem}[Quantified phase--velocity identity and boundary passage]\label{thm:phase-velocity-quant}
Let
\[
 u_\varepsilon(t):=\log\big|\dettwo(I-A(\tfrac12+\varepsilon+it))\big|-\log\big|\xi(\tfrac12+\varepsilon+it)\big|.
\]
Then \(u_\varepsilon\) is uniformly \(L^1\)-bounded and Cauchy on every compact \(I\Subset\R\) as \(\varepsilon\downarrow 0\), hence \(u_\varepsilon\to u\) in \(L^1_{\rm loc}(\R)\).
Let \(\mathcal O\) be the outer function on \(\Omega\) with boundary modulus \(e^{u}\) and normalization \(\mathcal O(\tfrac32)>0\), and set
\[
  \mathcal J(s):=\frac{\dettwo(I-A(s))}{\mathcal O(s)\,\xi(s)}.
\]
Then \(|\mathcal J(\tfrac12+it)|=1\) for a.e.\ \(t\in\R\).
By Lemma~\ref{lem:J-boundedtype-local}, $\mathcal J$ is of bounded type on every Whitney region $Q_{\alpha}(I)$.
Let \(w\in\mathcal D'(\R)\) denote the distributional boundary phase of \(\mathcal J\) (defined modulo an additive constant).
\begin{lemma}[Local bounded-type control for $\mathcal J$]\label{lem:J-boundedtype-local}
Fix a compact interval $I\Subset\R$. Assume that $\dettwo(I-A(s))$ is holomorphic and nonvanishing on a neighborhood of the Whitney region $Q_{\alpha}(I)\Subset\Omega$,
and that the Carleson energy bounds of Lemmas~\ref{lem:carleson-arith} and \ref{lem:carleson-xi} hold on $Q_{\alpha}(I)$.
Then $\mathcal J$ belongs to the Nevanlinna class (bounded type) on $Q_{\alpha}(I)$.
\end{lemma}

\begin{proof}
Write $\mathcal J=\dettwo(I-A)/(\mathcal O\,\xi)$ as above.
On each simply connected subdomain of $Q_{\alpha}(I)$ avoiding zeros, choose branches of $\log\dettwo(I-A)$ and $\log\xi$ and set
$U:=\Re\log\dettwo(I-A)$ and $V:=\Re\log\xi$.
By Lemmas~\ref{lem:carleson-arith} and \ref{lem:carleson-xi}, the measures $|\nabla U|^2\,\sigma\,dt\,d\sigma$ and $|\nabla V|^2\,\sigma\,dt\,d\sigma$ are Carleson on $Q_{\alpha}(I)$
(with $\sigma=\Re s-\tfrac12$). Standard half-plane theory (Carleson energy $\Rightarrow$ BMO boundary trace $\Rightarrow$ BMOA logarithm) implies that
$\dettwo(I-A)$ and $\xi$ are of bounded type on $Q_{\alpha}(I)$; see, e.g., Garnett, \emph{Bounded Analytic Functions}, Ch.~VI, and Koosis, \emph{The Logarithmic Integral}.
The outer function $\mathcal O$ with boundary modulus $e^{u}$ is an outer Smirnov function on $Q_{\alpha}(I)$ and hence of bounded type there.
Since the Nevanlinna class is closed under products and quotients (where defined), $\mathcal J$ is of bounded type on $Q_{\alpha}(I)$.
\end{proof}


Then, for every compact interval \(I\Subset\R\) and every nonnegative \(\phi\in C_c^\infty(I)\),
\begin{equation}\label{eq:pv-identity}
\int_I \phi(t)\,(-w'(t))\,dt
\ =\ \pi\!\int_I \phi(t)\,d\mu_{\rm off}(t)\ +\ \pi\!\int_I \phi(t)\,d\nu_{\rm sing}(t)\ +\ \pi\sum_{\gamma\in I} m_\gamma\,\phi(\gamma),
\end{equation}
where:
\begin{itemize}
\item \(\mu_{\rm off}\) is the Poisson balayage of the off--critical zeros \(\rho=\beta+i\gamma\) of \(\zeta\) with \(\beta>\tfrac12\), counted with multiplicity \(m_\rho\);
\item \(\nu_{\rm sing}\) is the (possibly zero) singular boundary measure of any singular inner factor in the canonical factorization of \(\mathcal J\) on \(\Omega\); and
\item the discrete sum ranges over boundary zeros/poles on \(\Re s=\tfrac12\), written \(s=\tfrac12+i\gamma\), with multiplicities \(m_\gamma\).
\end{itemize}
\end{theorem}
\begin{proof}
The \(L^1_{\rm loc}\) convergence \(u_\varepsilon\to u\) is as stated.
The outer function \(\mathcal O\) exists by Lemma~\ref{lem:outer-from-logmodulus}.

Under the bounded-type hypothesis, \(\mathcal J\) admits the canonical half-plane factorization into a unimodular constant, a Blaschke product over zeros in \(\Omega\), a (possibly trivial) singular inner factor, and an outer factor.
Since \(|\mathcal J(\tfrac12+it)|=1\) a.e., the outer factor is unimodular constant.
Taking the distributional boundary argument \(w\) and differentiating in \(\mathcal D'\), each factor contributes additively:
the Blaschke product yields the Poisson balayage measure \(\mu_{\rm off}\), the singular inner factor yields \(\nu_{\rm sing}\), and boundary zeros/poles yield atomic Dirac masses.
This is the standard phase-derivative computation for bounded-type functions on a half-plane; see, e.g., Garnett \emph{Bounded Analytic Functions}, Ch.~II, or Koosis \emph{The Logarithmic Integral}, Vol.~I.
\end{proof}

\subsection{Final assembly ingredients for the boundary wedge certificate (P+)}

This subsection collects the three final analytic ingredients—Poisson plateau, Whitney box pairing, and Carleson-energy control—used to conclude (P+).

\paragraph{Poisson plateau lower bound.}

We prove the key lower bound for the Poisson kernel averaged over admissible windows, which converts discrete off-axis contributions into a uniform wedge aperture.

\begin{lemma}[Poisson plateau lower bound]\label{lem:poisson-plateau}
Let \(\psi\in C_c^\infty(\R)\) be even with \(\psi\equiv 1\) on \([-1,1]\) and \(\operatorname{supp}\psi\subset[-2,2]\).
Then
\[
  c_0(\psi)\ :=\ \inf_{0<b\le 1,\ |x|\le 1} (P_b*\psi)(x)\ \ge\ \frac{1}{2\pi}\arctan 2\;>\;0.
\]
\end{lemma}
\begin{proof}
Since \(\psi\ge \mathbf 1_{[-1,1]}\), it suffices to compute \((P_b*\mathbf 1_{[-1,1]})(x)\).
For \(|x|\le 1\),
\[
 (P_b*\mathbf 1_{[-1,1]})(x)
 =\frac{1}{\pi}\int_{-1}^{1}\frac{b}{b^2+(x-y)^2}\,dy
 =\frac{1}{2\pi}\Big(\arctan\frac{1-x}{b}+\arctan\frac{1+x}{b}\Big).
\]
This expression is minimized over \(0<b\le 1\), \(|x|\le 1\), at \((x,b)=(1,1)\), yielding \(\frac{1}{2\pi}\arctan 2\).
\end{proof}

\paragraph{From phase--velocity and CR--Green to (P+).}

\label{app:assemble-pplus}


We combine the phase--velocity identity with a Cauchy--Riemann/Green pairing on Whitney boxes to obtain the windowed phase control that underlies (P+).

\paragraph{Carleson energy bound for the logarithmic derivative.}

\label{appA:carleson}

We prove the weighted $L^2$ (Carleson-energy) estimate for the relevant logarithmic derivative on Whitney boxes, including the neutralization of near-field zeros.

\begin{lemma}[Analytic (\(\xi\)) Carleson energy on Whitney boxes]\label{lem:carleson-xi}
There exist absolute constants \(c\in(0,1]\) and \(C_\xi<\infty\) such that for every interval \(I=[T-L,\,T+L]\) at Whitney scale \(L=c/\log\angles{T}\), the Poisson extension
\[
 U_{\xi}(\sigma,t):=\Re\log\xi\big(\tfrac12+\sigma+it\big)\qquad(\sigma>0)
\]
obeys the Carleson bound
\[
  \iint_{Q(I)} |\nabla U_{\xi}(\sigma,t)|^2\,\sigma\,dt\,d\sigma\ \le\ C_\xi\,|I|.
\]
\end{lemma}
\begin{proof}
Fix \(I=[T-L,T+L]\) with \(L=c/\log\angles{T}\) and a fixed aperture \(\alpha\in[1,2]\).
Neutralize near zeros by a local half-plane Blaschke product \(B_I\) removing zeros of \(\xi\) inside a fixed dilate \(Q(\alpha'I)\) (\(\alpha'>\alpha\)).
This yields a harmonic field \(\widetilde U_\xi\) on \(Q(\alpha I)\) and
\[
  \iint_{Q(\alpha I)} |\nabla U_\xi|^2\,\sigma\,dt\,d\sigma
  \ \asymp\
  \iint_{Q(\alpha I)} |\nabla \widetilde U_\xi|^2\,\sigma\,dt\,d\sigma\ +\ O_\alpha(|I|),
\]
so it suffices to bound the neutralized energy.

Choose a branch of \(\log\xi\) on each component of \(Q(\alpha I)\setminus Z(\xi)\) so that \(U_\xi=\Re\log\xi\) is harmonic there. Then
\[
  |\nabla U_\xi(\sigma,t)|^2\;=\;|(\log\xi)'(\tfrac12+\sigma+it)|^2\;=\;\Big|\frac{\xi'}{\xi}\big(\tfrac12+\sigma+it\big)\Big|^2.
\]
By the Hadamard product for \(\xi\),
\(\xi'/\xi(s)=A(s)+\sum_\rho (s-\rho)^{-1}\), where \(A\) is holomorphic and slowly varying on compact strips. Thus it suffices to bound the weighted \(L^2\)-norm of \(\xi'/\xi\) on \(Q(\alpha I)\); the contribution of \(A\) is \(O_\alpha(|I|)\), and the remaining term is handled annularly by summing the zero contributions.
Decompose the (neutralized) zeros into Whitney annuli
\(
\mathcal A_k:=\{\rho:2^kL<|\gamma-T|\le 2^{k+1}L\}
\), \(k\ge 1\).
For \(k\ge 1\) and \(t\in I\), any \(\rho=\beta+i\gamma\in\mathcal A_k\) satisfies
\(|t-\gamma|\ge 2^kL-L\ge 2^{k-1}L\).
Since \(|s-\rho|^2=(t-\gamma)^2+(\tfrac12+\sigma-\beta)^2\ge (t-\gamma)^2\), we have the pointwise bound
\(
  |(s-\rho)^{-1}|\le |t-\gamma|^{-1}\le (2^{k-1}L)^{-1}
\)
for all \(s=\tfrac12+\sigma+it\in Q_\alpha(I)\).
Therefore, writing \(S_k(s):=\sum_{\rho\in\mathcal A_k}(s-\rho)^{-1}\),
\[
  |S_k(s)|^2\ \le\ \nu_k\sum_{\rho\in\mathcal A_k}|s-\rho|^{-2}
  \ \le\ \nu_k\cdot \nu_k\,(2^{k-1}L)^{-2}
  \ =\ \nu_k^2\,(2^{k-1}L)^{-2}.
\]
Integrating this uniform bound over \(Q_\alpha(I)\) with the area weight \(\sigma\,dt\,d\sigma\) gives
\[
  \iint_{Q_\alpha(I)} |S_k(s)|^2\,\sigma\,dt\,d\sigma
  \ \le\ \nu_k^2\,(2^{k-1}L)^{-2}\cdot |I|\cdot \int_0^{\alpha L}\!\sigma\,d\sigma
  \ \ll_\alpha\ |I|\,L^2\,\frac{\nu_k^2}{4^k\,L^2}
  \ =\ \ll_\alpha\ |I|\,4^{-k}\,\nu_k^2.
\]
Now sum over annuli using Cauchy--Schwarz in \(k\):
\(
\big|\sum_{\rho\notin Q(\alpha'I)}(s-\rho)^{-1}\big|^2\le (\sum_k |S_k(s)|)^2\le (\sum_k 2^{-k})\,(\sum_k 2^{k}|S_k(s)|^2)
\), hence after integrating,
\[
  \iint_{Q_\alpha(I)} \Big|\sum_{\rho\notin Q(\alpha'I)}(s-\rho)^{-1}\Big|^2\,\sigma\,dt\,d\sigma
  \ \ll\ \sum_{k\ge1} 2^{k}\,\iint_{Q_\alpha(I)} |S_k(s)|^2\,\sigma\,dt\,d\sigma
  \ \ll_\alpha\ |I|\sum_{k\ge1} 2^{-k}\,\nu_k^2.
\]
To bound \(\nu_k\), use the short-interval zero-count bound above to obtain, for some absolute \(a_1(\alpha),a_2(\alpha)\),
\[
  \nu_k\ \le\ a_1(\alpha)\,2^k L\,\log\angles{T}\ +\ a_2(\alpha)\,\log\angles{T}.
\]
Therefore, using \(\nu_k\ll_\alpha 2^k L\log\angles{T}+\log\angles{T}\), we obtain
\[
  \sum_{k\ge1}2^{-k}\,\nu_k^2\ \ll\ \sum_{k\ge1}2^{-k}\big(4^kL^2\log^2\angles{T}+\log^2\angles{T}\big)
  \ \ll\ L^2\log^2\angles{T}+\log^2\angles{T}.
\]
On Whitney scale \(L=c/\log\angles{T}\), this is \(\ll 1\).
Adding the neutralized near-field \(O(|I|)\) and the smooth \(A\) contribution, we obtain
\[
  \iint_{Q(\alpha I)} |\nabla U_\xi|^2\,\sigma\,dt\,d\sigma\ \le\ C_\xi\,|I|,
\]
with \(C_\xi\) depending only on \((\alpha,c)\).
\end{proof}


\subsection{From windowed phase control to the wedge}

\label{appA:wedge}
We convert the established windowed phase bounds into an almost-everywhere wedge inclusion for the boundary values of $\mathcal J_{\rm out}$.

\begin{definition}[Admissible window class with atom avoidance]\label{def:adm-bumps}
Fix an even \(C^\infty\) window \(\psi\) with \(\psi\equiv1\) on \([-1,1]\) and \(\operatorname{supp}\psi\subset[-2,2]\).
For an interval \(I=[t_0-L,t_0+L]\), an aperture \(\alpha'>1\), and a parameter \(\varepsilon\in(0,\tfrac14]\), define \(\mathcal W_{\rm adm}(I;\varepsilon)\) to be the set of \(C^\infty\), nonnegative, mass-\(1\) bumps \(\phi\) supported in the fixed dilate \(2I=[t_0-2L,t_0+2L]\) that can be written as
\[
  \phi(t)\ =\ \frac{1}{Z}\,\frac{1}{L}\,\psi\!\left(\frac{t-t_0}{L}\right)\,m(t),
  \qquad Z=\int_{2I} \frac1L\psi\!\left(\frac{t-t_0}{L}\right)m(t)\,dt,
\]
where \(2I:=[t_0-2L,t_0+2L]\) and the mask \(m\in C^\infty(2I;[0,1])\) satisfies:
\begin{itemize}
\item[(i)] \emph{Atom avoidance.} There is a union of disjoint open subintervals \(E=\bigcup_{j=1}^{J} J_j\subset I\) with total length \(|E|\le \varepsilon L\) such that \(m\equiv0\) on \(E\) and \(m\equiv1\) on \(I\setminus E'\), where each transition layer \(E'\setminus E\) has thickness \(\le \varepsilon L\).
\item[(ii)] \emph{Uniform smoothness.} \(\|m'\|_\infty\lesssim (\varepsilon L)^{-1}\) and \(\|m''\|_\infty\lesssim (\varepsilon L)^{-2}\) with implicit constants independent of \(I,t_0,L\) and of the number/placement of the holes \(\{J_j\}\).
\end{itemize}
Every \(\phi\in\mathcal W_{\rm adm}(I;\varepsilon)\) is supported in \(2I\).
This class contains the unmasked profile \(\varphi_{L,t_0}(t)=Z_0^{-1}L^{-1}\psi((t-t_0)/L)\) with \(Z_0:=\int_{-2}^{2}\psi(x)\,dx\) (take \(E=\varnothing\), \(m\equiv1\)) and also allows dodging boundary atoms by punching out small neighborhoods while keeping total deleted length \(\le\varepsilon L\).
\end{definition}

\begin{lemma}[Uniform Poisson--energy bound for admissible tests]\label{lem:uniform-test-energy}
Let \(V_\phi\) be the Poisson extension of \(\phi\in\mathcal W_{\rm adm}(I;\varepsilon)\) to the half‑plane, and fix a cutoff to \(Q(\alpha' I)\) with \(\alpha'>1\) as in the CR--Green pairing.
Then there exists a finite constant \(\mathcal A_{\rm adm}(\psi,\varepsilon,\alpha')<\infty\), depending only on \((\psi,\varepsilon,\alpha')\), such that
\[
  \iint_{Q(\alpha' I)} |\nabla V_\phi(\sigma,t)|^2\,\sigma\,dt\,d\sigma\ \le\ \mathcal A_{\rm adm}(\psi,\varepsilon,\alpha')^2\; L.
\]
\end{lemma}
\begin{proof}
Let \(\phi(t)=Z^{-1}L^{-1}\psi((t-t_0)/L)m(t)\) be an admissible test.
By scaling of the Poisson kernel and the uniform bounds on \(m,m',m''\) from Definition~\ref{def:adm-bumps}, the \(H^1\)-size of \(\phi\) (equivalently the \(L^2(\sigma)\) Dirichlet energy of its Poisson extension on a fixed aperture box) is controlled uniformly by a constant depending only on \((\psi,\varepsilon,\alpha')\), times \(L^{1/2}\).
Squaring yields the stated \(\lesssim L\) energy bound with \(\mathcal A_{\rm adm}(\psi,\varepsilon,\alpha')\).
\end{proof}

\begin{lemma}[Cutoff pairing on boxes]\label{lem:cutoff-pairing}
Fix parameters \(\alpha'>\alpha>1\).
Let \(\chi_{L,t_0}\in C_c^\infty(\R^2_+)\) satisfy \(\chi\equiv1\) on \(Q(\alpha I)\), \(\operatorname{supp}\chi\subset Q(\alpha'I)\), \(\|\nabla\chi\|_\infty\lesssim L^{-1}\) and \(\|\nabla^2\chi\|_\infty\lesssim L^{-2}\).
Let \(V_\phi\) be the Poisson extension of \(\phi\in \mathcal W_{\rm adm}(I;\varepsilon)\).
Then one has the Green pairing identity
\[
 \int_{\R} u(t)\,\phi(t)\,dt
 \ =\ \iint_{Q(\alpha'I)} \nabla U\cdot \nabla\big(\chi_{L,t_0}\, V_\phi\big)\,dt\,d\sigma\ +\ \mathcal R_{\mathrm{side}}\ +\ \mathcal R_{\mathrm{top}},
\]
with remainders satisfying
\[
 |\mathcal R_{\mathrm{side}}|+|\mathcal R_{\mathrm{top}}|
 \ \lesssim\ \Big(\iint_{Q(\alpha'I)} |\nabla U|^2\,\sigma\Big)^{1/2}
               \cdot \Big(\iint_{Q(\alpha'I)} \big(|\nabla\chi|^2\,|V_\phi|^2+|\nabla V_\phi|^2\big)\,\sigma\Big)^{1/2}.
\]
\end{lemma}
\begin{proof}
Let \(Q:=Q(\alpha'I)\).
Assume \(U\) is \(C^2\) on \(\overline Q\) and harmonic on \(Q\), with boundary trace \(u(t)=U(0,t)\) on the bottom edge \(\{\sigma=0\}\).
Since \(\chi_{L,t_0}V_\phi\) is compactly supported in \(\overline Q\) and smooth on \(Q\), Green's identity gives
\[
  \iint_{Q} \nabla U\cdot \nabla(\chi V_\phi)\,dt\,d\sigma
  \,=\,
  \int_{\partial Q} (\chi V_\phi)\,\partial_n U\,ds
  \ -\ \iint_{Q} (\chi V_\phi)\,\Delta U\,dt\,d\sigma.
\]
Since \(\Delta U=0\) on \(Q\), only the boundary integral remains.
On the bottom edge one has \(\partial_n=-\partial_\sigma\), \(\chi\equiv1\), and \(V_\phi(0,t)=\phi(t)\), hence that contribution equals
\[
  \int_{I} \phi(t)\,(-\partial_\sigma U)(0,t)\,dt.
\]
\subsection{Whitney box pairing and local oscillation functional}

\label{appA:whitney}

We organize the remaining bookkeeping: pairing Whitney boxes across scales and bounding the oscillation functional needed to pass from local control to the global wedge statement.

If \(U\) is the real part of a holomorphic logarithm \(U=\Re\log J\) with \(|J(\tfrac12+it)|=1\) a.e., then \(U(0,t)=0\) a.e.\ and \(-\partial_\sigma U(0,t)=\partial_t \Arg J(\tfrac12+it)\) in distributions by Cauchy--Riemann; in particular, this term is the tested boundary phase derivative in Lemma~\ref{lem:CR-green-phase} beloww.
The remaining boundary pieces (two vertical sides and the top edge) are, by definition, the remainders \(\mathcal R_{\mathrm{side}}+\mathcal R_{\mathrm{top}}\).

For the remainder estimate, we apply Cauchy--Schwarz in the scale-invariant measure \(\sigma\,dt\,d\sigma\) on \(Q\):
\[
  \big|\mathcal R_{\mathrm{side}}\big|+\big|\mathcal R_{\mathrm{top}}\big|
  \ \lesssim\ \Big(\iint_Q |\nabla U|^2\,\sigma\Big)^{1/2}
               \Big(\iint_Q \big|\nabla(\chi V_\phi)\big|^2\,\sigma\Big)^{1/2}.
\]
Expanding \(\nabla(\chi V_\phi)=\chi\,\nabla V_\phi + (\nabla\chi)\,V_\phi\) yields
\[
  \iint_Q \big|\nabla(\chi V_\phi)\big|^2\,\sigma
  \ \lesssim\ \iint_Q \big(|\nabla V_\phi|^2 + |\nabla\chi|^2|V_\phi|^2\big)\,\sigma,
\]
which gives the displayed estimate.
\end{proof}

\begin{lemma}[CR--Green pairing for boundary phase]\label{lem:CR-green-phase}
Let \(J\) be analytic on \(\Omega\) with a.e.\ boundary modulus \(|J(\tfrac12+it)|=1\), and write \(\log J=U+iW\) on \(\Omega\), so \(U\) is harmonic with \(U(\tfrac12+it)=0\) a.e.
Fix a Whitney interval \(I=[t_0-L,t_0+L]\) and let \(V_\phi\) be the Poisson extension of \(\phi\in\mathcal W_{\rm adm}(I;\varepsilon)\).
Then, with a cutoff \(\chi_{L,t_0}\) as in Lemma~\ref{lem:cutoff-pairing},
\[
  \int_{\R} \phi(t)\,\big(-W'(t)\big)\,dt
  \ =\ \iint_{Q(\alpha'I)} \nabla U\cdot \nabla\big(\chi_{L,t_0}\,V_\phi\big)\,dt\,d\sigma\ +\ \mathcal R_{\mathrm{side}}\ +\ \mathcal R_{\mathrm{top}},
\]
and the remainders satisfy the same estimate as in Lemma~\ref{lem:cutoff-pairing}.
In particular, by Cauchy--Schwarz and Lemma~\ref{lem:uniform-test-energy}, there is a constant \(C_{\rm rem}(\alpha',\psi)\) such that
\[
  \int_{\R} \phi(t)\,\big(-w'(t)\big)\,dt\ \le\ C_{\rm rem}(\alpha',\psi)\,\Big(\iint_{Q(\alpha'I)} |\nabla U|^2\,\sigma\Big)^{1/2}.
\]
\end{lemma}
\begin{proof}
On the bottom edge \(\{\sigma=0\}\) the outward normal is \(\partial_n=-\partial_\sigma\).
By Cauchy--Riemann for \(\log J=U+iW\) on the boundary line \(\{\Re s=\tfrac12\}\) one has \(\partial_n U=-\partial_\sigma U=\partial_t W\).
Thus the bottom-edge term in Green's identity is
\[
  -\int_{\partial Q\cap\{\sigma=0\}} \chi\,V_\phi\,\partial_n U\,dt
  = -\int_{\R} \phi(t)\,\partial_t W(t)\,dt
  = \int_{\R} \phi(t)\,\big(-w'(t)\big)\,dt,
\]
which yields the stated identity after including the interior term and remainders.
The final inequality is Cauchy--Schwarz together with the uniform Poisson-energy bound from Lemma~\ref{lem:uniform-test-energy}.
\end{proof}

\begin{proposition}[Length‑independent upper bound for admissible tests]\label{prop:length-free}
Let \(J\) be holomorphic on \(\Omega\setminus Z(\zeta)\) with a.e.\ boundary modulus \(1\), write \(\log J=U+iW\) on \(\Omega\setminus Z(\zeta)\), and let \(-w'\) denote the boundary phase distribution.
For every interval \(I=[t_0-L,t_0+L]\), every \(\phi\in\mathcal W_{\rm adm}(I;\varepsilon)\), and every fixed cutoff to \(Q(\alpha' I)\),
\begin{equation}\label{eq:CRG-upper-adm}
\int_{\mathbb R}\!\phi(t)\,(-w')(t)\,dt\ \le\ C_{\rm test}(\psi,\varepsilon,\alpha')\,\Big(\iint_{Q(\alpha' I)}|\nabla U|^2\,\sigma\,dt\,d\sigma\Big)^{1/2}
\end{equation}
with \(C_{\rm test}(\psi,\varepsilon,\alpha'):=C_{\rm rem}(\alpha',\psi)\,\mathcal A_{\rm adm}(\psi,\varepsilon,\alpha')\) independent of \(I,t_0,L\).
In particular, defining the box-energy constant
\[
  C_{\rm box}^{(\zeta)}\ :=\ \sup_{I}\ \frac{1}{|I|}\iint_{Q(\alpha' I)}|\nabla U|^2\,\sigma\,dt\,d\sigma,
\]
one has the scale bound
\[
  \int_{\mathbb R}\!\phi\,(-w')\ \le\ C_{\rm test}(\psi,\varepsilon,\alpha')\,\sqrt{C_{\rm box}^{(\zeta)}}\,L^{1/2}.
\]
\end{proposition}
\begin{proof}
Apply Lemma~\ref{lem:CR-green-phase} with \(\phi\in\mathcal W_{\rm adm}(I;\varepsilon)\) and absorb the window-side constants into \(C_{\rm test}(\psi,\varepsilon,\alpha')\).
\end{proof}

\begin{lemma}[Whitney--uniform wedge]\label{lem:whitney-uniform-wedge}\label{lem:local-to-global-wedge}
Fix parameters \(\alpha'>1\) and \(\varepsilon\in(0,\tfrac14]\).
Fix the Whitney schedule and clip by \(L_\star\): set \(L_\star:=c/\log 2\) and henceforth
\[
  L(T)\ :=\ \min\Big\{\frac{c}{\log\angles{T}},\ L_\star\Big\}.
\]
Then for every Whitney interval \(I=[t_0-L,t_0+L]\) and the corresponding cutoff
\(\psi_{L,t_0}(t):=\psi((t-t_0)/L)=Z_0L\,\varphi_{L,t_0}(t)\) (so \(\psi_{L,t_0}\equiv 1\) on \(I\)),
\[
  \int_{\mathbb R} \psi_{L,t_0}(t)\,(-w'(t))\,dt\ \le\ Z_0\,L_\star\cdot C_{\rm test}(\psi,\varepsilon,\alpha')\,\sqrt{C_{\rm box}^{(\zeta)}}\,L_\star^{1/2}
  \ :=\ \pi\,\Upsilon_{\rm Whit}(c).
\]
Choosing \(c>0\) sufficiently small so that \(\Upsilon_{\rm Whit}(c)<\tfrac12\) yields the hypothesis of Lemma~\ref{lem:local-to-global-wedge} and hence \textup{(P+)}.
\end{lemma}
\begin{proof}
Since \(\psi_{L,t_0}=Z_0L\,\varphi_{L,t_0}\), apply Proposition~\ref{prop:length-free} with \(\phi=\varphi_{L,t_0}\), then multiply the resulting bound by \(Z_0L\) and use the Whitney clip \(L\le L_\star\).
\end{proof}

\begin{theorem}[Proof of Theorem~\ref{thm:Pplus}]\label{thm:pplus-proof-complete}
The boundary wedge \textup{(P+)} holds for \(\mathcal J_{\rm out}\).
\end{theorem}
\begin{proof}

By the definition \eqref{eq:J-out} and Theorem~\ref{thm:phase-velocity-quant}, the quantitative phase--velocity identity (Theorem~\ref{thm:phase-velocity-quant}) applies to the \(\zeta\)-normalized unimodular ratio \(J_\zeta\), and hence (by \eqref{eq:J-out}) to \(\mathcal J_{\rm out}\).
In particular, the associated boundary phase distribution \(-w'\) is positive.

Proposition~\ref{prop:length-free} (CR--Green pairing) supplies a uniform Whitney-scale bound for the windowed phase derivative in terms of the box energy \(C_{\rm box}^{(\zeta)}\).
Applying the Whitney schedule and choosing \(c>0\) small enough gives \(\Upsilon_{\rm Whit}(c)<\tfrac12\) in Lemma~\ref{lem:whitney-uniform-wedge}.
Lemma~\ref{lem:local-to-global-wedge} then yields \textup{(P+)}.
\end{proof}

\section{\textcolor{blue}{Supplementary computational audit manifest (verifier commands and expected fields)}}\label{app:audit}

\textcolor{blue}{This appendix provides a referee-facing audit checklist for the supplementary computational artifacts in Table~\ref{tab:artifact-data}.}
There are two audit modes:
\begin{itemize}
\item \textbf{Fast audit:} verify the shipped JSON artifacts match Table~\ref{tab:artifact-data}.
\item \textbf{\textcolor{blue}{Regeneration audit (supplementary):}} \textcolor{blue}{rerun the verifier to regenerate the artifacts from scratch.}
\end{itemize}

\subsection*{Prerequisites}
Install the ARB/ball-arithmetic bindings:
\begin{verbatim}
pip install python-flint==0.6.0
\end{verbatim}

\subsection*{Fast audit: check shipped JSON artifacts}
\begin{itemize}
\item \textbf{Rectangle artifact} \url{artifacts/theta_certify_sigma06_07_t0_20_outer_zeta_proj.json}. Check (at minimum):
  \begin{itemize}
  \item \texttt{results.ok = true}
  \item \texttt{results.theta\_hi = 0.9999928763... < 1}
  \item \texttt{results.processed\_boxes = 380764}
  \end{itemize}
\item \textbf{Pick artifact} \url{artifacts/pick_sigma0599_raw_zeta_N16.json}. Check (at minimum):
  \begin{itemize}
  \item \texttt{pick.delta\_cert = 0.594...}
  \item \texttt{pick.P\_spd\_at\_0 = true}
  \item \texttt{pick.tail\_l1\_partial\_hi} (diagnostic L1 tail sum)
  \end{itemize}
\item \textbf{Pick artifact} \url{artifacts/pick_sigma06_raw_zeta_N16.json}. Check (at minimum):
  \begin{itemize}
  \item \texttt{pick.delta\_cert = 0.594...}
  \item \texttt{pick.P\_spd\_at\_0 = true}
  \item \texttt{pick.tail\_l1\_partial\_hi} (diagnostic L1 tail sum)
  \end{itemize}
\item \textbf{Pick artifact} \url{artifacts/pick_sigma07_raw_zeta_N16.json}. Check (at minimum):
  \begin{itemize}
  \item \texttt{pick.delta\_cert = 0.627...}
  \item \texttt{pick.P\_spd\_at\_0 = true}
  \item \texttt{pick.tail\_l1\_partial\_hi} (diagnostic L1 tail sum)
  \end{itemize}
\end{itemize}

\subsection*{\textcolor{blue}{Regeneration audit (supplementary): exact command lines}}
Run the verifier from the bundle root (or repository root, if you have a checkout with the same layout).
The following commands reproduce the primary artifacts (line breaks are for readability):

\paragraph{1) Rectangle certification (\texttt{theta\_certify}).}
\begin{verbatim}
python scripts/verify_attachment_arb.py \
  --theta-certify \
  --arith-gauge outer_zeta_proj \
  --arith-P-cut 2000 \
  --rect-sigma-min 0.6 --rect-sigma-max 0.7 \
  --rect-t-min 0.0 --rect-t-max 20.0 \
  --outer-mode midpoint \
  --outer-P-cut 2000 \
  --outer-T 50.0 --outer-n 2001 \
  --theta-init-n-sigma 10 --theta-init-n-t 50 \
  --theta-min-sigma-width 0.0001 --theta-min-t-width 0.001 \
  --theta-max-boxes 500000 \
  --prec 260 \
  --theta-out artifacts/theta_certify_sigma06_07_t0_20_outer_zeta_proj.json \
  --progress
\end{verbatim}

\paragraph{2) Pick certification at $\sigma_0=0.599$ (\texttt{pick\_certify}).}
\begin{verbatim}
python scripts/verify_attachment_arb.py \
  --pick-certify \
  --pick-sigma0 0.599 \
  --pick-N 16 \
  --pick-coeff-count 128 \
  --pick-K 512 \
  --pick-rho 0.4 \
  --pick-rho-bound 0.5 \
  --arith-gauge raw_zeta \
  --arith-P-cut 2000 \
  --prec 1024 \
  --pick-out artifacts/pick_sigma0599_raw_zeta_N16.json
\end{verbatim}

\paragraph{3) Pick certification at $\sigma_0=0.6$ (\texttt{pick\_certify}).}
\begin{verbatim}
python scripts/verify_attachment_arb.py \
  --pick-certify \
  --pick-sigma0 0.6 \
  --pick-N 16 \
  --pick-coeff-count 128 \
  --pick-K 512 \
  --pick-rho 0.4 \
  --pick-rho-bound 0.5 \
  --arith-gauge raw_zeta \
  --arith-P-cut 2000 \
  --prec 1024 \
  --pick-out artifacts/pick_sigma06_raw_zeta_N16.json
\end{verbatim}

\paragraph{4) Pick certification at $\sigma_0=0.7$ (\texttt{pick\_certify}).}
\begin{verbatim}
python scripts/verify_attachment_arb.py \
  --pick-certify \
  --pick-sigma0 0.7 \
  --pick-N 16 \
  --pick-coeff-count 128 \
  --pick-K 512 \
  --pick-rho 0.4 \
  --pick-rho-bound 0.5 \
  --arith-gauge raw_zeta \
  --arith-P-cut 2000 \
  --outer-mode rigorous \
  --outer-P-cut 2000 \
  --prec 1024 \
  --pick-out artifacts/pick_sigma07_raw_zeta_N16.json
\end{verbatim}

\subsection*{What a successful audit means}
The verifier uses \emph{ball arithmetic}: each computed quantity is an interval enclosure (midpoint plus radius) and every operation propagates rounding error outward.
Thus each check is a formal inequality of the form ``upper bound $<1$'' or ``directed-rounding LDL$^\top$ succeeds with positive pivots''.
If the audit checks above pass, then the numerical inequalities summarized in Table~\ref{tab:artifact-data} are certified within the logic of ball arithmetic.

% Shared bibliography include for the three-paper split.
% Keep this file as a plain thebibliography environment to avoid toolchain friction.

\begin{thebibliography}{99}

\bibitem{IK}
H. Iwaniec and E. Kowalski,
\emph{Analytic Number Theory},
AMS Colloquium Publications, 2004.

\bibitem{MV}
H. L. Montgomery and R. C. Vaughan,
\emph{Multiplicative Number Theory I: Classical Theory},
Cambridge University Press, 2007.

\bibitem{Titchmarsh}
E. C. Titchmarsh,
\emph{The Theory of the Riemann Zeta-Function},
2nd ed., Oxford University Press, 1986.

\bibitem{Garnett}
J. B. Garnett,
\emph{Bounded Analytic Functions},
Graduate Texts in Mathematics, vol.~236, Springer, 2007.

\bibitem{RosenblumRovnyak}
M. Rosenblum and J. Rovnyak,
\emph{Hardy Classes and Operator Theory},
Oxford University Press, 1985.

\bibitem{Donoghue}
W. F. Donoghue,
\emph{Monotone Matrix Functions and Analytic Continuation},
Springer, 1974.

\bibitem{SimonTrace}
B. Simon,
\emph{Trace Ideals and Their Applications},
2nd ed., Mathematical Surveys and Monographs, vol.~120, American Mathematical Society, 2005.



\bibitem{Ahlfors}
L. V. Ahlfors,
\emph{Complex Analysis},
3rd ed., McGraw--Hill, 1979.

\end{thebibliography}



\begin{greentext}
\clearpage
\section*{Author response: status of the referee “blockers” list}
\noindent This draft has been revised to address the specific technical items flagged in the blue checklist as follows:
\begin{itemize}
\item \textbf{B1 (AC$_\mu$ upgrade)}: removed from the logical chain; \textup{(P+)} is proved Lebesgue-a.e.\ directly via oscillation control.
\item \textbf{B3 (singular inner)}: the phase--velocity identity is written with an explicit singular-inner boundary measure term $\nu_{\rm sing}$, so no hidden inner contribution is dropped.
\item \textbf{B4 (PV / distributional Arg)}: the distributional phase--velocity lemma is stated explicitly (hypotheses + conclusion), and \eqref{eq:pv-identity} records all terms (balayage, singular, and atomic).
\item \textbf{B2 (bounded type / Smirnov bridge)}: the Carleson-energy $\Rightarrow$ bounded-type step is stated with standard references (Garnett; Rosenblum--Rovnyak) and packaged as Lemma~\ref{lem:boundedtype-from-carleson} in the Appendix.
\end{itemize}
\noindent The remaining item \textbf{C-next} is the separate computational reproducibility pass (artifact regeneration + hash/field checks).
\end{greentext}
\begin{bluetext}
\clearpage
\section*{I listed the serious blockers in the following: }

\begin{greentext}
\noindent\textbf{Author revision.}
The bullets below are a \emph{historical snapshot} of the referee’s original blockers list.
In the current draft: the $\mu$-a.e.\(\Rightarrow\)Lebesgue-a.e.\ upgrade (and any (AC$_\mu$) domination hypothesis) has been removed from the logical chain, and the remaining items are addressed as summarized in the ``Author response'' section immediately above.
\end{greentext}

\begin{itemize}
  \item[\textbf{B0}] \textbf{Unconditionality vs.\ conditional hypotheses (global).}
  The abstract and Theorem~\ref{thm:farfield} state an unconditional zero-free half-plane result.
  As written, several boundary-to-interior implications are \emph{conditional} on the two explicitly stated hypotheses in the blue notes:
  \textbf{(AC$_\mu$ on the relevant boundary interval)} and \textbf{(bounded-type/Smirnov regularity for the determinant-ratio factor)}.
  The authors should either (i) prove these hypotheses from earlier material, or (ii) state Theorem~\ref{thm:farfield} and all downstream uses as conditional on them.

  \item[\textbf{B1}] \textbf{The (AC$_\mu$) upgrade.}
  The upgrade from $\mu$-a.e.\ statements to Lebesgue-a.e.\ statements (invoked around \eqref{eq:pv-identity} and Lemma~\ref{lem:mu-to-lebesgue} in the appendix)
  must be proved without circularity and with explicit dependence on the interval. This is load-bearing for the passage from \textup{(P+)} to the boundary inequality
  used in Theorem~\ref{thm:Pplus} and then in Theorem~\ref{thm:farfield}.

  \item[\textbf{B2}] \textbf{Carleson/Hardy control $\Rightarrow$ bounded type / Smirnov (boundary regularity bridge).}
  The step labeled as Smirnov/Hardy class regularity (Lemma~\ref{lem:smirnov-regularity} and its uses inside Proposition~\ref{prop:herglotz-schur-transport})
  needs a fully referenced theorem with hypotheses verified in the present setting, or a complete self-contained proof. In particular,
  any Carleson-measure estimate used must be matched to the exact domain/weight and traced to a standard source.

  \item[\textbf{B3}] \textbf{Blaschke / singular-inner handling for the determinant ratio.}
  Any factorization claim for the determinant-ratio object (appearing in the appendix where $F_\varepsilon$ is introduced) must explicitly address:
  (i) possible singular-inner factors on the boundary line, and (ii) conditions under which these factors vanish or are controlled.
  If the argument assumes they are absent, that assumption must be stated (and then the main theorem becomes conditional).

  \item[\textbf{B4}] \textbf{Principal-value / phase--velocity identity and differentiability on the boundary.}
  The identity relating $\frac{d}{dt}\Arg(\cdot)$ to a principal-value integral (referenced as \eqref{eq:pv-identity} in the appendix)
  must be supported by a precise statement: required smoothness, boundary limits, and justification of differentiating an argument (or of an appropriate distributional formulation).
  Any missing hypotheses should be stated explicitly at the point of use.

  \item[\textbf{C-next}] \textbf{Computational artifacts are not yet re-certified in this referee pass.}
  Appendix~\ref{app:audit} lists shipped JSON certificates and regeneration commands.
  A separate reproducibility pass should: (i) regenerate each certificate in a clean environment using the exact commands, and
  (ii) confirm either exact SHA256 match or documented field-by-field equivalence.
\end{itemize}

\bigskip
\noindent\textbf{Referee note:} The blue blocks in this file are \emph{clarifications and dependency mapping only}.
They are not intended to introduce new mathematical ideas; they mark where a standard theorem, a missing hypothesis, or a missing proof step must be supplied by the authors.
\end{bluetext}
\end{document}
