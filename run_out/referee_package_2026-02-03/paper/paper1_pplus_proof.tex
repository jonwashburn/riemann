% This file is \input{} from Appendix~\ref{app:pplus-proof} of paper1_farfield_v4.tex.
% It provides the full analytic proof of the boundary wedge (P+) for \mathcal J_{\rm out}.


\begin{bluetext}
\noindent\textbf{Referee status (scope of this appendix):}
All steps here are written to prove the boundary wedge (P$^+$) for $\mathcal J_{\rm out}$.
However, the final upgrade from a $\mu$-a.e.\ boundary statement to a Lebesgue-a.e.\ boundary statement requires the domination hypothesis (AC$_\mu$)
introduced in Lemma~\ref{lem:mu-to-lebesgue} (see also Lemma~\ref{lem:mu-dominates-lebesgue}).
Until (AC$_\mu$) is proved for the specific $\mu$ constructed here, the main-paper transport step is conditional on (AC$_\mu$).
\end{bluetext}
\begin{greentext}
\noindent\textbf{Author revision (removing the (AC$_\mu$) dependency).}
The proof of \textup{(P+)} used below proceeds via Whitney-local oscillation control and the local-to-global wedge lemma, which yields a \emph{Lebesgue-a.e.} wedge statement directly.
Accordingly, no $\mu$-a.e.\ $\Rightarrow$ Lebesgue-a.e.\ upgrade (and hence no (AC$_\mu$) domination hypothesis) is used in the logical chain proving \textup{(P+)}.
The (AC$_\mu$) discussion is retained only as background and is \emph{not load-bearing}.
\end{greentext}

\subsection*{Standing setup and notation}

\begin{bluetext}
\paragraph{Referee roadmap for Appendix (P+).}
This appendix is logically organized as follows (each arrow denotes a dependency that must be checked exactly as stated):
\begin{enumerate}
\item \emph{Wedge criterion from local control} (Whitney boxes): Lemma(s) in \S\,\ref{app:whitney-wedge} give a quantitative implication
\[
\text{(windowed phase variation + box Carleson control)} \Rightarrow \text{a.e.\ wedge inclusion for } 2e^{-im}\mathcal J_{\rm out}(\tfrac12+it).
\]
\item \emph{$\mu$-a.e.\ $\Rightarrow$ Lebesgue-a.e.} (domination): the upgrade is \textbf{conditional} unless (AC$_\mu$) is proved from the setup.
\item \emph{Phase--velocity identity}: Theorem~\ref{thm:phase-velocity-quant} (and the PV identity \eqref{eq:pv-identity}) give the windowed phase control,
provided $F_\varepsilon$ has boundary values in the sense required (Smirnov/bounded type on each relevant component).
\item \emph{Carleson energy $\Rightarrow$ bounded type} bridge: the ``bounded type'' step is \textbf{conditional} unless the Carleson-to-BMOA bridge is justified
with a precise reference and constants.
\item \emph{Summation of Blaschke contributions}: the Poisson balayage step is \textbf{conditional} unless Blaschke summability and absence (or explicit handling)
of singular inner factors are established.
\item \emph{Assemble}: \S\,\ref{app:assemble-pplus} combines the above to derive Theorem~\ref{thm:Pplus} in the main paper.
\end{enumerate}
\noindent\textbf{Referee requirement.} Each conditional arrow above must be either proved within this appendix (with explicit hypotheses) or stated as an assumption in the main theorem.
\end{bluetext}
\begin{greentext}
\noindent\textbf{Author revision (roadmap update).}
The proof of Theorem~\ref{thm:pplus-proof-complete} does \emph{not} use the ``$\mu$-a.e.\ $\Rightarrow$ Lebesgue-a.e.'' upgrade step; the wedge conclusion is obtained for Lebesgue-a.e.\ $t$ directly from oscillation control.
\end{greentext}

Throughout, let
\[
  \Omega:=\{\,s\in\C:\ \Re s>\tfrac12\,\},\qquad s=\tfrac12+\sigma+it\ (\sigma>0),
\]
and let
\[
  P_\sigma(x):=\frac{1}{\pi}\frac{\sigma}{\sigma^2+x^2}
\]
denote the Poisson kernel for the half-plane \(\Omega\) (shifted so that the boundary is \(\Re s=\tfrac12\)).
For an interval \(I=[t_0-L,t_0+L]\) we write the Carleson box
\[
  Q(I):=I\times(0,L]\subset \R\times(0,\infty).
\]

Recall from \eqref{eq:J-out} that \(\mathcal J_{\rm out}\) is holomorphic on \(\Omega\setminus Z(\zeta)\) and has a.e.\ boundary values on \(\Re s=\tfrac12\) with
\[
  \big|\mathcal J_{\rm out}(\tfrac12+it)\big|=1\qquad\text{for a.e.\ }t\in\R.
\]
Let
\[
  w(t):=\Arg \mathcal J_{\rm out}(\tfrac12+it)
\]
denote the boundary phase (defined for a.e.\ \(t\)), and write \(-w'\) for its boundary distributional derivative.
In the phase--velocity identity below, \(-w'\) is a positive distribution (measure plus atoms) encoding off--critical zeros.

\subsection*{A quantitative wedge criterion from Whitney-local control}\label{app:whitney-wedge}

\begin{bluetext}
\noindent\textbf{Scope and terminology for this subsection.}
We work on the boundary line $\Re s=\tfrac12$ and use the following conventions.
\begin{itemize}
\item \emph{Wedge.} For an angle parameter $\alpha\in(0,\tfrac\pi2)$ and center ray $e^{i m}$, write
\[
W_{m,\alpha}:=\{z\in\mathbb C:\ |\Arg(e^{-im}z)|\le \alpha\}.
\]
Thus the desired (P$^+$) boundary condition is the Lebesgue-a.e.\ inclusion
$\,2e^{-im}\mathcal J_{\rm out}(\tfrac12+it)\in W_{0,\alpha}\,$ for a fixed $\alpha<\tfrac\pi2$.
\item \emph{Whitney boxes / Carleson boxes.} For an interval $I\subset\mathbb R$, write the standard Carleson box
$S(I):=\{\sigma+it:\ 0<\sigma-\tfrac12\le |I|,\ t\in I\}$. A ``Whitney box'' means a box with comparable width and height,
e.g.\ $\{ \tfrac12+\sigma+it:\ \sigma\in[a|I|,b|I|],\ t\in I\}$ with fixed $0<a<b$.
\item \emph{Measure in the phrase ``a.e.''} Unless explicitly stated otherwise, ``a.e.'' refers to \emph{Lebesgue} measure $dt$ on $\mathbb R$.
When an auxiliary measure $\mu$ is used, statements are explicitly labeled ``$\mu$-a.e.'' and a separate upgrade step is required (see \S\,\ref{app:acmu-upgrade}).
\end{itemize}
\noindent\textbf{Referee warning.} Any implication of the form ``$\mu$-null $\Rightarrow$ Lebesgue-null'' requires a domination property (Lemma~\ref{lem:mu-dominates-lebesgue});
upper Carleson bounds alone do not suffice.
\end{bluetext}

\begin{bluetext}
\noindent\textbf{Dependency in the main manuscript (\texttt{paper1\_farfield.tex}).} \emph{Appendix Theorem~\ref{thm:phase-velocity-quant} (this file)} supplies the phase--velocity control needed to convert boundary wedge information into a Lebesgue-a.e.\ inequality for $\Re\!\bigl(2e^{-im}\mathcal J_{\rm out}(\tfrac12+it)\bigr)$. In the main manuscript \texttt{paper1\_farfield.tex}, this Appendix Theorem is used in the proof of \emph{Theorem~\ref{thm:Pplus}} (the boundary wedge certificate), and therefore is load-bearing for \emph{Proposition~\ref{prop:herglotz-schur-transport}} (Herglotz/Schur transport) and \emph{Theorem~\ref{thm:farfield}} (certified far-field zero-freeness).
\end{bluetext}

\begin{bluetext}
\noindent\textbf{Dependency in the main manuscript (\texttt{paper1\_farfield.tex}).} This principal-value identity is an internal sub-claim of Theorem~\ref{thm:phase-velocity-quant}. A referee must be able to verify it (including the precise meaning of $\frac{d}{dt}\Arg$ as a distribution) for the boundary transport step in Proposition~\ref{prop:herglotz-schur-transport}.
\end{bluetext}

\begin{bluetext}
\noindent\textbf{Dependency in the main manuscript (\texttt{paper1\_farfield.tex}).} This lemma links outer normalization to a Hilbert-transform phase relation. It is part of the chain used to justify boundary differentiation and the passage from modulus control to argument/phase control, feeding Theorem~\ref{thm:phase-velocity-quant} and ultimately Theorem~\ref{thm:Pplus}.
\end{bluetext}

\begin{bluetext}
\noindent\textbf{Dependency in the main manuscript (\texttt{paper1\_farfield.tex}).} This cutoff pairing lemma is used to localize CR--Green identities on Whitney boxes. It is an input to the Carleson/box-energy mechanism needed for the ``unconditional'' wedge globalization in Theorem~\ref{thm:Pplus}.
\end{bluetext}

\begin{bluetext}
\noindent\textbf{Dependency in the main manuscript (\texttt{paper1\_farfield.tex}).} This is the core CR--Green identity controlling phase increments by localized energy. It is a load-bearing step in the Appendix route to (P$^+$), hence indirectly required for the boundary-to-interior transport in Proposition~\ref{prop:herglotz-schur-transport}.
\end{bluetext}

\begin{bluetext}
\noindent\textbf{Dependency in the main manuscript (\texttt{paper1\_farfield.tex}).} This upper bound is the quantitative estimate extracted from Lemma~\ref{lem:CR-green-phase} and is used in the Carleson/Whitney energy-to-wedge chain. It must be fully justified for the (P$^+$) conclusion.
\end{bluetext}

\begin{bluetext}
\noindent\textbf{Dependency in the main manuscript (\texttt{paper1\_farfield.tex}).} This proposition provides the length-free Carleson estimate (uniform over interval scales) required to control the harmonic majorants that underpin both (i) the Appendix wedge globalization and (ii) the main-paper boundary admissibility claims (see Lemma~\ref{lem:F-boundary-admissible} in \texttt{paper1\_farfield.tex}).
\end{bluetext}

\begin{bluetext}
\noindent\textbf{Dependency in the main manuscript (\texttt{paper1\_farfield.tex}).} This definition sets up the admissible bump/test family used to formulate uniform box-energy bounds and Carleson estimates. It is foundational for Proposition~\ref{prop:length-free} and thus for the main-paper admissibility/Smirnov lemmas.
\end{bluetext}

\begin{bluetext}
\noindent\textbf{Dependency in the main manuscript (\texttt{paper1\_farfield.tex}).} This uniform test-energy bound is an essential quantitative input for Proposition~\ref{prop:length-free}. It is part of the chain that yields the Carleson control used in the main-paper Lemma~\ref{lem:F-boundary-admissible}.
\end{bluetext}

\begin{bluetext}
\noindent\textbf{Dependency in the main manuscript (\texttt{paper1\_farfield.tex}).} This arithmetic Carleson lemma controls the prime-sum measure contributions. It is an input to the box-energy/Carleson framework culminating in Proposition~\ref{prop:length-free}.
\end{bluetext}

\begin{bluetext}
\noindent\textbf{Dependency in the main manuscript (\texttt{paper1\_farfield.tex}).} This lemma is a key arithmetic-to-analytic bridge controlling the $\xi$-term in Carleson boxes. It is required for the unconditional Carleson bound used in Proposition~\ref{prop:length-free} and therefore is load-bearing for (P$^+$).
\end{bluetext}

\begin{bluetext}
\noindent\textbf{Dependency in the main manuscript (\texttt{paper1\_farfield.tex}).} This balayage lemma transfers annular control to box control and is used in the proof of the Carleson framework (ultimately Proposition~\ref{prop:length-free}). It is part of the chain needed for the main-paper admissibility lemma.
\end{bluetext}

\begin{bluetext}
\noindent\textbf{Dependency in the main manuscript (\texttt{paper1\_farfield.tex}).} This lemma is intended to connect a Poisson-smoothed measure control to boundary exceptional sets. It is used to justify that the exceptional set measured in the Appendix control measure translates to Lebesgue-null boundary exceptional sets (see Lemma~\ref{lem:mu-to-lebesgue}).
\end{bluetext}

\begin{bluetext}
\noindent\textbf{Dependency in the main manuscript (\texttt{paper1\_farfield.tex}).} This lemma is the mechanism that converts the Appendix ``bad set'' (measured in the auxiliary measure $\mu$) into a Lebesgue-null exceptional set on the boundary line. This conversion is necessary for the ``a.e.'' boundary inequalities used in the main-paper transport Proposition~\ref{prop:herglotz-schur-transport}.
\end{bluetext}

\begin{bluetext}
\noindent\textbf{Dependency in the main manuscript (\texttt{paper1\_farfield.tex}).} This normalization lemma supports the bookkeeping for $F(s)$ and $\mathcal J_{\rm out}(s)$ and is referenced in the main-paper Lemma~\ref{lem:F-boundary-admissible} (to justify boundary trace/log-modulus admissibility for $F$).
\end{bluetext}

\begin{bluetext}
\noindent\textbf{Dependency in the main manuscript (\texttt{paper1\_farfield.tex}).} This lemma is the Appendix-level justification for existence/stability of the outer normalizer used to define $\mathcal J_{\rm out}$. It is explicitly referenced by the main-paper Lemmas \ref{lem:F-boundary-admissible} and \ref{lem:smirnov-regularity} (in the revised manuscript).
\end{bluetext}

\begin{bluetext}
\noindent\textbf{Dependency in the main manuscript (\texttt{paper1\_farfield.tex}).} This lemma globalizes local wedge control to a global (P$^+$) statement. It is a near-final step in the Appendix proof of Theorem~\ref{thm:Pplus}.
\end{bluetext}

\begin{bluetext}
\noindent\textbf{Dependency in the main manuscript (\texttt{paper1\_farfield.tex}).} This lemma provides the uniform wedge bound across Whitney boxes, feeding the globalization step and the concluding Appendix theorem \ref{thm:pplus-proof-complete}. It is therefore load-bearing for Theorem~\ref{thm:Pplus} and hence Theorem~\ref{thm:farfield}.
\end{bluetext}

\begin{bluetext}
\noindent\textbf{Dependency in the main manuscript (\texttt{paper1\_farfield.tex}).} This is the concluding Appendix theorem whose sole purpose is to prove (P$^+$) in the main paper (Theorem~\ref{thm:Pplus} in \texttt{paper1\_farfield.tex}). Any gap here propagates directly to Proposition~\ref{prop:herglotz-schur-transport} and Theorem~\ref{thm:farfield}.
\end{bluetext}


\begin{bluetext}
\paragraph{Referee-tightening for Lemma~\ref{lem:det2-unsmoothed}.}
Whenever this appendix differentiates or takes boundary limits of $\log\dettwo(I-A(s))$,
the following elementary (but necessary) analytic facts are used.


\begin{bluetext}
\paragraph{Referee-tightening for Lemma~\ref{lem:mu-to-lebesgue}.}
As stated, an implication of the form ``$\mu(E)=0 \Rightarrow |E|=0$'' requires that $\mu$ \emph{dominate} Lebesgue measure.
We therefore isolate the exact condition needed.


\begin{bluetext}
\paragraph{Referee-tightening for Lemma~\ref{lem:outer-existence-stability}.}
The existence of an outer normalizer with prescribed boundary modulus is standard once the boundary log-modulus
belongs to $L^1((1+t^2)^{-1}dt)$ (or locally $L^1$ with mild growth control).
We record the exact analytic input.

\begin{lemma}[Outer normalizer from boundary log-modulus]
\label{lem:outer-from-logmodulus}
Let $u\in L^1(\mathbb R,(1+t^2)^{-1}dt)$ be real-valued. Then there exists an outer function $O$ on $\Omega$
(zero-free and holomorphic on $\Omega$) whose nontangential boundary values satisfy
\[
|O(\tfrac12+it)| = e^{u(t)} \quad\text{for a.e. }t\in\mathbb R.
\]
Moreover $O$ is unique up to a unimodular constant.
\end{lemma}

\begin{proof}[Reference]
This is the half-plane outer function construction via the Poisson integral of $u$ and its harmonic conjugate
(see, e.g.\ Duren, \emph{$H^p$ Spaces}, Ch.~II, or Garnett, \emph{Bounded Analytic Functions}, Ch.~II).
\end{proof}

\end{bluetext}
\begin{greentext}
\noindent\textbf{Author note.}
In this manuscript, the boundary log-modulus $u$ is defined as the $L^1_{\rm loc}$ limit of the regularized traces
$u_\varepsilon(t)=\log|\dettwo(I-A(\tfrac12+\varepsilon+it))|-\log|\xi(\tfrac12+\varepsilon+it)|$ as $\varepsilon\downarrow 0$
(see Theorem~\ref{thm:phase-velocity-quant}).
The weighted integrability requirement for the outer construction is then discharged by combining the Carleson/Whitney tested bounds for the two terms
(Lemmas~\ref{lem:det2-unsmoothed} and~\ref{lem:xi-deriv-L1}) together with standard logarithmic growth of $\log|\xi|$ at infinity.
\end{greentext}

\begin{lemma}[Sufficient condition for $\mu$-null $\Rightarrow$ Lebesgue-null]
\label{lem:mu-dominates-lebesgue}
Let $\mu$ be a Borel measure on $\mathbb R$. Fix an interval $I\subset\mathbb R$.
Assume there exists a constant $c_I>0$ such that
\[
\mu(E)\ \ge\ c_I\,|E|
\quad\text{for every measurable }E\subset I,
\]
where $|E|$ denotes Lebesgue measure. Then for any measurable $E\subset I$,
\[
\mu(E)=0 \ \Longrightarrow\ |E|=0.
\]
\end{lemma}

\begin{proof}
Immediate from $\mu(E)\ge c_I|E|$.
\end{proof}

\noindent\textbf{Required check to justify Lemma~\ref{lem:mu-to-lebesgue}:}
to deduce Lebesgue-a.e.\ boundary statements from $\mu$-a.e.\ statements,
one must prove that the specific auxiliary measure $\mu$ constructed in this appendix satisfies a domination estimate
of the type in Lemma~\ref{lem:mu-dominates-lebesgue} on the boundary interval(s) used.
If only an upper Carleson bound $\mu(S(I))\le C|I|$ is available, then $\mu(E)=0$ does \emph{not} imply $|E|=0$ in general,
so the argument must either (a) strengthen the measure property, or (b) replace the conclusion by a $\mu$-a.e.\ statement.
\end{bluetext}

\begin{bluetext}
\subsection*{Feasibility of upgrading $\mu$-a.e.\ to Lebesgue-a.e.}\label{app:acmu-upgrade}
\begin{greentext}
\noindent\textbf{Author note.} This subsection records the referee’s circularity concern for any approach that tries to deduce Lebesgue-a.e.\ conclusions from $\mu$-a.e.\ statements.
It is \emph{not used} in the proof of Theorem~\ref{thm:pplus-proof-complete}, which establishes \textup{(P+)} directly Lebesgue-a.e.\ by oscillation control.
\end{greentext}
\noindent\textbf{Minimal condition needed.}
For the implication ``$\mu(E)=0 \Rightarrow |E|=0$'' on an interval $I$, it is \emph{not} necessary to have the quantitative lower bound
$\mu(E)\ge c_I|E|$.
It suffices that $\mu$ be absolutely continuous with respect to Lebesgue measure on $I$ with Radon--Nikodym derivative
$f=\frac{d\mu}{dt}$ satisfying $f(t)>0$ for Lebesgue-a.e.\ $t\in I$.

\begin{lemma}[Null-set upgrade criterion]\label{lem:mu-null-upgrade-criterion}
Let $\mu$ be a Borel measure on $\mathbb R$ and fix an interval $I$.
Assume $\mu\ll dt$ on $I$ with density $f=\frac{d\mu}{dt}$ and $f(t)>0$ for Lebesgue-a.e.\ $t\in I$.
Then for every measurable $E\subset I$,
\[
\mu(E)=0\ \Longrightarrow\ |E|=0.
\]
\end{lemma}

\begin{proof}
If $\mu(E)=\int_E f(t)\,dt=0$ and $f>0$ a.e.\ on $I$, then $f=0$ a.e.\ on $E$, hence $|E|=0$.
\end{proof}

\noindent\textbf{What must be checked for the present $\mu$.}
Here $\mu$ is described as the Poisson balayage of the off--critical zeros (see Theorem~\ref{thm:pv-arg-identity}).
For a \emph{finite} positive measure $\nu$ supported in the open half-plane $\{\Re s>\tfrac12\}$, its Poisson balayage onto the boundary line
$\Re s=\tfrac12$ is absolutely continuous with a strictly positive density given by a Poisson-kernel integral.
If $\mu$ is literally such a balayage of a nonzero positive measure, then Lemma~\ref{lem:mu-null-upgrade-criterion} applies.

\noindent\textbf{Potential circularity.}
If the intent is to \emph{prove} absence of off--critical zeros, then $\mu$ may in fact be the \emph{zero} measure on the region being certified.
In that case, any upgrade ``$\mu$-a.e.\ $\Rightarrow$ Lebesgue-a.e.'' is invalid, since $\mu(E)=0$ holds for all $E$.
Therefore, an unconditional version of the main-paper transport step must either:
(i) work directly with Lebesgue (harmonic) measure on the boundary, or
(ii) prove a lower support/positivity property for $\mu$ from an independent source that does not assume what one is trying to prove.
\end{bluetext}



\begin{lemma}[Series representation and termwise differentiation for $\log\dettwo(I-A(s))$]
\label{lem:det2-series-diff}
Let $A(s)$ be the diagonal HS operator with eigenvalues $\lambda_p(s)=p^{-s}$ over primes $p$.
For any fixed $\sigma_0>\tfrac12$ and $s=\sigma+it$ with $\sigma\ge \sigma_0$:
\begin{enumerate}
\item The regularized determinant admits the absolutely convergent product
\[
\dettwo(I-A(s))=\prod_{p}(1-\lambda_p(s))\,e^{\lambda_p(s)},
\]
and hence
\[
\log\dettwo(I-A(s))=\sum_{p}\Big(\log(1-p^{-s})+p^{-s}\Big),
\]
where the branch of $\log(1-z)$ is the principal branch for $|z|<1$.
\item The series above converges absolutely and locally uniformly on $\{\Re s\ge \sigma_0\}$,
so $\log\dettwo(I-A(s))$ is holomorphic there.
\item On $\{\Re s\ge \sigma_0\}$, termwise differentiation is justified and yields
\[
\partial_s \log\dettwo(I-A(s))
=\sum_p\Big(\frac{p^{-s}\log p}{1-p^{-s}}-p^{-s}\log p\Big)
=\sum_p \frac{p^{-2s}\log p}{1-p^{-s}},
\]
with absolute/local uniform convergence.
\end{enumerate}
\end{lemma}

\begin{proof}
Since $\sum_p |p^{-s}|^2=\sum_p p^{-2\sigma}<\infty$ for $\sigma>\tfrac12$, the defining product for $\dettwo$ converges.
For $\sigma\ge\sigma_0>\tfrac12$, $|p^{-s}|\le p^{-\sigma_0}$ and $\sum_p p^{-2\sigma_0}<\infty$.
Also, for sufficiently large $p$, $|p^{-s}|\le \tfrac12$, and
$|\log(1-p^{-s})+p^{-s}|\lesssim |p^{-s}|^2$ by the Taylor remainder,
so absolute convergence follows by comparison to $\sum_p p^{-2\sigma_0}$.
Local uniform convergence and termwise differentiation follow by standard Weierstrass/M-test arguments, using the bound
$|\partial_s(\log(1-p^{-s})+p^{-s})|\lesssim |p^{-2s}|\log p$ for $\sigma\ge \sigma_0$,
and $\sum_p p^{-2\sigma_0}\log p<\infty$.
\end{proof}

\end{bluetext}
\begin{greentext}
\noindent\textbf{Author note.}
Boundary passage $\sigma\downarrow\tfrac12$ is handled in the ``tested'' (distributional) sense used throughout this appendix:
see Lemma~\ref{lem:det2-unsmoothed} (unsmoothed tested control for $\partial_\sigma\Re\log\dettwo$), Lemma~\ref{lem:xi-deriv-L1} (tested control for $\partial_\sigma\Re\log\xi$),
and Theorem~\ref{thm:phase-velocity-quant} (construction of the $L^1_{\rm loc}$ boundary log-modulus and the resulting phase--velocity identity).
\end{greentext}

\begin{lemma}[Local certificate \(\Rightarrow\) a.e.\ boundary wedge]\label{lem:local-to-global-wedge}
Let \(w\) be the boundary phase of a unimodular boundary function \(J\) with \(|J(\tfrac12+it)|=1\) a.e., and \(-w'\) its (positive) boundary distribution.
Fix an even cutoff profile \(\psi_{\rm cut}\in C_c^\infty([-2,2])\) with \(0\le \psi_{\rm cut}\le 1\) and \(\psi_{\rm cut}\equiv 1\) on \([-1,1]\).
For a Whitney interval \(I=[t_0-L,t_0+L]\), define the associated (smoothed triangular/hat) cutoff
\[
  \varphi_I(t)\ :=\ \psi_{\rm cut}\!\left(\frac{t-t_0}{L}\right),
\]
so \(0\le \varphi_I\le 1\), \(\varphi_I\equiv 1\) on \(I\), and \(\operatorname{supp}\varphi_I\subset 2I\).
Assume that for every Whitney interval \(I\) (with the fixed schedule) one has the local certificate bound
\[
  \int_\R \varphi_I(t)\,(-w')(t)\,dt\ \le\ \pi\,\Upsilon\qquad(\Upsilon<\tfrac12).
\]
Then, after a unimodular rotation of the outer, \(|w(t)|\le \pi\Upsilon\) for a.e.\ \(t\), hence \textup{(P+)} holds.
\end{lemma}
\begin{proof}
Let \(\Delta_I(w):=\operatorname*{ess\,sup}_I w-\operatorname*{ess\,inf}_I w\).
Since \(-w'\) is a positive distribution and \(\varphi_I\ge \mathbf 1_I\),
\[
  \Delta_I(w)\ \le\ \int_I (-w')\ \le\ \int_\R \varphi_I(t)\,(-w')(t)\,dt\ \le\ \pi\,\Upsilon
\]
uniformly on Whitney \(I\).
Whitney intervals shrink to points with bounded overlap; subtract a median to re-center \(w\), then pass \(I\downarrow\{t\}\) to get \(|w(t)|\le\pi\Upsilon\) a.e.
Since \(\Upsilon<\tfrac12\), \textup{(P+)} follows.
\end{proof}

\subsection*{Phase--velocity identity (quantitative form) and boundary passage}\label{app:phase-velocity}
\begin{lemma}[Outer--Hilbert boundary identity]\label{lem:outer-phase-HT}
Let \(u\in L^1_{\mathrm{loc}}(\mathbb R)\) and let \(O\) be the outer function on \(\Omega\) with boundary modulus \(|O(\tfrac12+it)|=e^{u(t)}\) a.e.
Then, in \(\mathcal D'(\mathbb R)\),
\[
  \frac{d}{dt}\Arg O\!\left(\tfrac12+it\right)=\Hilb[u'](t),
\]
where \(\Hilb\) is the boundary Hilbert transform on \(\R\) and \(u'\) is the distributional derivative.
\end{lemma}
\begin{proof}
Write \(\log O=U+iV\) on \(\Omega\), where \(U\) is the Poisson extension of \(u\) and \(V\) is its harmonic conjugate with \(V(\tfrac12+\cdot)=\Hilb[u]\) in \(\mathcal D'(\mathbb R)\).
Then \(\tfrac{d}{dt}\Arg O=\partial_t V=\Hilb[\partial_t U]=\Hilb[u']\) in distributions.
\end{proof}

\begin{lemma}[Smoothed distributional bound for \(\partial_\sigma\,\Re\log\dettwo\)]\label{lem:det2-unsmoothed}
Let \(I\Subset\R\) be a compact interval and fix \(\varepsilon_0\in(0,\tfrac12]\).
There exists a finite constant
\[
  C_*\ :=\ \sum_{p}\sum_{k\ge 2}\frac{p^{-k/2}}{k^2\,\log p}\ <\ \infty
\]
such that for all \(\sigma\in(\tfrac12,\tfrac12+\varepsilon_0]\) and every \(\varphi\in C_c^2(I)\),
\[
  \Big|\int_{\R} \varphi(t)\,\partial_\sigma\Re\log\dettwo\big(I-A(\sigma+it)\big)\,dt\Big|\ \le\ C_*\,\|\varphi''\|_{L^1(I)}.
\]
\end{lemma}
\begin{proof}
For \(\sigma>\tfrac12\) one has the absolutely convergent expansion
\[
  \partial_\sigma\,\Re\log\dettwo\big(I-A(\sigma+it)\big)
  \;=\; \sum_{p}\sum_{k\ge 2} (\log p)\,p^{-k\sigma}\cos(k t\log p).
\]
For each frequency \(\omega=k\log p\ge 2\log 2\), two integrations by parts give
\[
  \Big|\int_{\R}\!\varphi(t)\cos(\omega t)\,dt\Big|\ \le\ \frac{\|\varphi''\|_{L^1(I)}}{\omega^2}.
\]
Summing the resulting majorant yields
\[
  \Big|\int \varphi\,\partial_\sigma\Re\log\dettwo\,dt\Big|
  \ \le\ \|\varphi''\|_{L^1}\sum_{p}\sum_{k\ge 2}\frac{(\log p)\,p^{-k\sigma}}{(k\log p)^2}
  \ \le\ \|\varphi''\|_{L^1}\sum_{p}\sum_{k\ge 2}\frac{p^{-k/2}}{k^2\,\log p},
\]
uniformly for \(\sigma\in(\tfrac12,\tfrac12+\varepsilon_0]\), since the rightmost double series converges.
\end{proof}

\begin{lemma}[De-smoothing to \(L^1\) control]\label{lem:desmooth-L1}
Fix a compact interval \(I\Subset\R\).
Suppose a family \(g_\varepsilon\in\mathcal D'(I)\) satisfies
\[
  \big|\langle g_\varepsilon,\,\phi''\rangle\big|\ \le\ C_I\,\|\phi''\|_{L^1(I)}\qquad\forall\,\phi\in C_c^\infty(I),\ \forall\,\varepsilon\in(0,\varepsilon_0].
\]
Then \(g_\varepsilon\) is uniformly bounded in \(W^{-2,\infty}(I)\) and there exist primitives \(u_\varepsilon\in BV(I)\) with \(u_\varepsilon' = g_\varepsilon\) in \(\mathcal D'(I)\) such that, along a subsequence, \(u_\varepsilon\to u\) in \(L^1(I)\).
\end{lemma}
\begin{proof}
Define \(\Lambda_\varepsilon(\psi):=\langle g_\varepsilon,\,\psi\rangle\) for \(\psi\in C_c^\infty(I)\).
For any \(\psi\in C_c^\infty(I)\) let \(\Phi\in C_c^\infty(I)\) solve \(\Phi''=\psi\) with zero boundary data on \(I\) (obtainable by two integrations).
Then \(\|\Phi''\|_{L^1}=\|\psi\|_{L^1}\) and by hypothesis
\[
  |\Lambda_\varepsilon(\psi)|
  \ =\ |\langle g_\varepsilon,\Phi''\rangle|
  \ \le\ C_I\,\|\Phi''\|_{L^1}
  \ =\ C_I\,\|\psi\|_{L^1}.
\]
Thus \(\|g_\varepsilon\|_{W^{-2,\infty}(I)}\le C_I\) uniformly in \(\varepsilon\).

Fix any \(x_0\in I\).
Let \(G\) be the Green operator for \(\partial_t^2\) on \(I\) with homogeneous boundary data.
Define \(u_\varepsilon:=G[g_\varepsilon]+c_\varepsilon\), where \(c_\varepsilon\) makes \(\int_I u_\varepsilon=0\).
Then \(u_\varepsilon' = g_\varepsilon\) in distributions and the total variation \(\mathrm{Var}_I(u_\varepsilon)\) is uniformly bounded.
By the compact embedding \(BV(I)\hookrightarrow L^1(I)\) (Helly selection), a subsequence converges in \(L^1(I)\).
\end{proof}

\begin{lemma}[Arithmetic Carleson energy]\label{lem:carleson-arith}
Let
\[
 U_{\det_2}(\sigma,t)\ :=\ \Re\log\dettwo\!\Big(I-A\big(\tfrac12+\sigma+it\big)\Big)
 \ =\ -\sum_{p}\sum_{k\ge 2}\frac{p^{-k/2}}{k}\,e^{-k\log p\,\sigma}\,\cos\big(k\log p\,t\big),\qquad \sigma>0,
\]
where the series converges absolutely for every \(\sigma>0\).
Then for every interval \(I\subset\R\) with Carleson box \(Q(I):=I\times(0,|I|]\),
\[
 \iint_{Q(I)} |\nabla U_{\det_2}|^2\,\sigma\,dt\,d\sigma\ \le\ \frac{|I|}{4}\,\sum_{p}\sum_{k\ge 2}\frac{p^{-k}}{k^2}
 \ =:\ K_0\,|I|,\qquad K_0:=\frac{1}{4}\sum_{p}\sum_{k\ge 2}\frac{p^{-k}}{k^2}<\infty.
\]
\end{lemma}
\begin{proof}
For a single mode \(b\,e^{-\omega\sigma}\cos(\omega t)\) one has \(|\nabla|^2=b^2\omega^2e^{-2\omega\sigma}\), hence
\[
 \int_0^{|I|}\!\int_I |\nabla|^2\,\sigma\,dt\,d\sigma
 \ \le\ |I|\cdot\sup_{\omega>0}\int_0^{|I|}\sigma\,\omega^2e^{-2\omega\sigma}d\sigma\cdot b^2
 \ \le\ \tfrac14\,|I|\,b^2.
\]
With \(b=p^{-k/2}/k\) and \(\omega=k\log p\), summing over \((p,k)\) gives the claim and the finiteness of \(K_0\).
\end{proof}

\paragraph{Whitney scale and short--interval zero counts.}
Throughout the boundary-certificate route we work on Whitney boxes based at height \(T\) with
\[
  L=L(T):=\min\Big\{\frac{c}{\log\angles{T}},\ L_\star\Big\},\qquad
  \angles{T}:=\sqrt{1+T^2},\qquad c\in(0,1]\ \text{fixed}.
\]
The only input about the \emph{number} of zeros used below is the classical short-interval consequence of Riemann--von Mangoldt: there exist absolute constants \(A_0,A_1>0\) such that for \(T\ge 2\) and \(0<H\le 1\),
\[
  N(T;H)\ :=\ \#\{\rho=\beta+i\gamma:\ \gamma\in[T,T+H]\}\ \le\ A_0\ +\ A_1\,H\,\log\angles{T}.
\]

\begin{lemma}[Annular Poisson--balayage \(L^2\) bound]\label{lem:annular-balayage}
Let \(I=[T-L,T+L]\), \(Q_\alpha(I)=I\times(0,\alpha L]\), and fix \(k\ge 1\).
For
\(
\mathcal A_k:=\{\rho=\beta+i\gamma:\ 2^kL<|T-\gamma|\le 2^{k+1}L\}
\)
set
\[
  V_k(\sigma,t):=\sum_{\rho\in\mathcal A_k}\frac{\sigma}{(t-\gamma)^2+\sigma^2}.
\]
Then
\[
  \iint_{Q_\alpha(I)} V_k(\sigma,t)^2\,\sigma\,dt\,d\sigma\ \ll_\alpha\ |I|\,4^{-k}\,\nu_k,
\]
where \(\nu_k:=\#\mathcal A_k\), and the implicit constant depends only on \(\alpha\).
\end{lemma}
\begin{proof}
Write \(K_\sigma(x):=\sigma/(x^2+\sigma^2)\) and \(V_k=\sum_{\rho\in\mathcal A_k}K_\sigma(\cdot-\gamma)\).
Integrate over \(t\in I\) first.
For the diagonal terms, using \(|t-\gamma|\ge 2^kL-L\ge 2^{k-1}L\) for \(t\in I\) and \(k\ge 1\),
\[
 \int_I K_\sigma(t-\gamma)^2\,dt
 = \sigma^2\!\int_I \frac{dt}{\big((t-\gamma)^2+\sigma^2\big)^2}
 \ \le\ \frac{L}{(2^{k-1}L)^2}\,\sigma.
\]
Multiplying by the area weight \(\sigma\) and integrating \(\sigma\in(0,\alpha L]\) gives a contribution \(\ll_\alpha |I|\,4^{-k}\) per \(\gamma\), hence \(\ll_\alpha |I|\,4^{-k}\nu_k\) after summing.
For off-diagonal terms, for \(i\ne j\) one has on \(I\) that \(K_\sigma(t-\gamma_j)\le \sigma/(2^{k-1}L)^2\), hence
\[
 \int_I K_\sigma(t-\gamma_i)K_\sigma(t-\gamma_j)\,dt
 \ \le\ \frac{\sigma}{(2^{k-1}L)^2}\int_\R K_\sigma(t-\gamma_i)\,dt
 = \frac{\pi\sigma}{(2^{k-1}L)^2},
\]
and integrating \(\sigma\in(0,\alpha L]\) with the extra factor \(\sigma\) yields \(\ll_\alpha |I|\,4^{-k}\).
Summing over pairs \((i,j)\) via a Schur test gives the stated bound (absorbing constants into \(\ll_\alpha\)).
\end{proof}

\begin{lemma}[Analytic (\(\xi\)) Carleson energy on Whitney boxes]\label{lem:carleson-xi}
There exist absolute constants \(c\in(0,1]\) and \(C_\xi<\infty\) such that for every interval \(I=[T-L,\,T+L]\) at Whitney scale \(L=c/\log\angles{T}\), the Poisson extension
\[
 U_{\xi}(\sigma,t):=\Re\log\xi\big(\tfrac12+\sigma+it\big)\qquad(\sigma>0)
\]
obeys the Carleson bound
\[
  \iint_{Q(I)} |\nabla U_{\xi}(\sigma,t)|^2\,\sigma\,dt\,d\sigma\ \le\ C_\xi\,|I|.
\]
\end{lemma}
\begin{proof}
Fix \(I=[T-L,T+L]\) with \(L=c/\log\angles{T}\) and a fixed aperture \(\alpha\in[1,2]\).
Neutralize near zeros by a local half-plane Blaschke product \(B_I\) removing zeros of \(\xi\) inside a fixed dilate \(Q(\alpha'I)\) (\(\alpha'>\alpha\)).
This yields a harmonic field \(\widetilde U_\xi\) on \(Q(\alpha I)\) and
\[
  \iint_{Q(\alpha I)} |\nabla U_\xi|^2\,\sigma\,dt\,d\sigma
  \ \asymp\
  \iint_{Q(\alpha I)} |\nabla \widetilde U_\xi|^2\,\sigma\,dt\,d\sigma\ +\ O_\alpha(|I|),
\]
so it suffices to bound the neutralized energy.

Write \(\partial_\sigma U_\xi=\Re(\xi'/\xi)=\Re\sum_\rho (s-\rho)^{-1}+A\), where \(A\) is smooth on compact strips.
Since \(U_\xi\) is harmonic, \(|\nabla U_\xi|^2\asymp |\partial_\sigma U_\xi|^2\) on \(\R^2_+\); thus we bound the \(L^2(\sigma\,dt\,d\sigma)\) norm of \(\sum_\rho (s-\rho)^{-1}\) over \(Q(\alpha I)\).
Decompose the (neutralized) zeros into Whitney annuli
\(
\mathcal A_k:=\{\rho:2^kL<|\gamma-T|\le 2^{k+1}L\}
\), \(k\ge 1\).
For \(V_k(\sigma,t):=\sum_{\rho\in\mathcal A_k} K_\sigma(t-\gamma)\) with \(K_\sigma(x):=\sigma/(x^2+\sigma^2)\), Lemma~\ref{lem:annular-balayage} gives
\[
  \iint_{Q_\alpha(I)} V_k(\sigma,t)^2\,\sigma\,dt\,d\sigma\ \le\ C_\alpha\,|I|\,4^{-k}\,\nu_k,
\]
where \(\nu_k:=\#\mathcal A_k\) and \(C_\alpha\) depends only on \(\alpha\).
Summing Cauchy--Schwarz over annuli yields
\[
  \iint_{Q(\alpha I)} \Big|\sum_{\rho}(s-\rho)^{-1}\Big|^2\,\sigma\,dt\,d\sigma
  \ \le\ C_\alpha\,|I|\sum_{k\ge 1}4^{-k}\,\nu_k.
\]
To bound \(\nu_k\), use the short-interval zero-count bound above to obtain, for some absolute \(a_1(\alpha),a_2(\alpha)\),
\[
  \nu_k\ \le\ a_1(\alpha)\,2^k L\,\log\angles{T}\ +\ a_2(\alpha)\,\log\angles{T}.
\]
Therefore,
\[
  \sum_{k\ge1}4^{-k}\,\nu_k\ \ll\ L\,\log\angles{T}\ +\ 1.
\]
On Whitney scale \(L=c/\log\angles{T}\) this is \(\ll 1\).
Adding the neutralized near-field \(O(|I|)\) and the smooth \(A\) contribution, we obtain
\[
  \iint_{Q(\alpha I)} |\nabla U_\xi|^2\,\sigma\,dt\,d\sigma\ \le\ C_\xi\,|I|,
\]
with \(C_\xi\) depending only on \((\alpha,c)\).
\end{proof}

\begin{bluetext}
\subsection*{Referee bridge: from Carleson energy to ``bounded type'' for $F_\varepsilon$}\label{app:carleson-to-boundedtype}
Define on $\sigma>0$ the harmonic fields
\[
U_{\det_2}(\sigma,t)=\Re\log\dettwo\!\Big(I-A(\tfrac12+\sigma+it)\Big),
\qquad
U_{\xi}(\sigma,t)=\Re\log\xi(\tfrac12+\sigma+it),
\]
and set $U_{F} := U_{\det_2}-U_{\xi}=\Re\log F_\varepsilon$ on the half-plane $\{\Re s>\tfrac12+\varepsilon\}$ (with $\sigma>\varepsilon$).

\noindent\textbf{What the current lemmas give.}
Lemma~\ref{lem:carleson-arith} gives a \emph{global} Carleson bound for $U_{\det_2}$ on every box $Q(I)$.
Lemma~\ref{lem:carleson-xi} gives the corresponding Carleson bound for $U_\xi$ on Whitney boxes (after neutralizing near zeros).

\noindent\textbf{Standard implication (must be cited / checked).}
A harmonic function $U$ on the upper half-plane has boundary values in $\mathrm{BMO}(\mathbb R)$ iff
$|\nabla U|^2\,\sigma\,dt\,d\sigma$ is a Carleson measure (Fefferman--Stein).
Moreover, if $f$ is analytic and $U=\Re\log f$ has such a Carleson bound on $\sigma>\varepsilon$,
then $f$ belongs to the Smirnov/Nevanlinna class on that half-plane (``bounded type''),
and admits a canonical inner--outer factorization.
See, e.g., \cite[Ch.~VI]{Garnett} (BMOA/Carleson measures) and \cite[Ch.~2]{RosenblumRovnyak}.

\noindent\textbf{Concrete obligation to validate (H1).}
To justify (H1) in the factorization step, the manuscript should explicitly state and verify:
\begin{itemize}
\item[(B1)] that $\log F_\varepsilon$ is holomorphic on $\{\Re s>\tfrac12+\varepsilon\}$ (zeros/poles removed as needed);
\item[(B2)] that the Carleson measure bound for $|\nabla U_{F}|^2\,\sigma\,dt\,d\sigma$ holds on the boxes used in the argument
(typically Whitney boxes at scale $L(T)$), after accounting for the local Blaschke neutralizations in Lemma~\ref{lem:carleson-xi};
\item[(B3)] that any remaining singular inner contribution on the boundary is either absent or explicitly included (cf.\ (H3)).
\end{itemize}
\end{bluetext}

\begin{greentext}
\begin{lemma}[Bounded-type regularity from Carleson energy (usable form)]\label{lem:boundedtype-from-carleson}
Fix $\varepsilon>0$ and write $D_\varepsilon:=\{\Re s>\tfrac12+\varepsilon\}$.
Assume that on every Carleson box $Q(I)\subset D_\varepsilon$ the measure $|\nabla \Re\log F_\varepsilon|^2\,\sigma\,dt\,d\sigma$ is Carleson (after local neutralization near zeros of $\xi$ as in Lemma~\ref{lem:carleson-xi}).
Then $F_\varepsilon$ has bounded characteristic on $D_\varepsilon$ (hence is of bounded type on each connected component of $D_\varepsilon\setminus Z(F_\varepsilon)$) and admits the standard inner--outer factorization, including a possible singular inner part.
\end{lemma}
\begin{proof}[Reference]
This is a standard consequence of the Fefferman--Stein characterization (Carleson energy $\Leftrightarrow$ BMO boundary traces) together with the half-plane Hardy/Smirnov theory for bounded characteristic functions; see \cite[Ch.~VI]{Garnett} and \cite[Ch.~2]{RosenblumRovnyak}.
\end{proof}
\end{greentext}



\begin{lemma}[L$^1$-tested control for \(\partial_\sigma\Re\log\xi\)]\label{lem:xi-deriv-L1}
For each compact \(I\Subset\R\) there exists \(C'_I<\infty\) such that for all \(0<\sigma\le\varepsilon_0\) and all \(\phi\in C_c^2(I)\),
\[
  \Big|\int_I \phi(t)\,\partial_\sigma\Re\log\xi\!\big(\tfrac12+\sigma+it\big)\,dt\Big|
  \ \le\ C'_I\,\|\phi\|_{H^1(I)}.
\]
\end{lemma}
\begin{proof}
Let \(V\) be the Poisson extension of \(\phi\) on a fixed dilation \(Q(\alpha I)\).
Green's identity together with Cauchy--Riemann for \(U_\xi=\Re\log\xi\) gives
\[
  \int_I \phi(t)\,\partial_\sigma\Re\log\xi\!\big(\tfrac12+\sigma+it\big)\,dt
  \,=\, \iint_{Q(\alpha I)} \nabla U_\xi\cdot\nabla V\,dt\,d\sigma.
\]
By Cauchy--Schwarz and the scale-invariant bound \(\|\nabla V\|_{L^2(\sigma)}\lesssim \|\phi\|_{H^1(I)}\), together with Lemma~\ref{lem:carleson-xi}, we obtain the claim.
\end{proof}

\begin{theorem}[Quantified phase--velocity identity and boundary passage]\label{thm:phase-velocity-quant}\label{thm:pv-arg-identity}
Let
\[
 u_\varepsilon(t):=\log\big|\dettwo(I-A(\tfrac12+\varepsilon+it))\big|-\log\big|\xi(\tfrac12+\varepsilon+it)\big|.
\]
Then \(u_\varepsilon\) is uniformly \(L^1\)-bounded and Cauchy on compact \(I\Subset\R\) as \(\varepsilon\downarrow 0\), so \(u_\varepsilon\to u\) in \(L^1_{\rm loc}(\R)\).
Let \(\mathcal O\) be the outer on \(\Omega\) with boundary modulus \(e^{u}\), and set
\[
  \mathcal J(s):=\frac{\dettwo(I-A(s))}{\mathcal O(s)\,\xi(s)}.
\]
Then \(|\mathcal J(\tfrac12+it)|=1\) a.e.\ and, in the distributional sense on compact \(I\Subset\R\),
\begin{bluetext}
\paragraph{Referee-tightening for \eqref{eq:pv-identity}.}
To make \eqref{eq:pv-identity} checkable, we isolate the analytic statement being used.

\begin{greentext}
\begin{lemma}[Distributional phase--velocity identity for an outer function]\label{lem:pv-distributional}
Let \(u\in L^1_{\mathrm{loc}}(\mathbb R)\), and let \(g\) be an outer function on \(\Omega=\{\Re s>\tfrac12\}\) whose nontangential boundary values satisfy
\(|g(\tfrac12+it)|=e^{u(t)}\) for Lebesgue-a.e.\ \(t\).
Let \(w(t):=\Arg g(\tfrac12+it)\) be any measurable choice of boundary argument (defined for a.e.\ \(t\)).
Then, in \(\mathcal D'(\mathbb R)\),
\[
  \frac{d}{dt}w(t)\;=\;\Hilb[u'](t),
\]
where \(\Hilb\) is the boundary Hilbert transform and \(u'\) is the distributional derivative.
Equivalently, for any \(\varphi\in C_c^\infty(\mathbb R)\),
\[
  -\int_{\mathbb R} w(t)\,\varphi'(t)\,dt
  \;=\;
  \int_{\mathbb R} u(t)\,(\Hilb\varphi)'(t)\,dt .
\]
\end{lemma}

\begin{proof}
This is exactly Lemma~\ref{lem:outer-phase-HT} (Outer--Hilbert boundary identity) proved earlier in this appendix.
\end{proof}
\end{greentext}

\end{bluetext}
\begin{greentext}
\noindent\textbf{Author note (application of Lemma~\ref{lem:pv-distributional}).}
In the application to \eqref{eq:pv-identity}, $g$ is taken to be an outer (hence zero-free) holomorphic function on $\Omega$ built from the boundary log-modulus (Lemma~\ref{lem:outer-from-logmodulus} and Lemma~\ref{lem:outer-existence-stability}),
so $g^*(t)\neq 0$ a.e.\ and $\log|g^*|\in L^1_{\rm loc}$ on compact intervals by construction.
\end{greentext}

\begin{equation}\label{eq:pv-identity}
\int_I \phi(t)\,(-w'(t))\,dt
\ =\ \pi\!\int_I \phi(t)\,d\mu_{\rm off}(t)\ +\ \pi\!\int_I \phi(t)\,d\nu_{\rm sing}(t)\ +\ \pi\sum_{\gamma\in I} m_\gamma\,\phi(\gamma)
\end{equation}
for all \(\phi\in C_c^\infty(I)\), \(\phi\ge 0\), where:
\(\mu_{\rm off}\) is the Poisson balayage of off--critical zeros,
\(\nu_{\rm sing}\) is the (possibly zero) singular boundary measure associated to any singular inner factor,
and the discrete sum ranges over boundary zeros/poles on $\Re s=\tfrac12$ (in particular, critical-line ordinates), with multiplicities \(m_\gamma\).

\begin{bluetext}
\paragraph{Referee-tightening: explicit definition of $\mu_{\rm off}$ and the boundary terms in \eqref{eq:pv-identity}.}
Let $\rho=\beta+i\gamma$ range over zeros of $\zeta$ with $\beta>\tfrac12$ (``off--critical'' in $\Omega$), counted with multiplicity $m_\rho$.
Define a discrete positive measure in the open half-plane
\[
\nu_{\rm off}\ :=\ \sum_{\rho:\,\Re\rho>\frac12} m_\rho\,\delta_{\rho}.
\]
Its Poisson balayage onto the boundary line $\Re s=\tfrac12$ is the absolutely continuous measure
\[
d\mu_{\rm off}(t)\ :=\ \int_{\Re s>\frac12} P_{\beta-\frac12}(t-\gamma)\,d\nu_{\rm off}(\beta+i\gamma)
\quad\text{with}\quad
P_{y}(x)=\frac{1}{\pi}\frac{y}{x^2+y^2}.
\]
Equivalently, on any interval $I\Subset\mathbb R$,
\[
\mu_{\rm off}(I)\ =\ \sum_{\rho:\,\Re\rho>\frac12} m_\rho\int_I \frac{1}{\pi}\frac{\Re\rho-\frac12}{(t-\Im\rho)^2+(\Re\rho-\frac12)^2}\,dt,
\]
which is locally finite.
\medskip

\noindent\textbf{Singular inner term.}
If the bounded-type factorization of the relevant boundary unimodular function includes a singular inner factor, its associated singular boundary measure is denoted by $\nu_{\rm sing}$ in \eqref{eq:pv-identity}.
(If no singular inner factor is present on the boundary interval under discussion, take $\nu_{\rm sing}\equiv 0$.)

\noindent\textbf{Atomic term on $\Re s=\tfrac12$.}
The coefficients $m_\gamma$ in \eqref{eq:pv-identity} denote the multiplicities of boundary zeros/poles at $s=\tfrac12+i\gamma$ (in particular, critical-line zeros of $\xi$/$\zeta$) and produce the Dirac masses $\sum_\gamma m_\gamma\,\delta_\gamma$.
\end{bluetext}
\begin{greentext}
\noindent\textbf{Author note (``off--critical'' vs.\ left half-plane zeros).}
Throughout, ``off--critical zeros'' means zeros with $\Re\rho>\tfrac12$, i.e.\ interior zeros in $\Omega$.
These are exactly the zeros that contribute Blaschke factors (hence Poisson-kernel balayage terms) in the half-plane factorization on the $\Omega$ side.
Zeros with $\Re\rho<\tfrac12$ lie outside $\Omega$ and are handled separately (via the $\xi(s)=\xi(1-s)$ symmetry), and do not enter the $\Omega$-side boundary measure $\mu_{\rm off}$.
\end{greentext}

\end{theorem}
\begin{proof}
Fix a compact interval \(I\Subset\R\) and \(\varepsilon_0\in(0,\tfrac12]\).
By Lemma~\ref{lem:det2-unsmoothed} and Lemma~\ref{lem:xi-deriv-L1}, the family \(u_\varepsilon\) is Cauchy in \(L^1(I)\); the de-smoothing lemma (Lemma~\ref{lem:desmooth-L1}) yields \(u_\varepsilon\to u\) in \(L^1(I)\).
We now record the half-plane outer passage used here.
\begin{lemma}[Outer existence and stability under \(L^1\) convergence]\label{lem:outer-existence-stability}
Let \(I\Subset\R\) be compact and let \(u_n,u\in L^1(I)\) with \(u_n\to u\) in \(L^1(I)\).
For each \(n\), let \(O_n\) be the outer function on \(\Omega\) normalized by \(O_n(\tfrac32)>0\) and boundary modulus \(|O_n(\tfrac12+it)|=e^{u_n(t)}\) a.e.\ on \(I\).
Then there exists an outer \(O\) on \(\Omega\), normalized by \(O(\tfrac32)>0\), with \(|O(\tfrac12+it)|=e^{u(t)}\) a.e.\ on \(I\), and \(O_n\to O\) locally uniformly on compact subsets of \(\Omega\).
\end{lemma}
\begin{proof}
By the half-plane outer representation (see, e.g., \cite[Ch.~II]{Garnett} or \cite[Ch.~2]{RosenblumRovnyak}),
for each \(n\) one may write \(\log O_n = P[u_n] + i\,\Hilb[u_n]\) on \(\Omega\), where \(P[u_n]\) is the Poisson extension and \(\Hilb[u_n]\) its harmonic conjugate (normalized by the condition \(O_n(\tfrac32)>0\)).
Since \(u_n\to u\) in \(L^1(I)\), Poisson extension is continuous \(L^1(I)\to C^\infty_{\rm loc}(\Omega)\), hence \(P[u_n]\to P[u]\) locally uniformly, and similarly \(\Hilb[u_n]\to\Hilb[u]\) locally uniformly after fixing the same normalization.
Exponentiating gives local uniform convergence \(O_n\to O:=\exp(P[u]+i\Hilb[u])\), and \(O\) is outer with the stated boundary modulus.
\end{proof}
Applying Lemma~\ref{lem:outer-existence-stability} on each compact \(I\Subset\R\) and a diagonal subsequence yields an outer \(\mathcal O\) on \(\Omega\) with a.e.\ boundary modulus \(e^{u}\) and locally uniform convergence of \(\mathcal O_\varepsilon\to\mathcal O\).


\begin{bluetext}
\subsection*{Referee requirement: hypotheses needed to ``sum the Blaschke contributions''}\label{app:blaschke-sum}
The proof below uses canonical factorization in the half-plane $\{\Re s>\tfrac12+\varepsilon\}$ and then differentiates boundary arguments.
To justify the step ``Summing the Blaschke contributions of interior poles/zeros yields the Poisson balayage term'',
the manuscript must also address the following \emph{at this point} (not later):
\begin{itemize}
\item[(S1)] \textbf{No hidden singular inner factor.}  Either prove that $I_\varepsilon$ has no singular inner part on the boundary interval used,
or include the additional distributional term coming from the singular boundary measure in the identity (cf.\ (H3)).
\item[(S2)] \textbf{Explicit atomic term on $\Re s=\tfrac12$.}  Replace ``plus atoms at critical-line ordinates'' by an explicit formula:
are these Dirac masses arising from boundary zeros/poles of $\xi$/$\zeta$, or from the $\varepsilon\downarrow0$ limit?  Specify the measure.
\end{itemize}


the manuscript must explicitly verify the following standard hypotheses (Hardy/Nevanlinna theory in a half-plane; see, e.g.,
\cite[Ch.~II]{Garnett} and \cite[Ch.~2]{RosenblumRovnyak}):

\begin{enumerate}
\item[(H1)] \textbf{Bounded type / Hardy control.} For each fixed $\varepsilon>0$, the function
$F_\varepsilon=\dettwo/\xi$ is analytic and of bounded type in the half-plane $\{\Re s>\tfrac12+\varepsilon\}$,
so that it admits a canonical inner--outer factorization $F_\varepsilon = I_\varepsilon O_\varepsilon$.

\item[(H2)] \textbf{Blaschke summability for zeros/poles.} The zero/pole set $\{\rho\}$ of $F_\varepsilon$ in $\{\Re s>\tfrac12+\varepsilon\}$
(counted with multiplicity and with poles treated as negative multiplicity) satisfies the half-plane Blaschke condition, e.g.
\[
\sum_{\rho:\,\Re\rho>\frac12+\varepsilon} m_\rho\,
\frac{\Re\rho-(\frac12+\varepsilon)}{|\rho-(\frac12+\varepsilon)|^2}\ <\ \infty,
\]
which guarantees that the Blaschke product converges and that the boundary argument identity produces a Poisson-kernel sum.

\item[(H3)] \textbf{Singular inner part.} Either (a) $I_\varepsilon$ has no singular inner factor on the boundary intervals under consideration,
or (b) the singular inner factor is included explicitly, with its associated singular boundary measure contributing an additional distributional term
to $\frac{d}{dt}\Arg I_\varepsilon$.
(As written, the proof accounts only for Blaschke factors; any singular inner part would be an extra, presently untracked contribution.)

\item[(H4)] \textbf{Local finiteness / Carleson control of the induced measure.} The discrete measure of off--critical zeros
$\nu_{\rm off}$ used in the balayage definition is locally finite on the relevant strip,
so that the Poisson integral defining $\mu$ is finite for a.e.\ $t$ and yields a locally finite boundary measure.
If later steps require quantitative bounds on $\mu$ (Carleson or domination (AC$_\mu$)), these must be established separately.
\end{enumerate}

\noindent\textbf{Status:} the current Appendix states the conclusion of this summation but does not yet verify (H1)--(H4) from the operator/determinant setup.
\end{bluetext}

\begin{greentext}
\noindent\textbf{Author revision (resolving S1/S2; clarifying H1--H4).}
Item \textbf{(S1)} is resolved by writing the phase--velocity identity with an explicit singular-inner contribution $\nu_{\rm sing}$ (see \eqref{eq:pv-identity} and the paragraph immediately following it).
Item \textbf{(S2)} is resolved by making the atomic term $\sum_\gamma m_\gamma\,\delta_\gamma$ explicit in \eqref{eq:pv-identity}.
For \textbf{(H1)--(H2)}, the intended route is: Carleson-energy control of $\Re\log F_\varepsilon$ on $\{\Re s>\tfrac12+\varepsilon\}$ implies bounded characteristic (hence bounded type and canonical factorization) by standard Hardy/Smirnov theory; see \cite[Ch.~VI]{Garnett} and \cite[Ch.~2]{RosenblumRovnyak}.
\end{greentext}

For the phase--velocity identity, factor \(F_\varepsilon=\dettwo/\xi=I_\varepsilon\,O_\varepsilon\) (inner--outer) on \(\{\Re s>\tfrac12+\varepsilon\}\).
By Lemma~\ref{lem:outer-phase-HT}, the boundary argument of \(O_\varepsilon\) satisfies \(\frac{d}{dt}\Arg O_\varepsilon=\Hilb[u_\varepsilon']\) in \(\mathcal D'(\R)\).
Summing the Blaschke contributions of interior poles/zeros \emph{together with any singular inner contribution} yields the Poisson balayage term $\mu_{\rm off}$, the singular boundary measure $\nu_{\rm sing}$, and the boundary atoms $\sum_\gamma m_\gamma\,\delta_\gamma$; passage \(\varepsilon\downarrow 0\) gives \eqref{eq:pv-identity}.
\end{proof}

\begin{lemma}[\(\zeta\)–normalized outer and compensator]\label{lem:zeta-normalization}
Let \(\mathcal O_\zeta\) be the outer on \(\Omega\) with a.e.\ boundary modulus \(|\dettwo(I-A)/\zeta|\), and define
\[
  J_\zeta(s)\ :=\ \frac{\dettwo(I-A(s))}{\mathcal O_\zeta(s)\,\zeta(s)}\cdot B(s),
  \qquad B(s):=\frac{s-1}{s}.
\]
Then \(|J_\zeta(\tfrac12+it)|=1\) a.e.\ and the phase--velocity identity of Theorem~\ref{thm:phase-velocity-quant} holds for \(J_\zeta\) with the same Poisson/zero right-hand side.
\end{lemma}
\begin{proof}
Write \(\xi(s)=G(s)\zeta(s)\) where \(G(s)=\tfrac12 s(1-s)\pi^{-s/2}\Gamma(\tfrac s2)\) differs from the main-text completion by a unimodular constant.
Let \(\mathcal O_\xi\) be the outer with boundary modulus \(|\dettwo/\xi|\).
On \(\Re s=\tfrac12\) one has unimodularity of both \(\dettwo/(\mathcal O_\xi\xi)\) and \(\dettwo/(\mathcal O_\zeta\zeta)\).
The outer ratio \(\mathcal O_\xi/\mathcal O_\zeta\) cancels the boundary phase contribution of \(\log G\) (Lemma~\ref{lem:outer-phase-HT}); the remaining inner contribution at \(s=1\) is accounted for by the half-plane Blaschke factor \(B(s)=(s-1)/s\).
Thus the tested phase--velocity identity transfers from \(\dettwo/(\mathcal O_\xi\xi)\) to \(J_\zeta\).
\end{proof}

\subsection*{Poisson plateau lower bound}
\begin{lemma}[Poisson plateau lower bound]\label{lem:poisson-plateau}
Let \(\psi\in C_c^\infty(\R)\) be even with \(\psi\equiv 1\) on \([-1,1]\) and \(\operatorname{supp}\psi\subset[-2,2]\).
Then
\[
  c_0(\psi)\ :=\ \inf_{0<b\le 1,\ |x|\le 1} (P_b*\psi)(x)\ \ge\ \frac{1}{2\pi}\arctan 2\;>\;0.
\]
\end{lemma}
\begin{proof}
Since \(\psi\ge \mathbf 1_{[-1,1]}\), it suffices to compute \((P_b*\mathbf 1_{[-1,1]})(x)\).
For \(|x|\le 1\),
\[
 (P_b*\mathbf 1_{[-1,1]})(x)
 =\frac{1}{\pi}\int_{-1}^{1}\frac{b}{b^2+(x-y)^2}\,dy
 =\frac{1}{2\pi}\Big(\arctan\frac{1-x}{b}+\arctan\frac{1+x}{b}\Big).
\]
This expression is minimized over \(0<b\le 1\), \(|x|\le 1\), at \((x,b)=(1,1)\), yielding \(\frac{1}{2\pi}\arctan 2\).
\end{proof}

\subsection*{From phase--velocity and CR--Green to (P+)}\label{app:assemble-pplus}

\begin{bluetext}
\subsection*{Referee correction: $\mu$-null does not imply Lebesgue-null without domination}
Lemma~\ref{lem:mu-to-lebesgue} as currently stated concludes $|\mathcal Q|=0$ from the hypothesis $\mu(\mathcal Q)=0$.
This conclusion is \emph{not valid in general} unless one proves an additional domination/comparability property of $\mu$
with respect to Lebesgue measure (cf.\ Lemma~\ref{lem:mu-dominates-lebesgue}).

\medskip
\noindent\textbf{Corrected logical form needed for Paper I.}
To obtain the Lebesgue-a.e.\ boundary inequality required by Hardy/Smirnov transport in Proposition~\ref{prop:herglotz-schur-transport}
(of \texttt{paper1\_farfield.tex}), one must add and prove a hypothesis of the following type on each boundary interval $I$ used:

\begin{quote}
\textbf{(AC$_\mu$ on $I$)} There exists $c_I>0$ such that $\mu(E)\ge c_I|E|$ for all measurable $E\subset I$.
\end{quote}

Under (AC$_\mu$), the desired implication holds:

\begin{lemma}[Corrected: $\mu$-null $\Rightarrow$ Lebesgue-null under domination]\label{lem:mu-to-lebesgue-corrected}
Fix $I\Subset\R$ and let $\mathcal Q\subset I$ be measurable. If \textbf{(AC$_\mu$ on $I$)} holds, then
\[
\mu(\mathcal Q)=0 \quad\Longrightarrow\quad |\mathcal Q|=0.
\]
\end{lemma}
\begin{proof}
Immediate from Lemma~\ref{lem:mu-dominates-lebesgue}.
\end{proof}

\noindent\textbf{Referee action item (RED):} Either prove (AC$_\mu$) for the specific Poisson-balayage measure $\mu$
constructed from off-critical zeros, \emph{or} weaken downstream claims to $\mu$-a.e.\ boundary statements and rework the main-paper transport step accordingly.
\end{bluetext}

\begin{greentext}
\noindent\textbf{Author revision (closing the RED item).}
The proof of \textup{(P+)} in this appendix does not require Lemma~\ref{lem:mu-to-lebesgue} or any domination hypothesis (AC$_\mu$):
the Lebesgue-a.e.\ wedge statement is obtained directly from oscillation control via Lemma~\ref{lem:local-to-global-wedge}.
Accordingly, the (AC$_\mu$) upgrade issue is not load-bearing for the main-paper transport step.
\end{greentext}



\begin{lemma}[Poisson lower bound $\Rightarrow$ Lebesgue-a.e.\ wedge \emph{under domination}]\label{lem:mu-to-lebesgue}
Assume the phase--velocity identity \eqref{eq:pv-identity}. Fix a compact interval $I\Subset\R$ and let
\[
\mathcal Q_I := \{t\in I:\ |w(t)-m|\ge \pi/2\}.
\]
Assume additionally that $\mu$ dominates Lebesgue on $I$ in the sense of \textbf{(AC$_\mu$ on $I$)}:
there exists $c_I>0$ such that $\mu(E)\ge c_I|E|$ for all measurable $E\subset I$.
If $\mu(\mathcal Q_I)=0$, then $|\mathcal Q_I|=0$.
Consequently the boundary wedge inequality (P$^+$) holds for Lebesgue-a.e.\ $t\in I$.
\end{lemma}

\begin{proof}
By (AC$_\mu$ on $I$), $\mu(\mathcal Q_I)=0$ implies $|\mathcal Q_I|=0$ by Lemma~\ref{lem:mu-to-lebesgue-corrected}.
The final sentence is just the definition of $\mathcal Q_I$.
\end{proof}


\begin{definition}[Admissible window class with atom avoidance]\label{def:adm-bumps}
Fix an even \(C^\infty\) window \(\psi\) with \(\psi\equiv1\) on \([-1,1]\) and \(\operatorname{supp}\psi\subset[-2,2]\).
For an interval \(I=[t_0-L,t_0+L]\), an aperture \(\alpha'>1\), and a parameter \(\varepsilon\in(0,\tfrac14]\), define \(\mathcal W_{\rm adm}(I;\varepsilon)\) to be the set of \(C^\infty\), nonnegative, mass-\(1\) bumps \(\phi\) supported in the fixed dilate \(2I=[t_0-2L,t_0+2L]\) that can be written as
\[
  \phi(t)\ =\ \frac{1}{Z}\,\frac{1}{L}\,\psi\!\left(\frac{t-t_0}{L}\right)\,m(t),
  \qquad Z=\int_{2I} \frac1L\psi\!\left(\frac{t-t_0}{L}\right)m(t)\,dt,
\]
where \(2I:=[t_0-2L,t_0+2L]\) and the mask \(m\in C^\infty(2I;[0,1])\) satisfies:
\begin{itemize}
\item[(i)] \emph{Atom avoidance.} There is a union of disjoint open subintervals \(E=\bigcup_{j=1}^{J} J_j\subset I\) with total length \(|E|\le \varepsilon L\) such that \(m\equiv0\) on \(E\) and \(m\equiv1\) on \(I\setminus E'\), where each transition layer \(E'\setminus E\) has thickness \(\le \varepsilon L\).
\item[(ii)] \emph{Uniform smoothness.} \(\|m'\|_\infty\lesssim (\varepsilon L)^{-1}\) and \(\|m''\|_\infty\lesssim (\varepsilon L)^{-2}\) with implicit constants independent of \(I,t_0,L\) and of the number/placement of the holes \(\{J_j\}\).
\end{itemize}
Every \(\phi\in\mathcal W_{\rm adm}(I;\varepsilon)\) is supported in \(2I\).
This class contains the unmasked profile \(\varphi_{L,t_0}(t)=Z_0^{-1}L^{-1}\psi((t-t_0)/L)\) with \(Z_0:=\int_{-2}^{2}\psi(x)\,dx\) (take \(E=\varnothing\), \(m\equiv1\)) and also allows dodging boundary atoms by punching out small neighborhoods while keeping total deleted length \(\le\varepsilon L\).
\end{definition}

\begin{lemma}[Uniform Poisson--energy bound for admissible tests]\label{lem:uniform-test-energy}
Let \(V_\phi\) be the Poisson extension of \(\phi\in\mathcal W_{\rm adm}(I;\varepsilon)\) to the half‑plane, and fix a cutoff to \(Q(\alpha' I)\) with \(\alpha'>1\) as in the CR--Green pairing.
Then there exists a finite constant \(\mathcal A_{\rm adm}(\psi,\varepsilon,\alpha')<\infty\), depending only on \((\psi,\varepsilon,\alpha')\), such that
\[
  \iint_{Q(\alpha' I)} |\nabla V_\phi(\sigma,t)|^2\,\sigma\,dt\,d\sigma\ \le\ \mathcal A_{\rm adm}(\psi,\varepsilon,\alpha')^2\; L.
\]
\end{lemma}
\begin{proof}
Let \(\phi(t)=Z^{-1}L^{-1}\psi((t-t_0)/L)m(t)\) be an admissible test.
By scaling of the Poisson kernel and the uniform bounds on \(m,m',m''\) from Definition~\ref{def:adm-bumps}, the \(H^1\)-size of \(\phi\) (equivalently the \(L^2(\sigma)\) Dirichlet energy of its Poisson extension on a fixed aperture box) is controlled uniformly by a constant depending only on \((\psi,\varepsilon,\alpha')\), times \(L^{1/2}\).
Squaring yields the stated \(\lesssim L\) energy bound with \(\mathcal A_{\rm adm}(\psi,\varepsilon,\alpha')\).
\end{proof}

\begin{lemma}[Cutoff pairing on boxes]\label{lem:cutoff-pairing}
Fix parameters \(\alpha'>\alpha>1\).
Let \(\chi_{L,t_0}\in C_c^\infty(\R^2_+)\) satisfy \(\chi\equiv1\) on \(Q(\alpha I)\), \(\operatorname{supp}\chi\subset Q(\alpha'I)\), \(\|\nabla\chi\|_\infty\lesssim L^{-1}\) and \(\|\nabla^2\chi\|_\infty\lesssim L^{-2}\).
Let \(V_\phi\) be the Poisson extension of \(\phi\in \mathcal W_{\rm adm}(I;\varepsilon)\).
Then one has the Green pairing identity
\[
 \int_{\R} u(t)\,\phi(t)\,dt
 \ =\ \iint_{Q(\alpha'I)} \nabla U\cdot \nabla\big(\chi_{L,t_0}\, V_\phi\big)\,dt\,d\sigma\ +\ \mathcal R_{\mathrm{side}}\ +\ \mathcal R_{\mathrm{top}},
\]
with remainders satisfying
\[
 |\mathcal R_{\mathrm{side}}|+|\mathcal R_{\mathrm{top}}|
 \ \lesssim\ \Big(\iint_{Q(\alpha'I)} |\nabla U|^2\,\sigma\Big)^{1/2}
               \cdot \Big(\iint_{Q(\alpha'I)} \big(|\nabla\chi|^2\,|V_\phi|^2+|\nabla V_\phi|^2\big)\,\sigma\Big)^{1/2}.
\]
\end{lemma}
\begin{proof}
Let \(Q:=Q(\alpha'I)\).
Assume \(U\) is \(C^2\) on \(\overline Q\) and harmonic on \(Q\), with boundary trace \(u(t)=U(0,t)\) on the bottom edge \(\{\sigma=0\}\).
Since \(\chi_{L,t_0}V_\phi\) is compactly supported in \(\overline Q\) and smooth on \(Q\), Green's identity gives
\[
  \iint_{Q} \nabla U\cdot \nabla(\chi V_\phi)\,dt\,d\sigma
  \,=\,
  \int_{\partial Q} (\chi V_\phi)\,\partial_n U\,ds
  \ -\ \iint_{Q} (\chi V_\phi)\,\Delta U\,dt\,d\sigma.
\]
Since \(\Delta U=0\) on \(Q\), only the boundary integral remains.
On the bottom edge one has \(\partial_n=-\partial_\sigma\), \(\chi\equiv1\), and \(V_\phi(0,t)=\phi(t)\), hence that contribution equals
\[
  \int_{I} \phi(t)\,(-\partial_\sigma U)(0,t)\,dt.
\]
If \(U\) is the real part of a holomorphic logarithm \(U=\Re\log J\) with \(|J(\tfrac12+it)|=1\) a.e., then \(U(0,t)=0\) a.e.\ and \(-\partial_\sigma U(0,t)=\partial_t \Arg J(\tfrac12+it)\) in distributions by Cauchy--Riemann; in particular, this term is the tested boundary phase derivative in Lemma~\ref{lem:CR-green-phase} below.
The remaining boundary pieces (two vertical sides and the top edge) are, by definition, the remainders \(\mathcal R_{\mathrm{side}}+\mathcal R_{\mathrm{top}}\).

For the remainder estimate, we apply Cauchy--Schwarz in the scale-invariant measure \(\sigma\,dt\,d\sigma\) on \(Q\):
\[
  \big|\mathcal R_{\mathrm{side}}\big|+\big|\mathcal R_{\mathrm{top}}\big|
  \ \lesssim\ \Big(\iint_Q |\nabla U|^2\,\sigma\Big)^{1/2}
               \Big(\iint_Q \big|\nabla(\chi V_\phi)\big|^2\,\sigma\Big)^{1/2}.
\]
Expanding \(\nabla(\chi V_\phi)=\chi\,\nabla V_\phi + (\nabla\chi)\,V_\phi\) yields
\[
  \iint_Q \big|\nabla(\chi V_\phi)\big|^2\,\sigma
  \ \lesssim\ \iint_Q \big(|\nabla V_\phi|^2 + |\nabla\chi|^2|V_\phi|^2\big)\,\sigma,
\]
which gives the displayed estimate.
\end{proof}

\begin{lemma}[CR--Green pairing for boundary phase]\label{lem:CR-green-phase}
Let \(J\) be analytic on \(\Omega\) with a.e.\ boundary modulus \(|J(\tfrac12+it)|=1\), and write \(\log J=U+iW\) on \(\Omega\), so \(U\) is harmonic with \(U(\tfrac12+it)=0\) a.e.
Fix a Whitney interval \(I=[t_0-L,t_0+L]\) and let \(V_\phi\) be the Poisson extension of \(\phi\in\mathcal W_{\rm adm}(I;\varepsilon)\).
Then, with a cutoff \(\chi_{L,t_0}\) as in Lemma~\ref{lem:cutoff-pairing},
\[
  \int_{\R} \phi(t)\,\big(-W'(t)\big)\,dt
  \ =\ \iint_{Q(\alpha'I)} \nabla U\cdot \nabla\big(\chi_{L,t_0}\,V_\phi\big)\,dt\,d\sigma\ +\ \mathcal R_{\mathrm{side}}\ +\ \mathcal R_{\mathrm{top}},
\]
and the remainders satisfy the same estimate as in Lemma~\ref{lem:cutoff-pairing}.
In particular, by Cauchy--Schwarz and Lemma~\ref{lem:uniform-test-energy}, there is a constant \(C_{\rm rem}(\alpha',\psi)\) such that
\[
  \int_{\R} \phi(t)\,\big(-w'(t)\big)\,dt\ \le\ C_{\rm rem}(\alpha',\psi)\,\Big(\iint_{Q(\alpha'I)} |\nabla U|^2\,\sigma\Big)^{1/2}.
\]
\end{lemma}
\begin{proof}
On the bottom edge \(\{\sigma=0\}\) the outward normal is \(\partial_n=-\partial_\sigma\).
By Cauchy--Riemann for \(\log J=U+iW\) on the boundary line \(\{\Re s=\tfrac12\}\) one has \(\partial_n U=-\partial_\sigma U=\partial_t W\).
Thus the bottom-edge term in Green's identity is
\[
  -\int_{\partial Q\cap\{\sigma=0\}} \chi\,V_\phi\,\partial_n U\,dt
  = -\int_{\R} \phi(t)\,\partial_t W(t)\,dt
  = \int_{\R} \phi(t)\,\big(-w'(t)\big)\,dt,
\]
which yields the stated identity after including the interior term and remainders.
The final inequality is Cauchy--Schwarz together with the uniform Poisson-energy bound from Lemma~\ref{lem:uniform-test-energy}.
\end{proof}

\begin{proposition}[Length‑independent upper bound for admissible tests]\label{prop:length-free}
Let \(J\) be holomorphic on \(\Omega\setminus Z(\zeta)\) with a.e.\ boundary modulus \(1\), write \(\log J=U+iW\) on \(\Omega\setminus Z(\zeta)\), and let \(-w'\) denote the boundary phase distribution.
For every interval \(I=[t_0-L,t_0+L]\), every \(\phi\in\mathcal W_{\rm adm}(I;\varepsilon)\), and every fixed cutoff to \(Q(\alpha' I)\),
\begin{equation}\label{eq:CRG-upper-adm}
\int_{\mathbb R}\!\phi(t)\,(-w')(t)\,dt\ \le\ C_{\rm test}(\psi,\varepsilon,\alpha')\,\Big(\iint_{Q(\alpha' I)}|\nabla U|^2\,\sigma\,dt\,d\sigma\Big)^{1/2}
\end{equation}
with \(C_{\rm test}(\psi,\varepsilon,\alpha'):=C_{\rm rem}(\alpha',\psi)\,\mathcal A_{\rm adm}(\psi,\varepsilon,\alpha')\) independent of \(I,t_0,L\).
In particular, defining the box-energy constant
\[
  C_{\rm box}^{(\zeta)}\ :=\ \sup_{I}\ \frac{1}{|I|}\iint_{Q(\alpha' I)}|\nabla U|^2\,\sigma\,dt\,d\sigma,
\]
one has the scale bound
\[
  \int_{\mathbb R}\!\phi\,(-w')\ \le\ C_{\rm test}(\psi,\varepsilon,\alpha')\,\sqrt{C_{\rm box}^{(\zeta)}}\,L^{1/2}.
\]
\end{proposition}
\begin{proof}
Apply Lemma~\ref{lem:CR-green-phase} with \(\phi\in\mathcal W_{\rm adm}(I;\varepsilon)\) and absorb the window-side constants into \(C_{\rm test}(\psi,\varepsilon,\alpha')\).
\end{proof}

\begin{lemma}[Whitney--uniform wedge]\label{lem:whitney-uniform-wedge}
Fix parameters \(\alpha'>1\) and \(\varepsilon\in(0,\tfrac14]\).
Fix the Whitney schedule and clip by \(L_\star\): set \(L_\star:=c/\log 2\) and henceforth
\[
  L(T)\ :=\ \min\Big\{\frac{c}{\log\angles{T}},\ L_\star\Big\}.
\]
Then for every Whitney interval \(I=[t_0-L,t_0+L]\) and the corresponding cutoff
\(\psi_{L,t_0}(t):=\psi((t-t_0)/L)=Z_0L\,\varphi_{L,t_0}(t)\) (so \(\psi_{L,t_0}\equiv 1\) on \(I\)),
\[
  \int_{\mathbb R} \psi_{L,t_0}(t)\,(-w'(t))\,dt\ \le\ Z_0\,L_\star\cdot C_{\rm test}(\psi,\varepsilon,\alpha')\,\sqrt{C_{\rm box}^{(\zeta)}}\,L_\star^{1/2}
  \ :=\ \pi\,\Upsilon_{\rm Whit}(c).
\]
Choosing \(c>0\) sufficiently small so that \(\Upsilon_{\rm Whit}(c)<\tfrac12\) yields the hypothesis of Lemma~\ref{lem:local-to-global-wedge} and hence \textup{(P+)}.
\end{lemma}
\begin{proof}
Since \(\psi_{L,t_0}=Z_0L\,\varphi_{L,t_0}\), apply Proposition~\ref{prop:length-free} with \(\phi=\varphi_{L,t_0}\), then multiply the resulting bound by \(Z_0L\) and use the Whitney clip \(L\le L_\star\).
\end{proof}

\begin{theorem}[Proof of Theorem~\ref{thm:Pplus}]\label{thm:pplus-proof-complete}
The boundary wedge \textup{(P+)} holds for \(\mathcal J_{\rm out}\).
\end{theorem}
\begin{proof}
By Lemma~\ref{lem:zeta-normalization}, the quantitative phase--velocity identity (Theorem~\ref{thm:phase-velocity-quant}) applies to the \(\zeta\)-normalized unimodular ratio \(J_\zeta\), and hence (by \eqref{eq:J-out}) to \(\mathcal J_{\rm out}\).
In particular, the associated boundary phase distribution \(-w'\) is positive.

Proposition~\ref{prop:length-free} (CR--Green pairing) supplies a uniform Whitney-scale bound for the windowed phase derivative in terms of the box energy \(C_{\rm box}^{(\zeta)}\).
Applying the Whitney schedule and choosing \(c>0\) small enough gives \(\Upsilon_{\rm Whit}(c)<\tfrac12\) in Lemma~\ref{lem:whitney-uniform-wedge}.
Lemma~\ref{lem:local-to-global-wedge} then yields \textup{(P+)}.
\end{proof}

