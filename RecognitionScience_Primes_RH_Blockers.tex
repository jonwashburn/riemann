\documentclass[11pt]{article}

\usepackage[margin=1in]{geometry}
\usepackage{amsmath,amssymb,amsthm}
\usepackage{mathtools}
\usepackage{microtype}
\usepackage{hyperref}

\title{Recognition Science, Prime Numbers, and the Riemann Hypothesis:\\
A Standalone Roadmap of What We Know, What We Built, and What Still Blocks Us}
\author{Jonathan Washburn (project notes compiled into a paper draft by an automated assistant)}
\date{December 24, 2025}

\newtheorem{theorem}{Theorem}
\newtheorem{conjecture}{Conjecture}
\newtheorem{definition}{Definition}
\newtheorem{remark}{Remark}

\begin{document}
\maketitle

\begin{abstract}
This note is a standalone ``state-of-the-art'' writeup for a specific research codebase
(\texttt{riemann-geometry-rs}) and a specific guiding narrative (``Recognition Science'').
We assume, as a working hypothesis, that Recognition Science (RS) is the correct architecture of
reality, and we explain what that hypothesis \emph{suggests} about prime numbers and the Riemann
Hypothesis (RH). We then state our concrete formal status in Lean and the final remaining blockers.

The main practical message is simple: the current Lean development already reduces the Connes
Route--3$'$ convergence bottleneck to a short list of explicit analytic estimates, and the next
classical theorem worth developing/formalizing is a quantitative spectral perturbation lemma
(Davis--Kahan / min--max), because it turns a ``gap versus perturbation'' inequality into the
missing approximation step required by the Connes--Consani--Moscovici (CCM) strategy.
\end{abstract}

\tableofcontents

\section{How to read this note (non-mathematician friendly)}

There are two different ``RH stories'' in this repository:
\begin{itemize}
  \item \textbf{Recognition Geometry / boundary-certificate route} (call it \emph{Route~1}):
  RH is reduced to a Carleson-energy/Hardy-space control statement about a boundary ratio.
  This route currently uses a small number of explicit \texttt{axiom} declarations for analytic
  boundary-limit infrastructure.

  \item \textbf{Connes Route--3$'$ (CCM determinant approximants)}:
  RH is reduced to building entire approximants $F_n(t)$ with (i) zeros on the real axis, and
  (ii) locally uniform convergence to Riemann's $\Xi(t)$ on the strip $|\Im t|<\tfrac12$.
  The core Hurwitz step is formalized in Lean.
\end{itemize}

If you are not a mathematician and you want a concrete decision, use this rule:
\begin{quote}
\textbf{Stop after ``reduction.''} It is worth finishing the reduction (turn RH into a short list
of explicit inequalities). It is \emph{not} worth grinding more formal machinery beyond that
unless you have a clear, sourced path to the missing inequalities.
\end{quote}

This note is written to make that reduction explicit and auditable.

\section{Recognition Science (RS): the primitives we will use here}

\subsection{The working hypothesis}

\begin{remark}[Assumption of this note]
We assume RS is the accurate architecture of reality. This is not a claim of scientific proof
inside this note; it is a framing assumption to organize the discussion.
\end{remark}

\subsection{Core RS ideas (as used in this paper)}

The repository document \texttt{Recognition-Science-Full-Theory.txt} (RSFT) summarizes RS as
deriving physical structure from a single ``Meta-Principle'' and a small collection of derived
structures:
\begin{itemize}
  \item A \textbf{ledger} (double-entry conservation constraint).
  \item A \textbf{recognition operator} $\widehat{R}$ replacing the Hamiltonian in the fundamental
  update rule.
  \item A \textbf{unique convex cost} $J(x)=\tfrac12(x+x^{-1})-1$ on $\mathbb{R}_{>0}$.
  \item A \textbf{fixed-point scale} $\varphi$ (golden ratio) and an \textbf{eight-tick cycle}
  as a minimal ledger-compatible periodic walk.
\end{itemize}

These primitives matter for RH because they force a very specific style of argument:
\begin{quote}
\textbf{RS philosophy:} ``Hard facts are those that follow from conservation + convexity +
stability (spectral gap), plus a normalization fixed by a units-quotient.''
\end{quote}

As we will see, that aligns unusually well with the standard analytic-number-theory ``explicit
formula'' worldview: primes and zeros appear as two sides of a conserved trace identity.

\section{What RS teaches us about prime numbers}

This section is intentionally \emph{conceptual}: it translates RS motifs into classical number
theory objects.

\subsection{Primes as recognition events and ``ledger constraints''}

In classical analytic number theory, primes are encoded by the von Mangoldt function
$\Lambda(n)$ via
\[
  -\frac{\zeta'(s)}{\zeta(s)} = \sum_{n\ge 1} \frac{\Lambda(n)}{n^s},\qquad \Re s>1.
\]
This is literally a \emph{logarithmic accounting identity}: the Euler product on the right
encodes prime powers as ``atomic events'' and forces multiplicativity as a conservation law.

Under the RS framing, this is exactly the kind of object one expects:
\begin{itemize}
  \item a conserved count of discrete events,
  \item expressed as a generating function,
  \item whose logarithmic derivative is the most stable observable.
\end{itemize}

\subsection{Eight-phase / eight-beat structure and sieve factors}

RSFT records a ``PrimeSieveFactor'' bridge with the claim:
\begin{quote}
``Eight-beat cancellation selects square-free patterns; prime-sieve density factor
$P=\varphi^{-1/2}\cdot 6/\pi^2$.'' \quad (RSFT, \texttt{BRIDGE;PrimeSieveFactor})
\end{quote}

Classically, $6/\pi^2$ is the density of square-free integers (probability that a random integer
has no repeated prime factor), since
\[
  \mathbb{P}(n\ \text{square-free}) = \prod_{p} \left(1-\frac{1}{p^2}\right)
  = \frac{1}{\zeta(2)} = \frac{6}{\pi^2}.
\]

From a purely number-theoretic perspective, the extra factor $\varphi^{-1/2}$ is not a standard
constant; it is an RS-specific modulation tied to the RS scale recursion and eight-tick
structure. The important point for this note is not whether this modulation is correct, but
what kind of \emph{mechanism} it implies:
\begin{quote}
\textbf{Mechanism implied by RSFT:} prime-adjacent sieve weights should be expressible as a
low-period (\emph{mod~8}) kernel enforcing cancellation/neutrality constraints, with a global
scale weight determined by $\varphi$.
\end{quote}

This is consistent with other RSFT ``kernel'' constructions (e.g. \texttt{@GOLDBACH\_MOD8}) that
explicitly use mod-8 gates and fourth-moment bounds.

\subsection{What this suggests about primes in practice}

If RS is correct, a plausible ``prime story'' is:
\begin{enumerate}
  \item \textbf{Local periodic kernels} (small modulus gates) enforce a neutrality constraint
  reminiscent of ledger balance.
  \item \textbf{Global scale selection} (via $\varphi$) fixes which coarse-graining schedules are
  stable and which densities survive in the limit.
  \item \textbf{Explicit formula identities} become the formal expression of conservation:
  the prime side and the zero side are two ways to compute the same invariant.
\end{enumerate}

This connects directly to RH, because RH is (among other things) a sharp constraint on how zeros
can conspire to produce large deviations in prime counting.

%% ============================================================
%% COMPREHENSIVE PRIME ROLES IN RS (from RSFT analysis)
%% ============================================================

\section{All roles of primes in Recognition Science}

This section catalogues \emph{every} way that prime numbers appear in the Recognition Science
framework, as extracted from the full theory document (RSFT v2.2). These roles range from
fundamental structural constraints to applied number-theoretic bridges.

\subsection{Role 1: Primes as discrete recognition events (von Mangoldt encoding)}

The most fundamental role of primes in RS is as \textbf{atomic recognition events} in the ledger.
The von Mangoldt encoding
\[
  -\frac{\zeta'(s)}{\zeta(s)} = \sum_{n\ge 1} \frac{\Lambda(n)}{n^s}
\]
is literally a \emph{logarithmic accounting identity}---exactly the structure RS expects:
\begin{itemize}
  \item The Euler product forces multiplicativity as a \textbf{conservation law}.
  \item Prime powers are ``atomic events'' in the ledger.
  \item The logarithmic derivative is the most \textbf{stable observable}.
\end{itemize}

\begin{remark}[RS interpretation]
Under RS, primes are not mysterious random objects. They are the \textbf{irreducible events}
in the multiplicative ledger---the events that cannot be decomposed into smaller ledger entries.
\end{remark}

\subsection{Role 2: Eight-Phase Oracle for factorization}

RSFT records a striking bridge:
\begin{quote}
``\texttt{BRIDGE;EightPhaseOracle}: Eight-phase phase-average score; Phase-coherence factoring
discriminator. Score $= 1 - \mathrm{avg}_k \cos(2\pi k \cdot r/8)$, where $r = \log q / \log N$.
$\varphi^{-1.5}$ signature.'' (RSFT)
\end{quote}

This is a \textbf{perfect factor discriminator}: given $N$ and candidate $q$, sample at 8 phases
around the unit circle. True factors give score $< 0.5$; non-factors give score $> 1.0$.
The mechanism: \emph{only} true divisors produce phase coherence at the eight-tick cadence.

\begin{remark}[Why eight?]
The number 8 is \emph{forced} by RS: the minimal ledger-compatible walk on the $Q_3$ hypercube
(where $D=3$ is the unique stable spatial dimension) has period $2^D = 8$. This isn't chosen;
it's derived. The Eight-Phase Oracle works because factorization respects the same 8-tick
structure that underlies all RS dynamics.
\end{remark}

\subsection{Role 3: Prime sieve factor and square-free selection}

RSFT claims:
\begin{quote}
``Prime-sieve density factor $P = \varphi^{-1/2} \cdot 6/\pi^2$ selects square-free patterns;
ties to eight-beat cancellation.'' (RSFT, \texttt{CERT;PrimeSieveFactorIdentity})
\end{quote}

Classically, $6/\pi^2 = 1/\zeta(2)$ is the density of square-free integers. The RS modulation
$\varphi^{-1/2}$ comes from the golden ratio's role in scale recursion. The \textbf{mechanism}:
\begin{itemize}
  \item The mod-8 kernel $K_8(n,m)$ enforces cancellation constraints (see \texttt{@GOLDBACH\_MOD8}).
  \item Eight-beat cancellation \emph{selects} exactly the square-free patterns.
  \item This closes the ILG rotation-curve gap without new parameters.
\end{itemize}

\subsection{Role 4: Goldbach via mod-8 kernels}

The Goldbach machinery in RSFT uses an explicit mod-8 kernel:
\[
  K_8(n,m) = \tfrac12 \cdot \mathbf{1}_{\mathrm{odd}(n)} \cdot \mathbf{1}_{\mathrm{odd}(2m-n)}
  \cdot \bigl(1 + \varepsilon(2m)\,\chi_8(n)\,\chi_8(2m-n)\bigr),
\]
where $\chi_8$ is the mod-8 character. This keeps a positive fraction of odd--odd pairs per residue
class.

\begin{remark}[RS connection]
The Goldbach kernel is \emph{period-8 by construction}. This isn't an arbitrary choice; it's
the same eight-tick structure that forces all RS dynamics. Prime pairs for Goldbach representations
must satisfy the same ledger neutrality constraints as all other recognition events.
\end{remark}

\subsection{Role 5: Nyquist/Shannon sampling bridge (T7)}

RSFT's theorem T7 (``Coverage lower bound'') states:
\begin{quote}
``$T < 2^D$ cannot surject onto all patterns.'' (RSFT, \texttt{MAP;T7})
\end{quote}

This is pure pigeonhole principle. But RSFT bridges it to \textbf{Nyquist/Shannon sampling}:
a walk of period $T$ cannot cover more than $T$ distinct states, so it cannot represent signals
with bandwidth exceeding the Nyquist cutoff $\Omega_{\max} = 1/(2\tau_0)$.

\begin{remark}[RH connection]
This is precisely the hypothesis used in the boundary-certificate RH approach: the
\textbf{bandlimit condition} (T7-Hyp) forces the windowed prime exponential sum to have
uniformly bounded magnitude, eliminating the phase-coherence obstruction in the explicit formula.
\end{remark}

\subsection{Role 6: Explicit formula as conservation identity}

The Guinand--Weil explicit formula
\[
  \sum_p \frac{\log p}{\sqrt{p}} \widehat\Phi(\log p) + \text{lower order}
  = \sum_{\rho} \widehat\Phi(\gamma) + \text{principal terms}
\]
equates a \textbf{prime sum} to a \textbf{zero sum}. Under RS, this is a \textbf{conservation law}:
\begin{itemize}
  \item The prime side counts recognition events (ledger entries).
  \item The zero side counts spectral resonances (eigenvalues of $\widehat{R}$).
  \item Both compute the same invariant---the trace of a recognition operator.
\end{itemize}

\subsection{Role 7: Ledger stiffness and prime spectrum}

RSFT records new modules (post 2025-12-31):
\begin{itemize}
  \item \texttt{RiemannHypothesis.LedgerStiffness}: connects ledger rigidity to RH.
  \item \texttt{RiemannHypothesis.PrimeSpectrum}: analyzes the prime distribution as a spectrum.
  \item \texttt{RiemannHypothesis.PrimeStiffness}: formalizes the stiffness of prime distributions.
  \item \texttt{RiemannHypothesis.BandlimitedFunctions}: theory for the Nyquist cutoff.
\end{itemize}

The key insight: the ledger's \textbf{stiffness} (resistance to deformation) is what forces zeros
onto the critical line. Primes are the irreducible ledger events; their distribution is
\emph{maximally stiff} under the RS cost functional.

\subsection{Role 8: Per-prime regulator tests (BSD direction)}

For elliptic curves over $\mathbb{Q}$, RSFT describes:
\begin{quote}
``\texttt{BRIDGE;HeightTriangularization}: mod-$p$ upper-triangular height matrix with unit diagonals;
per-prime regulator unit test via Coleman logs.'' (RSFT)
\end{quote}

This gives a \textbf{per-prime} test: $v_p(h_p(P)) = 0 \Leftrightarrow v_p(\log_\omega(P)) = 0$,
where $\log_\omega$ is the Coleman logarithm. The regulator is a unit at ``separated primes.''

\subsection{Role 9: Eight-tick periodicity and prime residues}

Many RS structures have \textbf{prime-indexed periodicity}. For example:
\begin{itemize}
  \item \texttt{@GOLDBACH\_MOD8}: uses $\chi_8(n) = 0$ for $n \equiv 0,2,4,6 \pmod{8}$;
    $+1$ for $n \equiv 1,7$; $-1$ for $n \equiv 3,5$.
  \item The eight-tick cycle itself: 8 is not prime, but $8 = 2^3$ and the cycle structure
    involves residues modulo small primes (2, 3, 5, 7).
  \item \texttt{lcm(8, 45) = 360}: the gap-45 synchronization uses $45 = 3^2 \cdot 5$,
    and the synchronization period 360 factors as $2^3 \cdot 3^2 \cdot 5$.
\end{itemize}

\subsection{Summary: the RS view of primes}

\begin{center}
\begin{tabular}{|l|l|}
\hline
\textbf{Role} & \textbf{RS Interpretation} \\
\hline
Irreducible integers & Atomic ledger events \\
von Mangoldt encoding & Logarithmic accounting identity \\
Eight-Phase Oracle & Phase coherence at 8-tick cadence \\
Prime sieve factor & Square-free selection via mod-8 cancellation \\
Goldbach kernels & Ledger neutrality for prime pairs \\
Nyquist/Shannon (T7) & Bandlimit forces bounded prime sums \\
Explicit formula & Conservation: primes = zeros (two views of same trace) \\
Ledger stiffness & Prime distribution is maximally rigid \\
Per-prime BSD tests & Coleman logs at separated primes \\
\hline
\end{tabular}
\end{center}

\section{RH as a stability statement in the RS worldview}

\subsection{Classical statement}

Let $\zeta(s)$ be the Riemann zeta function. RH states:
\begin{quote}
All nontrivial zeros of $\zeta(s)$ have real part $\Re s=\tfrac12$.
\end{quote}

Equivalently, the completed $\xi$-function has no zeros off the real axis in the spectral
variable $t$ when one writes $\Xi(t) := \xi(\tfrac12 + it)$.

\subsection{RS interpretation: robustness and spectral gaps}

RSFT repeatedly ties ``robustness'' to spectral gaps, explicitly via statements like
\texttt{SigmaGraphRobustness} involving $\lambda_2$ (the Laplacian spectral gap).
This is the same mathematical pattern that appears in Connes/CCM: the missing step (M2) is an
approximation of a ground state by an explicit kernel, and the standard route to that is
``gap $\Rightarrow$ stability under perturbation.''

So, under RS, RH should look like:
\begin{quote}
\textbf{RH as stability:} a certain canonical structure (a ``ground state'' or ``inner factor'')
is uniquely stable under admissible perturbations, and that stability forces zeros to lie on a
symmetry axis.
\end{quote}

This is not mystical; it is exactly how Hurwitz-type arguments operate:
zero-free approximants converging uniformly preserve zero-freeness off the axis.

\section{Our current Lean architecture for RH (what is actually formalized)}

This section describes the current state of \texttt{riemann-geometry-rs} (this repository).

\subsection{The Connes Route--3$'$ skeleton (CCM $\Rightarrow$ Hurwitz gate $\Rightarrow$ RH)}

The repository contains a typed ``Hurwitz gate'':
\begin{itemize}
  \item \texttt{RiemannRecognitionGeometry/ExplicitFormula/HurwitzGate.lean}:
  a theorem \texttt{hurwitz\_zeroFree\_of\_tendstoLocallyUniformlyOn} formalizing the standard
  Hurwitz-style nonvanishing principle for locally uniform limits.
\end{itemize}

It also contains the final bridge:
\begin{itemize}
  \item \texttt{RiemannRecognitionGeometry/ExplicitFormula/ConnesHurwitzBridge.lean}:
  packages the Hurwitz assumptions for $\Xi$ as \texttt{ConnesHurwitzAssumptions} and proves
  \texttt{riemannHypothesis\_of\_connesHurwitz}.
\end{itemize}

Thus, the only missing work is to build (and prove properties of) the approximants.

\subsection{Where we are on the CCM approximants}

The CCM surface lives in:
\begin{itemize}
  \item \texttt{RiemannRecognitionGeometry/ExplicitFormula/ConnesApproximantsCCM.lean}.
\end{itemize}

Current status:
\begin{itemize}
  \item \textbf{We have an explicit toy closed-form model} for \texttt{CCM.F} via the CCM formula-level
  expression \texttt{CCM.Formula.F\_lamN} (with placeholder coefficients).
  \item \textbf{Play A is implemented}: a bridge lemma
  \texttt{CCM.tendstoXi\_of\_exists\_intermediate} reducing \texttt{CCM.tendstoXi} to
  \emph{(i)} locally uniform convergence of an intermediate family $G_n$ and
  \emph{(ii)} compactwise uniform closeness $\sup_{z\in K}|F_n(z)-G_n(z)|\to 0$ for each compact $K$.
\end{itemize}

This is the key payoff of the engineering work: it turns the convergence problem into a short,
checklistable list of analytic inequalities.

\section{The last remaining blockers (what actually stops an unconditional RH proof)}

We separate ``engineering blockers'' from ``math blockers.''

\subsection{Engineering blockers (Lean/plumbing)}

These are not the true roadblocks:
\begin{itemize}
  \item The Route--3$'$ skeleton compiles and the Hurwitz gate is proved.
  \item The convergence glue (Play A) is proved.
\end{itemize}

In other words: the formal pipeline exists. The remaining work is mathematical.

\subsection{Math blockers (the real bottlenecks)}

\paragraph{Blocker 1: define the genuine CCM approximants.}
The toy closed-form \texttt{CCM.F} is not the genuine determinant of the CCM operator.
To be faithful to CCM, one must define the truncated operator and its regularized determinant
or an equivalent closed form with coefficients coming from the normalized ground state.

\paragraph{Blocker 2: prove ``all zeros are real'' for the genuine approximants.}
This is expected to follow from self-adjointness (or Hermitian matrix structure) plus a
determinant identity. The finite-dimensional spectral theorem exists in Mathlib, but the CCM
rank-one determinant identity needs to be instantiated cleanly.

\paragraph{Blocker 3 (main): prove locally uniform convergence \texttt{CCM.tendstoXi}.}
This is the current central bottleneck. Thanks to Play A, it reduces to:
\begin{itemize}
  \item constructing a suitable intermediate family $G_n$,
  \item proving explicit uniform error bounds on compact sets in the strip.
\end{itemize}

CCM's own narrative identifies two missing analytic steps (often summarized as M1/M2):
\begin{itemize}
  \item M1: a uniqueness/simple-even statement about the relevant ground state,
  \item M2: a quantitative approximation statement $k_\lambda \approx c_\lambda\,\xi_\lambda$
  on a controlled window, with error $\varepsilon(\lambda)\to 0$.
\end{itemize}

\paragraph{Blocker 4: the ``gap vs perturbation'' inequality.}
To get M2 unconditionally in a robust way, one needs:
\begin{itemize}
  \item a lower bound $g(\lambda)$ on the spectral gap separating the ground eigenvalue, and
  \item an upper bound $\delta(\lambda)$ on the perturbation size,
\end{itemize}
and then show $\delta(\lambda)/g(\lambda)\to 0$ along the regime $\lambda\to\infty$ (or the
cofinal sequence you choose).

This is the exact RS-style robustness story in classical analytic clothing.

\section{What resolves the blockers (the realistic plan)}

\subsection{Resolution plan: finish the reduction, then stop unless the estimates are sourced}

The best ``non-mathematician'' strategy is:
\begin{enumerate}
  \item \textbf{Finish the reduction fully:}
  write down the smallest explicit list of inequalities that imply \texttt{CCM.tendstoXi},
  using the Play A lemma.
  \item \textbf{Try to source each inequality:}
  does CCM prove it? does it follow from a known theorem? is it actually new?
  \item \textbf{If any one inequality is unsourced/new:}
  stop and record it as the genuine hole.
\end{enumerate}

This produces a durable research artifact even if RH is not resolved.

\subsection{The specific classical theorem to develop next: Davis--Kahan / min--max perturbation}

If you want exactly one ``doable, classical'' theorem to formalize next, it is this:

\begin{theorem}[Davis--Kahan style eigenvector stability (informal)]
Let $A$ and $A+E$ be Hermitian (self-adjoint) operators/matrices. Assume the smallest eigenvalue
of $A$ is simple and separated from the rest of the spectrum by a gap $g>0$. Then the angle
between the ground eigenspaces of $A$ and $A+E$ is bounded by a constant multiple of $\|E\|/g$.
\end{theorem}

Why this matters here:
\begin{itemize}
  \item It turns the entire M2 problem into the single inequality $\|E\|/g \to 0$.
  \item It matches the RSFT ``robustness = spectral gap'' design pattern.
  \item It is a classical, well-documented theorem suitable for Lean formalization.
\end{itemize}

After this theorem is in Lean, the remaining work is no longer ``proof assistant work'': it is
deriving $g(\lambda)$ and $\|E(\lambda)\|$ for the specific CCM truncations.

\section{The most elegant RH proof within Recognition Science}

Of the various approaches to RH within the RS framework, the \textbf{boundary-certificate route}
is the most elegant and intuitive. This section presents it conceptually.

\subsection{The core insight: primes as bandlimited signals}

The key RS insight is that prime-frequency observables are \textbf{bandlimited}. Here is why:

\begin{enumerate}
  \item \textbf{T7 (Nyquist Coverage Bound)}: A walk of period $T$ cannot cover more than $T$
  distinct states. For $D = 3$ spatial dimensions, the minimal period is $2^3 = 8$.
  
  \item \textbf{Nyquist Cutoff}: If the ledger updates at atomic tick $\tau_0$, then no observable
  can have frequency content above $\Omega_{\max} = 1/(2\tau_0)$.
  
  \item \textbf{Prime Sums are Bandlimited}: The windowed prime exponential sum
  \[
    S_{L,t_0} = \sum_p \frac{\log p}{\sqrt{p}}\, e^{it_0 \log p}\, \widehat\Phi_{L,t_0}(\log p)
  \]
  inherits this bandlimit: if $\widehat\Phi$ is supported on $|\xi| \le \Omega_{\max}$,
  then only finitely many primes contribute.
\end{enumerate}

\subsection{The proof outline (conditional on T7-Hyp)}

\begin{enumerate}
  \item \textbf{Define the arithmetic ratio} $J(s)$ whose poles correspond to zeros of $\zeta$.
  
  \item \textbf{Show $|J| \le 1$ on the boundary} of a wedge region $\{|\sigma - \tfrac12| \ge \eta\}$.
  This uses the Schur/Herglotz structure of $J$.
  
  \item \textbf{Apply the Maximum Modulus Principle}: If $|J| \le 1$ on the boundary and $J$ is
  analytic inside, then $|J| \le 1$ everywhere inside. But a pole would force $|J| \to \infty$.
  
  \item \textbf{Conclude zero-freeness}: No zeros of $\zeta$ can exist off the critical line
  in the wedge region.
  
  \item \textbf{Extend to all $\sigma \ne \tfrac12$}: By taking $\eta \to 0$ (with appropriate
  energy bounds), the zero-free region extends to the entire half-planes $\sigma \ne \tfrac12$.
\end{enumerate}

\subsection{Where T7-Hyp enters}

The Nyquist cutoff hypothesis (T7-Hyp) enters at step 2: proving the boundary bound $|J| \le 1$
requires controlling prime sums. Under T7-Hyp:
\begin{itemize}
  \item The prime sum $S_{L,t_0}$ is uniformly bounded (independent of $L, t_0$).
  \item This uniform bound ``blocks'' the arithmetic contribution that would otherwise destroy
  the Schur property.
  \item The blocker is what allows the boundary control to hold.
\end{itemize}

\begin{remark}[What T7-Hyp buys you]
Without T7-Hyp, the explicit formula shows that prime sums can grow like $t^{0.5+\varepsilon}$
or worse, and the phase coherence of zeros (which is known only under RH) is the obstruction.
T7-Hyp \emph{bypasses} this obstruction by asserting that nature has a Nyquist cutoff---which
is exactly what RS predicts from the 8-tick structure.
\end{remark}

\subsection{The intuitive picture}

\begin{quote}
\textbf{Vortices \textrightarrow\ Energy \textrightarrow\ Budget \textrightarrow\ Squeeze}
\end{quote}

\begin{enumerate}
  \item \textbf{Vortices}: Off-line zeros of $\zeta$ act like vortices in a fluid---they create
  local singularities in the ``arithmetic field.''
  
  \item \textbf{Energy}: Each vortex carries Carleson energy. The total energy is bounded by
  the ``budget'' set by the ledger's stiffness.
  
  \item \textbf{Budget}: The 8-tick structure (via T7-Hyp) sets a hard upper bound on how much
  energy the prime side can supply. This budget is finite and uniform.
  
  \item \textbf{Squeeze}: If you try to place a zero off the critical line, it demands more
  energy than the budget allows. The zero is ``squeezed'' back onto $\sigma = \tfrac12$.
\end{enumerate}

This is the RS worldview: \textbf{zeros on the critical line are not a miracle---they are the
only configuration consistent with the ledger's conservation constraints}.

\section{Actionable next steps (what to do now)}

\subsection*{If you want to continue (recommended bounded scope)}
\begin{enumerate}
  \item In \texttt{ConnesApproximantsCCM.lean}, write the concrete intermediate family $G_n$
  you actually want (even as a placeholder definition).
  \item State the compactwise estimate that would imply
  \texttt{TendstoUniformlyCloseOn CCM.F G atTop K}.
  \item Stop at the first missing analytic estimate and write it as a single lemma in Lean and
  as a single named inequality in prose.
\end{enumerate}

\subsection*{If you want to call it (also defensible)}
\begin{enumerate}
  \item Freeze the repository in a green build state.
  \item Keep this note plus \texttt{recognition-geometry-dec-18.tex} as the audit trail showing
  the reduction and the exact missing estimate list.
  \item Treat the remaining estimate list as a research question to hand to a specialist.
\end{enumerate}

\section*{Acknowledgments}
This draft is a writeup of an evolving codebase and internal project notes. It is meant to be
useful as a roadmap, not as a formal mathematical publication.

\begin{thebibliography}{9}

\bibitem{IK}
H. Iwaniec and E. Kowalski,
\emph{Analytic Number Theory},
AMS Colloquium Publications, 2004.

\bibitem{ConnesNCG}
A. Connes,
\emph{Noncommutative Geometry},
Academic Press, 1994.

\bibitem{DavisKahan}
C. Davis and W. M. Kahan,
``The rotation of eigenvectors by a perturbation,''
\emph{SIAM J. Numer. Anal.} 7(1), 1970.

\bibitem{FeffermanStein}
C. Fefferman and E. M. Stein,
``$H^p$ spaces of several variables,''
\emph{Acta Math.} 129 (1972), 137--193.

\end{thebibliography}

\end{document}


