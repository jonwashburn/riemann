\documentclass[11pt]{amsart}
% ===== BEGIN inlined from riemann_common_preamble.tex =====
% Shared preamble for the three-paper split.
% Keep this file free of \documentclass and egin{document}.
% Do NOT mention proof assistants anywhere in the split papers.

\usepackage[margin=1in]{geometry}
\usepackage{booktabs}
\usepackage{float}
\usepackage{amsmath,amssymb,amsthm,mathtools}
\usepackage[T1]{fontenc}
\usepackage{lmodern}
\usepackage[utf8]{inputenc}
\usepackage{microtype}
\usepackage{hyperref}
\usepackage{xcolor}
\usepackage[numbers,sort&compress]{natbib}
\hypersetup{colorlinks=true,linkcolor=black,citecolor=black,urlcolor=black}

% Theorems
\newtheorem{theorem}{Theorem}
\newtheorem{proposition}[theorem]{Proposition}
\newtheorem{lemma}[theorem]{Lemma}
\newtheorem{corollary}[theorem]{Corollary}
\newtheorem{hypothesis}[theorem]{Hypothesis}
\theoremstyle{definition}
\newtheorem{definition}[theorem]{Definition}
\theoremstyle{remark}
\newtheorem{remark}[theorem]{Remark}

% Basic macros
\newcommand{\C}{\mathbb{C}}
\newcommand{\R}{\mathbb{R}}
\newcommand{\N}{\mathbb{N}}
\newcommand{\PP}{\mathcal{P}}
\newcommand{\Hilb}{\mathcal H}
\DeclareMathOperator{\dettwo}{det_2}

% Stable angle-bracket convention
\newcommand{\angles}[1]{\langle #1\rangle}

% Editorial markup (disabled)
\newcommand{\editblue}[1]{{#1}}
\newcommand{\editgreen}[1]{{#1}}
% ===== END inlined from riemann_common_preamble.tex =====

\DeclareMathOperator{\Arg}{Arg}
\DeclareMathOperator{\ntlim}{nt\!-\!lim}
\newcommand{\z}{\zeta} % compatibility: avoid undefined \z

\title[Zero-free region for $\zeta(s)$]{A zero-free region for the Riemann zeta function in the half-plane $\Re s\ge 0.6$}

\author{Jonathan Washburn}\thanks{Recognition Physics Research Institute, Austin, TX, USA; \email{jon@recognitionphysics.org}.}
\author{Amir Rahnamai Barghi}\thanks{Correspondence: \texttt{arahnama@gmail.com}.}\thanks{Recognition Physics Research Institute, Austin, TX, USA; \email{arahnamab@gmail.com}.}\thanks{Correspondence: arahnamab@gmail.com}

\date{}
\begin{document}
\begin{abstract}
We establish a  zero-free region for the Riemann zeta function: $\zeta(s)\neq 0$ for $\Re s\ge 0.6$.
On $\Omega=\{\Re s>\tfrac12\}$ we introduce an arithmetic ratio $\mathcal J(s)$ and the associated Cayley transform
$\Theta(s)=(2\mathcal J(s)-1)/(2\mathcal J(s)+1)$.
A boundary wedge condition (P$^+$) implies that $\Theta$ is Schur on $\Omega$, which in turn yields a half-plane Schur/Herglotz removability argument
excluding poles of $\mathcal J$ throughout $\{\Re s\ge 0.6\}$.
Since any pole of $\mathcal J$ in $\Omega$ forces a zero of $\zeta$, the pole exclusion gives the stated zero-free region.
Supplementary rigorous ball-arithmetic certificates are included as ancillary material for independent reproducibility checks on representative low-height rectangles.
\end{abstract}

\subjclass[2020]{Primary 11M26; Secondary 30H10, 42B30, 47B35.}
\maketitle

\section{Introduction}

The Riemann zeta function
\[
  \zeta(s)\;=\;\sum_{n\ge 1}\frac{1}{n^s},\qquad \Re s>1,
\]
extends meromorphically to $\C$ with a simple pole at $s=1$ and satisfies a functional equation after completion.
Its nontrivial zeros govern the finest fluctuations in the distribution of prime numbers, and the Riemann Hypothesis (RH) asserts that all such zeros lie on the critical line $\Re s=\tfrac12$; see \cite{Titchmarsh,IK} for background.

This paper isolates an unconditional, fixed half-plane statement in the direction of RH.
Unlike the classical zero-free regions near $\Re s=1$ (which are asymptotic in height), the result here is a \emph{uniform} half-plane exclusion at $\Re s\ge 0.6$.
\begin{theorem}[ far-field zero-freeness]\label{thm:farfield}
The Riemann zeta function has no zeros in the region $\{\,s\in\C:\ \Re s\ge 0.6\,\}$.

\noindent\textit{Proof.} See \S\ref{sec:proof-farfield}.
\end{theorem}

\noindent\textbf{Scope and supplementary computation.}
The proof of Theorem~\ref{thm:farfield} is analytic and self-contained in the present text, with the boundary wedge certificate \textup{(P+)} proved in §§\ref{app:phase-velocity}--\ref{app:assemble-pplus} of Appendix~\ref{app:pplus-proof}.
The shipped computational artifacts provide independently checkable ball-arithmetic cross-checks on representative low-height rectangles (Table~\ref{tab:artifact-data}) together with a verification protocol (Appendix~\ref{app:verify}).
They are included for transparency and reproducibility, but are not logically required for the all-heights implication chain in the proof.

The Appendix records the analytic function theory inputs used in the proof of \textup{(P+)} (boundary traces, factorization, and the phase--velocity identity).

\subsection*{Strategy: Schur pinching via a Cayley field}
We work on the right half-plane $\Omega=\{\,\Re s>\tfrac12\,\}$.
In Section~\ref{sec:defs} we define an arithmetic ratio $\mathcal J$ (in the default \emph{raw $\zeta$-gauge}) with the following two structural properties:
\begin{itemize}
\item \textbf{(normalization at $+\infty$)} $\mathcal J(\sigma+it)\to 1$ as $\sigma\to+\infty$, hence $\Theta(\sigma+it)\to \tfrac13$ (Remark~\ref{rem:Ocan-role});
\item \textbf{(normalization)} $\dettwo(I-A(s))$ is holomorphic and nonvanishing on $\Omega$, so any zero of $\zeta$ in $\Omega$ produces a pole of $\mathcal J$ (Lemma~\ref{lem:poles}).
\end{itemize}
We then pass to the Cayley transform
\[
  \Theta(s)\ :=\ \frac{2\mathcal J(s)-1}{2\mathcal J(s)+1}.
\]
The analytic mechanism is a \emph{Schur/Herglotz pinch} proved in Section~\ref{sec:pinch}:
if $\Theta$ is Schur on a domain (i.e.\ $|\Theta|\le 1$) and not identically $1$, then boundedness forces removability of any isolated singularity and prevents poles of $\mathcal J$.
Since $\Theta(\sigma+it)\to \tfrac13$ as $\sigma\to+\infty$, the degenerate possibility $\Theta\equiv 1$ is excluded on the half-planes relevant here.
Therefore, to prove Theorem~\ref{thm:farfield} it suffices to certify a Schur bound for the default Cayley field $\Theta_{\rm raw}$ on some open half-plane $\{\,\Re s>0.6-\varepsilon\,\}$.

\subsection*{Inputs (what is rigorously checked)}
The logical implication of Theorem~\ref{thm:farfield} rests on a boundary certificate:
we establish a boundary wedge \textup{(P+)} (Lebesgue-a.e.) for the boundary phase of $\mathcal J$ on $\Re s=\tfrac12$, which implies that $2\mathcal J$ is Herglotz and $\Theta$ is Schur on $\Omega\setminus Z(\zeta)$.
The Schur/Herglotz pinch mechanism then excludes poles of $\mathcal J$ on $\{\,\Re s\ge 0.6\,\}$ and hence excludes zeros of $\zeta$ there.

\medskip
\noindent\emph{Supplementary computational cross-checks (not used in the proof).}
The handoff bundle also contains rigorous ball-arithmetic rectangle checks and finite Pick artifacts on low-height regions; these are included as independent numerical corroboration but are not used in the all-heights proof.

\subsection*{Reproducibility and verification materials}
The handoff bundle (and repository) includes:
(i) the verifier script based on ARB ball arithmetic (\texttt{python-flint}), and
(ii) the JSON artifacts that record the  maxima, spectral separation bounds, and denominator checks for the supplementary certificates listed in Table~\ref{tab:artifact-data}.
The included \texttt{MANIFEST.md} and Appendix~\ref{app:verify} provide a verification manifest mapping certificates to exact commands and expected output fields.

\subsection*{Place in a series}
This paper is designed to stand alone as an unconditional  zero-free region.
Two companion papers (not required for Theorem~\ref{thm:farfield}) treat: (a) effective near-field energy barriers and Carleson budgets, and (b) a cutoff principle yielding conditional closure of RH.

\medskip
\noindent The remainder of the paper defines the arithmetic ratio $\mathcal J$ and Cayley field $\Theta$, proves the Schur pinch mechanism, and then discharges the Schur bound via the hybrid certification outlined above.

\section{Definitions and main objects}\label{sec:defs}

This section defines the analytic objects used throughout the proof and records the basic relationships
between zeros of $\zeta$ and the bounded-real (Schur/Herglotz) structure employed later.
\subsection*{The completed zeta function and the far half-plane}
Let $\zeta(s)$ denote the Riemann zeta function.
We write $\xi(s)$ for the completed zeta function
\[
  \xi(s)\ :=\ \tfrac12\,s(s-1)\,\pi^{-s/2}\Gamma(s/2)\,\zeta(s),
\]
which is entire and satisfies the functional equation $\xi(s)=\xi(1-s)$; see \cite{Titchmarsh}.
In this paper, when we say ``zero'' we mean a zero of $\zeta$ (equivalently of $\xi$ away from the
canceled singularities at $s=0,1$) lying in the half-plane
\[
  \Omega\ :=\ \{\,s\in\C:\ \Re s>\tfrac12\,\}.
\]
Theorem~\ref{thm:farfield} concerns the fixed far region $\{\,\Re s\ge 0.6\,\}\subset\Omega$.

\subsection*{The prime-diagonal operator and the regularized determinant}
Let $\PP$ denote the set of primes and write $\ell^2(\PP)$ for the Hilbert space with orthonormal basis
$\{e_p\}_{p\in\PP}$.
For $s\in\C$ define the prime-diagonal operator
\[
  A(s):\ell^2(\PP)\to\ell^2(\PP),\qquad A(s)e_p:=p^{-s}e_p.
\]
For $\Re s>1/2$,
\[
  \|A(s)\|_{\mathrm{HS}}^2=\sum_{p\in\PP}|p^{-s}|^2=\sum_{p\in\PP}p^{-2\Re s}\le \sum_{n\ge 2}n^{-2\Re s}<\infty,
\]
so $A(s)$ is Hilbert--Schmidt on $\Omega$.
In particular, the regularized determinant $\dettwo(I-A(s))$ is well-defined and holomorphic on $\Omega$
(see \cite[Ch.~III]{RosenblumRovnyak} and \cite[Ch.~9]{SimonTrace}).
\begin{lemma}[Diagonal product formula for $\det_2$]\label{lem:det2-diagonal}
Let $T$ be a diagonal Hilbert--Schmidt operator on $\ell^2$ with eigenvalues $\{\lambda_n\}$ satisfying
$\sum_n|\lambda_n|^2<\infty$. Then
\[
  \dettwo(I-T)\ =\ \prod_{n}(1-\lambda_n)\,e^{\lambda_n},
\]
where the product converges absolutely. In particular, $\det_2(I-T)=0$ iff $\lambda_n=1$ for some $n$.
\end{lemma}
\begin{proof}
This holds for the $\mathcal S_2$-regularized determinant; see \cite[Ch.~III]{RosenblumRovnyak}
or \cite[Ch.~9]{SimonTrace}. (We only use the diagonal case and the zero criterion $\lambda_n=1$.)
\end{proof}

Applying Lemma~\ref{lem:det2-diagonal} to $T=A(s)$ on $\Omega$ gives the explicit product
\begin{equation}\label{eq:det2-product}
  \dettwo(I-A(s))\ =\ \prod_{p\in\PP}(1-p^{-s})\,e^{p^{-s}}.
\end{equation}

The region $\Omega\subset\{\Re s>1/2\}$ lies away from $s=0$, so the compensator $1/s$ introduces no pole on the working domain.
The point $s=1$ lies in $\Omega$, but the factor $(s-1)$ cancels the simple pole of $\zeta$ there.
All holomorphy/pole assertions for $\mathcal J$ are made only on $\Omega$, and poles are tracked relative to zeros of $\zeta$ in $\Omega$.

Since $\Re s>1/2$ implies $|p^{-s}|<1$ for every prime $p$, each factor in \eqref{eq:det2-product} is nonzero.
Hence $\dettwo(I-A(s))$ is holomorphic and zero-free on $\Omega$.

\subsection*{The arithmetic ratio $\mathcal J$ and the Cayley field $\Theta$}
Fix a domain $D\subset\Omega$.
To allow numerically stable  bounds later, we permit a holomorphic nonvanishing \emph{normalizer}
(or \emph{gauge}) $\mathcal O$ on $D$, and define
\begin{equation}\label{eq:J-def}
  \mathcal{J}(s)\ :=\ \frac{\dettwo(I-A(s))}{\zeta(s)}\cdot \frac{s-1}{s}\cdot \frac{1}{\mathcal O(s)},\qquad s\in D.
\end{equation}
The factor $(s-1)$ cancels the simple pole of $\zeta$ at $s=1$; the factor $1/s$ plays no role on $D\subset\Omega$
(but is convenient in later normalization).
Unless explicitly stated otherwise, we work in the \emph{raw $\zeta$-gauge} $\mathcal O\equiv 1$ and denote the resulting
objects by $\mathcal J_{\rm raw}$ and $\Theta_{\rm raw}$; for readability we usually drop the subscript in this default gauge.
\begin{remark}[Gauge changes and what they do \emph{not} change]\label{rem:Ocan-role}
If $\mathcal O$ is holomorphic and nonvanishing on $D$, then multiplying by $\mathcal O^{-1}$ cannot introduce poles on $D$.
Thus the pole set of $\mathcal J$ on $D$ is independent of the choice of gauge.
However, quantitative bounds (e.g.\ Schur bounds on $\Theta$) are not gauge-invariant; when a nontrivial gauge is used for a
 bound, one also requires that $\mathcal O$ is holomorphic and nonvanishing on the domain.
In the raw gauge $\mathcal O\equiv 1$ one has $\mathcal J(s)\to 1$ as $\Re s\to+\infty$, and hence $\Theta(s)\to 1/3$.
\end{remark}

The associated Cayley transform is
\begin{equation}\label{eq:Theta-def}
  \Theta(s)\ :=\ \frac{2\mathcal J(s)-1}{2\mathcal J(s)+1}.
\end{equation}
Heuristically, $2\mathcal J$ plays the role of a Herglotz-type quantity and $\Theta$ the corresponding Schur function.
The next section makes precise the key implication used later: a Schur bound on $\Theta$ forces removability of any
isolated singularity of $\mathcal J$.
\begin{lemma}[Zeros of $\zeta$ produce poles of $\mathcal J$]\label{lem:poles}
Let $D\subset\Omega$ be a domain and assume the chosen gauge $\mathcal O$ is holomorphic and nonvanishing on $D$.
If $\rho\in D$ is a zero of $\zeta(s)$, then $\rho$ is a pole of $\mathcal J(s)$ defined in \eqref{eq:J-def}.
\end{lemma}
\begin{proof}
By \eqref{eq:J-def}, the only possible singularities of $\mathcal J$ on $D$ arise from zeros of $\zeta$ and from zeros of
$\mathcal O$. The latter do not occur by assumption. The factor $(s-1)/s$ is holomorphic and nonzero on $D\subset\Omega$.
Finally, $\dettwo(I-A(s))$ is holomorphic and nonzero on $\Omega$ by \eqref{eq:det2-product}. Hence a zero of $\zeta$ at $\rho$
forces a pole of $\mathcal J$ at $\rho$.
\end{proof}

\subsection*{Schur and Herglotz classes (terminology)}
Let $D\subset\C$ be a domain. A holomorphic function $\Theta$ on $D$ is called \emph{Schur} if $|\Theta|\le 1$ on $D$.
A holomorphic function $H$ on $D$ is called \emph{Herglotz} if $\Re H\ge 0$ on $D$.
The Cayley transform identifies these classes: if $H$ is Herglotz and $H\not\equiv -1$, then
\[
  \Theta=\frac{H-1}{H+1}
\]
is Schur. Conversely, if $\Theta$ is Schur and $\Theta\not\equiv 1$, then $(1+\Theta)/(1-\Theta)$ is Herglotz;
see \cite{Donoghue,RosenblumRovnyak}.

\subsection*{Outline of the far-field strategy in this language}
Theorem~\ref{thm:farfield} will follow once we establish that $\Theta$ is Schur on $\{\,\Re s>0.6\,\}$ (in the relevant gauge).
Indeed, if $|\Theta|\le 1$ holds on $\{\,\Re s>0.6\,\}$ away from the poles of $\mathcal J$, then $\Theta$ is bounded near any
isolated singularity, and thus extends holomorphically across it by the removable singularity theorem.
Since poles of $\mathcal J$ correspond to zeros of $\zeta$ in $\Omega$ by Lemma~\ref{lem:poles}, this rules out zeros of $\zeta$
in the far region. The precise pinch argument is proved in the next section.

\section{Schur/Herglotz pinch mechanism}\label{sec:pinch}
\par\noindent
This section records the analytic mechanism that converts a Schur bound for the Cayley field $\Theta$
into a zero-free region for $\zeta$.
The point is that a holomorphic function bounded by $1$ cannot have a pole, and any isolated singularity is removable.
In our setting, poles of $\mathcal J$ in $\Omega$ encode zeros of $\zeta$ (Lemma~\ref{lem:poles}), so a Schur bound forces those zeros to be absent.

\subsection*{Removable singularities under a Schur bound}
\begin{lemma}[Removable singularity under a strict Schur bound]\label{lem:removable-schur-p1}
Let $D\subset\C$ be a disc centered at $\rho$ and let $\Theta$ be holomorphic on $D\setminus\{\rho\}$ with $|\Theta(s)|<1$ there.
Then $\Theta$ extends holomorphically to $D$ and satisfies $|\Theta(s)|<1$ for all $s\in D$.
Consequently, the Cayley inverse
\[
  \mathcal H(s)\ :=\ \frac{1+\Theta(s)}{1-\Theta(s)}
\]
is holomorphic on $D$ and satisfies $\Re \mathcal H(s) > 0$ on $D$.
\end{lemma}
\begin{proof}
Since $\Theta$ is bounded on the punctured disc, Riemann's removable singularity theorem gives a holomorphic extension to $D$
(e.g.\ \cite[Ch.~2]{Ahlfors}).
The extension still satisfies $|\Theta|\le 1$ on $D$ by continuity.
If $|\Theta(\rho)|=1$, then $|\Theta|$ attains its maximum at an interior point, so $\Theta$ is constant unimodular on $D$
by the Maximum Modulus Principle (e.g.\ \cite[Ch.~2]{Ahlfors}); this contradicts the strict bound $|\Theta|<1$ on $D\setminus\{\rho\}$.
Hence $|\Theta(\rho)|<1$, and therefore $|\Theta|<1$ holds everywhere on $D$.
The Möbius map $w\mapsto (1+w)/(1-w)$ sends the unit disc into the right half-plane, so $\Re \mathcal H>0$ on $D$.
\end{proof}

\subsection*{From a Schur bound to absence of poles}
\begin{corollary}[Schur bound prevents poles of $\mathcal J$]\label{cor:no-poles}
Let $U\subset\Omega$ be a domain and let $S\subset U$ be a discrete set.
Assume $\Theta$ is holomorphic on $U\setminus S$ and satisfies $|\Theta(s)|\le 1$ for all $s\in U\setminus S$.
Then $\Theta$ extends holomorphically to $U$ and satisfies $|\Theta|\le 1$ on $U$.
If, moreover, $\Theta$ is not identically $1$ on any connected component of $U$, then
\[
  2\mathcal J(s)\;=\;\frac{1+\Theta(s)}{1-\Theta(s)}
\]
extends holomorphically to $U$ with $\Re(2\mathcal J)\ge 0$ there; in particular $\mathcal J$ has no poles in $U$.
\end{corollary}
\begin{proof}
Fix $\rho\in S$ and choose a disc $D\subset U$ centered at $\rho$ with $D\cap S=\{\rho\}$.
On $D\setminus\{\rho\}$ we have $|\Theta|\le 1$, hence $\Theta$ is bounded near $\rho$ and extends holomorphically across $\rho$
by Riemann's theorem (again \cite[Ch.~2]{Ahlfors}).
Doing this for each $\rho\in S$ yields a holomorphic extension of $\Theta$ to all of $U$.
The bound $|\Theta|\le 1$ persists by continuity.

If $\Theta(s_0)=1$ at an interior point $s_0$ of a connected component $U_0\subset U$, then $|\Theta|$ attains its maximum
at $s_0$, so $\Theta$ is constant unimodular on $U_0$ by the Maximum Modulus Principle \cite[Ch.~2]{Ahlfors}.
Thus the hypothesis excludes $\Theta\equiv 1$ on $U_0$, and therefore $1-\Theta$ is nonvanishing on $U$.
Hence the Cayley inverse $(1+\Theta)/(1-\Theta)$ is holomorphic on $U$.
Since the Cayley map sends the unit disc into the right half-plane, we have $\Re(2\mathcal J)\ge 0$ on $U$.
A holomorphic function cannot have a pole, so $\mathcal J$ has no poles in $U$.
\end{proof}

\subsection*{Conclusion: Schur on a far half-plane implies Theorem~\ref{thm:farfield}}
By Lemma~\ref{lem:poles}, any zero $\rho$ of $\zeta$ in $\Omega$ produces a pole of $\mathcal J$ in $\Omega$
(the numerator factors in \eqref{eq:J-def} are nonzero on $\Omega$).

On $\Omega=\{\Re s>\tfrac12\}$ we have $0\notin\Omega$, so the compensator $(s-1)/s$ introduces no pole on the working domain.
The point $s=1$ lies in $\Omega$ but the factor $(s-1)$ cancels the simple pole of $\zeta$ there, so $\mathcal J$ is holomorphic at $s=1$
in the raw gauge $\mathcal O\equiv 1$.
Whenever a nontrivial gauge $\mathcal O$ is introduced, we require (and, for  bounds, separately verify) that $\mathcal O$ is
holomorphic and nonvanishing on the stated domain; see Remark~\ref{rem:Ocan-role}.

Therefore, if we can certify a Schur bound for $\Theta$ on a half-plane
$U_\varepsilon=\{\,\Re s>0.6-\varepsilon\,\}$ with some $\varepsilon>0$, then Corollary~\ref{cor:no-poles} implies
$\mathcal J$ has no poles in $U_\varepsilon$, hence $\zeta$ has no zeros in $U_\varepsilon$.
Since $\{\,\Re s\ge 0.6\,\}\subset U_\varepsilon$, this yields Theorem~\ref{thm:farfield}.
The next section discharges the Schur bound by a boundary-certificate route and then specializes to $U_\varepsilon$.
\section{All-heights Schur bound via a boundary wedge certificate}\label{sec:hybrid}
We now discharge the Schur bound required in Corollary~\ref{cor:no-poles} on a half-plane $U_\varepsilon$.
The key input is an unconditional \emph{boundary wedge} \textup{(P+)} for a suitably outer-normalized version of $\mathcal J$
on the boundary line $\Re s=\tfrac12$. This route is analytic (no large-height asymptotics) and applies for all heights.

\subsection*{Outer normalization on $\Re s=\tfrac12$}
Define
\[
  F(s):=\frac{\dettwo(I-A(s))}{\zeta(s)}\cdot\frac{s-1}{s},\qquad (\Re s>\>\tfrac12),
\]
and extend $F$ to $\Omega\setminus Z(\zeta)$ by analytic continuation (removing the discrete pole set $Z(\zeta)$).
\begin{lemma}[Boundary admissibility and Smirnov class for $F$]\label{lem:F-boundary-admissible}
Let $F$ be as above. Then on each connected component of $\Omega\setminus Z(\zeta)$:
\begin{enumerate}
\item $F$ belongs to the Smirnov class $N^+$ (see, e.g., \cite[Ch.~10]{DurenHp}) and therefore admits nontangential boundary values
$F^*(t)=\ntlim_{\sigma\downarrow \tfrac12}F(\sigma+it)$ for Lebesgue-a.e.\ $t\in\mathbb R$.
\item The boundary log-modulus $u(t):=\log|F^*(t)|$ lies in $L^1_{\mathrm{loc}}(\mathbb R)$.
\end{enumerate}
Moreover, if $|u(t)|\le C\log(2+|t|)$ for $|t|\ge 1$, then $u\in L^1(\mathbb R,(1+t^2)^{-1}dt)$.
\end{lemma}
\begin{proof}
Fix a connected component $U$ of $\Omega\setminus Z(\zeta)$. By Lemma~\ref{lem:F-boundedtype-from-J}, for every compact interval
$I\Subset\mathbb R$ with $Q_\alpha(I)\Subset U$ the restriction of $F$ to $Q_\alpha(I)$ is of bounded type.
Since $U$ is covered by such Whitney regions and bounded type is local on simply connected subdomains, it follows that $F$ is of bounded type on $U$.

Next, on each such $Q_\alpha(I)\Subset U$, the boundary log-modulus of $\dettwo(I-A)$ lies in $L^1(I)$ by Lemma~\ref{lem:det2-logL1-from-carleson},
and $\log|\zeta(\tfrac12+it)|\in L^1(I)$ with $L^1$-convergence from the interior by Lemma~\ref{lem:zeta-logL1-components}.
Unwinding the definition of $F$ (as a holomorphic combination of $\dettwo(I-A)$ and $\zeta$ on $U$), this gives $\log|F^*|\in L^1_{\mathrm{loc}}$ on $\partial U\cap\{\Re s=\tfrac12\}$.
Applying Lemma~\ref{lem:BT-to-Nplus} on each Whitney region yields $F\in N^+(U)$, hence $F$ admits nontangential boundary values a.e.\ and
$u(t)=\log|F^*(t)|\in L^1_{\mathrm{loc}}(\mathbb R)$.

Finally, if $|u(t)|\le C\log(2+|t|)$ for $|t|\ge 1$, then
\[
\int_{\mathbb R}\frac{|u(t)|}{1+t^2}\,dt \ \le\ C\int_{\mathbb R}\frac{\log(2+|t|)}{1+t^2}\,dt \ <\ \infty,
\]
so $u\in L^1(\mathbb R,(1+t^2)^{-1}dt)$.
\end{proof}
\begin{lemma}[Local bounded-type control for $F$ from the Appendix normalizer]\label{lem:F-boundedtype-from-J}
Fix a compact interval $I\Subset\R$ and a Whitney region $Q_{\alpha}(I)\Subset\Omega$.
Assume the hypotheses of Appendix Lemma~\ref{lem:J-boundedtype-local} and Theorem~\ref{thm:phase-velocity-quant},
so that the Appendix constructs an outer function $\mathcal O$ on $\Omega$ with boundary modulus
$|\mathcal O(\tfrac12+it)|=ig|\dettwo(I-A(\tfrac12+it))|\,ig|\xi(\tfrac12+it)|^{-1}$ a.e.\ on $I$ and defines
\(
\mathcal J(s)=\dettwo(I-A(s))/(\mathcal O(s)\,\xi(s))
\)
of bounded type on $Q_{\alpha}(I)$.
Then $F$ is of bounded type on $Q_{\alpha}(I)$.
\end{lemma}
\begin{proof}
On $Q_\alpha(I)$, the Appendix constructs an outer function $\mathcal O$ with the stated boundary modulus on $I$ and defines
\(
\mathcal J=\dettwo(I-A)/(\mathcal O\,\xi)
\)
of bounded type on $Q_\alpha(I)$.
By the definition of $F$ in the main text, $F$ is obtained from $\mathcal J$ by composing with holomorphic operations that preserve bounded type on domains
(products, quotients by nonvanishing bounded-type functions, and linear fractional transformations with holomorphic coefficients).
Since $\mathcal O$ is outer and $\xi$ is holomorphic and nonvanishing on $Q_\alpha(I)\subset \Omega\setminus Z(\zeta)$, these operations are legitimate on $Q_\alpha(I)$.
Therefore $F$ is of bounded type on $Q_\alpha(I)$.
\end{proof}
\begin{lemma}[Smirnov upgrade from bounded type and boundary log-modulus]\label{lem:BT-to-Nplus}
Let $U\subset\Omega$ be a simply connected domain with rectifiable boundary segment on $\Re s=\tfrac12$ (e.g.\ a Whitney region $Q_\alpha(I)$ as in §\ref{appA:setup} of Appendix~\ref{app:pplus-proof}). 
Let $g$ be holomorphic on $U$ and of bounded type (Nevanlinna class) on $U$. 
Assume $g$ admits nontangential boundary values $g^*(t)$ for Lebesgue-a.e.\ $t$ along $\partial U\cap\{\Re s=\tfrac12\}$ and that $\log|g^*(t)|\in L^1_{\mathrm{loc}}(dt)$ on that boundary segment.
Then $g\in N^+(U)$, and in particular $g$ has nontangential boundary limits a.e.\ on $\partial U\cap\{\Re s=\tfrac12\}$.
\end{lemma}
\begin{proof}
By conformal mapping, it suffices to treat the case of the unit disk $\mathbb D$ (or upper half-plane) with boundary arc corresponding to the given rectifiable boundary segment.
Since $g$ is of bounded type on $U$, it belongs to the Nevanlinna class on $U$; equivalently, $g=h/k$ with $h,k\in H^\infty(U)$ and $k\not\equiv 0$.
The hypothesis $\log|g^*|\in L^1_{\mathrm{loc}}$ on the boundary segment implies that the boundary values of $\log|k^*|$ are locally integrable there as well (because $h$ is bounded),
so the outer-function construction on $U$ produces an outer function $k_{\mathrm{out}}$ with $|k_{\mathrm{out}}^*|=|k^*|$ a.e.\ on that segment.
Replacing $k$ by $k_{\mathrm{out}}$ and $h$ by $h\,k/k_{\mathrm{out}}$ (which remains bounded and holomorphic) yields a representation $g=\tilde h/k_{\mathrm{out}}$ with
$\tilde h\in H^\infty(U)$ and $k_{\mathrm{out}}$ outer. This is precisely $g\in N^+(U)$.
In particular, functions in $N^+(U)$ admit nontangential boundary limits a.e.\ on the corresponding boundary segment.
\end{proof}
\begin{lemma}[From Carleson energy to $L^1$ boundary control for $\log|\dettwo|$]\label{lem:det2-logL1-from-carleson}
Fix a compact interval $I\Subset\R$ and $\varepsilon_0\in(0,\tfrac12]$. Let
\[
U_{\det_2}(\sigma,t)=\log\Big|\dettwo\!\Big(I-A(\tfrac12+\sigma+it)\Big)\Big|,\qquad (\sigma,t)\in(0,\varepsilon_0]\times I,
\]
\editblue{where $\log|\dettwo(I-A)|$ is interpreted componentwise as the real part of any analytic branch $\operatorname{Log}(\dettwo(I-A))$ on each connected component of $\Omega\setminus Z(\dettwo(I-A))$ (so it is branch-independent). Further, $\log|\dettwo(I-A)|$ is subharmonic on $\Omega$ and harmonic on $\Omega\setminus Z(\dettwo(I-A))$; since the (discrete) zero set is polar, it does not affect harmonic-measure boundary trace statements used below.}
Assume the Carleson energy bound of Lemma~\ref{lem:carleson-arith} for $\nabla U_{\det_2}$ on $Q(I)$, uniformly up to height $\varepsilon_0$.
Then the boundary trace $u_{\det_2}(t):=\lim_{\sigma\downarrow0}U_{\det_2}(\sigma,t)$ exists in $\mathrm{BMO}(I)$ (hence in $L^1(I)$), and in particular
\[
\sup_{0<\sigma\le \varepsilon_0}\ \|U_{\det_2}(\sigma,\cdot)\|_{L^1(I)}\ <\ \infty.
\]
\end{lemma}
\begin{proof}
On $\Omega\setminus Z(\dettwo(I-A))$, the function $U_{\det_2}=\log|\dettwo(I-A)|$ is harmonic.
The Carleson energy hypothesis implies that the measure $|\nabla U_{\det_2}(\sigma,t)|^2\,\sigma\,d\sigma\,dt$ is Carleson on $Q(I)$.
By the Fefferman--Stein characterization of $\mathrm{BMO}$ boundary traces via Carleson measures for Poisson/harmonic extensions (see, for example, \cite[Ch.~IV, Thm.~3, p.~159]{SteinHA} and \cite[Ch.~VI, Thm.~3.4]{GarnettBAF}), $U_{\det_2}$ admits nontangential boundary values $U_{\det_2}^*\in\mathrm{BMO}(I)$, hence $U_{\det_2}^*\in L^1(I)$.
In particular, $U_{\det_2}(\sigma,\cdot)\to U_{\det_2}^*$ in $L^1(I)$ as $\sigma\downarrow 0$.
Moreover, $U_{\det_2}(\sigma,\cdot)$ admits a nontangential boundary trace in $\mathrm{BMO}(I)$ (hence in $L^1(I)$); see \cite{SteinHA,GarnettBAF}.
 and depends only on the modulus, hence is independent of any choice of analytic branch for $\operatorname{Log}(\dettwo(I-A))$.
The Carleson energy hypothesis in Lemma~\ref{lem:carleson-arith} gives a Carleson-measure bound for $|\nabla U_{\det_2}|^2\,\sigma\,d\sigma\,dt$ over the Carleson box above $I$.
By the Carleson-measure characterization of BMO boundary traces for harmonic functions on the upper half-plane, this implies that the nontangential boundary trace
$u_{\det_2}(t)=\lim_{\sigma\downarrow 0}U_{\det_2}(\sigma,t)$ exists in $\mathrm{BMO}(I)$; in particular $u_{\det_2}\in L^1(I)$.
Moreover, the same characterization yields the uniform $L^1(I)$ control
\(
\sup_{0<\sigma\le\varepsilon_0}\|U_{\det_2}(\sigma,\cdot)\|_{L^1(I)}<\infty.
\)
Since the zero set $Z(\dettwo(I-A))$ is discrete (hence polar), removing it does not affect harmonic-measure boundary trace statements.
\end{proof}
\begin{lemma}[Boundary log-modulus control for $\zeta$ on components]\label{lem:zeta-logL1-components}
Fix a compact interval $I\Subset\mathbb R$ and $\varepsilon_0\in(0,\tfrac12]$.
Let $U$ be a connected component of $\Omega\setminus Z(\zeta)$ intersecting $Q_{\varepsilon_0}(I)$.
Then $\zeta$ is holomorphic and nonvanishing on $U$, hence $u(s)=\log|\zeta(s)|$ is harmonic on $U$.
Moreover, the boundary trace $t\mapsto \log|\zeta(\tfrac12+it)|$ lies in $L^1(I)$ and
\[
\log|\zeta(\tfrac12+\varepsilon+it)|\to \log|\zeta(\tfrac12+it)| \quad\text{in }L^1(I)\ \text{as }\varepsilon\downarrow0.
\]
\end{lemma}
\begin{proof}
Let $U$ be a connected component of $\Omega\setminus Z(\zeta)$ intersecting $Q_{\varepsilon_0}(I)$. Then $\zeta$ is holomorphic and nonvanishing on $U$, hence $u(s)=\log|\zeta(s)|$ is harmonic on $U$.
On the compact strip segment $\{\sigma+it:\sigma\in[\tfrac12,\tfrac12+\varepsilon_0],\ t\in I\}$, $\zeta$ has only finitely many zeros (counted with multiplicity).
For each zero $s_k$ in this compact set, write $\zeta(s)=(s-s_k)^{m_k}g_k(s)$ with $g_k$ holomorphic and nonvanishing in a neighborhood of $s_k$.
Covering the compact strip by finitely many such neighborhoods and a zero-free remainder shows that on the strip
\[
\log|\zeta(s)|=\sum_k m_k\log|s-s_k| + O(1),
\]
with the $O(1)$ bounded on the strip.
For each fixed $s_k$, the functions $t\mapsto \log|(\tfrac12+\varepsilon+it)-s_k|$ are uniformly $L^1(I)$-bounded for $\varepsilon\in(0,\varepsilon_0]$ and converge in $L^1(I)$ as $\varepsilon\downarrow 0$.
Therefore dominated convergence yields the stated $L^1(I)$ convergence
\(
\log|\zeta(\tfrac12+\varepsilon+it)|\to \log|\zeta(\tfrac12+it)|
\)
as $\varepsilon\downarrow0$.
\end{proof}
\begin{lemma}[Local $L^1$ control of $\log|F^*|$ on boundary intervals]\label{lem:F-logL1-local}
Fix a compact interval $I\Subset\mathbb R$ and $\varepsilon_0\in(0,\tfrac12]$, and set
\[
Q_{\varepsilon_0}(I):=\{\,\tfrac12+\sigma+it:\ 0<\sigma\le \varepsilon_0,\ t\in I\,\}\Subset\Omega.
\]
Let
\[
F(s):=\dettwo(I-A(s))\,\frac{s-1}{s\,\zeta(s)},\qquad s\in\Omega\setminus Z(\zeta).
\]
Assume:
\begin{enumerate}
\item[(i)] $\log|\dettwo(I-A(\tfrac12+\varepsilon+it))|\in L^1(I)$ uniformly for $\varepsilon\in(0,\varepsilon_0]$, and the nontangential boundary limit
$\log|\dettwo(I-A(\tfrac12+it))|$ exists in $L^1(I)$;
\item[(ii)] for each connected component $U$ of $\Omega\setminus Z(\zeta)$ intersecting $Q_{\varepsilon_0}(I)$, the function $\log|\zeta(\tfrac12+\varepsilon+it)|$ has an $L^1(I)$-limit as $\varepsilon\downarrow0$ when restricted to $U$.
\end{enumerate}
Then on each such component $U$, the nontangential boundary values $F^*(t)$ exist for Lebesgue-a.e.\ $t\in I$, and $\log|F^*(t)|\in L^1_{\mathrm{loc}}(I)$ on $U$.
\end{lemma}
\begin{proof}
Fix a component $U$ as in the statement. For $s=\tfrac12+\varepsilon+it$ with $0<\varepsilon\le \varepsilon_0$ and $t\in I$, we have
\[
\log|F(s)|=\log|\dettwo(I-A(s))|+\log|s-1|-\log|s|-\log|\zeta(s)|.
\]
Since $I$ is compact and $\varepsilon\in(0,\varepsilon_0]$, the functions $t\mapsto \log|\tfrac12+\varepsilon+it|$ and $t\mapsto \log|-\tfrac12+\varepsilon+it|$
are bounded on $I$, uniformly in $\varepsilon$; hence $\log|s|$ and $\log|s-1|$ contribute uniformly bounded $L^1(I)$ terms.
Assumptions (i)--(ii) therefore imply that $\log|F(\tfrac12+\varepsilon+it)|$ is uniformly in $L^1(I)$ and has an $L^1(I)$ limit as $\varepsilon\downarrow0$ along $U$.
In particular, after passing to a subsequence if needed, $F(\tfrac12+\varepsilon+it)$ has a nontangential boundary limit for a.e.\ $t\in I$, and the limiting boundary modulus satisfies
$\log|F^*(t)|\in L^1_{\mathrm{loc}}(I)$ on $U$.
\end{proof}
\begin{lemma}[Outer factor from boundary modulus on $\Omega$]\label{lem:outer-factor-halfplane}
Assume Lemma~\ref{lem:F-boundary-admissible} together with $u\in L^1(\mathbb R,(1+t^2)^{-1}dt)$.
Then there exists a holomorphic function $\mathcal O_\zeta$ on $\Omega$, unique up to a unimodular constant,
with no zeros on $\Omega$, such that the nontangential boundary values satisfy
\[
  ig|\mathcal O_\zeta(\tfrac12+it)|=ig|F^*(t)|\qquad\text{for Lebesgue-a.e.\ }t\in\mathbb R.
\]
Moreover, $\log|\mathcal O_\zeta(s)|$ is the Poisson extension of $u(t)$ from the boundary line $\Re s=\tfrac12$.
\end{lemma}
\begin{proof}
Translate $\Omega$ to the right half-plane $\{\,\Re w>0\,\}$ via $w=s-\tfrac12$.
Since $u\in L^1(\mathbb R,(1+t^2)^{-1}dt)$, its Poisson extension $U=\mathcal P[u]$ is a harmonic function on $\Omega$
with nontangential boundary trace $u$ a.e.
Choose a harmonic conjugate $V$ of $U$ on $\Omega$ and set $\mathcal O_\zeta:=\exp(U+iV)$.
Then $\mathcal O_\zeta$ is holomorphic and zero-free on $\Omega$, and by Fatou theory its boundary modulus is $e^{u(t)}$
for a.e.\ $t$. Uniqueness up to a unimodular constant follows because the ratio of two such outer functions has boundary modulus $1$ a.e.\ and hence is an inner constant; see Garnett~\cite[Ch.~II]{GarnettBAF}.
\end{proof}

Define the outer-normalized ratio
\begin{equation}\label{eq:J-out}
  \mathcal J_{\rm out}(s):=\frac{F(s)}{\mathcal O_\zeta(s)}
  =\frac{\dettwo(I-A(s))}{\mathcal O_\zeta(s)\,\zeta(s)}\cdot\frac{s-1}{s}.
\end{equation}
Then $|\mathcal J_{\rm out}(\tfrac12+it)|=1$ for Lebesgue-a.e.\ $t$.

\subsection*{Boundary wedge (P+)}
Let $w(t):=\Arg \mathcal J_{\rm out}(\tfrac12+it)$ be the boundary phase (defined for a.e.\ $t$).
We say that \textup{(P+)} holds if there exists $m\in\mathbb R$ such that
\[
  |w(t)-m|<\frac{\pi}{2}\qquad\text{for Lebesgue-a.e.\ }t\in\mathbb R.
\]
Equivalently, $\Re\!ig(e^{-im}\mathcal J_{\rm out}(\tfrac12+it))\ge 0$ for Lebesgue-a.e.\ $t$.
\begin{theorem}[Boundary wedge \textup{(P+)}]\label{thm:Pplus}
The boundary wedge \textup{(P+)} holds for $\mathcal J_{\rm out}$.
\end{theorem}
\begin{proof}
See \S\S\ref{app:phase-velocity}--\ref{app:assemble-pplus} in Appendix~\ref{app:pplus-proof}.
\end{proof}

\subsection*{From (P+) to an interior Schur bound}
Fix $m$ witnessing \textup{(P+)} and set
\[
  \widetilde{\mathcal J}(s):=e^{-im}\mathcal J_{\rm out}(s).
\]
Then $|\widetilde{\mathcal J}(\tfrac12+it)|=1$ a.e.\ and $\Re\,\widetilde{\mathcal J}(\tfrac12+it)\ge 0$ a.e.
Define the corresponding Cayley field
\[
  \Theta_{\rm out}(s):=\frac{2\widetilde{\mathcal J}(s)-1}{2\widetilde{\mathcal J}(s)+1}.
\]
\begin{lemma}[Smirnov regularity for $\mathcal J_{\rm out}$ and $\Theta_{\rm out}$]\label{lem:smirnov-regularity}
Assume Lemmas~\ref{lem:F-boundary-admissible} and~\ref{lem:outer-factor-halfplane}.
Then $\mathcal J_{\rm out}\in N^+(\Omega\setminus Z(\zeta))$ and admits nontangential boundary values
$\mathcal J_{\rm out}(\tfrac12+it)$ for Lebesgue-a.e.\ $t$.
Consequently, $\widetilde{\mathcal J}=e^{-im}\mathcal J_{\rm out}\in N^+(\Omega\setminus Z(\zeta))$, and
$\Theta_{\rm out}$ admits Lebesgue-a.e.\ boundary values on $\Re s=\tfrac12$.
\end{lemma}
\begin{proof}
By Lemma~\ref{lem:F-boundary-admissible}, $F\in N^+(\Omega\setminus Z(\zeta))$.
By Lemma~\ref{lem:outer-factor-halfplane}, $\mathcal O_\zeta$ is holomorphic, zero-free, and outer on $\Omega$, hence lies in $N^+(\Omega)$.
Therefore $\mathcal J_{\rm out}=F/\mathcal O_\zeta\in N^+(\Omega\setminus Z(\zeta))$.
Multiplication by a unimodular constant preserves $N^+$ membership, so $\widetilde{\mathcal J}\in N^+(\Omega\setminus Z(\zeta))$.
The boundary-value statements follow from Smirnov boundary theory (e.g.\ Garnett~\cite[Ch.~II]{GarnettBAF}).
Finally, $\Theta_{\rm out}$ is a Möbius transform of $\widetilde{\mathcal J}$; its a.e.\ boundary values exist wherever
$2\widetilde{\mathcal J}(\tfrac12+it)\neq -1$, which holds a.e.\ since $\Re\,\widetilde{\mathcal J}(\tfrac12+it)\ge 0$ a.e.
\end{proof}
\begin{lemma}[Boundary-to-interior Schur transport on components of $\Omega\setminus Z(\zeta)$]\label{lem:schur-transport-omega}
Let $U$ be a connected component of $\Omega\setminus Z(\zeta)$ and let $\Theta\in N^+(U)$ admit nontangential boundary values
$\Theta(\tfrac12+it)$ for Lebesgue-a.e.\ $t$.
If $|\Theta(\tfrac12+it)|\le 1$ for Lebesgue-a.e.\ $t$, then $|\Theta(s)|\le 1$ for all $s\in U$.
\end{lemma}
\begin{proof}
For $\Theta\not\equiv 0$, the function $u:=\log|\Theta|$ is subharmonic on $U$ and has a harmonic majorant because $\Theta\in N^+(U)$.
At Lebesgue points where the nontangential boundary values exist, the boundary hypothesis gives $u(\tfrac12+it)\le 0$ a.e.
Since $Z(\zeta)$ is discrete, it is a polar set, hence has harmonic measure zero for each component $U$;
therefore the Poisson/harmonic-measure domination principle on $U$ yields $u(s)\le 0$ for all $s\in U$.
Thus $|\Theta(s)|\le 1$ on $U$.
See Garnett~\cite[Ch.~II]{GarnettBAF} and Ransford~\cite[Ch.~5]{RansfordPT}.
\end{proof}
\begin{lemma}[Boundary-to-interior Herglotz transport on components of $\Omega\setminus Z(\zeta)$]\label{lem:herglotz-transport-omega}
Let $U$ be a connected component of $\Omega\setminus Z(\zeta)$ and let $G\in N^+(U)$ admit nontangential boundary values
$G(\tfrac12+it)$ for Lebesgue-a.e.\ $t$.
If $\Re\,G(\tfrac12+it)\ge 0$ for Lebesgue-a.e.\ $t$, then $\Re\,G(s)\ge 0$ for all $s\in U$.
\end{lemma}
\begin{proof}
Apply Lemma~\ref{lem:schur-transport-omega} to the Cayley transform
$\Phi(s):=\frac{G(s)-1}{G(s)+1}$ (defined wherever $G\neq -1$) and use continuity to conclude $\Re\,G\ge 0$ on $U$.
Equivalently, one may apply the harmonic-measure domination principle directly to the harmonic function $\Re\,G$ using that $G\in N^+(U)$.
\end{proof}
\begin{proposition}[Herglotz/Schur transport]\label{prop:herglotz-schur-transport}
Assume \textup{(P+)} for $\mathcal J_{\rm out}$ and Lemma~\ref{lem:smirnov-regularity}.
Then $\Theta_{\rm out}$ is Schur on each connected component of $\Omega\setminus Z(\zeta)$, and
\[
  H(s):=\frac{1+\Theta_{\rm out}(s)}{1-\Theta_{\rm out}(s)}
\]
is Herglotz on $\Omega\setminus Z(\zeta)$ (i.e.\ $\Re H(s)\ge 0$ there). Moreover, $H(s)=2\widetilde{\mathcal J}(s)=2e^{-im}\mathcal J_{\rm out}(s)$.
\end{proposition}
\begin{proof}
Let $U$ be any connected component of $\Omega\setminus Z(\zeta)$.
By Lemma~\ref{lem:smirnov-regularity} we have $\widetilde{\mathcal J}\in N^+(U)$ and it admits nontangential boundary values on $\Re s=\tfrac12$ for Lebesgue-a.e.\ $t$.
By \textup{(P+)} (with the fixed phase $m$), $\Re\,\widetilde{\mathcal J}(\tfrac12+it)\ge 0$ for Lebesgue-a.e.\ $t$.
Therefore Lemma~\ref{lem:herglotz-transport-omega} yields $\Re\,\widetilde{\mathcal J}(s)\ge 0$ for all $s\in U$.
Since the Cayley map $z\mapsto \frac{2z-1}{2z+1}$ sends the closed right half-plane into the closed unit disc, it follows that
$|\Theta_{\rm out}(s)|\le 1$ for all $s\in U$. As $U$ was arbitrary, $\Theta_{\rm out}$ is Schur on each connected component of $\Omega\setminus Z(\zeta)$.
Finally, the Cayley inverse gives
\[
  H(s)=\frac{1+\Theta_{\rm out}(s)}{1-\Theta_{\rm out}(s)}=2\widetilde{\mathcal J}(s)=2e^{-im}\mathcal J_{\rm out}(s),
\]
so $H$ is Herglotz on $\Omega\setminus Z(\zeta)$.
\end{proof}

\subsection{Proof of the main theorem}\label{sec:proof-farfield}
\begin{proof}[Proof of Theorem~\ref{thm:farfield}]
By Proposition~\ref{prop:herglotz-schur-transport}, $\Theta_{\rm out}$ is Schur on $\Omega\setminus Z(\zeta)$; the constants in the transport argument are uniform on Whitney boxes by the length-free wedge assembly in Appendix~A (see §§\ref{lem:whitney-uniform-wedge}--\ref{app:assemble-pplus}).
Apply Corollary~\ref{cor:no-poles} to $\widetilde{\mathcal J}=e^{-im}\mathcal J_{\rm out}$ on $U_\varepsilon$ to conclude that
$\widetilde{\mathcal J}$ has no poles on $U_\varepsilon$; hence $\mathcal J_{\rm out}$ has no poles on $U_\varepsilon$ as well.
Since $\dettwo(I-A)$ and $\mathcal O_\zeta$ are holomorphic and nonvanishing on $\Omega$ (indeed, for $\Re s>\tfrac12$ the diagonal eigenvalues $p^{-s}$ satisfy $|p^{-s}|<1$, so $1$ is not an eigenvalue of $A(s)$ and hence $\dettwo(I-A(s))\neq0$; and $\mathcal O_\zeta$ is an outer factor and therefore zero-free on $\Omega$), poles of $\mathcal J_{\rm out}$ in $\Omega$
can only come from zeros of $\zeta$. Therefore $\zeta$ has no zeros in $U_\varepsilon$, and in particular none in $\{\,\Re s\ge 0.6\,\}$.
\end{proof}
\begin{table}[H]
\centering
\caption{Supplementary computational artifacts (not used in the proof).}\label{tab:artifact-data}
\small
\begin{tabular}{l l l}
\toprule
\textbf{Artifact} & \textbf{Parameter} & \textbf{Value} \\
\midrule
\multicolumn{3}{l}{\textit{Rectangle certification} (\texttt{theta\_certify})} \\
\quad Domain & $[\sigma_{\min}, \sigma_{\max}] \times [t_{\min}, t_{\max}]$ & $[0.6, 0.7] \times [0, 20]$ \\
\quad  upper bound & $\max |\Theta_{\rm proj}|$ & $0.9999928763$ \\
\quad Safety margin & $1 - \theta_{\rm hi}$ & $7.12 \times 10^{-6}$ \\
\quad Status & \texttt{ok} & \texttt{true} \\
\quad Boxes processed & & 380{,}764 \\
\quad Precision & (bits) & 260 \\
\quad Gauge & & \texttt{outer\_zeta\_proj} \\
\midrule
\multicolumn{3}{l}{\textit{Pick certificate} (\texttt{pick\_certify}, $\sigma_0 = 0.599$)} \\
\quad Matrix size & $N$ & 16 \\
\quad Spectral separation & $\delta_{\rm cert}$ & $0.594$ \\
\quad SPD at origin & $P_N \succ 0$ & \texttt{true} \\
\quad Coefficient count & $N_{\rm coeff}$ & 128 \\
\quad Tail sum (diagnostic) & $\sum_{16}^{127}|a_n|$ & $0.67$ \\
\quad Gauge & & \texttt{raw\_zeta} \\
\midrule
\multicolumn{3}{l}{\textit{Pick certificate} (\texttt{pick\_certify}, $\sigma_0 = 0.6$)} \\
\quad Matrix size & $N$ & 16 \\
\quad Spectral separation & $\delta_{\rm cert}$ & $0.594$ \\
\quad SPD at origin & $P_N \succ 0$ & \texttt{true} \\
\quad Coefficient count & $N_{\rm coeff}$ & 128 \\
\quad Tail sum (diagnostic) & $\sum_{16}^{127}|a_n|$ & $0.67$ \\
\quad Gauge & & \texttt{raw\_zeta} \\
\midrule
\multicolumn{3}{l}{\textit{Pick certificate} (\texttt{pick\_certify}, $\sigma_0 = 0.7$)} \\
\quad Matrix size & $N$ & 16 \\
\quad Spectral separation & $\delta_{\rm cert}$ & $0.627$ \\
\quad SPD at origin & $P_N \succ 0$ & \texttt{true} \\
\quad Coefficient count & $N_{\rm coeff}$ & 128 \\
\quad Tail sum (diagnostic) & $\sum_{16}^{127}|a_n|$ & $0.61$ \\
\quad Gauge & & \texttt{raw\_zeta} \\
ottomrule
\end{tabular}
\end{table}
\section*{Conclusion and limitations (unconditional status)}

We prove a fixed half-plane zero-exclusion for the Riemann zeta function: $\zeta(s)\neq 0$ for $\Re s\ge 0.6$ (Theorem~\ref{thm:farfield}).

The argument is analytic and function-theoretic: zeros are converted into poles of an arithmetic ratio $\mathcal J$, and a Schur bound $|\Theta|\le 1$ for the associated Cayley field forces removability and rules out poles (hence zeros).
For transparency, the handoff bundle also provides independently checkable ball-arithmetic artifacts on representative low-height rectangles together with a verification protocol in Appendix~\ref{app:verify}.
The technically hardest step is establishing the all-heights Schur bound, which is discharged by the boundary wedge certificate \textup{(P+)} (Section~\ref{sec:hybrid}).
The supplementary artifacts in Table~\ref{tab:artifact-data} are provided for independent verification of the low-height rectangle and Pick cross-checks listed in Appendix~\ref{app:verify}.

\paragraph{Computer assistance and verifiability.}
Although the core implications are analytic, the repository provides rigorous numerical certificates (ball arithmetic) together with a verifier and JSON outputs so that the finite checks can be independently verified.

\paragraph{Limitations and scope.}
We do not claim the Riemann Hypothesis here.

All conclusions are confined to the explicit region proved, and no statement is made about zeros with $\Re s<0.6$.
It isolates and certifies a fixed far-field exclusion $\Re s\ge 0.6$.
Pushing the boundary $0.6$ closer to $1/2$ within this framework would require sharpening the analytic boundary-certificate constants and the Carleson/box-energy bounds that enter the wedge criterion, which we do not pursue here.

\section*{Statements and Declarations}

\paragraph{Competing interests.}
The author declares no competing interests.

\paragraph{Data and materials availability.}
For reproducibility, the verification is with respect to the shipped bundle (and its SHA-256 manifest, if provided). If any file is updated after theory revision, the verification should be rerun and the manifest regenerated.
All computational artifacts used for supplementary cross-checks are included in the handoff bundle (and mirrored in the repository):
\begin{quote}\small\ttfamily
compute/artifacts/theta\_certify\_sigma06\_07\_t0\_20\_outer\_zeta\_proj.json\\
compute/artifacts/pick\_sigma0599\_raw\_zeta\_N16.json\\
compute/artifacts/pick\_sigma06\_raw\_zeta\_N16.json\\
compute/artifacts/pick\_sigma07\_raw\_zeta\_N16.json\\
compute/verify\_attachment\_arb.py
\end{quote}
\begin{remark}\label{rem:artifact-repro}
The verifier is based on rigorous ball arithmetic (ARB via \texttt{python-flint}) and is intended to be independently verifiable.
Appendix~\ref{app:verify} records the verification manifest and a minimal command sequence for running the verifier on a fresh machine.

\end{remark}

\appendix
\section{Proof of the boundary wedge certificate \textup{(P+)}}\label{app:pplus-proof}
This appendix proves Theorem~\ref{thm:Pplus}. The argument combines a quantitative phase--velocity identity with a Whitney-box Cauchy--Riemann/Green pairing and a Carleson-type energy bound, yielding a windowed phase estimate that implies the boundary wedge inclusion \textup{(P+)}.

% ===== BEGIN inlined from paper1_pplus_proof.tex =====
% ----------------------------------------------------------------------

% Appendix A input file: proof of the boundary wedge certificate (P+).
% This file is \input{} from Appendix~\ref{app:pplus-proof} in paper1_farfield.tex.
% ----------------------------------------------------------------------

\subsection{Statement, standing notation, and domains}

\label{appA:setup}

This subsection fixes the ambient domain, boundary conventions, Whitney geometry, and the meaning of boundary limits, so later phase and energy identities are unambiguous.

Throughout Appendix~\ref{app:pplus-proof} we work in the right half-plane
\[
  \Omega:=\{s\in\mathbb C:\Re s>\tfrac12\},
\]
with boundary line $\partial\Omega=\{\tfrac12+it:t\in\mathbb R\}$.
All analytic objects are understood componentwise on $\Omega\setminus Z$, where $Z$ denotes the relevant zero/pole set,
so that branches of $\log$ and $\Arg$ are well-defined on each connected component.

For a compact interval $I\subset\mathbb R$ and a dilation parameter $\alpha>1$ we write $Q_\alpha(I)$ for the Whitney box
based on $I$, and we use the weighted area measure $\sigma\,dt\,d\sigma$ on $\Omega$, where $\sigma:=\Re s-\tfrac12$.

The goal is to prove the boundary wedge certificate \textup{(P+)} stated in Theorem~\ref{thm:Pplus}.
The proof proceeds by:
(i) a quantitative phase--velocity identity for the boundary phase of $\mathcal J_{\rm out}$,
(ii) a Green/Cauchy--Riemann pairing on Whitney boxes,
(iii) a Carleson-type energy bound for a logarithmic derivative,
and (iv) a quantitative wedge criterion converting windowed phase control into a.e.\ wedge inclusion.

In the phase--velocity identity below, we will show that $-w'$ is positive (in the sense of distributions): a locally finite measure plus a discrete atomic part encoding the off--critical zero data that enter the boundary phase derivative.

\subsection{A quantitative wedge criterion from Whitney-local control}

\label{app:whitney-wedge}

We state the wedge target (P+) in a form suited to local Whitney-box estimates and record the boundary conventions used throughout Appendix~A.

We work on the boundary line $\Re s=\tfrac12$ and use the following conventions.
\begin{itemize}
\item \emph{Wedge.} For an aperture parameter $\alpha\in(0,\tfrac\pi2)$ and a center angle $m\in\mathbb R$, write
\[
  W_{m,\alpha}:=\{z\in\mathbb C:\ |\Arg(e^{-im}z)|\le \alpha\}.
\]
Thus \textup{(P+)} is the Lebesgue-a.e.\ inclusion $\,\mathcal J_{\rm out}(\tfrac12+it)\in W_{m,\alpha}\,$ for some fixed $\alpha<\tfrac\pi2$
and some $m\in\mathbb R$.

\item \emph{Whitney / Carleson boxes.} For an interval $I\subset\mathbb R$, write the Carleson box
$S(I):=\{\tfrac12+\sigma+it:\ 0<\sigma\le |I|,\ t\in I\}$.
A Whitney box means a box of comparable width and height, e.g.\ $\{\tfrac12+\sigma+it:\ \sigma\in[a|I|,b|I|],\ t\in I\}$ with fixed $0<a<b$.

\item \emph{Meaning of ``a.e.''} Unless explicitly stated otherwise, ``a.e.'' refers to Lebesgue measure $dt$ on $\mathbb R$.
\end{itemize}
\begin{lemma}[Outer normalizer from boundary log-modulus]
\label{lem:outer-from-logmodulus}
Let $u\in L^1(\mathbb R,(1+t^2)^{-1}dt)$ be real-valued. Then there exists an outer function $O$ on $\Omega$
(zero-free and holomorphic on $\Omega$) whose nontangential boundary values satisfy
\[
|O(\tfrac12+it)| = e^{u(t)} \quad\text{for a.e. }t\in\mathbb R.
\]
Moreover $O$ is unique up to a unimodular constant.
\end{lemma}
\begin{proof}
Define the Poisson extension $U$ of $u$ to $\Omega$ by
\[
U(\tfrac12+\sigma+it)\ :=\ \frac{1}{\pi}\int_{\mathbb R} u(\tau)\,\frac{\sigma}{\sigma^2+(t-\tau)^2}\,d\tau,
\qquad \sigma>0.
\]
The weighted integrability $u\in L^1(\mathbb R,(1+t^2)^{-1}dt)$ ensures the integral converges and that $U$ is harmonic on $\Omega$.
Let $V$ be a harmonic conjugate of $U$ on $\Omega$ (defined up to an additive constant), and set
\[
O(s)\ :=\ \exp\!ig(U(s)+iV(s)).
\]
Then $O$ is holomorphic and zero-free on $\Omega$. By the nontangential boundary limit theorem for Poisson extensions of $L^1_{\mathrm{loc}}$ boundary data, one has $U(\tfrac12+\varepsilon+it)\to u(t)$ for a.e.\ $t$ as $\varepsilon\downarrow 0$; hence the nontangential boundary values satisfy $|O(\tfrac12+it)|=e^{u(t)}$ for a.e.\ $t$; see Duren~\cite[Ch.~II]{DurenHp} or Garnett~\cite[Ch.~II]{GarnettBAF}.
Uniqueness up to unimodular constant follows because the ratio of two such outer functions has a.e.\ boundary modulus $1$ and hence is an inner constant.
(Source.) This is the outer-function construction in half-planes; see Duren \emph{$H^p$ Spaces}, Ch.~II, or Garnett \emph{Bounded Analytic Functions}, Ch.~II.
\end{proof}

\subsection{Phase--velocity identity (quantitative form) and boundary passage}

\label{app:phase-velocity}
We establish the boundary phase--velocity relation for the outer-normalized ratio, and record the precise sense in which derivatives and boundary traces are taken.
\begin{lemma}[Outer--Hilbert boundary identity]\label{lem:outer-phase-HT}
Let $u\in L^1_{\mathrm{loc}}(\mathbb R)$ and let $O$ be an outer function on $\Omega$ whose boundary modulus satisfies
$|O(\tfrac12+it)|=e^{u(t)}$ for a.e.\ $t$.
Let $w(t):=\Arg O(\tfrac12+it)$ denote the boundary argument (defined modulo an additive constant).
Then, in \(\mathcal D'(\mathbb R)\),
\[
  \frac{d}{dt}w(t)=\Hilb[u'](t),
\]
where \(\Hilb\) is the boundary Hilbert transform on \(\R\) (as a continuous operator \(\mathcal D'(\R)\to\mathcal D'(\R)\))
and \(u'\) is the distributional derivative.
\end{lemma}
\begin{proof}
Write \(\log O=U+iV\) on \(\Omega\), where \(U=\Re\log O\) is harmonic and \(V=\Im\log O\) is its harmonic conjugate
(on each component of \(\Omega\setminus Z(O)\), fixing a branch of \(\log\)).
The boundary trace satisfies \(U(\tfrac12+\cdot)=u\) in \(\mathcal D'(\R)\), and the conjugate boundary trace is
\(V(\tfrac12+\cdot)=\Hilb[u]\) in \(\mathcal D'(\R)\) (up to an additive constant).
Differentiating in \(t\) in the sense of distributions gives
\[
  \frac{d}{dt}\Arg O(\tfrac12+it)=\partial_t V(\tfrac12+it)=\Hilb[\partial_t u](t)=\Hilb[u'](t),
\]
since differentiation commutes with \(\Hilb\) on \(\mathcal D'(\R)\).
\end{proof}
\begin{lemma}[Smoothed distributional bound for \(\partial_\sigma\,\Re\log\dettwo\)]\label{lem:det2-unsmoothed}
Let \(I\Subset\R\) be a compact interval and fix \(\varepsilon_0\in(0,\tfrac12]\).
There exists a finite constant
\[
  C_*\ :=\ \sum_{p}\sum_{k\ge 2}\frac{p^{-k/2}}{k^2\,\log p}\ <\ \infty
\]
such that for all \(\sigma\in(\tfrac12,\tfrac12+\varepsilon_0]\) and every \(\varphi\in C_c^2(I)\),
\[
  \Big|\int_{\R} \varphi(t)\,\partial_\sigma\Re\log\det_2\!\big(I-A(\sigma+it)\big)\,dt\Big|\ \le\ C_*\,\|\varphi''\|_{L^1(I)}.
\]
\end{lemma}
\begin{proof}
For \(\sigma>\tfrac12\) one has the absolutely convergent expansion
\[
  \partial_\sigma\,\Re\log\det_2\!\big(I-A(\sigma+it)\big)
  \;=\; \sum_{p}\sum_{k\ge 2} (\log p)\,p^{-k\sigma}\cos(k t\log p).
\]
For each frequency \(\omega=k\log p\ge 2\log 2\), two integrations by parts give
\[
  \Big|\int_{\R}\!\varphi(t)\cos(\omega t)\,dt\Big|\ \le\ \frac{\|\varphi''\|_{L^1(I)}}{\omega^2}.
\]
Summing the resulting majorant yields
\[
  \Big|\int \varphi\,\partial_\sigma\Re\log\dettwo\,dt\Big|
  \ \le\ \|\varphi''\|_{L^1}\sum_{p}\sum_{k\ge 2}\frac{(\log p)\,p^{-k\sigma}}{(k\log p)^2}
  \ \le\ \|\varphi''\|_{L^1}\sum_{p}\sum_{k\ge 2}\frac{p^{-k/2}}{k^2\,\log p},
\]
uniformly for \(\sigma\in(\tfrac12,\tfrac12+\varepsilon_0]\), since the rightmost double series converges.
\end{proof}
\begin{lemma}[Arithmetic Carleson energy]\label{lem:carleson-arith}
Let
\[
 U_{\det_2}(\sigma,t)\ :=\ \Re\log\dettwo\!\Big(I-A(\tfrac12+\sigma+it)\Big)
 \ =\ -\sum_{p}\sum_{k\ge 2}\frac{p^{-k/2}}{k}\,e^{-k\log p\,\sigma}\,\cos(k\log p\,t),\qquad \sigma>0,
\]
where the series converges absolutely for every \(\sigma>0\).
Then for every interval \(I\subset\R\) with Carleson box \(Q(I):=I\times(0,|I|]\),
\[
 \iint_{Q(I)} |\nabla U_{\det_2}|^2\,\sigma\,dt\,d\sigma\ \le\ \frac{|I|}{4}\,\sum_{p}\sum_{k\ge 2}\frac{p^{-k}}{k^2}
 \ =:\ K_0\,|I|,\qquad K_0:=\frac{1}{4}\sum_{p}\sum_{k\ge 2}\frac{p^{-k}}{k^2}<\infty.
\]
\end{lemma}
\begin{proof}
For a single mode \(b\,e^{-\omega\sigma}\cos(\omega t)\) one has \(|\nabla|^2=b^2\omega^2e^{-2\omega\sigma}\), hence
\[
 \int_0^{|I|}\!\int_I |\nabla|^2\,\sigma\,dt\,d\sigma
 \ \le\ |I|\cdot\sup_{\omega>0}\int_0^{|I|}\sigma\,\omega^2e^{-2\omega\sigma}d\sigma\cdot b^2
 \ \le\ \tfrac14\,|I|\,b^2.
\]
With \(b=p^{-k/2}/k\) and \(\omega=k\log p\), summing over \((p,k)\) gives the claim and the finiteness of \(K_0\).
\end{proof}

\paragraph{Whitney scale and short--interval zero counts.}
Throughout the boundary-certificate route we work on Whitney boxes based at height \(T\) with
\[
  L=L(T):=\min\Big\{\frac{c}{\log\angles{T}},\ L_\star\Big\},\qquad
  \angles{T}:=\sqrt{1+T^2},\qquad c\in(0,1]\ \text{fixed}.
\]
The only input about the \emph{number} of zeros used below is the short-interval consequence of Riemann--von Mangoldt: there exist absolute constants \(A_0,A_1>0\) such that for \(T\ge 2\) and \(0<H\le 1\),
\[
  N(T;H)\ :=\ \#\{\rho=eta+i\gamma:\ \gamma\in[T,T+H]\}\ \le\ A_0\ +\ A_1\,H\,\log\angles{T}.
\]
\begin{lemma}[Annular Poisson--balayage \(L^2\) bound]\label{lem:annular-balayage}
Let \(I=[T-L,T+L]\), \(Q_\alpha(I)=I\times(0,\alpha L]\), and fix \(k\ge 1\).
For
\(
\mathcal A_k:=\{\rho=eta+i\gamma:\ 2^kL<|T-\gamma|\le 2^{k+1}L\}
\)
set
\[
  V_k(\sigma,t):=\sum_{\rho\in\mathcal A_k}\frac{\sigma}{(t-\gamma)^2+\sigma^2}.
\]
Then
\[
  \iint_{Q_\alpha(I)} V_k(\sigma,t)^2\,\sigma\,dt\,d\sigma\ \ll_\alpha\ |I|\,4^{-k}\,\nu_k,
\]
where \(\nu_k:=\#\mathcal A_k\), and the implicit constant depends only on \(\alpha\).
\end{lemma}
\begin{proof}
Write \(K_\sigma(x):=\sigma/(x^2+\sigma^2)\) and \(V_k=\sum_{\rho\in\mathcal A_k}K_\sigma(\cdot-\gamma)\).
Integrate over \(t\in I\) first.
For the diagonal terms, using \(|t-\gamma|\ge 2^kL-L\ge 2^{k-1}L\) for \(t\in I\) and \(k\ge 1\),
\[
 \int_I K_\sigma(t-\gamma)^2\,dt
 = \sigma^2\!\int_I \frac{dt}{ig((t-\gamma)^2+\sigma^2ig)^2}
 \ \le\ \frac{L}{(2^{k-1}L)^2}\,\sigma.
\]
Multiplying by the area weight \(\sigma\) and integrating \(\sigma\in(0,\alpha L]\) gives a contribution \(\ll_\alpha |I|\,4^{-k}\) per \(\gamma\), hence \(\ll_\alpha |I|\,4^{-k}\nu_k\) after summing.
For off-diagonal terms, for \(i\ne j\) one has on \(I\) that \(K_\sigma(t-\gamma_j)\le \sigma/(2^{k-1}L)^2\), hence
\[
 \int_I K_\sigma(t-\gamma_i)K_\sigma(t-\gamma_j)\,dt
 \ \le\ \frac{\sigma}{(2^{k-1}L)^2}\int_\R K_\sigma(t-\gamma_i)\,dt
 = \frac{\pi\sigma}{(2^{k-1}L)^2},
\]
and integrating \(\sigma\in(0,\alpha L]\) with the extra factor \(\sigma\) yields \(\ll_\alpha |I|\,4^{-k}\).
Summing over pairs \((i,j)\) via a Schur test gives the stated bound (absorbing constants into \(\ll_\alpha\)).
\end{proof}

\subsection{Quantitative phase--velocity identity}

\label{appA:phasevelocity}
This subsection derives the quantitative identity linking the distribution $-w'$ to the off-axis zero data, in a form usable under Whitney localization.
\begin{lemma}[Distributional phase--velocity identity for outer data]\label{lem:pv-distributional}
Let \(U^*\in L^1_{\mathrm{loc}}(\mathbb R)\) and define \(w:=\Hilb[U^*]\in\mathcal D'(\mathbb R)\) by the duality
\[
  -\langle w,\varphi'\rangle \;=\; \int_{\mathbb R} U^*(t)\,(\Hilb\varphi)'(t)\,dt\qquad\forall\,\varphi\in C_c^\infty(\mathbb R).
\]
Then \(w'=\Hilb[(U^*)']\) in \(\mathcal D'(\mathbb R)\).
\end{lemma}
\begin{proof}
Let \(\eta_\varepsilon\) be a mollifier and set \(U_\varepsilon^*:=U^**\eta_\varepsilon\in C^\infty(\mathbb R)\).
For smooth data one has \((\Hilb[U_\varepsilon^*])'=\Hilb[(U_\varepsilon^*)']\) pointwise.
Since \(U_\varepsilon^*\to U^*\) in \(L^1_{\mathrm{loc}}\), we have \((U_\varepsilon^*)'\to (U^*)'\) in \(\mathcal D'\), and the Hilbert transform is continuous on \(\mathcal D'\).
Passing to the limit yields \(w'=\Hilb[(U^*)']\) in \(\mathcal D'\).
\end{proof}

% --- Auxiliary L^1 control for the xi term (used only locally below) ---
\begin{lemma}[Local $L^1$ control for $\log|\xi|$ along vertical approach]\label{lem:xi-deriv-L1}
Fix a compact interval $I\Subset\mathbb R$. Then the family
$t\mapsto \log|\xi(\tfrac12+\varepsilon+it)|$ is bounded in $L^1(I)$ uniformly for $\varepsilon\in(0,1]$.
Moreover, for $\varepsilon,\varepsilon'\downarrow 0$ the difference
$\log|\xi(\tfrac12+\varepsilon+it)|-\log|\xi(\tfrac12+\varepsilon'+it)|$ tends to $0$ in $L^1(I)$.
\end{lemma}
\begin{proof}
Write $\xi$ in Hadamard form $\xi(s)=e^{a+bs}\prod_{\rho}igl(1-\frac{s}{\rho}igr)e^{s/\rho}$, where the product runs over nontrivial zeros $\rho$ of $\zeta$.
Fix $I=[T_0,T_1]\Subset\mathbb R$ and $\varepsilon\in(0,1]$.
Split the zeros into a finite set $\mathcal Z_R:=\{\rho:\ |\Im\rho|\le R\}$ and the complement, with $R\ge 2+\max(|T_0|,|T_1|)$.
For $\rho\in\mathcal Z_R$, the map $t\mapsto \log|(\tfrac12+\varepsilon+it)-\rho|$ lies in $L^1(I)$, with an $L^1(I)$ bound depending only on $I$ and $\mathcal Z_R$ (local integrability of $\log|t-\gamma|$ near $\gamma=\Im\rho$).
For $\rho\notin\mathcal Z_R$ and $t\in I$, one has $|(\tfrac12+\varepsilon+it)/\rho|\ll_I 1/|\rho|$, so
\[
\log\Bigl|\Bigl(1-\frac{\tfrac12+\varepsilon+it}{\rho}\Bigr)e^{(\tfrac12+\varepsilon+it)/\rho}\Bigr|
=O_Iigl(|\rho|^{-2}igr),
\]
uniformly in $t\in I$ and $\varepsilon\in(0,1]$.
Since $\sum_{\rho}|\rho|^{-2}<\infty$ (order $1$ entire function), the tail contributes an absolutely convergent $L^\infty(I)$ error uniformly in $\varepsilon$.
Combining these bounds gives $\sup_{\varepsilon\in(0,1]}\|\log|\xi(\tfrac12+\varepsilon+i\cdot)|\|_{L^1(I)}<\infty$.

For the Cauchy property, write the difference as a sum over the same factorization.
The finite set $\mathcal Z_R$ contributes a term that tends to $0$ in $L^1(I)$ as $\varepsilon,\varepsilon'\downarrow 0$ by dominated convergence away from the finitely many points $t=\Im\rho$ and the local integrability of $\log|t-\Im\rho|$.
The tail is uniformly $O_I\!\left(\sum_{\rho\notin\mathcal Z_R}|\rho|^{-2}\right)$ and hence uniformly small; letting $R\to\infty$ yields the $L^1(I)$-Cauchy claim.
\end{proof}
\begin{lemma}[Local bounded-type control for $\mathcal J$]\label{lem:J-boundedtype-local}
Fix a compact interval $I\Subset\R$. Assume that $\dettwo(I-A(s))$ is holomorphic and nonvanishing on a neighborhood of the Whitney region $Q_{\alpha}(I)\Subset\Omega$,
and that the Carleson energy bounds of Lemmas~\ref{lem:carleson-arith} and \ref{lem:carleson-xi} hold on $Q_{\alpha}(I)$.
Then $\mathcal J$ belongs to the Nevanlinna class (bounded type) on $Q_{\alpha}(I)$.
\end{lemma}
\begin{proof}
Let $D:=Q_{\alpha}(I)$ and write $\sigma=\Re s-\tfrac12$.
On $D$ (which is simply connected), the hypotheses ensure that $\dettwo(I-A)$ is holomorphic and nonvanishing, hence $U:=\log|\dettwo(I-A)|$ is harmonic on $D$.
Similarly, on each component of $D\setminus Z(\xi)$, $V:=\log|\xi|$ is harmonic.
The Carleson energy bounds in Lemmas~\ref{lem:carleson-arith} and \ref{lem:carleson-xi} imply that $|\nabla U|^2\,\sigma\,dt\,d\sigma$ and $|\nabla V|^2\,\sigma\,dt\,d\sigma$ are Carleson measures on $D$.
By the Carleson-measure characterization of BMO boundary traces for harmonic functions on the upper half-plane (see, e.g., Stein, \emph{Harmonic Analysis}, Ch.~IV, or Garnett, \emph{Bounded Analytic Functions}, Ch.~VI),
both $U$ and $V$ admit nontangential boundary traces $U^*,V^*$ on $I$ with $U^*,V^*\in\mathrm{BMO}(I)\subset L^1(I)$, and $U$ (resp. $V$) is the Poisson extension of its trace up to an additive constant on $D$.

Since $U^*\in L^1(I)$, there exists an outer Smirnov function $\mathcal O_{\dettwo}$ on $D$ with boundary modulus $|\mathcal O_{\dettwo}(\tfrac12+it)|=\exp(U^*(t))$ a.e.\ on $I$; likewise there is an outer Smirnov function $\mathcal O_{\xi}$ on $D$ with
$|\mathcal O_{\xi}(\tfrac12+it)|=\exp(V^*(t))$ a.e.\ on $I$.
Then $\dettwo(I-A)/\mathcal O_{\dettwo}$ and $\xi/\mathcal O_{\xi}$ have unimodular boundary values on $I$ and hence are bounded analytic on $D$.
Therefore $\dettwo(I-A)$ and $\xi$ are of bounded type on $D$.

By construction in Appendix~A, the normalizing outer function $\mathcal O$ is an outer Smirnov function on $D$, hence also of bounded type there.
Since the Nevanlinna class is closed under products and quotients (where defined), it follows that
\(
\mathcal J=\dettwo(I-A)/(\mathcal O\,\xi)
\)
is of bounded type on $D$.
\end{proof}
\begin{theorem}[Quantified phase--velocity identity and boundary passage]\label{thm:phase-velocity-quant}
Let
\[
 u_\varepsilon(t):=\log|\dettwo(I-A(\tfrac12+\varepsilon+it))|-\log|\xi(\tfrac12+\varepsilon+it)|.
\]
Then \(u_\varepsilon\) is uniformly \(L^1\)-bounded and Cauchy on every compact \(I\Subset\R\) as \(\varepsilon\downarrow 0\), hence \(u_\varepsilon\to u\) in \(L^1_{\rm loc}(\R)\).
Let \(\mathcal O\) be the outer function on \(\Omega\) with boundary modulus \(e^{u}\) and normalization \(\mathcal O(\tfrac32)>0\), and set
\[
  \mathcal J(s):=\frac{\dettwo(I-A(s))}{\mathcal O(s)\,\xi(s)}.
\]
Then \(|\mathcal J(\tfrac12+it)|=1\) for a.e.\ \(t\in\R\).
By Lemma~\ref{lem:J-boundedtype-local}, $\mathcal J$ is of bounded type on every Whitney region $Q_{\alpha}(I)$.
Let \(w\in\mathcal D'(\R)\) denote the distributional boundary phase of \(\mathcal J\) (defined modulo an additive constant).
Then, for every compact interval \(I\Subset\R\) and every nonnegative \(\phi\in C_c^\infty(I)\),
\begin{equation}\label{eq:pv-identity}
\int_I \phi(t)\,(-w'(t))\,dt
\ =\ \pi\!\int_I \phi(t)\,d\mu_{\rm off}(t)\ +\ \pi\!\int_I \phi(t)\,d\nu_{\rm sing}(t)\ +\ \pi\sum_{\gamma\in I} m_\gamma\,\phi(\gamma),
\end{equation}
where:
\begin{itemize}
\item \(\mu_{\rm off}\) is the Poisson balayage of the off--critical zeros \(\rho=eta+i\gamma\) of \(\zeta\) with \(eta>\tfrac12\), counted with multiplicity \(m_\rho\);
\item \(\nu_{\rm sing}\) is the (possibly zero) singular boundary measure of any singular inner factor in the canonical factorization of \(\mathcal J\) on \(\Omega\); and
\item the discrete sum ranges over boundary zeros/poles on \(\Re s=\tfrac12\), written \(s=\tfrac12+i\gamma\), with multiplicities \(m_\gamma\).
\end{itemize}
\end{theorem}
\begin{proof}
The \(L^1_{\rm loc}\) convergence \(u_\varepsilon\to u\) is as stated.
The outer function \(\mathcal O\) exists by Lemma~\ref{lem:outer-from-logmodulus}.

Under the bounded-type hypothesis, \(\mathcal J\) admits the canonical half-plane factorization into a unimodular constant, a Blaschke product over zeros in \(\Omega\), a (possibly trivial) singular inner factor, and an outer factor.
Since \(|\mathcal J(\tfrac12+it)|=1\) a.e., the outer factor is unimodular constant.
Taking the distributional boundary argument \(w\) and differentiating in \(\mathcal D'\), each factor contributes additively:
the Blaschke product yields the Poisson balayage measure \(\mu_{\rm off}\), the singular inner factor yields \(\nu_{\rm sing}\), and boundary zeros/poles yield atomic Dirac masses.
This is the phase-derivative computation for bounded-type functions on a half-plane; see, e.g., Garnett \emph{Bounded Analytic Functions}, Ch.~II, or Koosis \emph{The Logarithmic Integral}, Vol.~I.
\end{proof}

\subsection{Final assembly ingredients for the boundary wedge certificate (P+)}

This subsection collects the three final analytic ingredients—Poisson plateau, Whitney box pairing, and Carleson-energy control—used to conclude (P+).

\paragraph{Poisson plateau lower bound.}

We prove the key lower bound for the Poisson kernel averaged over admissible windows, which converts discrete off-axis contributions into a uniform wedge aperture.
\begin{lemma}[Poisson plateau lower bound]\label{lem:poisson-plateau}
Let \(\psi\in C_c^\infty(\R)\) be even with \(\psi\equiv 1\) on \([-1,1]\) and \(\operatorname{supp}\psi\subset[-2,2]\).
Then
\[
  c_0(\psi)\ :=\ \inf_{0<b\le 1,\ |x|\le 1} (P_b*\psi)(x)\ \ge\ \frac{1}{2\pi}\arctan 2\;>\;0.
\]
\end{lemma}
\begin{proof}
Since \(\psi\ge \mathbf 1_{[-1,1]}\), it suffices to compute \((P_b*\mathbf 1_{[-1,1]})(x)\).
For \(|x|\le 1\),
\[
 (P_b*\mathbf 1_{[-1,1]})(x)
 =\frac{1}{\pi}\int_{-1}^{1}\frac{b}{b^2+(x-y)^2}\,dy
 =\frac{1}{2\pi}\Big(\arctan\frac{1-x}{b}+\arctan\frac{1+x}{b}\Big).
\]
This expression is minimized over \(0<b\le 1\), \(|x|\le 1\), at \((x,b)=(1,1)\), yielding \(\frac{1}{2\pi}\arctan 2\).
\end{proof}

\paragraph{From phase--velocity and CR--Green to (P+).}

\label{app:assemble-pplus}

We combine the phase--velocity identity with a Cauchy--Riemann/Green pairing on Whitney boxes to obtain the windowed phase control that underlies (P+).

\paragraph{Carleson energy bound for the logarithmic derivative.}

\label{appA:carleson}

We prove the weighted $L^2$ (Carleson-energy) estimate for the relevant logarithmic derivative on Whitney boxes, including the neutralization of near-field zeros.
\begin{lemma}[Analytic (\(\xi\)) Carleson energy on Whitney boxes]\label{lem:carleson-xi}
There exist absolute constants \(c\in(0,1]\) and \(C_\xi<\infty\) such that for every interval \(I=[T-L,\,T+L]\) at Whitney scale \(L=c/\log\angles{T}\), the Poisson extension
\[
 U_{\xi}(\sigma,t):=\Re\log\xi(\tfrac12+\sigma+it)\qquad(\sigma>0)
\]
obeys the Carleson bound
\[
  \iint_{Q(I)} |\nabla U_{\xi}(\sigma,t)|^2\,\sigma\,dt\,d\sigma\ \le\ C_\xi\,|I|.
\]
\end{lemma}
\begin{proof}
Fix \(I=[T-L,T+L]\) with \(L=c/\log\angles{T}\) and a fixed aperture \(\alpha\in[1,2]\).
Neutralize near zeros by a local half-plane Blaschke product \(B_I\) removing zeros of \(\xi\) inside a fixed dilate \(Q(\alpha'I)\) (\(\alpha'>\alpha\)).
This yields a harmonic field \(\widetilde U_\xi\) on \(Q(\alpha I)\) and
\[
  \iint_{Q(\alpha I)} |\nabla U_\xi|^2\,\sigma\,dt\,d\sigma
  \ \asymp\
  \iint_{Q(\alpha I)} |\nabla \widetilde U_\xi|^2\,\sigma\,dt\,d\sigma\ +\ O_\alpha(|I|),
\]
so it suffices to bound the neutralized energy.

Choose a branch of \(\log\xi\) on each component of \(Q(\alpha I)\setminus Z(\xi)\) so that \(U_\xi=\Re\log\xi\) is harmonic there. Then
\[
  |\nabla U_\xi(\sigma,t)|^2\;=\;|(\log\xi)'(\tfrac12+\sigma+it)|^2\;=\;\Big|\frac{\xi'}{\xi}(\tfrac12+\sigma+it)\Big|^2.
\]
By the Hadamard product for \(\xi\),
\(\xi'/\xi(s)=A(s)+\sum_\rho (s-\rho)^{-1}\), where \(A\) is holomorphic and slowly varying on compact strips. Thus it suffices to bound the weighted \(L^2\)-norm of \(\xi'/\xi\) on \(Q(\alpha I)\); the contribution of \(A\) is \(O_\alpha(|I|)\), and the remaining term is handled annularly by summing the zero contributions.
Decompose the (neutralized) zeros into Whitney annuli
\(
\mathcal A_k:=\{\rho:2^kL<|\gamma-T|\le 2^{k+1}L\}
\), \(k\ge 1\).
For \(k\ge 1\) and \(t\in I\), any \(\rho=eta+i\gamma\in\mathcal A_k\) satisfies
\(|t-\gamma|\ge 2^kL-L\ge 2^{k-1}L\).
Since \(|s-\rho|^2=(t-\gamma)^2+(\tfrac12+\sigma-eta)^2\ge (t-\gamma)^2\), we have the pointwise bound
\(
  |(s-\rho)^{-1}|\le |t-\gamma|^{-1}\le (2^{k-1}L)^{-1}
\)
for all \(s=\tfrac12+\sigma+it\in Q_\alpha(I)\).
Therefore, writing \(S_k(s):=\sum_{\rho\in\mathcal A_k}(s-\rho)^{-1}\),
\[
  |S_k(s)|^2\ \le\ \nu_k\sum_{\rho\in\mathcal A_k}|s-\rho|^{-2}
  \ \le\ \nu_k\cdot \nu_k\,(2^{k-1}L)^{-2}
  \ =\ \nu_k^2\,(2^{k-1}L)^{-2}.
\]
Integrating this uniform bound over \(Q_\alpha(I)\) with the area weight \(\sigma\,dt\,d\sigma\) gives
\[
  \iint_{Q_\alpha(I)} |S_k(s)|^2\,\sigma\,dt\,d\sigma
  \ \le\ \nu_k^2\,(2^{k-1}L)^{-2}\cdot |I|\cdot \int_0^{\alpha L}\!\sigma\,d\sigma
  \ \ll_\alpha\ |I|\,L^2\,\frac{\nu_k^2}{4^k\,L^2}
  \ =\ \ll_\alpha\ |I|\,4^{-k}\,\nu_k^2.
\]
Now sum over annuli using Cauchy--Schwarz in \(k\):
\(
ig|\sum_{\rho\notin Q(\alpha'I)}(s-\rho)^{-1}ig|^2\le (\sum_k |S_k(s)|)^2\le (\sum_k 2^{-k})\,(\sum_k 2^{k}|S_k(s)|^2)
\), hence after integrating,
\[
  \iint_{Q_\alpha(I)} \Big|\sum_{\rho\notin Q(\alpha'I)}(s-\rho)^{-1}\Big|^2\,\sigma\,dt\,d\sigma
  \ \ll\ \sum_{k\ge1} 2^{k}\,\iint_{Q_\alpha(I)} |S_k(s)|^2\,\sigma\,dt\,d\sigma
  \ \ll_\alpha\ |I|\sum_{k\ge1} 2^{-k}\,\nu_k^2.
\]
To bound \(\nu_k\), use the short-interval zero-count bound above to obtain, for some absolute \(a_1(\alpha),a_2(\alpha)\),
\[
  \nu_k\ \le\ a_1(\alpha)\,2^k L\,\log\angles{T}\ +\ a_2(\alpha)\,\log\angles{T}.
\]
Therefore, using \(\nu_k\ll_\alpha 2^k L\log\angles{T}+\log\angles{T}\), we obtain
\[
  \sum_{k\ge1}2^{-k}\,\nu_k^2\ \ll\ \sum_{k\ge1}2^{-k}ig(4^kL^2\log^2\angles{T}+\log^2\angles{T}ig)
  \ \ll\ L^2\log^2\angles{T}+\log^2\angles{T}.
\]
On Whitney scale \(L=c/\log\angles{T}\), this is \(\ll 1\).
Adding the neutralized near-field \(O(|I|)\) and the smooth \(A\) contribution, we obtain
\[
  \iint_{Q(\alpha I)} |\nabla U_\xi|^2\,\sigma\,dt\,d\sigma\ \le\ C_\xi\,|I|,
\]
with \(C_\xi\) depending only on \((\alpha,c)\).
\end{proof}

\subsection{From windowed phase control to the wedge}

\label{appA:wedge}
We convert the established windowed phase bounds into an almost-everywhere wedge inclusion for the boundary values of $\mathcal J_{\rm out}$.
\begin{definition}[Admissible window class with atom avoidance]\label{def:adm-bumps}
Fix an even \(C^\infty\) window \(\psi\) with \(\psi\equiv1\) on \([-1,1]\) and \(\operatorname{supp}\psi\subset[-2,2]\).
For an interval \(I=[t_0-L,t_0+L]\), an aperture \(\alpha'>1\), and a parameter \(\varepsilon\in(0,\tfrac14]\), define \(\mathcal W_{\rm adm}(I;\varepsilon)\) to be the set of \(C^\infty\), nonnegative, mass-\(1\) bumps \(\phi\) supported in the fixed dilate \(2I=[t_0-2L,t_0+2L]\) that can be written as
\[
  \phi(t)\ =\ \frac{1}{Z}\,\frac{1}{L}\,\psi\!\left(\frac{t-t_0}{L}\right)\,m(t),
  \qquad Z=\int_{2I} \frac1L\psi\!\left(\frac{t-t_0}{L}\right)m(t)\,dt,
\]
where \(2I:=[t_0-2L,t_0+2L]\) and the mask \(m\in C^\infty(2I;[0,1])\) satisfies:
\begin{itemize}
\item[(i)] \emph{Atom avoidance.} There is a union of disjoint open subintervals \(E=igcup_{j=1}^{J} J_j\subset I\) with total length \(|E|\le \varepsilon L\) such that \(m\equiv0\) on \(E\) and \(m\equiv1\) on \(I\setminus E'\), where each transition layer \(E'\setminus E\) has thickness \(\le \varepsilon L\).
\item[(ii)] \emph{Uniform smoothness.} \(\|m'\|_\infty\lesssim (\varepsilon L)^{-1}\) and \(\|m''\|_\infty\lesssim (\varepsilon L)^{-2}\) with implicit constants independent of \(I,t_0,L\) and of the number/placement of the holes \(\{J_j\}\).
\end{itemize}
Every \(\phi\in\mathcal W_{\rm adm}(I;\varepsilon)\) is supported in \(2I\).
This class contains the unmasked profile \(\varphi_{L,t_0}(t)=Z_0^{-1}L^{-1}\psi((t-t_0)/L)\) with \(Z_0:=\int_{-2}^{2}\psi(x)\,dx\) (take \(E=\varnothing\), \(m\equiv1\)) and also allows dodging boundary atoms by punching out small neighborhoods while keeping total deleted length \(\le\varepsilon L\).
\end{definition}
\begin{lemma}[Uniform Poisson--energy bound for admissible tests]\label{lem:uniform-test-energy}
Let \(V_\phi\) be the Poisson extension of \(\phi\in\mathcal W_{\rm adm}(I;\varepsilon)\) to the half‑plane, and fix a cutoff to \(Q(\alpha' I)\) with \(\alpha'>1\) as in the CR--Green pairing.
Then there exists a finite constant \(\mathcal A_{\rm adm}(\psi,\varepsilon,\alpha')<\infty\), depending only on \((\psi,\varepsilon,\alpha')\), such that
\[
  \iint_{Q(\alpha' I)} |\nabla V_\phi(\sigma,t)|^2\,\sigma\,dt\,d\sigma\ \le\ \mathcal A_{\rm adm}(\psi,\varepsilon,\alpha')^2\; L.
\]
\end{lemma}
\begin{proof}
Let \(\phi(t)=Z^{-1}L^{-1}\psi((t-t_0)/L)m(t)\) be an admissible test.
By scaling of the Poisson kernel and the uniform bounds on \(m,m',m''\) from Definition~\ref{def:adm-bumps}, the \(H^1\)-size of \(\phi\) (equivalently the \(L^2(\sigma)\) Dirichlet energy of its Poisson extension on a fixed aperture box) is controlled uniformly by a constant depending only on \((\psi,\varepsilon,\alpha')\), times \(L^{1/2}\).
Squaring yields the stated \(\lesssim L\) energy bound with \(\mathcal A_{\rm adm}(\psi,\varepsilon,\alpha')\).
\end{proof}
\begin{lemma}[Cutoff pairing on boxes]\label{lem:cutoff-pairing}
Fix parameters \(\alpha'>\alpha>1\).
Let \(\chi_{L,t_0}\in C_c^\infty(\R^2_+)\) satisfy \(\chi\equiv1\) on \(Q(\alpha I)\), \(\operatorname{supp}\chi\subset Q(\alpha'I)\), \(\|\nabla\chi\|_\infty\lesssim L^{-1}\) and \(\|\nabla^2\chi\|_\infty\lesssim L^{-2}\).
Let \(V_\phi\) be the Poisson extension of \(\phi\in \mathcal W_{\rm adm}(I;\varepsilon)\).
Then one has the Green pairing identity
\[
 \int_{\R} u(t)\,\phi(t)\,dt
 \ =\ \iint_{Q(\alpha'I)} \nabla U\cdot \nabla(\chi_{L,t_0}\, V_\phi)\,dt\,d\sigma\ +\ \mathcal R_{\mathrm{side}}\ +\ \mathcal R_{\mathrm{top}},
\]
with remainders satisfying
\[
 |\mathcal R_{\mathrm{side}}|+|\mathcal R_{\mathrm{top}}|
 \ \lesssim\ \Big(\iint_{Q(\alpha'I)} |\nabla U|^2\,\sigma\Big)^{1/2}
               \cdot \Big(\iint_{Q(\alpha'I)} ig(|\nabla\chi|^2\,|V_\phi|^2+|\nabla V_\phi|^2ig)\,\sigma\Big)^{1/2}.
\]
\end{lemma}
\begin{proof}
Let \(Q:=Q(\alpha'I)\).
Assume \(U\) is \(C^2\) on \(\overline Q\) and harmonic on \(Q\), with boundary trace \(u(t)=U(0,t)\) on the bottom edge \(\{\sigma=0\}\).
Since \(\chi_{L,t_0}V_\phi\) is compactly supported in \(\overline Q\) and smooth on \(Q\), Green's identity gives
\[
  \iint_{Q} \nabla U\cdot \nabla(\chi V_\phi)\,dt\,d\sigma
  \,=\,
  \int_{\partial Q} (\chi V_\phi)\,\partial_n U\,ds
  \ -\ \iint_{Q} (\chi V_\phi)\,\Delta U\,dt\,d\sigma.
\]
Since \(\Delta U=0\) on \(Q\), only the boundary integral remains.
On the bottom edge one has \(\partial_n=-\partial_\sigma\), \(\chi\equiv1\), and \(V_\phi(0,t)=\phi(t)\), hence that contribution equals
\[
  \int_{I} \phi(t)\,(-\partial_\sigma U)(0,t)\,dt.
\]
\subsection{Whitney box pairing and local oscillation functional}

\label{appA:whitney}

We organize the remaining bookkeeping: pairing Whitney boxes across scales and bounding the oscillation functional needed to pass from local control to the global wedge statement.

If \(U\) is the real part of a holomorphic logarithm \(U=\Re\log J\) with \(|J(\tfrac12+it)|=1\) a.e., then \(U(0,t)=0\) a.e.\ and \(-\partial_\sigma U(0,t)=\partial_t \Arg J(\tfrac12+it)\) in distributions by Cauchy--Riemann; in particular, this term is the tested boundary phase derivative in Lemma~\ref{lem:CR-green-phase} below.
The remaining boundary pieces (two vertical sides and the top edge) are, by definition, the remainders \(\mathcal R_{\mathrm{side}}+\mathcal R_{\mathrm{top}}\).

For the remainder estimate, we apply Cauchy--Schwarz in the scale-invariant measure \(\sigma\,dt\,d\sigma\) on \(Q\):
\[
  ig|\mathcal R_{\mathrm{side}}ig|+ig|\mathcal R_{\mathrm{top}}ig|
  \ \lesssim\ \Big(\iint_Q |\nabla U|^2\,\sigma\Big)^{1/2}
               \Big(\iint_Q ig|\nabla(\chi V_\phi)|^2\,\sigma\Big)^{1/2}.
\]
Expanding \(\nabla(\chi V_\phi)=\chi\,\nabla V_\phi + (\nabla\chi)\,V_\phi\) yields
\[
  \iint_Q ig|\nabla(\chi V_\phi)|^2\,\sigma
  \ \lesssim\ \iint_Q ig(|\nabla V_\phi|^2 + |\nabla\chi|^2|V_\phi|^2ig)\,\sigma,
\]
which gives the displayed estimate.
\end{proof}
\begin{lemma}[CR--Green pairing for boundary phase]\label{lem:CR-green-phase}
Let \(J\) be analytic on \(\Omega\) with a.e.\ boundary modulus \(|J(\tfrac12+it)|=1\), and write \(\log J=U+iW\) on \(\Omega\), so \(U\) is harmonic with \(U(\tfrac12+it)=0\) a.e.
Fix a Whitney interval \(I=[t_0-L,t_0+L]\) and let \(V_\phi\) be the Poisson extension of \(\phi\in\mathcal W_{\rm adm}(I;\varepsilon)\).
Then, with a cutoff \(\chi_{L,t_0}\) as in Lemma~\ref{lem:cutoff-pairing},
\[
  \int_{\R} \phi(t)\,ig(-W'(t))\,dt
  \ =\ \iint_{Q(\alpha'I)} \nabla U\cdot \nabla(\chi_{L,t_0}\,V_\phi)\,dt\,d\sigma\ +\ \mathcal R_{\mathrm{side}}\ +\ \mathcal R_{\mathrm{top}},
\]
and the remainders satisfy the same estimate as in Lemma~\ref{lem:cutoff-pairing}.
In particular, by Cauchy--Schwarz and Lemma~\ref{lem:uniform-test-energy}, there is a constant \(C_{\rm rem}(\alpha',\psi)\) such that
\[
  \int_{\R} \phi(t)\,ig(-w'(t))\,dt\ \le\ C_{\rm rem}(\alpha',\psi)\,\Big(\iint_{Q(\alpha'I)} |\nabla U|^2\,\sigma\Big)^{1/2}.
\]
\end{lemma}
\begin{proof}
On the bottom edge \(\{\sigma=0\}\) the outward normal is \(\partial_n=-\partial_\sigma\).
By Cauchy--Riemann for \(\log J=U+iW\) on the boundary line \(\{\Re s=\tfrac12\}\) one has \(\partial_n U=-\partial_\sigma U=\partial_t W\).
Thus the bottom-edge term in Green's identity is
\[
  -\int_{\partial Q\cap\{\sigma=0\}} \chi\,V_\phi\,\partial_n U\,dt
  = -\int_{\R} \phi(t)\,\partial_t W(t)\,dt
  = \int_{\R} \phi(t)\,ig(-w'(t))\,dt,
\]
which yields the stated identity after including the interior term and remainders.
The final inequality is Cauchy--Schwarz together with the uniform Poisson-energy bound from Lemma~\ref{lem:uniform-test-energy}.
\end{proof}
\begin{proposition}[Length‑independent upper bound for admissible tests]\label{prop:length-free}
Let \(J\) be holomorphic on \(\Omega\setminus Z(\zeta)\) with a.e.\ boundary modulus \(1\), write \(\log J=U+iW\) on \(\Omega\setminus Z(\zeta)\), and let \(-w'\) denote the boundary phase distribution.
For every interval \(I=[t_0-L,t_0+L]\), every \(\phi\in\mathcal W_{\rm adm}(I;\varepsilon)\), and every fixed cutoff to \(Q(\alpha' I)\),
\begin{equation}\label{eq:CRG-upper-adm}
\int_{\mathbb R}\!\phi(t)\,(-w')(t)\,dt\ \le\ C_{\rm test}(\psi,\varepsilon,\alpha')\,\Big(\iint_{Q(\alpha' I)}|\nabla U|^2\,\sigma\,dt\,d\sigma\Big)^{1/2}
\end{equation}
with \(C_{\rm test}(\psi,\varepsilon,\alpha'):=C_{\rm rem}(\alpha',\psi)\,\mathcal A_{\rm adm}(\psi,\varepsilon,\alpha')\) independent of \(I,t_0,L\).
In particular, defining the box-energy constant
\[
  C_{\rm box}^{(\zeta)}\ :=\ \sup_{I}\ \frac{1}{|I|}\iint_{Q(\alpha' I)}|\nabla U|^2\,\sigma\,dt\,d\sigma,
\]
one has the scale bound
\[
  \int_{\mathbb R}\!\phi\,(-w')\ \le\ C_{\rm test}(\psi,\varepsilon,\alpha')\,\sqrt{C_{\rm box}^{(\zeta)}}\,L^{1/2}.
\]
\end{proposition}
\begin{proof}
Apply Lemma~\ref{lem:CR-green-phase} with \(\phi\in\mathcal W_{\rm adm}(I;\varepsilon)\) and absorb the window-side constants into \(C_{\rm test}(\psi,\varepsilon,\alpha')\).
\end{proof}
\begin{lemma}[Whitney--uniform wedge]\label{lem:whitney-uniform-wedge}\label{lem:local-to-global-wedge}
Fix parameters \(\alpha'>1\) and \(\varepsilon\in(0,\tfrac14]\).
Fix the Whitney schedule and clip by \(L_\star\): set \(L_\star:=c/\log 2\) and henceforth
\[
  L(T)\ :=\ \min\Big\{\frac{c}{\log\angles{T}},\ L_\star\Big\}.
\]
Then for every Whitney interval \(I=[t_0-L,t_0+L]\) and the corresponding cutoff
\(\psi_{L,t_0}(t):=\psi((t-t_0)/L)=Z_0L\,\varphi_{L,t_0}(t)\) (so \(\psi_{L,t_0}\equiv 1\) on \(I\)),
\[
  \int_{\mathbb R} \psi_{L,t_0}(t)\,(-w'(t))\,dt\ \le\ Z_0\,L_\star\cdot C_{\rm test}(\psi,\varepsilon,\alpha')\,\sqrt{C_{\rm box}^{(\zeta)}}\,L_\star^{1/2}
  \ :=\ \pi\,\Upsilon_{\rm Whit}(c).
\]
Choosing \(c>0\) sufficiently small so that \(\Upsilon_{\rm Whit}(c)<\tfrac12\) yields the hypothesis of Lemma~\ref{lem:local-to-global-wedge} and hence \textup{(P+)}.
\end{lemma}
\begin{proof}
Since \(\psi_{L,t_0}=Z_0L\,\varphi_{L,t_0}\), apply Proposition~\ref{prop:length-free} with \(\phi=\varphi_{L,t_0}\), then multiply the resulting bound by \(Z_0L\) and use the Whitney clip \(L\le L_\star\).
\end{proof}
\begin{theorem}[Proof of Theorem~\ref{thm:Pplus}]\label{thm:pplus-proof-complete}
The boundary wedge \textup{(P+)} holds for \(\mathcal J_{\rm out}\).
\end{theorem}
\begin{proof}
By the definition \eqref{eq:J-out} and Theorem~\ref{thm:phase-velocity-quant}, the quantitative phase--velocity identity (Theorem~\ref{thm:phase-velocity-quant}) applies to the \(\zeta\)-normalized unimodular ratio \(J_\zeta\), and hence (by \eqref{eq:J-out}) to \(\mathcal J_{\rm out}\).
In particular, the associated boundary phase distribution \(-w'\) is positive.

Proposition~\ref{prop:length-free} (CR--Green pairing) supplies a uniform Whitney-scale bound for the windowed phase derivative in terms of the box energy \(C_{\rm box}^{(\zeta)}\).
Applying the Whitney schedule and choosing \(c>0\) small enough gives \(\Upsilon_{\rm Whit}(c)<\tfrac12\) in Lemma~\ref{lem:whitney-uniform-wedge}.
Lemma~\ref{lem:local-to-global-wedge} then yields \textup{(P+)}.
\end{proof}
% ===== END inlined from paper1_pplus_proof.tex =====

\section{Supplementary computational verification protocol (verifier commands and expected fields)}\label{app:verify}

There are two verification modes:
\begin{itemize}
\item \textbf{Verification using shipped artifacts:} verify the shipped JSON artifacts match Table~\ref{tab:artifact-data}.
\item \textbf{Optional regeneration check (supplementary):} rerun the verifier to regenerate the artifacts from scratch.
\end{itemize}

\subsection*{Prerequisites}
From the bundle's \texttt{compute/} directory, install the ARB/ball-arithmetic bindings:
\begin{verbatim}
pip install -r requirements.txt
\end{verbatim}

\subsection*{Verification using shipped artifacts: check shipped JSON artifacts}
\begin{itemize}
\item \textbf{Rectangle artifact} \url{artifacts/theta_certify_sigma06_07_t0_20_outer_zeta_proj.json}. Check (at minimum):
\begin{itemize}
  \item \texttt{results.ok = true}
  \item \texttt{results.theta\_hi = 0.9999928763... < 1}
  \item \texttt{results.processed\_boxes = 380764}
  \end{itemize}
\item \textbf{Pick artifact} \url{artifacts/pick_sigma0599_raw_zeta_N16.json}. Check (at minimum):
\begin{itemize}
  \item \texttt{pick.delta\_cert = 0.594...}
  \item \texttt{pick.P\_spd\_at\_0 = true}
  \item \texttt{pick.tail\_l1\_partial\_hi} (diagnostic L1 tail sum)
  \end{itemize}
\item \textbf{Pick artifact} \url{artifacts/pick_sigma06_raw_zeta_N16.json}. Check (at minimum):
\begin{itemize}
  \item \texttt{pick.delta\_cert = 0.594...}
  \item \texttt{pick.P\_spd\_at\_0 = true}
  \item \texttt{pick.tail\_l1\_partial\_hi} (diagnostic L1 tail sum)
  \end{itemize}
\item \textbf{Pick artifact} \url{artifacts/pick_sigma07_raw_zeta_N16.json}. Check (at minimum):
\begin{itemize}
  \item \texttt{pick.delta\_cert = 0.627...}
  \item \texttt{pick.P\_spd\_at\_0 = true}
  \item \texttt{pick.tail\_l1\_partial\_hi} (diagnostic L1 tail sum)
  \end{itemize}
\end{itemize}

\subsection*{Optional regeneration check (supplementary): exact command lines}
Run the verifier from the bundle's \texttt{compute/} directory (or the corresponding directory in a repository checkout).
The following commands reproduce the primary artifacts (line breaks are for readability):

\paragraph{1) Rectangle certification (\texttt{theta\_certify}).}
\begin{verbatim}
python verify_attachment_arb.py \
  --theta-certify \
  --arith-gauge outer_zeta_proj \
  --arith-P-cut 2000 \
  --rect-sigma-min 0.6 --rect-sigma-max 0.7 \
  --rect-t-min 0.0 --rect-t-max 20.0 \
  --outer-mode midpoint \
  --outer-P-cut 2000 \
  --outer-T 50.0 --outer-n 2001 \
  --theta-init-n-sigma 10 --theta-init-n-t 50 \
  --theta-min-sigma-width 0.0001 --theta-min-t-width 0.001 \
  --theta-max-boxes 500000 \
  --prec 260 \
  --theta-out artifacts/theta_certify_sigma06_07_t0_20_outer_zeta_proj.json \
  --progress
\end{verbatim}

\paragraph{2) Pick certification at $\sigma_0=0.599$ (\texttt{pick\_certify}).}
\begin{verbatim}
python verify_attachment_arb.py \
  --pick-certify \
  --pick-sigma0 0.599 \
  --pick-N 16 \
  --pick-coeff-count 128 \
  --pick-K 512 \
  --pick-rho 0.4 \
  --pick-rho-bound 0.5 \
  --arith-gauge raw_zeta \
  --arith-P-cut 2000 \
  --prec 1024 \
  --pick-out artifacts/pick_sigma0599_raw_zeta_N16.json
\end{verbatim}

\paragraph{3) Pick certification at $\sigma_0=0.6$ (\texttt{pick\_certify}).}
\begin{verbatim}
python verify_attachment_arb.py \
  --pick-certify \
  --pick-sigma0 0.6 \
  --pick-N 16 \
  --pick-coeff-count 128 \
  --pick-K 512 \
  --pick-rho 0.4 \
  --pick-rho-bound 0.5 \
  --arith-gauge raw_zeta \
  --arith-P-cut 2000 \
  --prec 1024 \
  --pick-out artifacts/pick_sigma06_raw_zeta_N16.json
\end{verbatim}

\paragraph{4) Pick certification at $\sigma_0=0.7$ (\texttt{pick\_certify}).}
\begin{verbatim}
python verify_attachment_arb.py \
  --pick-certify \
  --pick-sigma0 0.7 \
  --pick-N 16 \
  --pick-coeff-count 128 \
  --pick-K 512 \
  --pick-rho 0.4 \
  --pick-rho-bound 0.5 \
  --arith-gauge raw_zeta \
  --arith-P-cut 2000 \
  --outer-mode rigorous \
  --outer-P-cut 2000 \
  --prec 1024 \
  --pick-out artifacts/pick_sigma07_raw_zeta_N16.json
\end{verbatim}

\subsection*{Meaning of successful verification}
The verifier uses \emph{ball arithmetic}: each computed quantity is an interval enclosure (midpoint plus radius) and every operation propagates rounding error outward.
Thus each check is a formal inequality of the form \texttt{upper bound < 1} or \texttt{directed-rounding }LDL$^\top$\texttt{ succeeds with positive pivots}.
If the verification checks above pass, then the numerical inequalities summarized in Table~\ref{tab:artifact-data} are  within the logic of ball arithmetic.

% ===== BEGIN inlined from riemann_bibliography.tex =====
% Shared bibliography include for the three-paper split.
% Keep this file as a plain thebibliography environment to avoid toolchain friction.
\begin{thebibliography}{99}

\bibitem{IK}
H. Iwaniec and E. Kowalski,
\emph{Analytic Number Theory},
AMS Colloquium Publications, 2004.

\bibitem{MV}
H. L. Montgomery and R. C. Vaughan,
\emph{Multiplicative Number Theory I: Classical Theory},
Cambridge University Press, 2007.

\bibitem{Titchmarsh}
E. C. Titchmarsh,
\emph{The Theory of the Riemann Zeta-Function},
2nd ed., Oxford University Press, 1986.

\bibitem{RosenblumRovnyak}
M. Rosenblum and J. Rovnyak,
\emph{Hardy Classes and Operator Theory},
Oxford University Press, 1985.

\bibitem{Donoghue}
W. F. Donoghue,
\emph{Monotone Matrix Functions and Analytic Continuation},
Springer, 1974.

\bibitem{SimonTrace}
B. Simon,
\emph{Trace Ideals and Their Applications},
2nd ed., Mathematical Surveys and Monographs, vol.~120, American Mathematical Society, 2005.

\bibitem{Ahlfors}
L. V. Ahlfors,
\emph{Complex Analysis},
3rd ed., McGraw--Hill, 1979.

\bibitem{RansfordPT}
T. Ransford,
\emph{Potential Theory in the Complex Plane},
London Mathematical Society Student Texts, vol.~28, Cambridge University Press, 1995.

\bibitem{DurenHp}
P. L. Duren,
\emph{Theory of $H^p$ Spaces},
Academic Press, 1970.

\bibitem{SteinHA}
E. M. Stein,
\emph{Harmonic Analysis: Real-Variable Methods, Orthogonality, and Oscillatory Integrals},
Princeton University Press, 1993.

\bibitem{GarnettBAF}
J. B. Garnett,
\emph{Bounded Analytic Functions},
Graduate Texts in Mathematics, vol.~236, Springer, 2007.

\end{thebibliography}
% ===== END inlined from riemann_bibliography.tex =====

\end{document}