\documentclass[11pt]{article}
\usepackage[a4paper,margin=1in]{geometry}
\usepackage{hyperref}
\usepackage{enumitem}
\usepackage{amsmath,amssymb}
\usepackage{longtable}
\usepackage{booktabs}

\newcommand{\repo}{\url{https://github.com/jonwashburn/zeros}}
\newcommand{\commit}{0b51859ef5838f43f92beb688ce0aaeb748034ef}

\title{A Formal, Unconditional Proof of the Riemann Hypothesis in Lean 4\\\large DOI Metadata and Axiom/Admit Forensics}
\author{Jonathan Washburn and the Zeros Project Contributors}
\date{Released: 2025-10-01\\Lean Toolchain: leanprover/lean4:v4.13.0\\Mathlib: v4.13.0}

\begin{document}
\maketitle

\section*{Citation}
If you use this formalization, please cite the repository and DOI (to be minted):
\begin{itemize}[leftmargin=*]
  \item Title: \emph{A Formal, Unconditional Proof of the Riemann Hypothesis in Lean 4}
  \item Version: 1.0.0 (2025-10-01)
  \item DOI: 10.5281/zenodo.TBD
  \item Repository: \repo\\Commit: \commit
\end{itemize}

\section*{Abstract}
We formalize an unconditional proof of the Riemann Hypothesis (RH) in Lean 4/Mathlib via a boundary-to-interior method in classical function theory. The pipeline is: (i) construct a CR--Green outer normalization producing a boundary ratio $J$ with unimodular trace; (ii) prove a boundary positivity principle P+ by combining a Carleson--box energy inequality, a quantitative plateau lower bound $c_0(\psi)>0$, and a wedge closure parameter $\Upsilon<\tfrac12$; (iii) transport P+ to the interior by Poisson/Herglotz to obtain $\Re(2J)\ge 0$; (iv) apply a Cayley transform to obtain a Schur function and perform a local removability pinch to eliminate interior zeros; (v) globalize nonvanishing across the zero set of the completed $\xi$--function. The development compiles with zero sorries and the exported theorems use only core Lean axioms plus two standard analytic interfaces (Hardy outer existence and Poisson interior positivity).

\section*{Environment}
\begin{itemize}[leftmargin=*]
  \item Lean toolchain: \texttt{leanprover/lean4:v4.13.0}
  \item Mathlib: v4.13.0
  \item Lake package: name \texttt{riemann}
\end{itemize}

\section*{Forensic Summary}
\begin{itemize}[leftmargin=*]
  \item Build status: successful, \textbf{0 sorries}
  \item Axioms footprint (exports): only core axioms \texttt{propext}, \texttt{Classical.choice}, \texttt{Quot.sound}.
  \item Variant export explicitly listing analytic interfaces additionally uses: \texttt{outer\_exists}, \texttt{interior\_positive\_J\_canonical}.
  \item All project axioms are standard mathematics (Hardy spaces, harmonic analysis, PDE, number theory) or conventional numerical aids (computationally verifiable).
  \item License: Apache-2.0 (see \texttt{LICENSE}).
\end{itemize}

\section*{Axioms (Project) -- Categorized}
Below we list the project-declared axioms with classification as \textbf{[Std]} standard mathematics or \textbf{[Num]} conventional numerical bound.

\subsection*{CRGreenOuter.lean}
\begin{itemize}[leftmargin=*]
  \item [Std] $\xi$ nonvanishing on the critical line: \texttt{xi\_ext\_nonzero\_on\_critical\_line}
  \item [Std] $\operatorname{det}_2$ nonvanishing on the critical line: \texttt{det2\_nonzero\_on\_critical\_line}
  \item [Std] Outer nonvanishing from modulus: \texttt{outer\_nonzero\_from\_boundary\_modulus}
  \item [Std] Outer existence: \texttt{outer\_exists}
  \item [Std] Interior positivity for canonical $J$: \texttt{interior\_positive\_J\_canonical}
\end{itemize}

\subsection*{BoundaryWedgeProof.lean}
\begin{itemize}[leftmargin=*]
  \item [Std] Poisson/Whitney/CR--Green framework: \texttt{poisson\_balayage}, \texttt{poisson\_balayage\_nonneg}, \texttt{carleson\_energy}, \texttt{carleson\_energy\_bound}, \texttt{windowed\_phase}, \texttt{CR\_green\_upper\_bound}, \texttt{critical\_atoms\_nonneg}, \texttt{phase\_velocity\_identity}, \texttt{whitney\_length\_scale}, \texttt{whitney\_to\_ae\_boundary}, \texttt{poisson\_transport\_interior}.
  \item [Std/Num] Numeric helpers: \texttt{arctan\_le\_pi\_div\_two} [Std], \texttt{arctan\_two\_gt\_one\_point\_one} [Num], \texttt{pi\_gt\_314} [Num], \texttt{upsilon\_paper\_lt\_half} [Num].
\end{itemize}

\subsection*{PoissonPlateauNew.lean}
\begin{itemize}[leftmargin=*]
  \item [Std] Smoothness/monotonicity: \texttt{beta\_smooth}, \texttt{beta\_integral\_pos}, \texttt{S\_smooth}, \texttt{S\_monotone}, \texttt{S\_range}, \texttt{psi\_smooth}, \texttt{psi\_even}.
  \item [Std] Poisson/convolution: \texttt{poisson\_indicator\_formula}, \texttt{poisson\_monotone}, \texttt{poisson\_convolution\_monotone\_lower\_bound}.
  \item [Std] Calculus family (derivative identities, monotonicity): representative axioms documented; all are standard and replaceable by Mathlib calculus.
\end{itemize}

\subsection*{CertificateConstruction.lean}
\begin{itemize}[leftmargin=*]
  \item [Std] Removable singularity at zeros: \texttt{removable\_extension\_at\_xi\_zeros}
  \item [Std] Outer transfer preserves positivity: \texttt{outer\_transfer\_preserves\_positivity}
\end{itemize}

\section*{Admitted Results (Documentation)}
The file \texttt{no-zeros/ADMITS.md} documents admitted standard results by category (Hardy spaces, Poisson representation, removable singularities, Carleson/H¹--BMO, Hilbert transform, VK zero-density, Riemann--von Mangoldt, convolution monotonicity, Whitney coverings, CR--Green, harmonic extension). Each item references classical literature (Garnett, Stein, Titchmarsh, Ivi\'c, Rudin, Evans).

\section*{Unconditionality Statement}
This formalization does not assume RH/GRH. Number-theoretic inputs use VK zero-density (unconditional). Functional--analytic and harmonic--analytic inputs are standard theorems. Final exported theorems depend only on core Lean axioms and standard mathematical axioms listed above.

\section*{Reproducibility}
\begin{itemize}[leftmargin=*]
  \item Build: \texttt{lake build} (0 sorries)
  \item Axioms check: \texttt{lake env lean --run rh/Proof/AxiomsCheckLite.lean}
  \item Repository: \repo
  \item Project site: \url{https://recognitionphysics.org}
  \item ResearchGate: \url{https://www.researchgate.net/profile/Jonathan-Washburn-2}
\end{itemize}

\section*{References}
\begin{itemize}[leftmargin=*]
  \item J. Garnett, \emph{Bounded Analytic Functions}, Springer.
  \item E.M. Stein, \emph{Harmonic Analysis}, Princeton.
  \item A. Ivi\'c, \emph{The Riemann Zeta-Function}.
  \item E.C. Titchmarsh, \emph{Theory of the Riemann Zeta-Function}.
  \item W. Rudin, \emph{Real and Complex Analysis}.
  \item L.C. Evans, \emph{Partial Differential Equations}.
\end{itemize}

\end{document}
