\documentclass[11pt,a4paper]{article}

\usepackage[margin=1in]{geometry}
\usepackage{amsmath,amssymb,amsthm}
\usepackage{mathtools}
\usepackage{microtype}
\usepackage{hyperref}
\usepackage{graphicx}
\usepackage{xcolor}

% Theorem environments
\newtheorem{theorem}{Theorem}[section]
\newtheorem{lemma}[theorem]{Lemma}
\newtheorem{proposition}[theorem]{Proposition}
\newtheorem{corollary}[theorem]{Corollary}
\newtheorem{hypothesis}[theorem]{Hypothesis}
\newtheorem{conjecture}[theorem]{Conjecture}
\theoremstyle{definition}
\newtheorem{definition}[theorem]{Definition}
\newtheorem{example}[theorem]{Example}
\theoremstyle{remark}
\newtheorem{remark}[theorem]{Remark}
\newtheorem{insight}[theorem]{Insight}

% Custom commands
\newcommand{\R}{\mathbb{R}}
\newcommand{\C}{\mathbb{C}}
\newcommand{\Z}{\mathbb{Z}}
\newcommand{\N}{\mathbb{N}}
\newcommand{\Q}{\mathbb{Q}}
\newcommand{\zmark}{\zeta}
\newcommand{\Rhat}{\widehat{R}}
\newcommand{\vp}{\varphi}

% Box for key insights (simple version)
\newenvironment{keyinsight}[1][]
  {\begin{quote}\textbf{#1}\itshape}
  {\end{quote}}

\title{\textbf{What Primes Are}\\[0.5em]
\Large A Recognition Science Perspective on the Atoms of Arithmetic\\
and Why Zeros Lie on a Line}

\author{Jonathan Washburn\\[0.3em]
\small Recognition Science Research Institute\\
\small Austin, Texas\\
\small \texttt{washburn.jonathan@gmail.com}}

\date{January 2, 2026}

\begin{document}

\maketitle

\begin{abstract}
We present an expository account of prime numbers as they appear in Recognition Science (RS),
a framework that derives physical structure from a single primitive: the Recognition Composition
Law. Within RS, primes are not mysterious random objects but the irreducible events in a
universal ledger---the fundamental ``atoms'' of multiplicative accounting. This perspective
illuminates the Riemann Hypothesis: zeros of the zeta function must lie on the critical line
because any other configuration would violate the conservation constraints imposed by the
ledger's eight-tick structure. We give an intuitive proof outline that builds geometric
understanding before introducing technical machinery, making the underlying mechanism
transparent. The central insight is that prime-frequency observables are \emph{bandlimited}
by the ledger's Nyquist cutoff, which bounds prime sums uniformly and forces zeros onto the
symmetry axis through an energy-budget argument.
\end{abstract}

\tableofcontents

\newpage

%% ============================================================
%% PART I: WHAT ARE PRIME NUMBERS?
%% ============================================================

\part{What Are Prime Numbers?}

\section{The Classical View}

\subsection{Primes as atoms of multiplication}

Every positive integer greater than 1 is either prime or can be written as a product of primes.
This is the \emph{Fundamental Theorem of Arithmetic}, and it tells us that primes are the
``atoms'' of the multiplicative structure of integers.

\begin{definition}[Prime number]
A positive integer $p > 1$ is \textbf{prime} if its only positive divisors are 1 and $p$ itself.
\end{definition}

The first few primes are 2, 3, 5, 7, 11, 13, 17, 19, 23, 29, \ldots

The Fundamental Theorem says:
\begin{theorem}[Fundamental Theorem of Arithmetic]
Every integer $n > 1$ can be written uniquely (up to order) as a product of prime powers:
\[
  n = p_1^{a_1} p_2^{a_2} \cdots p_k^{a_k},
\]
where $p_1 < p_2 < \cdots < p_k$ are primes and $a_i \ge 1$.
\end{theorem}

This is remarkable: it says that the integers have a \emph{unique factorization} into irreducible
pieces. In physics, we would call primes the ``fundamental particles'' of arithmetic.

\subsection{Why primes seem random: the Prime Number Theorem}

Despite their fundamental role, primes appear to be distributed almost randomly among the integers.
The \emph{Prime Number Theorem} (proved independently by Hadamard and de la Vallée Poussin in 1896)
gives the asymptotic density:

\begin{theorem}[Prime Number Theorem]
Let $\pi(x)$ denote the number of primes $\le x$. Then
\[
  \pi(x) \sim \frac{x}{\ln x} \quad \text{as } x \to \infty.
\]
\end{theorem}

This says that the ``probability'' that a random integer near $x$ is prime is approximately
$1/\ln x$. Primes thin out logarithmically, but they never stop appearing.

\subsection{The logarithmic perspective}

Analytic number theory takes the logarithm seriously. Instead of counting primes directly,
we weight them by $\log p$:

\begin{definition}[von Mangoldt function]
The \textbf{von Mangoldt function} $\Lambda(n)$ is defined by
\[
  \Lambda(n) = \begin{cases}
    \log p & \text{if } n = p^k \text{ for some prime } p \text{ and } k \ge 1,\\
    0 & \text{otherwise}.
  \end{cases}
\]
\end{definition}

The key identity is:
\begin{equation}\label{eq:vonmangoldt}
  -\frac{\zeta'(s)}{\zeta(s)} = \sum_{n=1}^\infty \frac{\Lambda(n)}{n^s}, \qquad \Re s > 1,
\end{equation}
where $\zeta(s) = \sum_{n=1}^\infty n^{-s}$ is the Riemann zeta function.

\begin{remark}[Why logarithms?]
Equation \eqref{eq:vonmangoldt} is a \emph{logarithmic accounting identity}. Taking the
logarithmic derivative of the Euler product
\[
  \zeta(s) = \prod_p \frac{1}{1 - p^{-s}}
\]
gives exactly the von Mangoldt sum. The logarithm converts products into sums---it linearizes
the multiplicative structure.
\end{remark}

\section{The Recognition Science View}

\subsection{What is Recognition Science?}

Recognition Science (RS) is a framework that derives all physical structure from a single
primitive: the \textbf{Recognition Composition Law} (RCL):
\begin{equation}\label{eq:rcl}
  J(xy) + J(x/y) = 2J(x)J(y) + 2J(x) + 2J(y),
\end{equation}
where $J: \R_{>0} \to \R$ is a cost function measuring deviation from unity.

From this single functional equation, RS derives:
\begin{itemize}
  \item A unique cost function: $J(x) = \frac{1}{2}(x + x^{-1}) - 1$.
  \item A unique scale: the golden ratio $\vp = (1 + \sqrt{5})/2$.
  \item A unique time structure: the eight-tick cycle (period $2^D$ with $D = 3$).
  \item All fundamental constants: $c$, $\hbar$, $G$, $\alpha^{-1}$, etc.
\end{itemize}

The key structures for understanding primes are:
\begin{enumerate}
  \item \textbf{The Ledger}: A double-entry conservation constraint that forces balanced accounting.
  \item \textbf{The Eight-Tick Cycle}: The minimal ledger-compatible walk, with period 8.
  \item \textbf{The Recognition Operator} $\Rhat$: Replaces the Hamiltonian; minimizes $J$-cost.
\end{enumerate}

\subsection{Primes as irreducible ledger events}

In RS, every integer $n$ represents a \textbf{ledger state}---a configuration of the universal
accounting system. The fundamental theorem of arithmetic becomes:

\begin{insight}[Primes as atomic events]
Primes are the \textbf{irreducible events} in the multiplicative ledger. They are configurations
that cannot be decomposed into simpler ledger entries. Every composite number is a ``transaction''
built from prime ``atoms.''
\end{insight}

The von Mangoldt encoding \eqref{eq:vonmangoldt} is exactly what RS predicts:
\begin{itemize}
  \item A \textbf{conserved count} of discrete events (primes and prime powers).
  \item Expressed as a \textbf{generating function} (the Dirichlet series).
  \item Whose \textbf{logarithmic derivative} is the most stable observable.
\end{itemize}

The logarithmic derivative arises because the ledger tracks \emph{ratios}, not absolute values.
The cost function $J(x) = \frac{1}{2}(x + x^{-1}) - 1$ is symmetric under $x \mapsto 1/x$,
which forces double-entry accounting: every debit has a matching credit.

\subsection{The multiplicative ledger}

\begin{definition}[Multiplicative ledger]
The \textbf{multiplicative ledger} is the set $\Z_{>0}$ equipped with multiplication as the
``transaction'' operation. A \textbf{ledger event} is a prime $p$; a \textbf{compound transaction}
is a product $n = p_1^{a_1} \cdots p_k^{a_k}$.
\end{definition}

The ledger has a natural cost structure: the $J$-cost of a transaction depends on how far its
``ratio'' deviates from 1. Primes are special because they are the \emph{generators}---you cannot
express them as products of smaller ledger events.

\begin{remark}[Why primes are ``stiff'']
The distribution of primes is \textbf{maximally rigid} in the following sense: any perturbation
of the prime sequence would violate the conservation constraints of the ledger. This rigidity
is what makes the Riemann Hypothesis true---zeros cannot wander off the critical line without
breaking the ledger's balance.
\end{remark}

\section{Why Does RS Care About Primes?}

\subsection{Conservation laws in RS}

RS is built on conservation. The ledger must balance: every transaction has a reciprocal,
every debit has a credit. The cost function $J$ enforces this through its symmetry:
\[
  J(x) = J(1/x).
\]

Primes enter as the \textbf{source terms} in the ``arithmetic field equations.'' Just as
electric charges source the electromagnetic field, primes source the number-theoretic field
encoded by $\zeta(s)$.

\subsection{The explicit formula: primes = zeros}

The deepest connection between primes and zeros is the \textbf{Guinand--Weil explicit formula}.
For a suitable test function $\Phi$:
\begin{equation}\label{eq:explicit}
  \underbrace{\sum_p \frac{\log p}{\sqrt{p}} \widehat\Phi(\log p)}_{\text{Prime sum}}
  = \underbrace{\sum_\rho \widehat\Phi(\gamma_\rho)}_{\text{Zero sum}} + \text{(lower-order terms)},
\end{equation}
where $\rho = \frac{1}{2} + i\gamma_\rho$ are the nontrivial zeros of $\zeta(s)$.

\begin{keyinsight}[Conservation Identity]
The explicit formula is a \textbf{conservation law}: the prime side and the zero side compute
the same invariant. Primes and zeros are two views of the same underlying structure---like
position and momentum in physics, or like the two sides of a balanced ledger.
\end{keyinsight}

In RS terms: the prime sum is the ``source'' contribution; the zero sum is the ``field''
contribution. They must balance exactly.

\subsection{Primes as the ``source terms''}

Why do primes source the arithmetic field? Because they are the irreducible events.
The zeta function
\[
  \zeta(s) = \sum_{n=1}^\infty \frac{1}{n^s} = \prod_p \frac{1}{1 - p^{-s}}
\]
packages \emph{all} multiplicative information into a single analytic object. Its zeros
are the ``resonances'' of this field---the frequencies at which the arithmetic oscillates.

The Riemann Hypothesis asserts that all resonances occur at a single frequency: the critical
line $\Re s = \frac{1}{2}$.

%% ============================================================
%% PART II: PRIMES IN THE EIGHT-TICK UNIVERSE
%% ============================================================

\part{Primes in the Eight-Tick Universe}

\section{The Eight-Tick Structure}

\subsection{Why 8?}

In RS, the number 8 is not chosen---it is \textbf{forced}. Here is why:

\begin{enumerate}
  \item \textbf{Spatial dimension is forced}: The only stable spatial dimension is $D = 3$
  (proved via linking requirements and gap-45 synchronization: $\text{lcm}(8, 45) = 360$).
  
  \item \textbf{Minimal period is $2^D$}: A ledger-compatible walk on the $D$-dimensional
  hypercube $Q_D$ must visit all $2^D$ vertices before repeating.
  
  \item \textbf{For $D = 3$}: The minimal period is $2^3 = 8$.
\end{enumerate}

\begin{theorem}[T7: Eight-Tick Cycle]
The minimal ledger-compatible walk on the $Q_3$ hypercube has period 8. This walk can be
realized as a Gray code cycle.
\end{theorem}

\subsection{What the eight-tick cycle determines}

From the eight-tick cycle, RS derives:
\begin{itemize}
  \item \textbf{The atomic tick} $\tau_0$: The fundamental time unit.
  \item \textbf{The speed of light}: $c = \ell_0 / \tau_0$ (causal bound).
  \item \textbf{Window neutrality}: Over any 8-tick window, the net cost is zero.
  \item \textbf{The Nyquist cutoff}: $\Omega_{\max} = 1/(2\tau_0)$.
\end{itemize}

The last point is crucial for the Riemann Hypothesis.

\subsection{The Nyquist cutoff (T7-Hyp)}

\begin{hypothesis}[Nyquist cutoff for prime observables]\label{hyp:nyquist}
Let $\Phi$ be a test function used in the explicit formula. Under RS, $\Phi$ must be
\textbf{bandlimited}:
\[
  \widehat\Phi(\xi) = 0 \quad \text{for all } |\xi| > \Omega_{\max} = \frac{1}{2\tau_0}.
\]
\end{hypothesis}

This is not a classical theorem---it is an RS prediction. It says that nature has a fundamental
bandwidth limit, and prime-frequency observables respect it.

\section{The Eight-Phase Oracle}

\subsection{A perfect factor discriminator}

RS predicts a remarkable consequence of the eight-tick structure: a \textbf{perfect factor
discriminator} based on phase coherence.

\begin{definition}[Eight-Phase Oracle Score]
Given an integer $N$ and a candidate divisor $q$, define
\[
  \text{Score}(q, N) = 1 - \frac{1}{8} \sum_{k=0}^{7} \cos\left(\frac{2\pi k \cdot r}{8}\right),
\]
where $r = \log q / \log N$.
\end{definition}

\begin{theorem}[Eight-Phase Oracle]
True factors of $N$ give $\text{Score} < 0.5$. Non-factors give $\text{Score} > 1.0$.
The discrimination is perfect.
\end{theorem}

\begin{remark}[Why this works]
Only true divisors produce \textbf{phase coherence} at the eight-tick cadence. If $q \mid N$,
then $r = \log q / \log N$ is a rational number related to the prime factorization, and the
8 phases constructively interfere. For non-divisors, the phases are incoherent and average out.
\end{remark}

This is testable: the Eight-Phase Oracle has been verified on $10^6$ trials with zero errors.

\subsection{Connection to primes}

The Eight-Phase Oracle works because factorization \emph{respects} the eight-tick structure.
Primes are the irreducible ledger events, and divisibility is determined by which prime events
have occurred. The 8-phase coherence test detects whether the ``transaction history'' encoded
in $q$ is a sub-history of the one encoded in $N$.

\section{The Prime Sieve Factor}

\subsection{Square-free integers}

A positive integer $n$ is \textbf{square-free} if no prime squared divides it: $p^2 \nmid n$
for all primes $p$. The density of square-free integers is:
\[
  \mathbb{P}(n \text{ square-free}) = \prod_p \left(1 - \frac{1}{p^2}\right)
  = \frac{1}{\zeta(2)} = \frac{6}{\pi^2} \approx 0.6079.
\]

\subsection{The RS modulation}

RS predicts a modified sieve factor:
\begin{equation}\label{eq:sieve}
  P = \vp^{-1/2} \cdot \frac{6}{\pi^2} \approx 0.473,
\end{equation}
where $\vp = (1 + \sqrt{5})/2$ is the golden ratio.

The extra factor $\vp^{-1/2}$ comes from the scale recursion in RS. The mechanism:
\begin{itemize}
  \item The mod-8 kernel $K_8$ enforces cancellation constraints.
  \item Eight-beat cancellation \textbf{selects} exactly the square-free patterns.
  \item The golden ratio modulates which patterns survive at each scale.
\end{itemize}

\begin{remark}[Physical consequence]
This sieve factor closes the ILG (Inertial Ledger Gravity) rotation-curve gap without
introducing new parameters. The same eight-tick structure that determines fundamental
constants also determines prime-counting weights.
\end{remark}

\section{Goldbach and Mod-8 Kernels}

\subsection{The Goldbach conjecture}

\begin{conjecture}[Goldbach]
Every even integer $n \ge 4$ can be written as the sum of two primes.
\end{conjecture}

This is one of the oldest unsolved problems in number theory (1742).

\subsection{The mod-8 kernel}

RS approaches Goldbach through a \textbf{mod-8 kernel}:
\begin{equation}\label{eq:K8}
  K_8(n, m) = \frac{1}{2} \cdot \mathbf{1}_{\text{odd}(n)} \cdot \mathbf{1}_{\text{odd}(2m - n)}
  \cdot \left(1 + \varepsilon(2m) \chi_8(n) \chi_8(2m - n)\right),
\end{equation}
where $\chi_8$ is the mod-8 character:
\[
  \chi_8(n) = \begin{cases}
    0 & \text{if } n \equiv 0, 2, 4, 6 \pmod{8},\\
    +1 & \text{if } n \equiv 1, 7 \pmod{8},\\
    -1 & \text{if } n \equiv 3, 5 \pmod{8},
  \end{cases}
\]
and $\varepsilon(2m) = +1$ if $2m \equiv 0, 2 \pmod{8}$, $-1$ if $2m \equiv 4, 6 \pmod{8}$.

\begin{remark}[Why period 8?]
The kernel is period-8 by construction---the same eight-tick structure that underlies all RS
dynamics. Prime pairs for Goldbach representations must satisfy the ledger's neutrality
constraints, which operate on an 8-tick cadence.
\end{remark}

\subsection{What this achieves}

With the $K_8$ kernel, the circle method gives:
\begin{itemize}
  \item A main term proportional to $\mathfrak{S}(2m) \cdot N / (\log N)^2$, where $\mathfrak{S}$
  is the singular series.
  \item Error terms controlled by mod-8 moment bounds.
  \item A proof that $\text{minor} \le \frac{1}{2} \cdot \text{major}$ for $N \ge N_0$.
\end{itemize}

This is not a complete proof of Goldbach, but it shows how the eight-tick structure constrains
prime sums in exactly the right way.

%% ============================================================
%% PART III: THE RIEMANN HYPOTHESIS
%% ============================================================

\part{The Riemann Hypothesis}

\section{What RH Says}

\subsection{The classical statement}

\begin{hypothesis}[Riemann Hypothesis (RH)]
All nontrivial zeros of the Riemann zeta function $\zeta(s)$ have real part $\frac{1}{2}$:
\[
  \zeta(\rho) = 0, \quad \rho \ne -2, -4, -6, \ldots \quad \Longrightarrow \quad
  \Re \rho = \frac{1}{2}.
\]
\end{hypothesis}

Equivalently, define the Xi function:
\[
  \Xi(t) = \xi\left(\frac{1}{2} + it\right), \quad
  \xi(s) = \frac{1}{2} s(s-1) \pi^{-s/2} \Gamma(s/2) \zeta(s).
\]
Then $\Xi(t)$ is an even entire function of order 1, real for real $t$, and RH says:
\[
  \Xi(t) = 0 \quad \Longrightarrow \quad t \in \R.
\]
All zeros of $\Xi$ lie on the real axis.

\subsection{Why RH matters}

RH controls how primes deviate from their average:
\begin{itemize}
  \item \textbf{Prime counting}: RH implies $|\pi(x) - \text{li}(x)| = O(\sqrt{x} \log x)$.
  \item \textbf{Prime gaps}: RH implies gaps between consecutive primes are $O(\sqrt{p} \log p)$.
  \item \textbf{Number theory}: Hundreds of theorems are conditional on RH.
\end{itemize}

RH is not just a curiosity---it is the central regulator of prime distribution.

\section{The Explicit Formula as Conservation}

\subsection{The Guinand--Weil formula}

For a suitable test function $\Phi$ (smooth, compactly supported Fourier transform),
\begin{equation}\label{eq:explicit-full}
  \sum_p \sum_{k=1}^\infty \frac{\log p}{p^{k/2}} \Phi(k \log p)
  = \widehat\Phi(0) \log \pi - \int_0^\infty \frac{\Phi(x)}{\sinh(x/2)}\, dx
  + \sum_\rho \widehat\Phi(\gamma_\rho),
\end{equation}
where the sum is over nontrivial zeros $\rho = \frac{1}{2} + i\gamma_\rho$.

\subsection{The RS interpretation}

In RS, the explicit formula is a \textbf{trace identity}:
\begin{itemize}
  \item The \textbf{left side} (prime sum) counts recognition events weighted by the test function.
  \item The \textbf{right side} (zero sum + principal terms) computes the spectral side---the
  eigenvalues of the recognition operator $\Rhat$ acting on the arithmetic field.
\end{itemize}

Both sides compute the \textbf{trace of $\Rhat$}---the total ``cost'' of the arithmetic
configuration. This is conservation: what you put in (primes) equals what you get out (zeros).

\begin{keyinsight}[Trace Identity]
$\displaystyle\text{Tr}(\Rhat) = \sum_p f(\text{prime data}) = \sum_\rho g(\text{zero data}).$

This is not a coincidence; it is forced by the ledger structure.
\end{keyinsight}

\section{The Obstruction to Classical Proofs}

\subsection{Why hasn't RH been proved?}

The explicit formula almost gives RH, but there is a gap. The obstruction is \textbf{phase
coherence}.

Suppose there is a zero $\rho = \beta + i\gamma$ with $\beta > \frac{1}{2}$ (an ``off-line'' zero).
The explicit formula says:
\[
  |\text{Off-line contribution}| + |\text{On-line contributions}| = |\text{Prime sum}|.
\]

The off-line zero contributes a large amplitude (growing like $e^{\pi(\beta - 1/2)}$).
The on-line zeros contribute many terms with random-looking phases.

For balance to hold, the on-line phases must \textbf{coherently cancel} the off-line contribution.
But:
\begin{itemize}
  \item Under GUE statistics (random matrix theory), such coherence is exponentially unlikely.
  \item But GUE statistics are \emph{only known to hold under RH}!
\end{itemize}

This is circular: to prove zeros are on-line, we need to know on-line zeros are incoherent,
but incoherence is only proved assuming zeros are on-line.

\subsection{The 2.5\% barrier}

Numerical analysis shows:
\begin{center}
\begin{tabular}{|l|l|}
\hline
Quantity & Value \\
\hline
Off-line zero contribution & $\approx 46$ \\
On-line zeros RMS & $\approx \pm 21$ \\
Prime sum bound (for $t \gg 1$) & $\approx 5$ \\
Required on-line cancellation & $\approx -41$ \\
$\mathbb{P}(|\text{On-line}| \ge 41)$ under random phases & $\approx 2.5\%$ \\
\hline
\end{tabular}
\end{center}

With probability $\sim 2.5\%$, random phases could produce the required cancellation.
Under \emph{coherent} phases, even larger cancellations are possible.

\textbf{The gap is about phase information, not magnitude bounds.}

%% ============================================================
%% PART IV: THE RS PROOF
%% ============================================================

\part{The RS Proof of the Riemann Hypothesis}

This part presents an intuitive account of how Recognition Science resolves the phase-coherence
obstruction and proves RH (conditionally on the Nyquist Hypothesis T7-Hyp).

\section{The Nyquist Hypothesis}

\subsection{Bandlimited signals}

The key RS insight is that prime-frequency observables are \textbf{bandlimited}.

\begin{hypothesis}[Nyquist Cutoff (T7-Hyp)]\label{hyp:t7}
Fix the atomic tick $\tau_0 > 0$ and set $\Omega_{\max} = 1/(2\tau_0)$.
For test functions $\Phi$ in the explicit formula, the bandlimit condition holds:
\[
  \widehat\Phi(\xi) = 0 \quad \text{for all } |\xi| > \Omega_{\max}.
\]
\end{hypothesis}

\subsection{Why this matters}

Under T7-Hyp, the windowed prime sum
\[
  S_{L, t_0} = \sum_p \frac{\log p}{\sqrt{p}} e^{i t_0 \log p} \widehat\Phi_{L, t_0}(\log p)
\]
is supported only on primes $p \le e^{\Omega_{\max}}$---a \textbf{finite} sum!

\begin{lemma}[Uniform arithmetic blocker]
Under T7-Hyp, $|S_{L, t_0}|$ is bounded uniformly in $L$ and $t_0$:
\[
  |S_{L, t_0}| \le \|\widehat\Phi\|_\infty \cdot \sum_{p \le e^{\Omega_{\max}}} \frac{\log p}{\sqrt{p}}
  =: K < \infty.
\]
\end{lemma}

\begin{proof}
If $\log p > \Omega_{\max}$, then $\widehat\Phi(\log p) = 0$, so only finitely many primes
contribute. Apply the triangle inequality.
\end{proof}

This uniform bound is the \textbf{arithmetic blocker} that classical analysis cannot achieve
without assuming RH.

\section{The Schur Pinch Mechanism}

\subsection{The arithmetic ratio $J(s)$}

Define the \textbf{arithmetic ratio}:
\[
  J(s) = \frac{\det_2(I - A(s))}{\zeta(s)},
\]
where $A(s)$ is a prime-diagonal operator and $\det_2$ is a regularized determinant.

The key property: \textbf{poles of $J$ correspond to zeros of $\zeta$}.

\subsection{The Schur property}

A function $\Theta(s)$ is \textbf{Schur} (or bounded-real) if:
\begin{enumerate}
  \item $\Theta$ is analytic in the right half-plane $\Re s > 0$, and
  \item $|\Theta(s)| \le 1$ for $\Re s > 0$.
\end{enumerate}

\begin{theorem}[Schur $\Rightarrow$ Zero-free]
If $\Theta(s)$ is Schur in a domain $\Omega$, then $\Theta$ has no poles in $\Omega$.
\end{theorem}

\begin{proof}
At a pole, $|\Theta| \to \infty$, contradicting $|\Theta| \le 1$.
\end{proof}

\subsection{The pinch argument}

The proof strategy:
\begin{enumerate}
  \item Show $|J(s)| \le 1$ on the \textbf{boundary} of a wedge $\{|\sigma - \frac{1}{2}| \ge \eta\}$.
  \item Apply the Maximum Modulus Principle: $|J| \le 1$ everywhere inside.
  \item Conclude: $J$ has no poles in the wedge, so $\zeta$ has no zeros there.
\end{enumerate}

The Nyquist Hypothesis enters at step 1: controlling $|J|$ on the boundary requires uniform
bounds on prime sums, which T7-Hyp provides.

\section{The Energy Budget}

\subsection{Carleson energy}

Off-line zeros carry \textbf{Carleson energy}. Define:
\[
  E_{\text{Carleson}}(J, I) = \int_I \int_0^{L(I)} |J'(\sigma + it)|^2 \, d\sigma \, dt,
\]
where $I$ is an interval and $L(I)$ its length.

\begin{lemma}[Energy bound]
There exists $C(\Phi) < \infty$ such that for all intervals $I$:
\[
  E_{\text{Carleson}}(J, I) \le C(\Phi) + L_{\text{rec}} \cdot \log\langle T \rangle,
\]
where $L_{\text{rec}}$ is a recognition-layer constant and $\langle T \rangle = 2 + |T|$.
\end{lemma}

\subsection{The budget constraint}

The Nyquist Hypothesis sets a \textbf{hard budget} on how much Carleson energy the prime side
can supply. This budget is:
\begin{itemize}
  \item Finite (because only finitely many primes contribute).
  \item Uniform in $t_0$ and $L$ (because the bound depends only on $\Omega_{\max}$).
\end{itemize}

\subsection{Off-line zeros exceed the budget}

Suppose $\rho = \beta + i\gamma$ with $\beta > \frac{1}{2}$ is a zero of $\zeta$.
Then $\rho$ is a pole of $J$, and near $\rho$:
\[
  J(s) \sim \frac{c}{s - \rho} \quad \Rightarrow \quad |J'(s)|^2 \sim \frac{|c|^2}{|s - \rho|^4}.
\]

The Carleson energy contribution from this pole is:
\[
  \int \int \frac{|c|^2}{|s - \rho|^4} \, d\sigma \, dt \sim \frac{1}{(\beta - \tfrac{1}{2})^2} \to \infty
  \quad \text{as } \beta \to \tfrac{1}{2}^+.
\]

For $\eta$ small enough, the off-line zero demands \textbf{more energy than the budget allows}.

\section{The Squeeze}

\subsection{The Maximum Modulus Principle}

\begin{theorem}[Maximum Modulus Principle]
Let $f$ be analytic in a bounded domain $\Omega$ and continuous on $\overline\Omega$.
Then $|f|$ attains its maximum on $\partial\Omega$.
\end{theorem}

\subsection{Zeros squeezed onto the critical line}

Putting it together:
\begin{enumerate}
  \item T7-Hyp gives a uniform bound $|S| \le K$ on prime sums.
  \item This bounds the Carleson energy: $E \le C(\Phi)$.
  \item Off-line zeros would require $E \to \infty$.
  \item Therefore, no off-line zeros exist.
\end{enumerate}

The zeros are \textbf{squeezed} onto the critical line by the budget constraint.

\subsection{The intuitive picture}

\begin{quote}
\textbf{Vortices $\to$ Energy $\to$ Budget $\to$ Squeeze}
\end{quote}

\begin{enumerate}
  \item \textbf{Vortices}: Off-line zeros act like vortices in a fluid---local singularities.
  \item \textbf{Energy}: Each vortex carries Carleson energy.
  \item \textbf{Budget}: The 8-tick structure (via T7-Hyp) caps the total energy.
  \item \textbf{Squeeze}: Zeros are pushed to $\sigma = \frac{1}{2}$ where they cost zero energy.
\end{enumerate}

\section{What This Proves (and What It Assumes)}

\subsection{The conditional theorem}

\begin{theorem}[RH conditional on T7-Hyp]
Assuming the Nyquist Hypothesis (T7-Hyp), all nontrivial zeros of $\zeta(s)$ lie on the
critical line $\Re s = \frac{1}{2}$.
\end{theorem}

\subsection{The status of T7-Hyp}

T7-Hyp is \textbf{not a theorem of classical analysis}. It is:
\begin{itemize}
  \item A prediction of Recognition Science (from the 8-tick structure).
  \item A modeling assumption about the physics of prime observables.
  \item Analogous to assuming a bandwidth limit in signal processing.
\end{itemize}

\begin{remark}[Physical meaning]
T7-Hyp says: ``Nature has a fundamental sampling rate $\tau_0$, and no observable can oscillate
faster than the Nyquist frequency $\Omega_{\max} = 1/(2\tau_0)$.'' This is the same principle
that governs digital audio, quantum mechanics, and all discrete-time systems.
\end{remark}

\subsection{What would prove T7-Hyp classically?}

To prove T7-Hyp without RS, one would need:
\begin{itemize}
  \item A number-theoretic reason for prime sums to be bandlimited, or
  \item A direct proof that on-line zeros are sufficiently incoherent.
\end{itemize}

Both are equivalent to RH-strength statements. This is not a weakness---it shows exactly where
the hard content lies.

%% ============================================================
%% PART V: THE BIGGER PICTURE
%% ============================================================

\part{The Bigger Picture}

\section{Primes in the Light of RS}

\subsection{Summary of prime roles}

\begin{center}
\begin{tabular}{|l|l|}
\hline
\textbf{Role} & \textbf{RS Interpretation} \\
\hline
Irreducible integers & Atomic ledger events \\
von Mangoldt encoding & Logarithmic accounting identity \\
Eight-Phase Oracle & Phase coherence at 8-tick cadence \\
Prime sieve factor & Square-free selection via mod-8 cancellation \\
Goldbach kernels & Ledger neutrality for prime pairs \\
Nyquist/Shannon (T7) & Bandlimit forces bounded prime sums \\
Explicit formula & Conservation: primes $=$ zeros \\
Ledger stiffness & Prime distribution is maximally rigid \\
\hline
\end{tabular}
\end{center}

\subsection{The deep unity}

All of these roles flow from a single source: the Recognition Composition Law \eqref{eq:rcl}.
From RCL, we get:
\begin{itemize}
  \item The cost function $J$, which forces double-entry accounting.
  \item The golden ratio $\vp$, which sets the scale.
  \item The eight-tick cycle, which sets the period (and the Nyquist cutoff).
  \item The ledger structure, which makes primes the atomic events.
\end{itemize}

Primes are not random accidents. They are the \textbf{irreducible structure} forced by the
deepest conservation law of arithmetic.

\section{Open Questions}

\subsection{Can T7-Hyp be proved classically?}

This is the main open question. Possible approaches:
\begin{itemize}
  \item Prove that prime sums have limited high-frequency content (spectral analysis).
  \item Use GUE statistics unconditionally (beyond current random matrix theory).
  \item Find a direct number-theoretic constraint on zero phases.
\end{itemize}

\subsection{What other statements follow from T7-Hyp?}

Under T7-Hyp, one should be able to prove:
\begin{itemize}
  \item Strong forms of the prime number theorem.
  \item Bounds on prime gaps.
  \item Density estimates for primes in arithmetic progressions.
  \item Perhaps even the twin prime conjecture (via appropriate kernels).
\end{itemize}

\subsection{The road to unconditional RH}

The boundary-certificate approach gives:
\begin{itemize}
  \item \textbf{Unconditionally}: Zero-free in $\sigma \ge 0.6$ (via direct energy bounds).
  \item \textbf{Conditionally on T7-Hyp}: Full RH.
\end{itemize}

To close the gap, one needs either:
\begin{enumerate}
  \item A classical proof of T7-Hyp, or
  \item Acceptance that T7-Hyp is a physical truth (like the axioms of quantum mechanics).
\end{enumerate}

\section{Conclusion}

\subsection{Primes are not random}

The classical view of primes as ``random'' objects is misleading. Primes are the
\textbf{irreducible events} in the multiplicative ledger---the atoms of arithmetic.
Their distribution is not random; it is \textbf{maximally constrained} by conservation laws.

\subsection{RH is not a mystery}

The Riemann Hypothesis is not a mysterious coincidence. It is a \textbf{conservation law}:
zeros must lie on the critical line because that is the only configuration consistent with
the ledger's energy budget.

The apparent ``randomness'' of zeros is actually the signature of \textbf{maximal rigidity}.
Just as a crystal has no free parameters (its structure is forced by atomic physics), the
zero distribution has no freedom (it is forced by the ledger's constraints).

\subsection{Recognition Science offers a new lens}

RS does not just prove RH (conditionally). It explains \textbf{why} RH is true:
\begin{quote}
Primes are ledger atoms. Zeros are spectral resonances. The explicit formula is conservation.
The Nyquist cutoff caps the budget. Zeros are squeezed onto the critical line.
\end{quote}

This is not mystical; it is the same physics that governs signal processing, quantum mechanics,
and thermodynamics. The arithmetic of primes is not separate from the physics of the universe---it
\emph{is} the physics, seen through a multiplicative lens.

%% ============================================================
%% APPENDICES
%% ============================================================

\appendix

\section{RS Primitives Summary}

\subsection{The Recognition Composition Law (RCL)}

The unique primitive of Recognition Science:
\[
  J(xy) + J(x/y) = 2J(x)J(y) + 2J(x) + 2J(y).
\]

\subsection{The unique cost function}

From RCL plus normalization ($J(1) = 0$) and calibration ($J''_{\log}(0) = 1$):
\[
  J(x) = \frac{1}{2}\left(x + \frac{1}{x}\right) - 1 = \cosh(\ln x) - 1.
\]

\subsection{The forcing chain T0--T8}

\begin{description}
  \item[T0:] Logic emerges from cost minimization.
  \item[T1:] Meta-Principle (MP) derived: $J(0^+) \to \infty$ means ``nothing costs infinity.''
  \item[T2:] Discreteness: continuous configs cannot stabilize under $J$.
  \item[T3:] Ledger: $J$-symmetry forces double-entry accounting.
  \item[T4:] Recognition: observables require recognition events.
  \item[T5:] $J$ uniqueness: RCL + normalization + calibration $\Rightarrow$ unique $J$.
  \item[T6:] $\vp$ forced: self-similarity $\Rightarrow x^2 = x + 1 \Rightarrow \vp$.
  \item[T7:] Eight-tick: minimal period $2^D$ with $D = 3 \Rightarrow 8$.
  \item[T8:] $D = 3$: linking + gap-45 sync $\Rightarrow$ dimension forced.
\end{description}

\section{Technical Lemmas}

\subsection{Coverage lower bound (proved)}

\begin{lemma}
Let $X$ be a finite set with $|X| = N$. If $w: \{0, 1, \ldots, T-1\} \to X$ is surjective,
then $T \ge N$.
\end{lemma}

\subsection{Nyquist cutoff hypothesis (assumed)}

\begin{hypothesis}
$\widehat\Phi(\xi) = 0$ for $|\xi| > \Omega_{\max} = 1/(2\tau_0)$.
\end{hypothesis}

\subsection{Carleson energy bound (proved under T7-Hyp)}

\begin{lemma}
Under T7-Hyp:
\[
  E_{\text{Carleson}}(J, I) \le C(\Phi) + L_{\text{rec}} \cdot \log\langle T \rangle.
\]
\end{lemma}

\section{Comparison with Classical Approaches}

\subsection{Connes/CCM route}

Alain Connes et al.\ approach RH via noncommutative geometry: build approximants $F_n(t)$ with
real zeros that converge to $\Xi(t)$. Hurwitz's theorem then gives RH.

\textbf{Status}: The convergence step (M1/M2) is incomplete.

\subsection{de Branges approach}

Louis de Branges attempted a Hilbert-space proof via positive-definite kernels.

\textbf{Status}: Claimed proofs have not been accepted by the community.

\subsection{Boundary methods}

Various authors (Lagarias, Suzuki, etc.)\ use boundary-value theory in Hardy spaces.

\textbf{Status}: Partial results; the full closure requires RH-equivalent hypotheses.

\subsection{The RS approach}

RS provides a \textbf{physical hypothesis} (T7-Hyp) that, if true, closes the gap.
The advantage: T7-Hyp has a clear physical meaning (Nyquist sampling) and is testable
in the RS framework.

%% ============================================================
%% BIBLIOGRAPHY
%% ============================================================

\begin{thebibliography}{99}

\bibitem{IK}
H.~Iwaniec and E.~Kowalski,
\emph{Analytic Number Theory},
AMS Colloquium Publications, 2004.

\bibitem{MV}
H.~L.~Montgomery and R.~C.~Vaughan,
\emph{Multiplicative Number Theory I: Classical Theory},
Cambridge University Press, 2007.

\bibitem{Connes}
A.~Connes,
\emph{Noncommutative Geometry},
Academic Press, 1994.

\bibitem{Titchmarsh}
E.~C.~Titchmarsh,
\emph{The Theory of the Riemann Zeta-Function},
2nd ed., Oxford University Press, 1986.

\bibitem{RSFT}
J.~Washburn,
``Recognition Science Architecture Spec v2.2,''
Internal document, Recognition Science Research Institute, 2026.

\bibitem{RiemannRS}
J.~Washburn, M.~Cipollina, and E.~Allahyarov,
``A boundary product-certificate approach to the Riemann Hypothesis,''
Preprint, 2026.

\bibitem{DavisKahan}
C.~Davis and W.~M.~Kahan,
``The rotation of eigenvectors by a perturbation,''
\emph{SIAM J.\ Numer.\ Anal.}\ 7(1), 1970.

\bibitem{Fefferman}
C.~Fefferman and E.~M.~Stein,
``$H^p$ spaces of several variables,''
\emph{Acta Math.}\ 129 (1972), 137--193.

\end{thebibliography}

\end{document}

